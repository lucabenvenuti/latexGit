\section{Results and discussion}
\label{sec:results}
%************************************************

\subsection{Experiments}
\label{subsec:experiments}

Initially, experimental values identifying the bulk behavior, $\mu_{psh}$, $\mu_{sh}$ and $\rho_{b}$, 
for sinter fine have been acquired through the $SRSCT$, e.g. in Table
\ref{tab:05sinterTableExperimental}
these values for three load conditions are presented.
In the $\mu_{psh}$ a descending path is evident. 
Instead, the $\mu_{sh}$ is oscillating.
The $\rho_b$ presents a clear average of $1760 [kg/m^3]$ with a $42 [kg/m^3]$
deviation.
The stress path for the second load condition of Table
\ref{tab:05sinterTableExperimental} is shown in Fig.
\ref{fig:20experimental}.
In the first 300 seconds the $\sigma_n$ was kept constant. After 250 seconds a
plateau was reached. 
The $\mu_{psh}$ was calculated as average of the $\mu_{ie}$ in this plateau.
Later, the $\sigma_n$ was reduced to $80 \%$ of its initial value.
After approximately 30 seconds, a second plateau started.
As average of $\mu_{ie}$ in this second plateau we obtained $\mu_{sh}$.
%\begin{table}[h]
\centering
\begin{tabular}{cccccc}
$\sigma_n$ [Pa] & $\tau$ [Pa] & $\mu_{psh}$ [-] & $\mu_{sh}$ [-] &
$\rho_b$ [kg/m3] & AOR $\circ$ \\
\hline
    1068  & 1059  & 0.9916 & 0.9916 & 1718  & 38.85 \\
    2069  & 1818  & 0.8787 & 0.8787 & 1759  & 38.85 \\
    10070 & 8232  & 0.8175 & 0.8175 & 1802  & 38.85 \\

\hline
\end{tabular}
\caption{Experimental values for sinter fine}
\label{tab:05sinterTableExperimental}
\end{table}
Later, two $AOR$ test have been performed, thus identifying an average angle of
$38.85 ^\circ$.
We also realized the sieving, obtaining the radius ($R$) mean and standard
deviation, already shown in Table \ref{tab:09DEMFixedinputvalues}.
\begin{figure}[htp] \centering
    \begin{subfigure}[b]{2cm}
        \includegraphics[width=\textwidth]{images/original/20experimental}
        \caption{Experimental shear cell tester stress path - $\sigma_n = 2000
        [Pa]$}
        \label{fig:20experimental} 
    \end{subfigure}\\
        \begin{subfigure}[b]{2cm}
        \includegraphics[width=\textwidth]{images/original/21simexample}
        \caption{Numerical shear cell tester stress path - $\sigma_n = 10000
        [Pa]$}
        \label{fig:21simexample} 
    \end{subfigure}
    \caption[Stress path]{Sample of the stress path for
	the Schulze ring shear cell tester, experimental and numerical.
	Time is normalized: $\tilde{t} = t/t_{change}$, where $t_{change}$ is the
	time when the normal stress ($\sigma_n$) is modified during the tests.
	Until $\tilde{t}=1$ the $\sigma_n = 2000 ~[Pa]$ is kept constant. 
	In Fig. \ref{fig:20experimental} at $\tilde{t}~=0.91$
 	a plateau is reached.
	The $\mu_{psh}$ is calculated as average of the $\mu_{ie}$ in this first
	plateau.
	Later, at $\tilde{t}=1$, the $\sigma_n$ is reduced to $80 \%$ of its initial
	value.
	Soon, a second plateau starts.
	As average of $\mu_{ie}$ in this second plateau we obtain $\mu_{sh}$.
	The stress path is in the numerical simulation is comparable to the
	experimental one, especially the plateaux.
	They were clearly relevant because there we collected the numerical bulk
	behaviour representative values. }
    \label{fig:40experimentalsimulation}
\end{figure}

\begin{table}[h]
\centering
\begin{tabular}{cccccc}
$\sigma_n$ [Pa] & $\tau$ [Pa] & $\mu_{psh}$ [-] & $\mu_{sh}$ [-] &
$\rho_b$ [kg/m3] & AOR $\circ$ \\
\hline
    1068  & 1059  & 0.9916 & 0.9916 & 1718  & 38.85 \\
    2069  & 1818  & 0.8787 & 0.8787 & 1759  & 38.85 \\
    10070 & 8232  & 0.8175 & 0.8175 & 1802  & 38.85 \\

\hline
\end{tabular}
\caption{Experimental values for sinter fine}
\label{tab:05sinterTableExperimental}
\end{table}

\subsection{DEM Simulations}
\label{subsec:simulations}

For sinter fine 546 shear cell and 81 static angle of repose simulations have
been realized with the variations described in table
\ref{tab:10DEMVariableinputvalues}.
A representative stress path can be seen in Fig. \ref{fig:21simexample}.
Although the duration is smaller by two order of magnitude, the stress path is
comparable to the experimental one, especially the plateaux.
They were clearly relevant because there we collected the numerical bulk
behaviour representative values.\\
The computational time resulted in 1 hour with 32 AMD cores for a benchmark
shear cell simulation and 9 hours for a benchmark $AOR$ simulation, both with 50K particles. 
Simulations with large $dCylDp$ required a greater time amount (e.g. with 400K
particles about 12 hours for the shear cell). \\
%\begin{figure}[!htb] 
\centering 
\includegraphics[width=.96\textwidth]{images/original/21simexample} 
\caption[Numerical stress path]{NEW SHORTER GRAPH!!!!! Sample of the numerical
stress path for a shear cell tester simulation, $\sigma_n = 10000 ~[Pa]$.
Although the duration is smaller by two order of magnitude, the stress path is comparable to the experimental one, especially the plateaux.
They were clearly relevant because there we collected the numerical bulk
behaviour representative values.}
\label{fig:21simexample} 
\end{figure}


% \begin{figure}[htp]
%     \centering
%     \includegraphics[width=.2\textwidth]{images/vitae/lbenvenuti}
%     \caption{OpenMP, MPI, MPI/OpenMP Hybrid runs of Box in a box testcase on 32
%     cores. The OpenMP-only run suffers from limited memory bandwidth in
%     memory-bound algorithms inside of the Modify section of the code. MPI-only has
%     low averaged runtimes for each section, but a very large Other timing, which
%     hints for a large amount of load-imbalance. Hybrid timings are a bit worse
%     on average, but because of better balancing, processes have lower wait times
%     inside of Other timing.}
% 	\label{fig:boxInBoxComparison}


\subsection{ANN model development}
\label{subsec:annmodeldev}

First, we controlled the regression of the bulk behaviour parameters, e.g. the
$\mu_{psh}$, see Fig. \ref{fig:22regression}, where the corresponding plot for
the $NN$ with the maximum $R^2$ in shown. Each circle represents one of the 546
simulations.
The plot presents a consistent agreement between the $DEM$ results distribution
(T in the legend) and the $NN$ regression (or fitting) line.
% \begin{figure}[!h] 
\centering 
\includegraphics[width=.96\textwidth]{22regression}
%[width=.96\textwidth]
\caption[Comparison between prediction of the trained NN and full DEM
simulation]{Comparison between prediction of the trained Neural Network ($NN$)
and 546 full DEM simulations of the coefficient of pre-shear ($\mu_{psh}$). In
this case the regression line is nearly linear, and demonstrates the accurate
prediction of the $NN$.}
\label{fig:22regression} 
\end{figure}

% Regression plot - $\mu_{psh}$. In this case the
% $Output = 0.96 \cdot Target + 0.046$. With 546 simulations the $R^2 = 0.98044$. The plot
% presents a consistent agreement between the $DEM$ results distribution and the $NN$ regression line.
% \begin{figure}[htp]
%     \centering
%     \includegraphics[width=.2\textwidth]{images/vitae/lbenvenuti}
%     \caption{OpenMP, MPI, MPI/OpenMP Hybrid runs of Box in a box testcase on 32
%     cores. The OpenMP-only run suffers from limited memory bandwidth in
%     memory-bound algorithms inside of the Modify section of the code. MPI-only has
%     low averaged runtimes for each section, but a very large Other timing, which
%     hints for a large amount of load-imbalance. Hybrid timings are a bit worse
%     on average, but because of better balancing, processes have lower wait times
%     inside of Other timing.}
% 	\label{fig:boxInBoxComparison}

The linear relationship between the
training values have been evaluated in Table \ref{tab:06inputRelationshipTable}.
The clearest connections were between $\mu_s$ and $\mu_{psh}$, and
$\rho_p$ and $\rho_b$.
Instead, for $\mu_{sh}$ and $AOR$ the $\mu_r$ balanced the influence of the 
$\mu_s$, and further parameters were worthly correlated. \\
% \begin{table}[H!]                                                                                                                                                          
\centering                                                                                                                                                                 
\begin{tabular}{|c|c|c|c|c|c|c|c|c|c|c|c|}                                                                                                                                 
\hline                                                                                                                                                                     
 & sf & rf & rest & dt & dCylDp & ctrlStress & shearperc & dens & mush & mupsh & rhob \\                                                                                   
\hline                                                                                                                                                                     
sf & 1 & 5.549787e-03 & -3.818461e-04 & -1.268763e-15 & -1.628657e-02 & 1.282025e-15 & 4.517397e-03 & 0 & 3.838826e-02 & 8.725701e-01 & -8.393464e-02 \\                   
\hline                                                                                                                                                                     
rf & 5.549787e-03 & 1 & -1.523330e-03 & -2.349289e-15 & -5.968531e-02 & 2.322503e-15 & 1.802162e-02 & 3.348007e-18 & 5.891756e-01 & 3.370233e-01 & -3.101856e-02 \\        
\hline                                                                                                                                                                     
rest & -3.818461e-04 & -1.523330e-03 & 1 & -1.555718e-15 & -2.758674e-01 & 1.568359e-15 & 8.090707e-02 & 6.680307e-18 & 1.551852e-01 & -5.671687e-03 & -1.712429e-02 \\    
\hline                                                                                                                                                                     
dt & -1.268763e-15 & -2.349289e-15 & -1.555718e-15 & 1 & -1.026312e-16 & -1.000000e+00 & -2.681936e-17 & 0 & 6.168810e-16 & -4.320958e-15 & 1.243669e-14 \\                
\hline                                                                                                                                                                     
dCylDp & -1.628657e-02 & -5.968531e-02 & -2.758674e-01 & -1.026312e-16 & 1 & 7.853515e-17 & -2.939311e-01 & 2.688281e-17 & -2.879551e-01 & -1.916393e-01 & 9.603603e-02 \\ 
\hline                                                                                                                                                                     
ctrlStress & 1.282025e-15 & 2.322503e-15 & 1.568359e-15 & -1 & 7.853515e-17 & 1 & -3.731389e-17 & 0 & -6.100950e-16 & 4.292811e-15 & -1.234126e-14 \\                      
\hline                                                                                                                                                                     
shearperc & 4.517397e-03 & 1.802162e-02 & 8.090707e-02 & -2.681936e-17 & -2.939311e-01 & -3.731389e-17 & 1 & -3.512479e-17 & 5.730199e-02 & 5.380657e-02 & -5.095294e-03 \\
\hline                                                                                                                                                                     
dens & 0 & 3.348007e-18 & 6.680307e-18 & 0 & 2.688281e-17 & 0 & -3.512479e-17 & 1 & -4.980664e-02 & 5.709445e-02 & 9.900341e-01 \\                                         
\hline                                                                                                                                                                     
mush & 3.838826e-02 & 5.891756e-01 & 1.551852e-01 & 6.168810e-16 & -2.879551e-01 & -6.100950e-16 & 5.730199e-02 & -4.980664e-02 & 1 & 2.603411e-01 & -9.516313e-02 \\      
\hline                                                                                                                                                                     
mupsh & 8.725701e-01 & 3.370233e-01 & -5.671687e-03 & -4.320958e-15 & -1.916393e-01 & 4.292811e-15 & 5.380657e-02 & 5.709445e-02 & 2.603411e-01 & 1 & -4.329071e-02 \\     
\hline                                                                                                                                                                     
rhob & -8.393464e-02 & -3.101856e-02 & -1.712429e-02 & 1.243669e-14 & 9.603603e-02 & -1.234126e-14 & -5.095294e-03 & 9.900341e-01 & -9.516313e-02 & -4.329071e-02 & 1 \\   
\hline                                                                                                                                                                     
\end{tabular}                                                                                                                                                              
\caption{MyTableCaption}                                                                                                                                                   
\label{table:MyTableLabel}                                                                                                                                                 
\end{table}               
Then we observed how the $R^2$ changed with the different number of neurons for
the $\mu_{psh}$.
Then we observed how the $R^2$ changed with the different number of neurons for the $\mu_{psh}$. 
In this case we reached a $R^2 = 0.96$ for a $NN$ with fifteen neurons. 
Increasing the number of neurons did not improve the $R^2$, that even started to oscillate with the neuron number. 
Later, we processed the random combinations (table
\ref{tab:10DEMVariableinputvalues}) with the $NN$.
The $NN$ evaluation was incredibly faster compared to the $DEM$ simulations. The
individuation of all the tabbed $DEM$ combinations for the shear cell did not take more than a few seconds on a single core. 
We represented the tabbed combinations ($TC1$) for one load condition of the
shear cell ($\sigma_n=10070 ~[Pa]$, $P=1.0$) in Fig.
\ref{fig:24radarpirker1schulze10070}.
% \input{images/texCaller/29schulzeradarandcloud}
Here, the minimum and maximum values, together with the mean and the confidence
range, provided by the square deviation, are shown. Notably, the confidence range is large, 
especially for the $COR$, highlighting its scarce influence over the characterization. 
Instead, both the $\rho_p$  and the $\mu_s$ show a narrow confidence range, 
displaying at the same time their influence and the validity of this procedure to find valid $DEM$ parameters. 
That agrees with the examination of the ratio of the standard deviation to the
range, see table \ref{tab:13DEMvalidvalues}.
Further, we could see how different $DEM$ parameters
combinations could reproduce the experimental behaviour and evaluate their mutual dependencies. 
This is clearer in a cloud plot, as in Fig. 
\ref{fig:25cloudpirker1schulze10070}. While the $COR$ varied, multiple
combinations ($250407 --> 4\% $ of the total) of $\mu_s$ and $\mu_r$ reproduced
the experimental behaviour.
This underlines once more their correlation, as already stated by Wensrich and 
Katterfeld \cite{RefWorks:87}.
To further demonstrate the validity of the procedure, we modified the product
coefficient. In the first attempt we set it to $P=0.8$ and we obtained another
series of tabbed combinations ($TC2$).
We can see in the radar plot in Fig.
\ref{fig:26radarpirker08schulze10070} that the confidence range is narrower
compared to $P=1.0$, while in the cloud plot in Fig. 
\ref{fig:27cloudpirker08schulze10070} the area
appears larger, although slightly less densely populated. Finally, for $P=1.2$
and its tabbed combinations ($TC3$) the radar plot in Fig.
\ref{fig:28radarpirker12schulze10070} shows a largely different confidence
range, while the cloud plot in Fig. \ref{fig:30cloudpirker12schulze10070} 
illustrates a smaller area. As expected, the procedure was highly sensible to the variations of the experimental data. 
Thus, it could be effectively handled for a wide range of bulk materials.\\
% \begin{figure}[htp] \centering
    \begin{subfigure}[b]{0.48\columnwidth}
        \includegraphics[width=\textwidth]{images/original/31radarpirker1aor}
        \caption{Radar P1 AOR}
        \label{fig:31radarpirker1aor} 
    \end{subfigure}
    \begin{subfigure}[b]{0.48\columnwidth}
        \includegraphics[width=\textwidth]{images/original/32cloudpirker1aor}
        \caption{Cloud P1 AOR}
        \label{fig:32cloudpirker1aor} 
    \end{subfigure}\\
        \begin{subfigure}[b]{0.48\columnwidth}
        \includegraphics[width=\textwidth]{images/original/33radarpirker1schulze10070aor}
        \caption{Radar P1 Schulze 10070 & AOR}
        \label{fig:33radarpirker1schulze10070aor} 
    \end{subfigure}
    \begin{subfigure}[b]{0.48\columnwidth}
        \includegraphics[width=\textwidth]{images/original/34cloudpirker1schulze10070aor}
        \caption{Cloud P1 Schulze 10070 & AOR}
        \label{fig:34cloudpirker1schulze10070aor} 
    \end{subfigure}
    \caption{AOR and merge results.}
    \label{fig:35schulze10070aorradarandcloud}
\end{figure}
We then processed the random combinations with the $AOR$ $NN$. In Fig.
\ref{fig:31radarpirker1aor} the radar plot realized with the same criteria as
before can be seen.
In accordance with the theory (Wensrich and Katterfeld \cite{RefWorks:87}), in a simulation dominated
by the particles rolling the coefficient of rolling friction has the maximum influence. 
Further, in the cloud plot in Fig. \ref{fig:32cloudpirker1aor}
we could see that there are valid combinations also with slight $\mu_s$. \\
Finally, we extracted from the $TC1$ values the $AOR$ $NN$ behaviour
and compared it with the experimental one.
As can be seen in the radar plot in Fig.
\ref{fig:33radarpirker1schulze10070aor}, the confidence range is meager, indicating that all the parameters but the $COR$ 
had an important role and the reliability of these parameters combinations to represent the bulk behaviour. 
Also in the cloud plot in Fig. \ref{fig:34cloudpirker1schulze10070aor} we
observed a condensed distribution of valid parameters.
From the initial 6250000 combinations, only 3884 of them were valid (0.0621 \%),
see table \ref{tab:13DEMvalidvalues}.
\begin{figure}[!h] 
\centering 
\includegraphics[width=.96\textwidth]{22regression}
%[width=.96\textwidth]
\caption[Comparison between prediction of the trained NN and full DEM
simulation]{Comparison between prediction of the trained Neural Network ($NN$)
and 546 full DEM simulations of the coefficient of pre-shear ($\mu_{psh}$). In
this case the regression line is nearly linear, and demonstrates the accurate
prediction of the $NN$.}
\label{fig:22regression} 
\end{figure}

% Regression plot - $\mu_{psh}$. In this case the
% $Output = 0.96 \cdot Target + 0.046$. With 546 simulations the $R^2 = 0.98044$. The plot
% presents a consistent agreement between the $DEM$ results distribution and the $NN$ regression line.
% \begin{figure}[htp]
%     \centering
%     \includegraphics[width=.2\textwidth]{images/vitae/lbenvenuti}
%     \caption{OpenMP, MPI, MPI/OpenMP Hybrid runs of Box in a box testcase on 32
%     cores. The OpenMP-only run suffers from limited memory bandwidth in
%     memory-bound algorithms inside of the Modify section of the code. MPI-only has
%     low averaged runtimes for each section, but a very large Other timing, which
%     hints for a large amount of load-imbalance. Hybrid timings are a bit worse
%     on average, but because of better balancing, processes have lower wait times
%     inside of Other timing.}
% 	\label{fig:boxInBoxComparison}

\begin{table}[H!]                                                                                                                                                          
\centering                                                                                                                                                                 
\begin{tabular}{|c|c|c|c|c|c|c|c|c|c|c|c|}                                                                                                                                 
\hline                                                                                                                                                                     
 & sf & rf & rest & dt & dCylDp & ctrlStress & shearperc & dens & mush & mupsh & rhob \\                                                                                   
\hline                                                                                                                                                                     
sf & 1 & 5.549787e-03 & -3.818461e-04 & -1.268763e-15 & -1.628657e-02 & 1.282025e-15 & 4.517397e-03 & 0 & 3.838826e-02 & 8.725701e-01 & -8.393464e-02 \\                   
\hline                                                                                                                                                                     
rf & 5.549787e-03 & 1 & -1.523330e-03 & -2.349289e-15 & -5.968531e-02 & 2.322503e-15 & 1.802162e-02 & 3.348007e-18 & 5.891756e-01 & 3.370233e-01 & -3.101856e-02 \\        
\hline                                                                                                                                                                     
rest & -3.818461e-04 & -1.523330e-03 & 1 & -1.555718e-15 & -2.758674e-01 & 1.568359e-15 & 8.090707e-02 & 6.680307e-18 & 1.551852e-01 & -5.671687e-03 & -1.712429e-02 \\    
\hline                                                                                                                                                                     
dt & -1.268763e-15 & -2.349289e-15 & -1.555718e-15 & 1 & -1.026312e-16 & -1.000000e+00 & -2.681936e-17 & 0 & 6.168810e-16 & -4.320958e-15 & 1.243669e-14 \\                
\hline                                                                                                                                                                     
dCylDp & -1.628657e-02 & -5.968531e-02 & -2.758674e-01 & -1.026312e-16 & 1 & 7.853515e-17 & -2.939311e-01 & 2.688281e-17 & -2.879551e-01 & -1.916393e-01 & 9.603603e-02 \\ 
\hline                                                                                                                                                                     
ctrlStress & 1.282025e-15 & 2.322503e-15 & 1.568359e-15 & -1 & 7.853515e-17 & 1 & -3.731389e-17 & 0 & -6.100950e-16 & 4.292811e-15 & -1.234126e-14 \\                      
\hline                                                                                                                                                                     
shearperc & 4.517397e-03 & 1.802162e-02 & 8.090707e-02 & -2.681936e-17 & -2.939311e-01 & -3.731389e-17 & 1 & -3.512479e-17 & 5.730199e-02 & 5.380657e-02 & -5.095294e-03 \\
\hline                                                                                                                                                                     
dens & 0 & 3.348007e-18 & 6.680307e-18 & 0 & 2.688281e-17 & 0 & -3.512479e-17 & 1 & -4.980664e-02 & 5.709445e-02 & 9.900341e-01 \\                                         
\hline                                                                                                                                                                     
mush & 3.838826e-02 & 5.891756e-01 & 1.551852e-01 & 6.168810e-16 & -2.879551e-01 & -6.100950e-16 & 5.730199e-02 & -4.980664e-02 & 1 & 2.603411e-01 & -9.516313e-02 \\      
\hline                                                                                                                                                                     
mupsh & 8.725701e-01 & 3.370233e-01 & -5.671687e-03 & -4.320958e-15 & -1.916393e-01 & 4.292811e-15 & 5.380657e-02 & 5.709445e-02 & 2.603411e-01 & 1 & -4.329071e-02 \\     
\hline                                                                                                                                                                     
rhob & -8.393464e-02 & -3.101856e-02 & -1.712429e-02 & 1.243669e-14 & 9.603603e-02 & -1.234126e-14 & -5.095294e-03 & 9.900341e-01 & -9.516313e-02 & -4.329071e-02 & 1 \\   
\hline                                                                                                                                                                     
\end{tabular}                                                                                                                                                              
\caption{MyTableCaption}                                                                                                                                                   
\label{table:MyTableLabel}                                                                                                                                                 
\end{table}               
\begin{table}[h]
\centering
\scalebox{0.7}{
\begin{tabular}{llccc}
\hline

          & type  & SCT & AOR   & SCT \& AOR \\
          \hline

    $\mu_s$ & mean  & 0.831 & 0.177 & 0.664 \\
    $[-]$   & std. dev. (SD) & 0.097 & 0.095 & 0.029 \\
          & range ($R$) & 0.9   & 0.9   & 0.9 \\
          & SD / R & 0.108 & 0.106 & 0.032 \\
          \hline
    $\mu_r$ & mean  & 0.692 & 0.830 & 0.916 \\
    $[-]$   & std. dev. (SD) & 0.215 & 0.193 & 0.042 \\
          & range ($R$) & 0.9   & 0.9   & 0.9 \\
          & SD / R & 0.239 & 0.214 & 0.046 \\
          \hline
              COR   & mean  & 0.708 & 0.590 & 0.590 \\
   $ [-]$   & std. dev. (SD) & 0.104 & 0.073 & 0.065 \\
          & range ($R$) & 0.4   & 0.4   & 0.4 \\
          & SD / R & 0.259 & 0.183 & 0.161 \\
          \hline
    $\rho_p$ & mean  & 2245.7 & 3192.8 & 2283.9 \\
    $[kg/m3]$ & std. dev. (SD) & 80.5  & 277.4 & 67.1 \\
          & range ($R$) & 1500  & 1500  & 1500 \\
          & SD / R & 0.054 & 0.185 & 0.045 \\
          \hline
    valid & number & 290203 & 816552 & 3884 \\
    combinations & [$\%$] & 4.64  & 13.06 & 0.06 \\
    

\hline
\end{tabular}}
\caption{DEM valid values.}
\label{tab:13DEMvalidvalues}
\end{table}
%\begin{figure}[!h] 
\centering 
\includegraphics[width=.96\textwidth]{images/original/23regressiongraph}
%[width=.96\textwidth]
\caption[Regression graph]{Regression graph. The variation of $R^2$ for
$\mu_{psh}$ with the number of neurons in the NN is shown. The
straight line, referred only to the test simulations, is more sensible to
variations of number of neurons, compared
to the total line. In the latter, also the correlated simulations used for
training participate in the $R^2$ evaluation.}
\label{fig:23regressiongraph} 
\end{figure}

%SCT: sn = 10070 [Pa], coeff. P. = 1
% \begin{figure}[htp]
%     \centering
%     \includegraphics[width=.2\textwidth]{images/vitae/lbenvenuti}
%     \caption{OpenMP, MPI, MPI/OpenMP Hybrid runs of Box in a box testcase on 32
%     cores. The OpenMP-only run suffers from limited memory bandwidth in
%     memory-bound algorithms inside of the Modify section of the code. MPI-only has
%     low averaged runtimes for each section, but a very large Other timing, which
%     hints for a large amount of load-imbalance. Hybrid timings are a bit worse
%     on average, but because of better balancing, processes have lower wait times
%     inside of Other timing.}
% 	\label{fig:boxInBoxComparison}

%\input{images/texCaller/29schulzeradarandcloud}
%\begin{figure}[htp] \centering
    \begin{subfigure}[b]{0.48\columnwidth}
        \includegraphics[width=\textwidth]{images/original/31radarpirker1aor}
        \caption{Radar P1 AOR}
        \label{fig:31radarpirker1aor} 
    \end{subfigure}
    \begin{subfigure}[b]{0.48\columnwidth}
        \includegraphics[width=\textwidth]{images/original/32cloudpirker1aor}
        \caption{Cloud P1 AOR}
        \label{fig:32cloudpirker1aor} 
    \end{subfigure}\\
        \begin{subfigure}[b]{0.48\columnwidth}
        \includegraphics[width=\textwidth]{images/original/33radarpirker1schulze10070aor}
        \caption{Radar P1 Schulze 10070 & AOR}
        \label{fig:33radarpirker1schulze10070aor} 
    \end{subfigure}
    \begin{subfigure}[b]{0.48\columnwidth}
        \includegraphics[width=\textwidth]{images/original/34cloudpirker1schulze10070aor}
        \caption{Cloud P1 Schulze 10070 & AOR}
        \label{fig:34cloudpirker1schulze10070aor} 
    \end{subfigure}
    \caption{AOR and merge results.}
    \label{fig:35schulze10070aorradarandcloud}
\end{figure}
\begin{figure}[htp] \centering
        \begin{subfigure}[b]{0.5\columnwidth}
        \includegraphics[width=\textwidth]{26radarpirker08schulze10070}
        \caption{Parameter space plot, $SSC$, $\sigma_n=10070 ~[Pa]$, $P=0.8$}
        \label{fig:26radarpirker08schulze10070} 
    \end{subfigure}\\
     \begin{subfigure}[b]{0.5\columnwidth}
        \includegraphics[width=\textwidth]{24radarpirker1schulze10070}
        \caption{Parameter space plot, $SSC$, $\sigma_n=10070 ~[Pa]$, $P=1.0$}
        \label{fig:24radarpirker1schulze10070}
    \end{subfigure} \\
        \begin{subfigure}[b]{0.5\columnwidth}
        \includegraphics[width=\textwidth]{28radarpirker12schulze10070}
        \caption{Parameter space plot, $SSC$, $\sigma_n=10070 ~[Pa]$, $P=1.2$}
        \label{fig:28radarpirker12schulze10070} 
    \end{subfigure}
    \caption[Parameter space plot of valid simulations parameters for three different
    bulk behaviours measured by SCT]{Parameter space plot of valid simulations
    parameters for three different bulk behaviours measured by shear cell tester ($SSC$).
    Each axis of the parameter space plot represents one simulation parameters.
    Furthermore, the shaded area represents valid parameters combinations.
    Dark shaded values stand for the confidence range.
    We represent the marked combinations for one load condition of the shear
    cell.
    Further explanation in the text.
   }
    \label{fig:29schulzeradarandcloud}
\end{figure}
\begin{figure}[htp] \centering

    \begin{subfigure}[b]{0.96\columnwidth}
        \includegraphics[width=\textwidth]{images/original/27cloudpirker08schulze10070}
        \caption{Cloud plot, $SSC$, $\sigma_n=10070 ~[Pa]$, $P=0.8$}
        \label{fig:27cloudpirker08schulze10070} 
    \end{subfigure}\\
    \begin{subfigure}[b]{0.96\columnwidth}
        \includegraphics[width=\textwidth]{images/original/25cloudpirker1schulze10070}
        \caption{Cloud plot, $SSC$, $\sigma_n=10070 ~[Pa]$, $P=1.0$}
        \label{fig:25cloudpirker1schulze10070}
    \end{subfigure}\\

    \begin{subfigure}[b]{0.96\columnwidth}
        \includegraphics[width=\textwidth]{images/original/30cloudpirker12schulze10070}
        \caption{Cloud plot, $SSC$, $\sigma_n=10070 ~[Pa]$, $P=1.2$}
        \label{fig:30cloudpirker12schulze10070} 
    \end{subfigure}
    \caption[Density plot comparison of SCT results]{Density plot comparison of
    shear cell tester ($SSC$) results. We represent the marked combinations for
    one load condition of the shear cell. 
    Density plot of the particles' coefficient of restitution (COR) in dependence
	of coefficient of sliding friction and coefficient of rolling friction; in the
	white area no valid sets of simulation parameter can be found.
	In each cell the valid sets are grouped accordingly to the 4 different COR
	ranges.
	Each cell is colored accordingly to the group with the most members. 
    Here, the values plotted are selected between the numerical
    values from the Neural Network with initially the original experimental
    results for the $SSC$, with a product coefficient $P=1.0$ (Fig.
    \ref{fig:25cloudpirker1schulze10070}). 
        Later, they have been chosen with  
    the virtual decreased results $P=0.8$
    (\ref{fig:27cloudpirker08schulze10070}).
    The last image (Fig. \ref{fig:30cloudpirker12schulze10070}) represents
    instead the selection with the the virtual increased results $P=1.2$.    }
    \label{fig:29schulzeradarandcloud}
\end{figure}
\begin{figure}[htp] \centering
    \begin{subfigure}[b]{0.96\columnwidth}
        \includegraphics[width=\textwidth]{images/original/31radarpirker1aor}
        \caption{Radar plot, $AoR_{exp} = 38.85 ^\circ$}
        \label{fig:31radarpirker1aor} 
    \end{subfigure}\\
        \begin{subfigure}[b]{0.96\columnwidth}
        \includegraphics[width=\textwidth]{images/original/33radarpirker1schulze10070aor}
        \caption{Radar plot, $AoR_{exp} = 38.85
        ^\circ$ \& $SSC$: $\sigma_n=10070 ~[Pa]$}
        \label{fig:33radarpirker1schulze10070aor} 
    \end{subfigure}
    \caption[Radar plot of valid simulations parameters for the AOR and
    the merge between AOR and SCT valid parameters]{Radar plot of valid
    simulations parameters for the angle of repose tester ($AoR$) and the merge
    between AOR and shear cell tester ($SSC$).
    Each axes of the radar plot represents one simulation parameters.
    Furthermore, the shaded area represents valid parameters combinations.
    Dark shaded values stand for the confidence range.
    We represent the marked combinations for one load condition of the shear
    cell.
    Further explanation in the text. }
    \label{fig:35schulze10070aorradarandcloud}
\end{figure}
\begin{figure}[htp] \centering
    \begin{subfigure}[b]{0.96\columnwidth}
        \includegraphics[width=\textwidth]{images/original/32cloudpirker1aor}
        \caption{Cloud plot, $AOR_{exp} = 38.85 ^\circ$}
        \label{fig:32cloudpirker1aor} 
    \end{subfigure}\\
    \begin{subfigure}[b]{0.96\columnwidth}
        \includegraphics[width=\textwidth]{images/original/34cloudpirker1schulze10070aor}
        \caption{Cloud plot, $AOR_{exp} = 38.85
        ^\circ$, $SCT$: $\sigma_n=10070 ~[Pa]$}
        \label{fig:34cloudpirker1schulze10070aor} 
    \end{subfigure}
    \caption[Density plot comparison of AOR and SCT results]{Density plot
    comparison of $AOR$ and $SCT$ results. We represent the tabbed combinations for the
    $AOR$ test.
    Density plot of the particles' coefficient of restitution (COR) in dependence
	of coefficient of sliding friction and coefficient of rolling friction; in the
	white area no valid sets of simulation parameter can be found.
	In each cell the valid sets are grouped accordingly to the 4 different COR
	ranges.
	Each cell is colored accordingly to the group with the most members. 
    Here, the values plotted are selected between the numerical
    values from the $NN$ with initially the original experimental results for
    the $AOR$, $P=1.0$ (Fig.
    \ref{fig:32cloudpirker1aor}). 
    The last image (Fig. \ref{fig:34cloudpirker1schulze10070aor}) represents
    instead the values valid for both the $AOR$ test and the $SCT$, with a
    $\sigma_n=10070 ~[Pa]$, both for $P=1.0$. }
    \label{fig:35schulze10070aorradarandcloud}
\end{figure}


%\input{images/texCaller/24radarpirker1schulze10070}
%\input{images/texCaller/25cloudpirker1schulze10070}
%\input{images/texCaller/26radarpirker08schulze10070}
%\input{images/texCaller/27cloudpirker08schulze10070}
%\input{images/texCaller/28radarpirker12schulze10070}
% \ref{eq:rsquare}.
% \begin{equation}
R^2 = \frac {SSR}{SST} = 1 - \frac {SSE}{SST} ,
 \label{eq:rsquare}
\end{equation}

% 
% \ref{eq:rootMeanSquareError}.
% \begin{equation}
RMSE = \sqrt{\frac{\sum _{i=1}^{n} (x_{i}-\widehat{x}_{i})^{2}}{n}} ,
\label{eq:rootMeanSquareError}
\end{equation}



% \lipsum[1]
% \begin{equation}
m \ddot{x}_{ij} + c \dot{x}_{ij} + k x_{ij} =  F_{i} .
\label{equ:newtonlaw}
\end{equation}

% \subsection{ANN identification}
% \label{subsec:annmodeliden}
% \subsection{ANN application}
% \label{subsec:annapplication}
% 
% Later, each of these three trained $NN$ received as insertion $100M$ different
% combinations $DEM-micro$ parameters.
% So, we gained the numerical bulk behavior for each of this combination. 
% We then compared the values of these behaviors against the experimental bulk
% values, $SRSCT$ and $AOR$, obtaining a narrow range of valid DEM-micro
% combinations (about 80K).
% These results have been showed through radar plots (Figs. \ref{fig:15Schulze}
% and \ref{fig:16aorSchulzeIntersectionWorking}).
% To highlight an eventual $clumping$ behavior, we also plot the results in a
% cloud shape, see Fig. \ref{fig:17aorSchulzeIntersectionCloudSFRFCOR}.
% 
% %\input{images/texCaller/14aorSchulzeIntersection}
% \input{images/texCaller/15Schulze}
% % \lipsum[1]
% \input{images/texCaller/16aorSchulzeIntersectionWorking}
% % \begin{equation}
\begin{aligned}
\phi_{e-psh} &= \arctan \left(\frac{\tau_{psh}}{\sigma_{n,psh}} \right) ,\\
\mu_{psh} &=\tan(\phi_{e-psh}) .
\end{aligned}
 \label{eq:phi_ps}
\end{equation}

% \input{images/texCaller/17aorSchulzeIntersectionCloudSFRFCOR}
