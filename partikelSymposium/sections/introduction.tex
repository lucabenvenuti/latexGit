%************************************************
\section{Introduction}
\label{sec:introduction}
%************************************************


Discrete Element Method ($DEM$) simulations are widely used to picture particle
behaviour in these granular processes.
$LIGGGHTS$ is one of the most powerful open source $DEM$ simulation software packages available. 
The models it can analyze are described in detail in the literature, see Kloss
et al. \cite{RefWorks:136}, while a useful example is provided by the shear cell tester 
simulation developed by Aigner et al. \cite{RefWorks:139}.
This approach try to solve the issue of particle-particle interaction. 
From the experimental point of view, we are focusing
on the bulk behaviour of the materials analysed.
From that we investigated the relationship with the particle-particle $DEM$
contact parameters.
We tried to obtain simulations ideally perform, with the correct parameters, 
the same macroscopic behaviour of the experiments, the static angle of repose
($AOR$) and Schulze ring shear cell tester ($SRSCT$). That allowed us to later
compare numerical and experimental results, as suggested by Ai et al.
\cite{RefWorks:131}.
We could identify the most suitable value for each $DEM$ parameter 
investigated performing huge numbers of simulations. Each will have a different
value. Regrettably, many parameters contribute to define the numerical bulk behaviour. Performing and investigate the 
more than $10^8$ simulations required was out of the scope of this paper.
Instead, as suggested by Vaferi et al. \cite{RefWorks:150}, we harnessed Artificial Neural Networks ($NN$) for their
stability and reliability with non-linear systems like ours.
Furthermore, the main aim of this work was to improve the characterization 
of several $DEM$ parameters for non-spherical particles. 
From the performed simulations bulk representative parameters were extracted. 
They provide the information for the output layer of the $NN$. Instead the $DEM$ 
parameters of the same simulations provided the information for the input layer. 
Once trained, these $NN$ were fed with random combinations of $DEM$ parameters,
and they provided numerical bulk representative parameters for each
of these combinations.
Those were compared with the experimental results. A portion of the combinations
(ca. 0.1\%) had parameters matching with the experiments.
We kept these combinations as working solutions.