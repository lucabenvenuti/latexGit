\section{Modelling Pre-requisites}
\label{sec:modellingprerequisites}

\subsection{Macroscopic Experiments}
\label{subsec:Macroscopicexperiments}

The first step of the procedure was using a SRSCT (see Schulze
\cite{RefWorks:142}) to characterize particle flow properties, especially the complete yield locus.
We obtained for each of the twelve load conditions three values representative of the bulk behaviour: bulk density ($\rho_b$),
coefficient of internal friction in the pre-shear phase $ (\mu_{psh})$ and
coefficient of internal friction in the shear phase  $ (\mu_{sh})$.
Furthermore, to recreate the repose angle observed in a pile of the real material, 
we performed angle of repose ($AOR$) tests, as the $AOR$ was the fourth
behaviour value.
Moreover, we sieved the materials samples to obtain the size distribution of the
particles. Six different sifters have been used.

\subsection{Discrete element method}
\label{subsec:dem}
For the sinter fine used in this work 
Di Renzo and Di Maio \cite{RefWorks:145} suggested using the non-linear Hertzian model without cohesion for 
the particle-particle and particle-wall contacts. 
In this granular model the shear force is a "history" effect that accounts for the tangential displacement 
("tangential overlap") between the particles for the duration of contact. 
The tangential force component is truncated to fulfil $F_{t,ij} \leq \mu_s
F_{n,ij}$,
where $\mu_s$ is the coefficient of sliding friction, one of the particle based
$DEM$ parameter we investigated. 
An ulterior parameter was the coefficient of rolling friction ($\mu_r$). 
For coarse not round particles is a critical parameter and describes inter-particle 
friction in medium to dense granular flows simulations. It is proportional to the 
torque counteracting the rotation of the particle. The $\mu_r$ parameter enters the 
equations according to the elasto-rolling resistance model presented by Wensrich and 
Katterfeld \cite{RefWorks:87}. 
The maximum magnitude of rolling resistance torque is $T_{r~max} = \mu_r R_r
|\tilde{F_n}|$, where $R_r$ is the equivalent radius and $F_n$ the normal force.
The last two particle based $DEM$ parameter we investigated were the particle density 
($\rho_p$) and the coefficient of restitution ($COR$).

\subsection{Artificial Neural Networks}
\label{subsec:ann}

In this paper, we first use Neural Networks ($NN$) to fit the $DEM$ numerical
simulation data, and then to process vast amount of parameters combinations. 
They map combinations of input data into convenient outputs (fitting). 
There is a variety of types of $NN$, remarkably the Feedforward ($FF$) . 
To recognize not linearly separable data the standard linear perceptron $NN$ 
has been modified into \textit{FF Multilayer Perceptron Neural Networks (MLPNN)}. 
Here, each processing units or node (neuron) possesses a nonlinear activation function. 
Together, they are interconnected into layers, also linked together. 
The trustworthiness of the $MLPNN$, with a backpropagation reinforcement learning 
training algorithm (scaled conjugate gradient), has been widely demonstrated in the 
literature, see Haykin \cite{RefWorks:158}. 
In fact, $MLPNN$ are built with three different layers. 
The input layer has a number of neurons equal to the number of different inputs
of the network.
Following the best practice suggested by Vaferi et al. \cite{RefWorks:150} $MLPNN$ have been handled.
Similarly, the best practice also demands to establish the most appropriate number of neurons inside the 
hidden layer of each $NN$. This check has been handled through mean square maximization ($R^2$). 
For each investigated output we chose the number of neurons with the greater
$R^2$.
We should question the quality of the $NN$ data, both the $NN$ training process and the following data
generation from provided inputs.
The particles in each of our simulations were created through a random
algorithm, and the training pool was extensive.
For massive training data the effect of noise-corrupted patterns is negligible. 
Instead the latter was a challenging aspect of our work. Once trained, as input for the $NN$ we imposed 
combinations of $DEM$ parameters. 
Random values generators created values in the defined ranges and in the requested 
number for each of the investigated parameter. Then, they were combined and imposed as input.