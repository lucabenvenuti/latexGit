\section{Methodology of DEM Parameter Identification}
\label{sec:methodology}

The first bulk behaviour representative value ($\rho_b$) was directly provided. 
The second bulk behaviour representative value ($\mu_{psh}$) was calculated by
averaging the coefficient in the first plateau.
Further, the normal load was modified, and then a second plateau was reached. The third value ($\mu_{sh}$) was 
determined by averaging the coefficient in this plateau. 
Next, we performed two $AOR$ tests. 
The average of the repose angles provided us the fourth bulk value, allowing us
to define the experimental bulk behaviour.
We decided to fix one univocal contact law for all the simulations performed. 
Some coefficients, $COR$, $\mu_s$, $\mu_r$,
$\rho_p$ and $dCylDp$
were constant in each simulation, but their combination differed between
simulations.
The normal stress $\sigma_n$ and its
percentage during the incipient flow condition $\tau_{\%}$
varied to replicate the twelve shear cell load conditions. 
In total, we realized $546$ shear cell and $81$ angle of repose simulations.
A Matlab script allowed us to extract from the simulations output the numerical
bulk representative values ($\mu_{psh}$, $\mu_{sh}$, $\rho_b$ and $AOR$) for each $simulation-DEM$ parameter combination. 
So, we could use the $DEM$ parameter combinations and their corresponding bulk values to train the $NN$. 
We then created random values.
The total number of combinations of these random values was $6250000$. These
combinations were then processed by the selected $NN$, granting for each three bulk representative parameters for the shear cell and one for the $AOR$. Later, we confronted the $NN$ and experimental bulk behaviours for the twelve shear cell load conditions. 
If in a $DEM-parameter$ combination all the three bulk representative parameters differed less 
than 5\% from the corresponding experiments,
then the combination was tabbed. The latter combinations were handled by the $AOR$ $NN$, and then confronted with the experiment. 
Were branded as valid only those that differed less than $5\%$ also in this
comparison.