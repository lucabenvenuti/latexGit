              %******************************************%
              %                                          %
              %     Modello di articolo scientifico      %
              %            di Lorenzo Pantieri �         %
              %                                          %
              %         versione: 26 agosto 2012         %
              %                                          %
              %******************************************%
       

% I seguenti commenti speciali impostano:
% 1. utf8 come codifica di input,
% 2. PDFLaTeX come motore di composizione;
% 3. Articolo.tex come documento principale;
% 4. il controllo ortografico italiano per l'editor.

% !TEX encoding = UTF-8
% !TEX TS-program = pdflatex
% !TEX root = Articolo.tex
% !TEX spellcheck = en-EN

\documentclass[12pt,%                       % corpo del font principale
               a4paper,%                    % carta A4
               oneside,%                    % solo fronte
%              twoside,%                    % fronte-retro
               ]{article}                  % classe report di KOMA-Script;
         
\usepackage{helvet}
\renewcommand*\familydefault{\sfdefault} 	   
\usepackage[T1]{fontenc}                    % codifica dei font:
                                            % NOTA BENE! richiede una distribuzione *completa* di LaTeX,
                                            % per esempio TeXLive o MiKTeX *complete*

\usepackage[utf8]{inputenc}                 % codifica di input; anche [latin1] va bene
                                            % NOTA BENE! va accordata con le preferenze dell'editor

\usepackage[english]{babel}         % per scrivere in italiano e in inglese;
                                            % l'ultima lingua (l'italiano) risulta predefinita

%\usepackage[binding=20mm]{layaureo}          % margini ottimizzati per l'A4;
% rilegatura di 5 mm
\usepackage[a4paper,margin=2cm,footskip=2cm]{geometry}



\usepackage{indentfirst}                    % rientra il primo capoverso di ogni sezione

\usepackage{booktabs}                       % tabelle

\usepackage{tabularx}                       % tabelle di larghezza prefissata

\usepackage{graphicx}                       % immagini
\usepackage{epstopdf}

\usepackage{subfig}                         % sottofigure, sottotabelle
\usepackage{subfigure}   

\usepackage{caption}                        % didascalie

\usepackage{listings}                       % codici

\usepackage[font=small]{quoting}            % citazioni

\usepackage{amsmath,amssymb,amsthm}         % matematica

\usepackage[english]{varioref}              % riferimenti completi della pagina

% \usepackage{mparhack,fixltx2e,relsize}      % finezze tipografiche

\usepackage[style=numeric-comp,backref=false,backend=bibtex]{biblatex}
%\usepackage{refcheck}
                                            % eccellente pacchetto per la
                                            % bibliografia;hyperref, produce uno stile di citazione autore-anno; 
                                            % lo stile "numeric-comp" produce riferimenti numerici
                                          
\bibliography{Bibliografia}                 % database di biblatex 
                                          
\usepackage[dvipsnames]{xcolor}             % colori

\usepackage{lipsum}                         % testo fittizio

\usepackage{eurosym}                        % simbolo dell'euro

%\usepackage{hyperref}                       % collegamenti ipertestuali

%\usepackage{bookmark}                       % segnalibri

%*********************************************************************************
% impostazioni-articolo.tex
% di Luca Benvenuti (2013)
% file che contiene le impostazioni dell'articolo
%*********************************************************************************


%*********************************************************************************
% Comandi personali
%*******************************************************
\newcommand{\myName}{Luca \textsc{Benvenuti}}                            % autore
\newcommand{\myMatricola}{16457}
\newcommand{\myTitle}{Quarterly report} % titolo
\newcommand{\myDegree}{Tesi di laurea}                       % tipo di tesi
\newcommand{\myUni}{JKU} % universit\`a
\newcommand{\myFaculty}{Strongmuslehre}    % facolt\`a
\newcommand{\myDepartment}{Department of \\ Particulate Flow Modelling}        
% dipartimento
\newcommand{\myProf}{Christoph \textsc{Kloss}}    %DI~Dr.~
\newcommand{\myOtherProf}{Stefan \textsc{Pirker}}     %DI~Dr.~         %
% eventuale correlatore \newcommand{\myOtherProff}{Ing.~Gabriele Frigerio}              % eventuale correlatore
%\newcommand{\myCounterProf}{Chiar.mo Prof.~Mister x}    
\newcommand{\myLocation}{Linz}                         % dove
\newcommand{\myTime}{\today}                          % quando
\newcommand{\myPhd}{Materials2Simulation2Application} % titolo
\newcommand{\myemail}{luca.benvenuti@jku.at} % titolo








%*********************************************************************************
% Impostazioni di amsmath, amssymb, amsthm
%*********************************************************************************

% comandi per gli insiemi numerici (serve il pacchetto amssymb)
\newcommand{\numberset}{\mathbb} 
\newcommand{\N}{\numberset{N}} 
\newcommand{\R}{\numberset{R}} 

% un ambiente per i sistemi
\newenvironment{sistema}%
  {\left\lbrace\begin{array}{@{}l@{}}}%
  {\end{array}\right.}

% definizioni (serve il pacchetto amsthm)
\theoremstyle{definition} 
\newtheorem{definizione}{Definizione}

% teoremi, leggi e decreti (serve il pacchetto amsthm)
\theoremstyle{plain} 
\newtheorem{teorema}{Teorema}
\newtheorem{legge}{Legge}
\newtheorem{decreto}[legge]{Decreto}
\newtheorem{murphy}{Murphy}[section]

%simboli matematici vari
\newcommand{\Rot}[1]{\nabla\times\vec{#1}}
\newcommand{\Div}[1]{\nabla\cdot\vec{#1}}
\newcommand{\Grad}[1]{\nabla #1}
\newcommand{\Lap}[1]{\nabla^2#1}
\newcommand{\parder}[2]{\frac{\partial #1}{\partial #2}}
\newcommand{\braket}[3]{\langle #1\,\vert\,\hat{#2}\,\vert\,#3\rangle}
\newcommand{\ud}{\mathrm{d}}
\newcommand{\total}{\mathrm{D}}

%********************************************************************************
% Impostazioni di biblatex
%*********************************************************************************

% \renewcommand\bibname{References} s'incazza
% \addto\captionsenglish{\renewcommand\refname{References}} 
% \addto\captionsitalian{\renewcommand\refname{References}} 

\DefineBibliographyStrings{english}{%
  bibliography = {Bibliography},
  references = {References},
}

\DefineBibliographyStrings{italian}{%
  bibliography = {Bibliography},
  references = {References},
}

\defbibheading{bibliography}{%
\cleardoublepage
\phantomsection 
\addcontentsline{toc}{section}{\refname}

\section*{\refname\markboth{\refname}
{\refname}}}









%*********************************************************************************
% Impostazioni di listings
%*********************************************************************************
\lstset{language=[LaTeX]Tex,%C++,
    keywordstyle=\color{RoyalBlue},%\bfseries,
    basicstyle=\small\ttfamily,
    %identifierstyle=\color{NavyBlue},
    commentstyle=\color{Green}\ttfamily,
    stringstyle=\rmfamily,
    numbers=none,%left,%
    numberstyle=\scriptsize,%\tiny
    stepnumber=5,
    numbersep=8pt,
    showstringspaces=false,
    breaklines=true,
    frameround=ftff,
    frame=single
	tabsize=2,                      % sets default tabsize to 2 spaces
    captionpos=b,                   % sets the caption-position to bottom
} 





%*********************************************************************************
% Impostazioni di hyperref
%*********************************************************************************
\hypersetup{%
    hyperfootnotes=false,pdfpagelabels,
    %draft,	% = elimina tutti i link (utile per stampe in bianco e nero)
    colorlinks=true, linktocpage=true, pdfstartpage=1, pdfstartview=FitV,%
    % decommenta la riga seguente per avere link in nero (per esempio per la stampa in bianco e nero)
    %colorlinks=false, linktocpage=false, pdfborder={0 0 0}, pdfstartpage=1, pdfstartview=FitV,% 
    breaklinks=true, pdfpagemode=UseNone, pageanchor=true, pdfpagemode=UseOutlines,%
    plainpages=false, bookmarksnumbered, bookmarksopen=true, bookmarksopenlevel=1,%
    hypertexnames=true, pdfhighlight=/O,%nesting=true,%frenchlinks,%
    urlcolor=webbrown, linkcolor=RoyalBlue, citecolor=webgreen, %pagecolor=RoyalBlue,%
    %urlcolor=Black, linkcolor=Black, citecolor=Black, %pagecolor=Black,%
    pdftitle={\myTitle},%
    pdfauthor={\textcopyright\ \myName},%
    pdfsubject={},%
    pdfkeywords={},%
    pdfcreator={pdfLaTeX},%
    pdfproducer={LaTeX with hyperref and ClassicThesis}%
}



%*********************************************************************************
% Impostazioni di graphicx
%*********************************************************************************
\graphicspath{{Immagini/}} % cartella dove sono riposte le immagini



%*********************************************************************************
% Impostazioni di xcolor
%*********************************************************************************
\definecolor{webgreen}{rgb}{0,.5,0}
\definecolor{webbrown}{rgb}{.6,0,0}



%*********************************************************************************
% Impostazioni di caption
%*********************************************************************************
\captionsetup{tableposition=top,figureposition=bottom,font=small,format=hang,labelfont=bf}





%*********************************************************************************
% Impostazioni di fancyhdr
%*********************************************************************************
\pagestyle{fancy}
\renewcommand{\sectionmark}[1]{\markboth{\sectionname\ \thesection.\ #1}{}}

\fancyhf{}


\fancyhead[LE,RO]{\thepage}
\fancyhead[RE]{\nouppercase{\leftmark}}
\fancyhead[LO]{\nouppercase{\rightmark}}


\renewcommand{\headrulewidth}{0.5pt}

\renewcommand{\footrulewidth}{0pt}
\fancyheadoffset{0\columnwidth}	


%*********************************************************************************
% Altro
%*********************************************************************************

% [...] ;-)
\newcommand{\omissis}{[\dots\negthinspace]}

% eccezioni all'algoritmo di sillabazione
\hyphenation{Fortran ma-cro-istru-zio-ne nitro-idrossil-amminico}



\newcommand{\HRule}{\rule{\linewidth}{0.5mm}}
\newcommand{\RM}[1]{\MakeUppercase{\romannumeral #1}} 
               % file con le impostazioni personali

\begin{document}

%\begin{frontmatter}
% !TEX encoding = UTF-8
% !TEX TS-program = pdflatex
% !TEX root = ../elsarticle-template-num.tex
% !TEX spellcheck = en-EN

%************************************************
%\section{front}
%\label{sec:introduction}
%************************************************

%% Title, authors and addresses

%% use the tnoteref command within \title for footnotes;
%% use the tnotetext command for theassociated footnote;
%% use the fnref command within \author or \address for footnotes;
%% use the fntext command for theassociated footnote;
%% use the corref command within \author for corresponding author footnotes;
%% use the cortext command for theassociated footnote;
%% use the ead command for the email address,
%% and the form \ead[url] for the home page:
\title{Title\tnoteref{label1}}
%\tnotetext[label1]{}
%% \author{Name\corref{cor1}\fnref{label2}}
%% \ead{email address}
%% \ead[url]{home page}
%% \fntext[label2]{}
%% \cortext[cor1]{}
%% \address{Address\fnref{label3}}
%% \fntext[label3]{}

\title{A novel approach to particle characterization for discrete element method by means of artificial neural networks}

%% use optional labels to link authors explicitly to addresses:
%% \author[label1,label2]{}
%% \address[label1]{}
%% \address[label2]{}

\author[JKU PFM]{L. Benvenuti \corref{benvenuti}}
\ead{luca.benvenuti@jku.at}

\author[JKU PFM]{S. Pirker}

\address[JKU PFM]{JKU Department of Particulate Flow Modelling - Linz}


%\ead{stefan.pirker@jku.at}
%\address[JKU PFM]{JKU Department of Particulate Flow Modelling - Linz}


\author[DCS]{C. Kloss}
%\ead{christoph.kloss@dcs-computing.com}
\address[DCS]{DCS Computing - Linz}

\cortext[benvenuti]{Corresponding author}

\begin{abstract}
In Discrete Element Method ($DEM$) simulations, particle-particle contact laws
determine the macroscopic simulation results. Particle based contact laws, in
turn, commonly rely on semi-empirical parameters, which can be hardly obtained by direct measurements.
In this study we present a methodology for the identification of particle based
DEM simulation parameters by means of macroscopic experiments and dedicated
artificial neural networks. 
In a first step, the macroscopic results of a series
of DEM simulations with varying simulation parameters are used to train a feed
forward artificial neural network by backward propagation reinforcement. In a
second step, this artificial neural network can be harnessed to predict the
macroscopic ensemble behaviour in dependence of additional sets of particle
based simulation parameters.
As a result, a comprehensive database is obtained,
which links particle based simulation parameters to a specific macroscopic ensemble output.
The trained artificial neural network is able to predict the behaviour of
additional sets of input parameters accurately and highly efficient.
Furthermore, this methodology can be applied to different macroscopic behaviours
picturing dedicated calibration experiments.
By the help of these experiments, the DEM simulation parameters of a specific granular material can be identified.

\end{abstract}


\begin{keyword}
%% keywords here, in the form: keyword \sep keyword
Meshless methods (DEM) \sep Rheology \sep experimental validation studies \sep process industries \sep process metallurgy \sep LIGGGHTS \sep Material characterization 
\sep Artificial Neural Networks
%% PACS codes here, in the form: \PACS code \sep code

%% MSC codes here, in the form: \MSC code \sep code
%% or \MSC[2008] code \sep code (2000 is the default)

\end{keyword}
%\end{frontmatter}

% %************************************************
\section{Highlights}
\label{sec:highlights}
%************************************************
\begin{itemize}
  \item{We trained an Artificial Neural Network by DEM simulations with varying
  parameters}
  \item{The Artificial Neural Network then predicts granular bulk behaviour}
  \item{By comparison with bulk experiments DEM simulation parameters are
  identified}
  \item{This DEM parameter identification can be applied to different bulk
  behaviours}
  \item{This DEM parameter identification can be applied to different granular
  materials}
\end{itemize}

%************************************************
\section{Introduction}
\label{sec:introduction}
%************************************************


Discrete Element Method ($DEM$) simulations are widely used to picture particle
behaviour in these granular processes.
$LIGGGHTS$ is one of the most powerful open source $DEM$ simulation software packages available. 
The models it can analyze are described in detail in the literature, see Kloss
et al. \cite{RefWorks:136}, while a useful example is provided by the shear cell tester 
simulation developed by Aigner et al. \cite{RefWorks:139}.
This approach try to solve the issue of particle-particle interaction. 
From the experimental point of view, we are focusing
on the bulk behaviour of the materials analysed.
From that we investigated the relationship with the particle-particle $DEM$
contact parameters.
We tried to obtain simulations ideally perform, with the correct parameters, 
the same macroscopic behaviour of the experiments, the static angle of repose
($AOR$) and Schulze ring shear cell tester ($SRSCT$). That allowed us to later
compare numerical and experimental results, as suggested by Ai et al.
\cite{RefWorks:131}.
We could identify the most suitable value for each $DEM$ parameter 
investigated performing huge numbers of simulations. Each will have a different
value. Regrettably, many parameters contribute to define the numerical bulk behaviour. Performing and investigate the 
more than $10^8$ simulations required was out of the scope of this paper.
Instead, as suggested by Vaferi et al. \cite{RefWorks:150}, we harnessed Artificial Neural Networks ($NN$) for their
stability and reliability with non-linear systems like ours.
Furthermore, the main aim of this work was to improve the characterization 
of several $DEM$ parameters for non-spherical particles. 
From the performed simulations bulk representative parameters were extracted. 
They provide the information for the output layer of the $NN$. Instead the $DEM$ 
parameters of the same simulations provided the information for the input layer. 
Once trained, these $NN$ were fed with random combinations of $DEM$ parameters,
and they provided numerical bulk representative parameters for each
of these combinations.
Those were compared with the experimental results. A portion of the combinations
(ca. 0.1\%) had parameters matching with the experiments.
We kept these combinations as working solutions.

\section{Modelling Pre-requisites}
\label{sec:modellingprerequisites}

\subsection{Macroscopic Experiments}
\label{subsec:Macroscopicexperiments}

The first step of the procedure was using a SRSCT (see Schulze
\cite{RefWorks:142}) to characterize particle flow properties, especially the complete yield locus.
We obtained for each of the twelve load conditions three values representative of the bulk behaviour: bulk density ($\rho_b$),
coefficient of internal friction in the pre-shear phase $ (\mu_{psh})$ and
coefficient of internal friction in the shear phase  $ (\mu_{sh})$.
Furthermore, to recreate the repose angle observed in a pile of the real material, 
we performed angle of repose ($AOR$) tests, as the $AOR$ was the fourth
behaviour value.
Moreover, we sieved the materials samples to obtain the size distribution of the
particles. Six different sifters have been used.

\subsection{Discrete element method}
\label{subsec:dem}
For the sinter fine used in this work 
Di Renzo and Di Maio \cite{RefWorks:145} suggested using the non-linear Hertzian model without cohesion for 
the particle-particle and particle-wall contacts. 
In this granular model the shear force is a "history" effect that accounts for the tangential displacement 
("tangential overlap") between the particles for the duration of contact. 
The tangential force component is truncated to fulfil $F_{t,ij} \leq \mu_s
F_{n,ij}$,
where $\mu_s$ is the coefficient of sliding friction, one of the particle based
$DEM$ parameter we investigated. 
An ulterior parameter was the coefficient of rolling friction ($\mu_r$). 
For coarse not round particles is a critical parameter and describes inter-particle 
friction in medium to dense granular flows simulations. It is proportional to the 
torque counteracting the rotation of the particle. The $\mu_r$ parameter enters the 
equations according to the elasto-rolling resistance model presented by Wensrich and 
Katterfeld \cite{RefWorks:87}. 
The maximum magnitude of rolling resistance torque is $T_{r~max} = \mu_r R_r
|\tilde{F_n}|$, where $R_r$ is the equivalent radius and $F_n$ the normal force.
The last two particle based $DEM$ parameter we investigated were the particle density 
($\rho_p$) and the coefficient of restitution ($COR$).

\subsection{Artificial Neural Networks}
\label{subsec:ann}

In this paper, we first use Neural Networks ($NN$) to fit the $DEM$ numerical
simulation data, and then to process vast amount of parameters combinations. 
They map combinations of input data into convenient outputs (fitting). 
There is a variety of types of $NN$, remarkably the Feedforward ($FF$) . 
To recognize not linearly separable data the standard linear perceptron $NN$ 
has been modified into \textit{FF Multilayer Perceptron Neural Networks (MLPNN)}. 
Here, each processing units or node (neuron) possesses a nonlinear activation function. 
Together, they are interconnected into layers, also linked together. 
The trustworthiness of the $MLPNN$, with a backpropagation reinforcement learning 
training algorithm (scaled conjugate gradient), has been widely demonstrated in the 
literature, see Haykin \cite{RefWorks:158}. 
In fact, $MLPNN$ are built with three different layers. 
The input layer has a number of neurons equal to the number of different inputs
of the network.
Following the best practice suggested by Vaferi et al. \cite{RefWorks:150} $MLPNN$ have been handled.
Similarly, the best practice also demands to establish the most appropriate number of neurons inside the 
hidden layer of each $NN$. This check has been handled through mean square maximization ($R^2$). 
For each investigated output we chose the number of neurons with the greater
$R^2$.
We should question the quality of the $NN$ data, both the $NN$ training process and the following data
generation from provided inputs.
The particles in each of our simulations were created through a random
algorithm, and the training pool was extensive.
For massive training data the effect of noise-corrupted patterns is negligible. 
Instead the latter was a challenging aspect of our work. Once trained, as input for the $NN$ we imposed 
combinations of $DEM$ parameters. 
Random values generators created values in the defined ranges and in the requested 
number for each of the investigated parameter. Then, they were combined and imposed as input.
\section{Methodology of DEM Parameter Identification}
\label{sec:methodology}

We now illustrate the methodology used, also shown in Fig.
\ref{fig:19methodology}.
%\begin{figure}[!htb] 
\centering 
\includegraphics[width=.96\textwidth]{images/19methodology} 
\caption[Method]{Method. 
In the training phase (dashed lines)
$DEM$ simulations are performed
with random initial input parameters.
The behaviours obtained are used to train the
Artificial Neural Networks ($ANNs$) in a loop that continues until the
difference between the outputs of each $ANN$ and its simulations is below the
limit ($\Delta$) (see Section \ref{subsec:ann}).
In the parameters identification phase (solid
lines) we identify valid input parameters by comparing (\textbf{=}) $ANNs$ and
experimental behaviours.
Further explanations can be found in Section \ref{sec:methodology}.
}
\label{fig:19methodology} 
\end{figure}
The experimental characterization has been performed as described in
\ref{subsec:srsctexperiment} and \ref{subsec:aorexperiment}. We performed
three tests with the $SRSCT$ for the sinter fine bulk, for a total of twelve
load conditions. An example for three of them can be seen in Tab. \ref{tab:05sinterTableExperimental}.
% \begin{table}[h]
\centering
\begin{tabular}{cccccc}
$\sigma_n$ [Pa] & $\tau$ [Pa] & $\mu_{psh}$ [-] & $\mu_{sh}$ [-] &
$\rho_b$ [kg/m3] & AOR $\circ$ \\
\hline
    1068  & 1059  & 0.9916 & 0.9916 & 1718  & 38.85 \\
    2069  & 1818  & 0.8787 & 0.8787 & 1759  & 38.85 \\
    10070 & 8232  & 0.8175 & 0.8175 & 1802  & 38.85 \\

\hline
\end{tabular}
\caption{Experimental values for sinter fine}
\label{tab:05sinterTableExperimental}
\end{table}
The first bulk behaviour representative value ($\rho_b$) is directly provided. 
Concerning the stresses for a sample test, we observed a linear increase in the
coefficient of internal friction, see Fig. \ref{fig:20experimental}.
Later, the first plateau is reached. 
The second bulk behaviour representative value ($\mu_{ie,psh}$) is calculated by averaging the coefficient in this plateau. 
Further, the normal load is modified, and then a second plateau is reached. The third value ($\mu_{ie,sh}$) is 
determined by averaging the coefficient in this plateau. 
Next, we performed two $AOR$ tests. 
Their average provided us the fourth bulk value, allowing us to define the experimental bulk behaviour. 
For simulations purposes, we also sieved the bulk to know the size distribution.
We could then focus on the numerical section of the characterization. 
As stated in the modelling section of this paper, we decided to fix one univocal contact law for all the simulations performed. 
Furthermore, we locked the size distribution, as provided by the sieving, the
elastic coefficients and the time step, see Tab.
\ref{tab:09DEMFixedinputvalues}.
% \begin{table}[h]
\centering
\begin{tabular}{c|c|c|c|c}
\hline
average & std dev & constant & DEM   & DEM \\
    particle & particle & ring  & Young's & Poisson's \\
    radius & radius & velocity & modulus & ratio \\
    [mm]  & [mm]  & [mm/s] & [Gpa] & [-] \\
    \hline
    0.732 & 0.41  & 2.196 & 10    & 0.40 \\


\hline
\end{tabular}
\caption{DEM fixed input values}
\label{tab:09DEMFixedinputvalues}
\end{table}
% \begin{table}[h]
\centering
\begin{tabular}{c|c|c|c|c}
\hline
	DEM   & DEM   & DEM   & average & simulation \\
    sliding & rolling & coefficient & particle & domain diameter \\
    friction & friction & restitution & density & to particle mean \\
    	$[-]$  & $[-]$   & $[-]$   & $[kg/m3]$ & diameter ratio \\
    \hline
    0.4 - 0.6 - 0.8 & 0.4 - 0.6 - 0.8 & 0.5 - 0.7 - 0.9 & 2500 - 3000 - 3500 & 20 - 36 - 38 - 40 \\

\hline
\end{tabular}
\caption{DEM variable input values}
\label{tab:10DEMVariableinputvalues}
\end{table}
The latter was smaller than the Rayleigh time. Instead, $COR$, $\mu_s$, $\mu_r$,
$\rho_p$ and $dCylDp$, as indicated in Tab. \ref{tab:10DEMVariableinputvalues},
were constant in each simulation, but their combination differed between
simulations, see e.g. Tab. \ref{tab:11DEMSimExampleinputvalues}.
% \begin{table}[h]
\centering
\begin{tabular}{lccccc}
\hline
 sim &  $\mu_s$ & $\mu_r$ & $COR$ & $\rho_p$ & $dCylDp$ \\
  \#  &	$[-]$  & $[-]$   & $[-]$   & $[kg/m3]$ & $[-]$ \\
          \hline
    1st & 0.40  & 0.40  & 0.50  & 2500  & 20 \\
    2nd & 0.60  & 0.40  & 0.50  & 2500  & 20 \\


\hline
\end{tabular}
\caption{DEM simulation examples input values}
\label{tab:11DEMSimExampleinputvalues}
\end{table}
Further, $dCylDp$ was used to evaluate the wall effect, but only $~10\%$ of the
all simulations had $dCylDp$ larger than $20$. The normal stress $\sigma_n$ and its
percentage during the incipient flow condition $\tau_{\%}$
varied to replicate the twelve shear cell load conditions. 
In total, we realized $546$ shear cell and $81$ angle of repose simulations.
A Matlab script allowed us to extract from the simulations output the numerical
bulk representative values ($\mu_{ie-ps}$, $\mu_{ie-s}$, $\rho_b$ and $AOR$) for each $simulation-DEM$ parameter combination. 
So, we could use the $DEM$ parameter combinations and their corresponding bulk values to train the $NN$. 
Notably, we excluded 15\% of the simulations ($test ~ simulations$), casually
picked, from the training processes.
First, we started with all the $DEM$ parameter combinations and their corresponding numerical $\mu_{ie-ps}$ to create 36 $NN$. 
They differed because they have from five to forty neurons in the hidden layer. 
Later, we controlled the square regression error between the $bulk-macro$ behaviours in the output of 
the $NN$ and the 15\% $test ~ simulations$, granted uncorrelated. 
So, we could select for $\mu_{ie-ps}$ the $NN$ with the maximum $R^2$, and we noted its number of neurons. 
We repeated the same steps from the $NN$ creations for $\mu_{ie-s}$, $\rho_b$ and $AOR$, 
obtaining one trained $NN$ for each bulk representative value. \\
% \begin{table}[h]
\centering
\begin{tabular}{lcccc}
\hline
 &  \ac{mus} & \ac{mur} & \ac{CoR} & \ac{rhop}  \\
  &	$[-]$  & $[-]$   & $[-]$   & $[kg/m3]$ \\
          \hline
    range & $[0.1 \ldots 1.0]$ & $[0.1 \ldots 1.0]$ & $[0.5 \ldots 0.9]$ &
    $[2000 \ldots 3500]$     \\
    \# rnd & 100   & 100   & 25    & 25    \\

\hline
\end{tabular}
\caption[DEM random input values]{DEM random input values. Within each range \#
random values are chosen.}
\label{tab:12DEMRandominputvalues}
\end{table}
Since $\mu_{ie-ps}$, $\mu_{ie-s}$ and $\rho_b$ belonged to the shear cell
simulations their $NN$ were handled together. We then created random values in the range
and number defined in Tab. \ref{tab:12DEMRandominputvalues}.
The total number of combinations of these random values was $6250000$. These
combinations were then processed by the selected $NN$, granting for each three bulk representative parameters for the shear cell and one for the $AOR$. Later, we confronted the $NN$ and experimental bulk behaviours for the twelve shear cell load conditions. 
If in a $DEM-parameter$ combination all the three bulk representative parameters differed less 
than 5\% from the corresponding experiments, see Eq. \ref{eq:check2}:
 \begin{equation}
 \begin{cases}
\text{if } & \lvert{1-\frac{\mu_{psh,num}}{\mu_{psh,exp}}}\rvert < 5\%  ,\\
\text{and if } & \lvert{1-\frac{\mu_{sh,num}}{\mu_{sh,exp}}}\rvert < 5\% , \\ 
\text{and if } & \lvert{1-\frac{\rho_{p,num}}{\rho_{p,exp}}}\rvert < 5\% ,\\ 
\end{cases}
 \label{eq:check2}
\end{equation}

then the combination was tabbed. The latter combinations were handled by the $AOR$ $NN$, and then confronted with the experiment. 
Only those that differed less than $5\%$ also in this comparison (Eq.
\ref{eq:checkaor}) were branded as valid:
\begin{equation}
\text{if} ~~~~~~ \lvert{1-\frac{AoR_{num}}{AoR_{exp}}}\rvert < 5\% .
\label{eq:checkaor}
\end{equation}
%************************************************
Further, to prove the system validity, we tested the tabbed combinations by modifying the experimental bulk
behaviour representative values of the shear cell. 
We artificially decreased or increased all of them by a product coefficient ($P$).
\begin{figure}[!htb] 
\centering 
\includegraphics[width=.96\textwidth]{images/19methodology} 
\caption[Method]{Method. 
In the training phase (dashed lines)
$DEM$ simulations are performed
with random initial input parameters.
The behaviours obtained are used to train the
Artificial Neural Networks ($ANNs$) in a loop that continues until the
difference between the outputs of each $ANN$ and its simulations is below the
limit ($\Delta$) (see Section \ref{subsec:ann}).
In the parameters identification phase (solid
lines) we identify valid input parameters by comparing (\textbf{=}) $ANNs$ and
experimental behaviours.
Further explanations can be found in Section \ref{sec:methodology}.
}
\label{fig:19methodology} 
\end{figure}
\begin{table}[h]
\centering
\begin{tabular}{cccccc}
$\sigma_n$ [Pa] & $\tau$ [Pa] & $\mu_{psh}$ [-] & $\mu_{sh}$ [-] &
$\rho_b$ [kg/m3] & AOR $\circ$ \\
\hline
    1068  & 1059  & 0.9916 & 0.9916 & 1718  & 38.85 \\
    2069  & 1818  & 0.8787 & 0.8787 & 1759  & 38.85 \\
    10070 & 8232  & 0.8175 & 0.8175 & 1802  & 38.85 \\

\hline
\end{tabular}
\caption{Experimental values for sinter fine}
\label{tab:05sinterTableExperimental}
\end{table}
\begin{table}[h]
\centering
\begin{tabular}{c|c|c|c|c}
\hline
average & std dev & constant & DEM   & DEM \\
    particle & particle & ring  & Young's & Poisson's \\
    radius & radius & velocity & modulus & ratio \\
    [mm]  & [mm]  & [mm/s] & [Gpa] & [-] \\
    \hline
    0.732 & 0.41  & 2.196 & 10    & 0.40 \\


\hline
\end{tabular}
\caption{DEM fixed input values}
\label{tab:09DEMFixedinputvalues}
\end{table}
\begin{table}[h]
\centering
\begin{tabular}{c|c|c|c|c}
\hline
	DEM   & DEM   & DEM   & average & simulation \\
    sliding & rolling & coefficient & particle & domain diameter \\
    friction & friction & restitution & density & to particle mean \\
    	$[-]$  & $[-]$   & $[-]$   & $[kg/m3]$ & diameter ratio \\
    \hline
    0.4 - 0.6 - 0.8 & 0.4 - 0.6 - 0.8 & 0.5 - 0.7 - 0.9 & 2500 - 3000 - 3500 & 20 - 36 - 38 - 40 \\

\hline
\end{tabular}
\caption{DEM variable input values}
\label{tab:10DEMVariableinputvalues}
\end{table}
\begin{table}[h]
\centering
\begin{tabular}{lccccc}
\hline
 sim &  $\mu_s$ & $\mu_r$ & $COR$ & $\rho_p$ & $dCylDp$ \\
  \#  &	$[-]$  & $[-]$   & $[-]$   & $[kg/m3]$ & $[-]$ \\
          \hline
    1st & 0.40  & 0.40  & 0.50  & 2500  & 20 \\
    2nd & 0.60  & 0.40  & 0.50  & 2500  & 20 \\


\hline
\end{tabular}
\caption{DEM simulation examples input values}
\label{tab:11DEMSimExampleinputvalues}
\end{table}
\begin{table}[h]
\centering
\begin{tabular}{lcccc}
\hline
 &  \ac{mus} & \ac{mur} & \ac{CoR} & \ac{rhop}  \\
  &	$[-]$  & $[-]$   & $[-]$   & $[kg/m3]$ \\
          \hline
    range & $[0.1 \ldots 1.0]$ & $[0.1 \ldots 1.0]$ & $[0.5 \ldots 0.9]$ &
    $[2000 \ldots 3500]$     \\
    \# rnd & 100   & 100   & 25    & 25    \\

\hline
\end{tabular}
\caption[DEM random input values]{DEM random input values. Within each range \#
random values are chosen.}
\label{tab:12DEMRandominputvalues}
\end{table}
\section{Results and discussion}
\label{sec:results}
%************************************************

\subsection{Experiments}
\label{subsec:experiments}

%\InsTab{tables/05sinterTableExperimental}
Initially, experimental values identifying the bulk behavior, $\phi_{e-psh}$, 
$\phi_{e-sh}$ and $\rho_{b}$, for sinter fine have been acquired though the
SRSCT, see Table \ref{tab:05sinterTableExperimental}.
\begin{table}[h]
\centering
\begin{tabular}{cccccc}
$\sigma_n$ [Pa] & $\tau$ [Pa] & $\mu_{psh}$ [-] & $\mu_{sh}$ [-] &
$\rho_b$ [kg/m3] & AOR $\circ$ \\
\hline
    1068  & 1059  & 0.9916 & 0.9916 & 1718  & 38.85 \\
    2069  & 1818  & 0.8787 & 0.8787 & 1759  & 38.85 \\
    10070 & 8232  & 0.8175 & 0.8175 & 1802  & 38.85 \\

\hline
\end{tabular}
\caption{Experimental values for sinter fine}
\label{tab:05sinterTableExperimental}
\end{table}
Later, two AOR test have been performed, given an average angle of $38.85
^\circ$.
 
ONE TABLE FOR EACH MATERIAL

% a
% 
% \begin{equation}
F_{t} \leq \mu_s F_{n},
 \label{eq:force_t}
\end{equation}


\subsection{DEM Simulations}
\label{subsec:simulations}
% \lipsum[1]
% \begin{equation}
\begin{aligned}
	k_n &= \frac{4}{3} E_{eq} \sqrt{R_{eq} \delta_n} ,\\
	\gamma_n &= 2 \sqrt{\frac{5}{6}} \beta \sqrt{S_n m_{eq}} ,\\
	k_t &= 8 G_{eq} \sqrt{R_{eq}} \delta_n ,\\
	\gamma_t &= 2 \sqrt{\frac{5}{6}} \beta \sqrt{S_t m_{eq}} .
\end{aligned}
\label{eq:hertz}
\end{equation}

For sinter fine 704 shear cell and 54 static angle of repose simulations have
been realized with the variation described in Table \ref{tab:07DEMinputvalues},
gaining four values identifying the numerical bulk behavior, $\phi_{e-psh}$, 
$\phi_{e-sh}$, $AOR$ and $\rho_{b}$ for each simulated combination.
The shear cell simulations made are more than the angle of repose to account for
the different normal load applied.
\begin{table}[h]
\centering
\begin{tabular}{c|c|c|c|c}
\hline
average & std dev & average & DEM   & DEM \\
particle & particle & particle & Young's & Poisson's \\
    radius & radius & density & modulus & ratio \\
    [mm]  & [mm]  & [kg/m3] & [Gpa] & [-] \\
    \hline
    0.732 & 0.41  & 2500 - 3000 - 3500 & 10    & 0.40 \\
    \hline
    DEM   & DEM   & DEM   & constant & simulation \\
    sliding & rolling & coefficient & ring  & domain diameter \\
    friction & friction & restitution & velocity & to particle mean \\
    [-]   & [-]   & [-]   & [mm/s] & diameter ratio \\
    \hline
    0.4 - 0.6 - 0.8 & 0.4 - 0.6 - 0.8 & 0.5 - 0.7 - 0.9 & 2.196 & 20 - 36 - 38 - 40 \\

\hline
\end{tabular}
\caption{DEM input values}
\label{tab:07DEMinputvalues}
\end{table}


\subsection{ANN model development}
\label{subsec:annmodeldev}

We then fed the $NN$ with the same $DEM-micro$ parameters of these simulations
as input.
A series of NN have been realized for each bulk behavior as target, ranging from
5 to 40 neurons in the hidden layer.
The linear relationship between the
training values have been evaluated in Table \ref{tab:06inputRelationshipTable}.
\begin{table}[H!]                                                                                                                                                          
\centering                                                                                                                                                                 
\begin{tabular}{|c|c|c|c|c|c|c|c|c|c|c|c|}                                                                                                                                 
\hline                                                                                                                                                                     
 & sf & rf & rest & dt & dCylDp & ctrlStress & shearperc & dens & mush & mupsh & rhob \\                                                                                   
\hline                                                                                                                                                                     
sf & 1 & 5.549787e-03 & -3.818461e-04 & -1.268763e-15 & -1.628657e-02 & 1.282025e-15 & 4.517397e-03 & 0 & 3.838826e-02 & 8.725701e-01 & -8.393464e-02 \\                   
\hline                                                                                                                                                                     
rf & 5.549787e-03 & 1 & -1.523330e-03 & -2.349289e-15 & -5.968531e-02 & 2.322503e-15 & 1.802162e-02 & 3.348007e-18 & 5.891756e-01 & 3.370233e-01 & -3.101856e-02 \\        
\hline                                                                                                                                                                     
rest & -3.818461e-04 & -1.523330e-03 & 1 & -1.555718e-15 & -2.758674e-01 & 1.568359e-15 & 8.090707e-02 & 6.680307e-18 & 1.551852e-01 & -5.671687e-03 & -1.712429e-02 \\    
\hline                                                                                                                                                                     
dt & -1.268763e-15 & -2.349289e-15 & -1.555718e-15 & 1 & -1.026312e-16 & -1.000000e+00 & -2.681936e-17 & 0 & 6.168810e-16 & -4.320958e-15 & 1.243669e-14 \\                
\hline                                                                                                                                                                     
dCylDp & -1.628657e-02 & -5.968531e-02 & -2.758674e-01 & -1.026312e-16 & 1 & 7.853515e-17 & -2.939311e-01 & 2.688281e-17 & -2.879551e-01 & -1.916393e-01 & 9.603603e-02 \\ 
\hline                                                                                                                                                                     
ctrlStress & 1.282025e-15 & 2.322503e-15 & 1.568359e-15 & -1 & 7.853515e-17 & 1 & -3.731389e-17 & 0 & -6.100950e-16 & 4.292811e-15 & -1.234126e-14 \\                      
\hline                                                                                                                                                                     
shearperc & 4.517397e-03 & 1.802162e-02 & 8.090707e-02 & -2.681936e-17 & -2.939311e-01 & -3.731389e-17 & 1 & -3.512479e-17 & 5.730199e-02 & 5.380657e-02 & -5.095294e-03 \\
\hline                                                                                                                                                                     
dens & 0 & 3.348007e-18 & 6.680307e-18 & 0 & 2.688281e-17 & 0 & -3.512479e-17 & 1 & -4.980664e-02 & 5.709445e-02 & 9.900341e-01 \\                                         
\hline                                                                                                                                                                     
mush & 3.838826e-02 & 5.891756e-01 & 1.551852e-01 & 6.168810e-16 & -2.879551e-01 & -6.100950e-16 & 5.730199e-02 & -4.980664e-02 & 1 & 2.603411e-01 & -9.516313e-02 \\      
\hline                                                                                                                                                                     
mupsh & 8.725701e-01 & 3.370233e-01 & -5.671687e-03 & -4.320958e-15 & -1.916393e-01 & 4.292811e-15 & 5.380657e-02 & 5.709445e-02 & 2.603411e-01 & 1 & -4.329071e-02 \\     
\hline                                                                                                                                                                     
rhob & -8.393464e-02 & -3.101856e-02 & -1.712429e-02 & 1.243669e-14 & 9.603603e-02 & -1.234126e-14 & -5.095294e-03 & 9.900341e-01 & -9.516313e-02 & -4.329071e-02 & 1 \\   
\hline                                                                                                                                                                     
\end{tabular}                                                                                                                                                              
\caption{MyTableCaption}                                                                                                                                                   
\label{table:MyTableLabel}                                                                                                                                                 
\end{table}               
EVALUTE IF ADD AOR AS ENTRY IN THE TABLE OR MAKE A NEW SEPARATE TABLE



% \lipsum[1]
% \begin{equation}
 \mu_r =  \tan(\iota) .
\label{equ:mur}
\end{equation}


\subsection{ANN identification}
\label{subsec:annmodeliden}
Subsequently, we controlled the square regression error between the $bulk-macro$
behavior in these simulations and in the output of the $NN$, see Eq.
\ref{eq:rsquare}.
\begin{equation}
R^2 = \frac {SSR}{SST} = 1 - \frac {SSE}{SST} ,
 \label{eq:rsquare}
\end{equation}

SSE is the sum of squared error, SSR is the sum of squared regression, SST is
the sum of squared total.
As in literature, this goal has been achieved by first excluding 15\% of the
simulations from the training processes.\\

RMSE amplifies and severely punishes large errors, see Eq.
\ref{eq:rootMeanSquareError}.
\begin{equation}
RMSE = \sqrt{\frac{\sum _{i=1}^{n} (x_{i}-\widehat{x}_{i})^{2}}{n}} ,
\label{eq:rootMeanSquareError}
\end{equation}


Finally, we selected for each $bulk-macro$ behavior property ($\mu_{ie-ps}$, $\mu_{ie-s}$ and $\rho_b$) the $NN$ with the 
number of neurons that provided the maximum $R^2$.


% \lipsum[1]
% \begin{equation}
m \ddot{x}_{ij} + c \dot{x}_{ij} + k x_{ij} =  F_{i} .
\label{equ:newtonlaw}
\end{equation}


\subsection{ANN application}
\label{subsec:annapplication}

Later, each of these three trained $NN$ received as insertion $100M$ different
combinations $DEM-micro$ parameters.
So, we gained the numerical bulk behavior for each of this combination. 
We then compared the values of these behaviors against the experimental bulk
values, $SRSCT$ and $AOR$, obtaining a narrow range of valid DEM-micro
combinations (about 80K).
These results have been showed through radar plots (Figs. \ref{fig:15Schulze}
and \ref{fig:16aorSchulzeIntersectionWorking}).
To highlight an eventual $clumping$ behavior, we also plot the results in a
cloud shape, see Fig. \ref{fig:17aorSchulzeIntersectionCloudSFRFCOR}.

%\input{images/texCaller/14aorSchulzeIntersection}
\input{images/texCaller/15Schulze}
% \lipsum[1]
\input{images/texCaller/16aorSchulzeIntersectionWorking}
% \begin{equation}
\begin{aligned}
\phi_{e-psh} &= \arctan \left(\frac{\tau_{psh}}{\sigma_{n,psh}} \right) ,\\
\mu_{psh} &=\tan(\phi_{e-psh}) .
\end{aligned}
 \label{eq:phi_ps}
\end{equation}

\input{images/texCaller/17aorSchulzeIntersectionCloudSFRFCOR}

%************************************************
\section{Conclusions}
\label{sec:conclusions}
%************************************************

We presented a two phase methodology for DEM simulation parameter
identification. In a first phase an artificial neural network is 
trained by dedicated DEM simulation in order to predict bulk 
behaviour in dependence on a set of DEM simulation parameters. 
In a second phase this artificial neural network is then utilized 
to predict the bulk behaviour of a huge number of additional DEM parameter sets. 
The main findings of this study can be summarized as:
\begin{itemize}
  \item{An artificial neural network can be trained by a limited number of dedicated DEM simulations. 
  		Subsequently, the trained artificial neural network is able to predict
  		granular bulk behaviour.}
  \item{This prediction of granular bulk behaviour is highly efficient if
  		compared to computationally expensive DEM simulations.
  		Thus, a huge number of parameter sets can be studied with respect to their
  		macroscopic output.}
  \item{If the predictions of the artificial neural network are compared to a bulk experiment, 
  		valid sets of DEM simulation parameters can be readily deduced for a
  		specific granular material.}
  \item{This DEM parameter identification methodology can be applied to arbitrary bulk experiments. 
  		Combining two artificial neural networks, which predict two different bulk
  		behaviours, leads to focusing the set of valid DEM simulation parameters.}
\end{itemize}
In future we will further develop this methodology by considering different
fractions of granular materials, which will lead to size dependent sets of DEM simulation parameters.

%\section{Acknowledgements}
This study was funded by Christian Doppler Forschungsgesellschaft, Siemens VAI Metals Technologies and Voestalpine Stahl. The authors gratefully acknowledge their support.
\input{sections/bibliografy}
Copyright note: The authors confirm that they do hold the full copyrights on all parts of their submissions and have permission from all contributors, authors, project partners etc. to publish this material.
The authors allow the organizers of the Minisymposium/Partikelforum to publish their submission online and in print and grant therefore temporary unlimited non-exclusive publication rights free of any charge/fee to the organizers and to chemical-engineering.at.
This does not prevent to publish the submission elsewhere, but the authors need to make sure that they cannot grant exclusive rights to other publishers.

\end{document}