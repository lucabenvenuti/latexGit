\begin{table}[h]
\centering
\scalebox{1.0}{
\begin{tabular}{c|cccccccc}
\hline
          & $\mu_s$ & $\mu_r$ & $COR$ & $\rho_p$ & $\mu_{sh}$ & $\mu_{psh}$ & $\rho_{b}$ & $AOR$ \\
          \hline
    $\mu_s$ & 100.00 & 0.55  & 0.04  & 0.00  & 3.84  & 87.26 & 8.39  & 49.48 \\
    $\mu_r$ & 0.55  & 100.00 & 0.15  & 0.00  & 58.92 & 33.70 & 3.10  & 60.20 \\
    $COR$ & 0.04  & 0.15  & 100.00 & 0.00  & 15.52 & 0.57  & 1.71  & 0.00 \\
    $\rho_p$ & 0.00  & 0.00  & 0.00  & 100.00 & 4.98  & 5.71  & 99.00 & 0.00 \\
    $\mu_{sh}$ & 3.84  & 58.92 & 15.52 & 4.98  & 100.00 & 26.03 & 9.52  & 0.00 \\
    $\mu_{psh}$ & \textbf{87.26} & 33.70 & 0.57  & 5.71  & 26.03 & 100.00 & 4.33 
    & 0.00
    \\
    $\rho_{b}$ & 8.39  & 3.10  & 1.71  & \textbf{99.00} & 9.52  & 4.33  & 100.00
    & 0.00 \\
    $AOR$ & 49.48 & \textbf{60.20} & 0.00  & 0.00  & 0.00  & 0.00  & 0.00  &
    100.00 \\
    
\hline
\end{tabular}}
\caption[Values of linear relationship between considered variables]{Values of
linear relationship between considered variables multiplied for 100. Sliding
friction ($\mu_s$), rolling friction ($\mu_r$) and particle density ($\rho_p$)
influence the most, respectively, the coefficient of pre-shear ($\mu_{psh}$),
the angle of repose  ($AOR$) and the bulk density ($\rho_b$). Notably, $\rho_p$
is not used as training parameter for the $AOR$ bulk behaviour.}
\label{tab:06inputRelationshipTable}
\end{table}