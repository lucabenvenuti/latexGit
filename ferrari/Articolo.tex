              %******************************************%
              %                                          %
              %     Modello di articolo scientifico      %
              %            di Luca Benvenuti             %
              %                                          %
              %         versione: 22 maggio 2013         %
              %                                          %
              %******************************************%
       

% I seguenti commenti speciali impostano:
% 1. utf8 come codifica di input,
% 2. PDFLaTeX come motore di composizione;
% 3. Articolo.tex come documento principale;
% 4. il controllo ortografico italiano per l'editor.

% !TEX encoding = UTF-8
% !TEX TS-program = pdflatex
% !TEX root = Articolo.tex
% !TEX spellcheck = it-IT

\documentclass[10pt,%                       % corpo del font principale
               a4paper,%                    % carta A4
               oneside,%                    % solo fronte
%              twoside,%                    % fronte-retro
               ]{article}                  % classe report di KOMA-Script;
	   
			   
\usepackage{fancyhdr}
               
\usepackage[T1]{fontenc}                    % codifica dei font:
                                            % NOTA BENE! richiede una distribuzione *completa* di LaTeX,
                                          % per esempio TeXLive o MiKTeX *complete*

\usepackage[utf8]{inputenc}              % codifica di input; anche [latin1] va beneutf8
                                           % NOTA BENE! va accordata con le preferenze dell'editor


\usepackage{microtype}                     % microtipografia

\usepackage[italian,english]{babel}        % per scrivere in italiano e in inglese;
                                           % l'ultima lingua (l'italiano) risulta predefinita
                                           
\usepackage[binding=5mm]{layaureo}         % margini ottimizzati per l'A4; rilegatura di 5 mm


\usepackage{lmodern}  					   %new gerahrd
\usepackage{cleveref}  					   %new gerahrd
\usepackage{emptypage}     
\usepackage{indentfirst}                    % rientra il primo capoverso di ogni sezione

\usepackage{booktabs}                       % tabelle
\usepackage{tabularx}                       % tabelle di larghezza prefissata
\usepackage{longtable}
\usepackage{graphicx}                       % immagini

\usepackage{subfig}                         % sottofigure, sottotabelle
\usepackage{caption}                        % didascalie

\usepackage{listings}                       % codici

\usepackage[font=small]{quoting}            % citazioni

\usepackage{amsmath,amssymb,amsthm}        % matematica
\usepackage{amsxtra,amstext,amsfonts}
\usepackage{amscd}						   %chimica

\usepackage[english]{varioref}              % riferimenti completi della pagina

\usepackage{mparhack,fixltx2e,relsize}      % finezze tipografiche
\usepackage[babel]{csquotes}
      
\usepackage[style=numeric-comp,hyperref,backref,backend=bibtex]{biblatex}
\bibliography{Bibliografia}                 % database di biblatex 
                                          
\usepackage{chngpage,calc}                 % centra il frontespizio
\usepackage[dvipsnames]{xcolor}             % colori
\usepackage{color}

\usepackage{lipsum}                        % testo fittizio

\usepackage{eurosym}                       % simbolo dell'euro

\usepackage[pdfa]{hyperref}                      % collegamenti ipertestuali

\usepackage{bookmark}                      % segnalibri
\usepackage[figuresright]{rotating}

\usepackage{array}

\usepackage{enumitem}

\usepackage{colortbl}

\usepackage{floatflt}
\usepackage{titlesec}

\usepackage{eurosym}  %indovina

\setcounter{tocdepth}{5} %to make it appears in TOC
\setcounter{secnumdepth}{5} %to make it numbered

\usepackage{acronym}
\usepackage{rotating}
\usepackage{pdflscape}


%*********************************************************************************
% impostazioni-articolo.tex
% di Luca Benvenuti (2013)
% file che contiene le impostazioni dell'articolo
%*********************************************************************************


%*********************************************************************************
% Comandi personali
%*******************************************************
\newcommand{\myName}{Luca \textsc{Benvenuti}}                            % autore
\newcommand{\myMatricola}{16457}
\newcommand{\myTitle}{Quarterly report} % titolo
\newcommand{\myDegree}{Tesi di laurea}                       % tipo di tesi
\newcommand{\myUni}{JKU} % universit\`a
\newcommand{\myFaculty}{Strongmuslehre}    % facolt\`a
\newcommand{\myDepartment}{Department of \\ Particulate Flow Modelling}        
% dipartimento
\newcommand{\myProf}{Christoph \textsc{Kloss}}    %DI~Dr.~
\newcommand{\myOtherProf}{Stefan \textsc{Pirker}}     %DI~Dr.~         %
% eventuale correlatore \newcommand{\myOtherProff}{Ing.~Gabriele Frigerio}              % eventuale correlatore
%\newcommand{\myCounterProf}{Chiar.mo Prof.~Mister x}    
\newcommand{\myLocation}{Linz}                         % dove
\newcommand{\myTime}{\today}                          % quando
\newcommand{\myPhd}{Materials2Simulation2Application} % titolo
\newcommand{\myemail}{luca.benvenuti@jku.at} % titolo








%*********************************************************************************
% Impostazioni di amsmath, amssymb, amsthm
%*********************************************************************************

% comandi per gli insiemi numerici (serve il pacchetto amssymb)
\newcommand{\numberset}{\mathbb} 
\newcommand{\N}{\numberset{N}} 
\newcommand{\R}{\numberset{R}} 

% un ambiente per i sistemi
\newenvironment{sistema}%
  {\left\lbrace\begin{array}{@{}l@{}}}%
  {\end{array}\right.}

% definizioni (serve il pacchetto amsthm)
\theoremstyle{definition} 
\newtheorem{definizione}{Definizione}

% teoremi, leggi e decreti (serve il pacchetto amsthm)
\theoremstyle{plain} 
\newtheorem{teorema}{Teorema}
\newtheorem{legge}{Legge}
\newtheorem{decreto}[legge]{Decreto}
\newtheorem{murphy}{Murphy}[section]

%simboli matematici vari
\newcommand{\Rot}[1]{\nabla\times\vec{#1}}
\newcommand{\Div}[1]{\nabla\cdot\vec{#1}}
\newcommand{\Grad}[1]{\nabla #1}
\newcommand{\Lap}[1]{\nabla^2#1}
\newcommand{\parder}[2]{\frac{\partial #1}{\partial #2}}
\newcommand{\braket}[3]{\langle #1\,\vert\,\hat{#2}\,\vert\,#3\rangle}
\newcommand{\ud}{\mathrm{d}}
\newcommand{\total}{\mathrm{D}}

%********************************************************************************
% Impostazioni di biblatex
%*********************************************************************************

% \renewcommand\bibname{References} s'incazza
% \addto\captionsenglish{\renewcommand\refname{References}} 
% \addto\captionsitalian{\renewcommand\refname{References}} 

\DefineBibliographyStrings{english}{%
  bibliography = {Bibliography},
  references = {References},
}

\DefineBibliographyStrings{italian}{%
  bibliography = {Bibliography},
  references = {References},
}

\defbibheading{bibliography}{%
\cleardoublepage
\phantomsection 
\addcontentsline{toc}{section}{\refname}

\section*{\refname\markboth{\refname}
{\refname}}}









%*********************************************************************************
% Impostazioni di listings
%*********************************************************************************
\lstset{language=[LaTeX]Tex,%C++,
    keywordstyle=\color{RoyalBlue},%\bfseries,
    basicstyle=\small\ttfamily,
    %identifierstyle=\color{NavyBlue},
    commentstyle=\color{Green}\ttfamily,
    stringstyle=\rmfamily,
    numbers=none,%left,%
    numberstyle=\scriptsize,%\tiny
    stepnumber=5,
    numbersep=8pt,
    showstringspaces=false,
    breaklines=true,
    frameround=ftff,
    frame=single
	tabsize=2,                      % sets default tabsize to 2 spaces
    captionpos=b,                   % sets the caption-position to bottom
} 





%*********************************************************************************
% Impostazioni di hyperref
%*********************************************************************************
\hypersetup{%
    hyperfootnotes=false,pdfpagelabels,
    %draft,	% = elimina tutti i link (utile per stampe in bianco e nero)
    colorlinks=true, linktocpage=true, pdfstartpage=1, pdfstartview=FitV,%
    % decommenta la riga seguente per avere link in nero (per esempio per la stampa in bianco e nero)
    %colorlinks=false, linktocpage=false, pdfborder={0 0 0}, pdfstartpage=1, pdfstartview=FitV,% 
    breaklinks=true, pdfpagemode=UseNone, pageanchor=true, pdfpagemode=UseOutlines,%
    plainpages=false, bookmarksnumbered, bookmarksopen=true, bookmarksopenlevel=1,%
    hypertexnames=true, pdfhighlight=/O,%nesting=true,%frenchlinks,%
    urlcolor=webbrown, linkcolor=RoyalBlue, citecolor=webgreen, %pagecolor=RoyalBlue,%
    %urlcolor=Black, linkcolor=Black, citecolor=Black, %pagecolor=Black,%
    pdftitle={\myTitle},%
    pdfauthor={\textcopyright\ \myName},%
    pdfsubject={},%
    pdfkeywords={},%
    pdfcreator={pdfLaTeX},%
    pdfproducer={LaTeX with hyperref and ClassicThesis}%
}



%*********************************************************************************
% Impostazioni di graphicx
%*********************************************************************************
\graphicspath{{Immagini/}} % cartella dove sono riposte le immagini



%*********************************************************************************
% Impostazioni di xcolor
%*********************************************************************************
\definecolor{webgreen}{rgb}{0,.5,0}
\definecolor{webbrown}{rgb}{.6,0,0}



%*********************************************************************************
% Impostazioni di caption
%*********************************************************************************
\captionsetup{tableposition=top,figureposition=bottom,font=small,format=hang,labelfont=bf}





%*********************************************************************************
% Impostazioni di fancyhdr
%*********************************************************************************
\pagestyle{fancy}
\renewcommand{\sectionmark}[1]{\markboth{\sectionname\ \thesection.\ #1}{}}

\fancyhf{}


\fancyhead[LE,RO]{\thepage}
\fancyhead[RE]{\nouppercase{\leftmark}}
\fancyhead[LO]{\nouppercase{\rightmark}}


\renewcommand{\headrulewidth}{0.5pt}

\renewcommand{\footrulewidth}{0pt}
\fancyheadoffset{0\columnwidth}	


%*********************************************************************************
% Altro
%*********************************************************************************

% [...] ;-)
\newcommand{\omissis}{[\dots\negthinspace]}

% eccezioni all'algoritmo di sillabazione
\hyphenation{Fortran ma-cro-istru-zio-ne nitro-idrossil-amminico}



\newcommand{\HRule}{\rule{\linewidth}{0.5mm}}
\newcommand{\RM}[1]{\MakeUppercase{\romannumeral #1}} 
               % file con le impostazioni personali



\begin{document}
\pagestyle{headings} 
%******************************************************************
% Materiale iniziale
%******************************************************************
%% !TEX encoding = UTF-8
% !TEX TS-program = pdflatex
% !TEX root = ../Articolo.tex
% !TEX spellcheck = it-IT

%*******************************************************
% Frontespizio
%*******************************************************
%\maketitle
\begin{center}
\Large \textbf {\myTitle}\\
\bigskip
\normalsize {Author One and Author Two}\\
\normalsize {Your College\\Your University\\{\today}}
\end{center}
% !TEX encoding = UTF-8
% !TEX TS-program = pdflatex
% !TEX root = ../Tesi.tex
% !TEX spellcheck = it-IT

%*******************************************************
% Indici
%*******************************************************


\begingroup

\renewcommand{\chaptermark}[1]{\markboth{#1}{}}

\fancyhf{}
%\fancyfoot[C]{\thepage}
%%%\fancyhead[LO]{\rightmark}
%\fancyhead[LE,RO]{\leftmark}
%\fancyhead[RE,LO]{}

\fancyhead[LE,RO]{\thepage}
\fancyhead[RE]{\nouppercase{\leftmark}}
\fancyhead[LO]{\nouppercase{\rightmark}}


\renewcommand{\headrulewidth}{0.5pt}

\renewcommand{\footrulewidth}{0pt}
\fancyheadoffset{0\columnwidth}


\cleardoublepage
\pdfbookmark{\contentsname}{tableofcontents}
%\setcounter{tocdepth}{2}
\tableofcontents
%\markboth{\contentsname}{\contentsname} 
\clearpage

%\begingroup 
    %\let\clearpage\relax
    \let\cleardoublepage\relax
    \let\cleardoublepage\relax
    %*******************************************************
    % Elenco delle figure
    %*******************************************************    
    \phantomsection
    \pdfbookmark{\listfigurename}{lof}
    \listoffigures

 %   \vspace*{8ex}
\clearpage
    %*******************************************************
    % Elenco delle tabelle
    %*******************************************************
    \phantomsection
    \pdfbookmark{\listtablename}{lot}
    \listoftables
        
    \vspace*{8ex}
       
\endgroup

\cleardoublepage

%
%% !TEX encoding = UTF-8
% !TEX TS-program = pdflatex
% !TEX root = ../Tesi.tex
% !TEX spellcheck = it-IT

%*******************************************************
% Sommario+Abstract
%*******************************************************
\cleardoublepage
\phantomsection
\pdfbookmark{Abstract}{Abstract}
\begingroup
\let\clearpage\relax
\let\cleardoublepage\relax
\let\cleardoublepage\relax


\renewcommand{\chaptermark}[1]{\markboth{#1}{}}

\fancyhf{}
\fancyhead[LE,RO]{\thepage}
\fancyhead[RE]{\nouppercase{\leftmark}}
\fancyhead[LO]{\nouppercase{\rightmark}}
\renewcommand{\headrulewidth}{0.5pt}

\renewcommand{\footrulewidth}{0pt}
\fancyheadoffset{0\columnwidth}



\chapter*{Abstract}

L'ex cava Prete Santo di San Lazzaro di Savena, complesso estrattivo ormai abbandonato, � da tempo oggetto di attenzione da parte delle autorit�, in quanto si � osservato che il materiale gesso che costituisce la miniera � soggetto a degradazione progressiva.\\
Infatti ci� che le autorit� temono � la propagazione del processo sino a coinvolgere, seppure marginalmente, le unit� abitative poste in superficie.\\
In questa tesi sono contestualizzate e discusse le evidenze sperimentali ottenute esaminando il materiale prelevato dall'ex cava. Esse hanno permesso di determinare i valori di resistenza del materiale in sito e di effettuare analisi numeriche 3D agli Elementi Finiti, cos� da definire i possibili scenari di collasso.\\
Sono stati anche considerati gli interventi di mitigazione del rischio, basati sull'utilizzo di cerchiature in calcestruzzo e geosintetici ad alta resistenza.
%\vfill

\selectlanguage{english}
\pdfbookmark{Abstract in english}{Abstract in english}
\chapter*{Abstract in english}

An abandoned gypsum cave, Prete Santo in the town of San Lazzaro di Savena (BO), has been under observation by local authorities for many years, because the gypsum, which is the mine's constituent, is subjected to gradual degradation.\\
The authorities believe that the propagation could extend until the involvement, even if marginally, of the the residential area on its surface.\\
Here we discussed the experimental evidence gained through focused investigation of the directly collected cave materials, and calculated
the in situ strengh values allowing to perform a 3D Finite Element analysis to model the possible scenarios.\\
Moreover, we examined the main risk reduction action plans, based on the application of circular structures in concrete and
high-performance geocomposites.

\selectlanguage{italian}

\endgroup			

\vfill

%, whose contributions were then further modeled in case of collapse, adopting a reduction-of-external-subsidence perspective
%, a seguito della quale i pilastri che sostengono il sistema perdono rigidezza e resistenza, con il rischio di propagare cedimenti in %superficie.\\
%
%constitutes a significant hydrogeological risk because of the presence of residential areas on its top. Gypsum degradation leads to loss %of stiffness and strengh in the mine pillars, thus potentially propagating subsidence on surface.\\
%******************************************************************
% Materiale principale
%******************************************************************

%% !TEX encoding = UTF-8
% !TEX TS-program = pdflatex
% !TEX root = ../Articolo.tex
% !TEX spellcheck = it-IT

%************************************************
\section{Scope of work}
\label{sec:scopeofwork}
%************************************************

The purpose of this PhD thesis is to utilize, develop and set up a handful of characterization devices.
These should be user friendly, as mechanized as possible (to avoid operator bias) and easy to set up, of course with the proper manuals that should also be written.\\
After the data collection physical key parameters of particle flows will be identified; 
this physical information should allow the calibration of series of trustworthy numerical parameters for \ac{dem} (\ac{liggghts}) and FEM (CFDEMcoupling) reliable simulations.
The knowledge gained so far should be applied to industrial processes.\\

%% !TEX encoding = UTF-8
% !TEX TS-program = pdflatex
% !TEX root = ../Articolo.tex
% !TEX spellcheck = it-IT

%************************************************
\section{Stages}
\label{section:stages}
%************************************************

The work stages defined until now are the following:

\begin{enumerate}
\item{\textbf{References collection}, \textit{and literature study};}
\item{\textbf{DEM Characterization Workflow Coarse Particles}, \textit{based on existing experiments};}
\item{\textbf{Set up of a Jenike shear cell}, \textit{through a training trip to the University of Braunschweig};}
\item{\textbf{DEM Characterization Workflow Powder}, \textit{temptative approach with the hollow sphere experiment};}
\item{\textbf{CFD Characterization Workflow Coarse Particles}, \textit{pressure drop in packed bed; lateral jet experiment};}
\item{\textbf{CFD Characterization Workflow Powder}, \textit{maybe thanks to the fluidized bed experiment};}
\item{\textbf{Sinter plant simulation}, \textit{the first application};}
\item{\textbf{Blast furnace simulation}, \textit{the second application};}
\item{\textbf{Conclusions writing};}
\end{enumerate}

%% !TEX encoding = UTF-8
% !TEX TS-program = pdflatex
% !TEX root = ../Articolo.tex
% !TEX spellcheck = it-IT

%************************************************
\section{References collection}
\label{section:referencescollection}
%************************************************

The first step of work has been the collection of some of the books and articles that concern the subject analyzed.
They can be found in the \hyperref[section:bibliography]{bibliography section (\ref{section:bibliography})}.\\


%% !TEX encoding = UTF-8
% !TEX TS-program = pdflatex
% !TEX root = ../Articolo.tex
% !TEX spellcheck = it-IT

%************************************************
\section{Materials}
\label{section:materials}
%************************************************

With the agreement of the industrial partners, the range of the raw materials to be characterized has been extended until 50 mm of diameter.
Especially, we have also added lump ore.\\

%% !TEX encoding = UTF-8
% !TEX TS-program = pdflatex
% !TEX root = ../Tesi.tex
% !TEX spellcheck = it-IT

%************************************************
\chapter{DEM Parameters}
\label{cap:demparameters}
%************************************************

\lipsum[1]

\section{Literature Values}
\label{sec:literaturevalues}

\lipsum[1]

\section{Particle Distribution}
\label{sec:particledistribution}

\lipsum[1]

\subsection{coke}
\label{subsec:coke}

\lipsum[2]

\section{Bulk Density}
\label{sec:bulkdensity}


\lipsum[1]


\section{Angle of Repose (p-p) - Small Scale}
\label{sec:aor}


\lipsum[1]

\section{Angle of Repose (p-p) - Large Scale}
\label{sec:aorlargescale}


\lipsum[1]

\section{Angle of Repose Simulation}
\label{sec:aorsimulation}


\lipsum[1]

\section{Maximum Static Angle (p-w)}
\label{sec:msa}
%************************************************

\lipsum[1]

\section{Maximum Static Angle Simulation}
\label{sec:msasimulation}
%************************************************

\lipsum[1]

\section{Schulze Ring Shear Cell tester (p-p)}
\label{sec:SRSCT}
%************************************************

\lipsum[1]

\section{Jenike Shear Cell tester}
\label{sec:jsct}
%************************************************

\lipsum[1]

\subsection{p-p}
\label{subsec:JSCTpp}

\lipsum[2]

\subsubsection{Instructions}
\label{subsubsec:instructions}

1. Scope*
1.1 This method covers the apparatus and procedures for measuring the cohesive strength of bulk solids during both continuous flow and after storage at rest. In addition, measurements of internal friction, bulk density, and wall friction on various wall surfaces are included.\\
1.2 This standard is not applicable to testing bulk solids that do not reach the steady state requirement within the travel limit of the shear cell. It is impossible to classify ahead of time which bulk solids cannot be tested, but one example may be those consisting of highly elastic particles. \\
1.3 The values stated in SI units are to be regarded as standard.\\
1.4 The most common use of this information is in the design of storage bins and hoppers to prevent flow stoppages due to arching and ratholing, including the slope and smoothness of hopper walls to provide mass flow. Parameters for structural design of such equipment also may be derived from this data.\\
3. Terminology \\
3.1 Definitions: \\
3.1.1 Definitions of terms used in this test method are in accordance with Terminology D653. \\
3.1.2 adhesion test, a static wall friction test with time consolidation. \\
3.1.3 angle of internal friction, $\phi_e$, the angle between the axis of normal stress (abscissa) and the tangent to the yield locus. \\
3.1.4 angle of wall friction, $\phi_w$, the arctan of the ratio of the wall shear stress to the wall normal stress. \\
3.1.5 bin, a container or vessel for holding a bulk solid, frequently consisting of a vertical cylinder with a converging hopper. Sometimes referred to as silo, bunker, or elevator. \\
3.1.6 bulk density,  $\rho_b$, the mass of a quantity of a bulk solid divided by its total volume. \\
3.1.7 bulk solid, an assembly of solid particles handled in sufficient quantities that its characteristics can be described by the properties of the mass of particles rather than the characteristics of each individual particle. May also be referred to as granular material, particulate solid, or powder. Examples are sugar, flour, ore, and coal. \\
3.1.8 bunker, synonym for bin, but sometimes understood as being a bin without any or only a small vertical part at the top of the hopper. \\
3.1.9 cohesive strength, synonym for unconfined yield strength. \\
3.1.10 consolidation, the process of increasing the strength of a bulk solid. \\
3.1.11 critical state, a state of stress in which the bulk density of a bulk solid and the shear stress in the shear zone remain constant. \\
3.1.12 effective angle of friction, $\delta$, the inclination of the effective yield locus (EYL). \\
3.1.13 effective yield locus (EYL), straight line passing through the origin of the $\sigma, \tau$-plane and tangential to the steady state Mohr circle, corresponding to steady state flow conditions of a bulk solid of given bulk density. \\
3.1.14 elevator, synonym for bin, commonly used in the grain industry. \\
3.1.15 failure (of a bulk solid), plastic deformation of an overconsolidated bulk solid subject to shear, causing dilation and a decrease in strength. \\
3.1.16 flow, steady state, continuous plastic deformation of a bulk solid at critical state.  \\
3.1.17 flow function, FF, the plot of unconfined yield strength versus major consolidation stress for one specific bulk solid. \\
3.1.18 granular material, synonym for bulk solid. \\
3.1.19 hopper, the converging portion of a bin. \\
3.1.20 major consolidation stress, $\sigma_1$, the major principal stress given by the Mohr stress circle of steady state flow. This Mohr stress circle is tangential to the effective yield locus. \\
3.1.21 Mohr stress circle, the graphical representation of a state of stress in coordinates of normal and shear stress, that is, in the $\sigma, \tau$-plane. \\
3.1.22 normal stress, $\sigma$, the stress acting normally to the considered plane. \\
3.1.23 overconsolidated specimen, a condition in which the shear force passes through a maximum and then decreases during preshear. \\
3.1.24 particulate solid, synonym for bulk solid. \\
3.1.25 powder, synonym for bulk solid, particularly when the particles of the bulk solid are fine. \\
3.1.26 silo, synonym for bin. \\
3.1.27 shear test, an experiment to determine the flow properties of a bulk solid by applying different states of stress and strain to it. \\
3.1.28 shear tester, an apparatus for performing shear tests. \\
%##3.1.29 time angle of internal friction, ft, inclination of the time yield locus of the tangency point with the Mohr stress circle passing through the origin. \\
%##3.1.30 time yield locus, the yield locus of a bulk solid which has remained at rest under a given normal stress for a certain time. \\ 
3.1.31 unconfined yield strength, $f_c$, the major principal stress of the Mohr stress circle being tangential to the yield locus with the minor principal stress being zero.A synonym for compressive strength. \\
3.1.32 underconsolidated specimen, a condition in which the shear force increases continually during preshear. \\
3.1.33 wall normal stress, $\sigma_w$, the normal stress present at a confining wall. \\
3.1.34 wall shear stress, $\tau_w$, the shear stress present at a confining wall. \\
3.1.35 wall yield locus,  a plot of the wall shear stress versus wall normal stress. The angle of wall friction is obtained from the wall yield locus as the arctan of the ratio of the wall shear stress to wall normal stress. \\
3.1.36 yield locus, plot of shear stress versus normal stress at failure. The yield locus (YL) is sometimes called the instantaneous yield locus to differentiate it from the time yield locus. \\
 
4. Summary of Test Method \\
4.1 A representative sample of bulk solid is placed in a shear cell of specific dimensions. This specimen is preconsolidated by twisting the shear cell cover while applying a compressive load normal to the cover.
4.2 When running an instantaneous
% or time shear test
, a normal load is applied to the cover, and the specimen is presheared until a steady state shear value has been reached. \\
4.3 An instantaneous test is run by shearing the specimen under a reduced normal load until the shear force goes through a maximum value and then begins to decrease. \\
%4.4 A time shear test is run similarly to an instantaneous
%shear test, except that the specimen is placed in a consolidation
%bench between preshear and shear.
4.5 A wall friction test is run by sliding the specimen over a coupon of wall material and measuring the frictional resistance as a function of normal, compressive load. \\
%4.6 A wall friction time test involves sliding the specimen
%over the coupon of wall material, leaving the load on the
%specimen for a predetermined period of time, then sliding it
%again to see if the shearing force has increased.

5. Significance and Use \\
5.1 Reliable, controlled flow of bulk solids from bins and hoppers is essential in almost every industrial facility. Unfortunately, flow stoppages due to arching and ratholing are common. Additional problems include uncontrolled flow (flooding) of powders, segregation of particle mixtures, useable capacity which is significantly less than design capacity, caking and spoilage of bulk solids in stagnant zones, and structural failures. \\
5.2 By measuring the flow properties of bulk solids, and designing bins and hoppers based on these flow properties, most flow problems can be prevented or eliminated. \\
5.3 For bulk solids with a significant percentage of particles (typically, one third or more) finer than about 6 mm (1/4 in.), the cohesive strength is governed by the fines (6mm fraction). For such bulk solids, cohesive strength and wall friction tests may be performed on the fine fraction only. \\
NOTE 1: The quality of the result produced by this test method is dependent on the competence of the personnel performing it, and the suitability of the equipment and facilities used. \\


6. Apparatus
6.1 The Jenike shear cell is shown in \textbf{Fig. 1}. It consists of a base (1), shear ring (2), and shear lid (3), the latter having a bracket (4) and pin (5). Before shear, the ring is placed in an offset position as shown in \textbf{Fig. 1}, and a vertical force $F_v$ is applied to the lid, and hence, to the particulate solid within the cell by means of a weight hanger (6) and weights (7). A horizontal force is applied to the bracket by a mechanically driven measuring stem (8). \\
6.2 It is especially important that the shear force measuring stem acts on the bracket in the shear plane (plane between base and shear ring) and not above or below this plane. \\
6.3 The dimensions of the Jenike shear cells supplied by Jenike and Johanson, Inc. are given in the first two columns of the table in \textbf{Fig. 4}. These dimensions have been derived from English units. The standard size Jenike shear cell is made from aluminum or stainless steel, and a smaller 63mm diameter cell made from stainless steel is also available. Since the actual dimensions are not believed to be critical, the same results could be obtained with a shear cell of the dimensions listed in the third column of the table in \textbf{Fig. 4} or with other shear cells of different sizes provided that proportions of these dimensions are maintained approximately. In addition, \textbf{the shear cell diameter must be at least 20 times the maximum particle size of the bulk solid being tested}. Besides the shear cell, the complete shear tester includes a force transducer which measures the shear force $F_s$, an amplifier and a recorder, a motor driving the force measuring stem, a twisting wrench, a weight hanger, 
%a time consolidation bench, 
an accessory for mounting wall material sample plates, and a calibrating device. A spatula having a blade at least 50 % longer than the diameter of the shear cell, 
%and at least a 10-mm width, 
is needed. The force  transducer should be capable of measuring a force up to 300 N with a precision of 0.1 % of full scale. The signal from the force transducer is conditioned by an amplifier and shown on
a recorder. The motor driving the force measuring stem advances the stem at a constant speed in the range from 1 to 3 mm/min. \\

7. Specimen Preparation \\
7.1 Filling the Cell \textbf{Fig. 8}: \\
7.1.1 Place the shear ring on the base in the offset position shown in \textbf{Fig. 1} and gently press the ring with the fingers against the locating screws (10) as shown in Fig. 3 and Fig. 9. Set these  screws to give an overlap of approximately 3 mm for standard cell sizes and to ensure that the axis of the cell is aligned with the force measuring stem. Then place the mould ring (11) on the shear ring. \\
7.1.2 Fill the assembled cell uniformly in small horizontal layers by a spoon or spatula without applying force to the surface of the material until the material is somewhat over the top of the mould ring. The filling should be conducted in such a way as to ensure that there are no voids within the cell, particularly at ??????? \textbf{Fig. 8} where the ring and the base overlap. Remove excess material in small quantities by scraping off with a blade (1). The blade should be scraped across the ring in a zig-zag motion. Take care not to disturb the position of the ring on the base. For scraping, a rigid sharp  straight blade should be used, and, during scraping, the blade should be tilted as shown in \textbf{Fig. 8}. \\
7.2 Preconsolidation:\\
7.2.1 Place the twisting or consolidation lid (12) shown in \textbf{Fig. 9} on the leveled surface of the material in the mould, then place the hanger (6) on the twisting lid with weights (7) of mass $m_{Wtw}$ being hung from the hanger. See \textbf{Fig. 1}. Lower the lid, hanger, and weights as slowly as possible to minimize aerated material being ejected from the cell. \\
7.2.2 Visually observe the vertical movement of the lid as the material of the cell is compressed.Wait until this movement appears to stop. \\
7.2.3 Remove the weights, hanger, and twisting lid. Fill and level the space above the compressed material as during filling. \\
NOTE 3: As will be mentioned later, this refilling procedure may not be necessary at all or may need to be performed several times, depending on the compressibility of the powder being tested. This operation determines what height of compacted material will have to be scraped off the ring after twisting. \\
7.3 Twisting: \\
7.3.1 Place the twisting lid (12) with a smooth bottom surface on the leveled surface of material in the mould after filling or refilling. Place the hanger with weights of $m_{Wtw}$ on the twisting lid. The weights on the hanger should correspond to a pressure of $\sigma_{tw}$, approximately equal to $\sigma_{p}$. \\
7.3.2 Empty the cell and repeat the filling operation if the surface of material in the cell does not appear to the naked eye to be level. \\
7.3.3 Having filled the cell, the twisting lid is usually twisted through 20 cycles by means of the twisting wrench (spanner) (13) or twisting device. Each twisting cycle consists of a $90 ~ degrees$ rotation of the lid which is then reversed. Care must be taken not to apply vertical forces to the lid during twisting. While twisting, press the ring against the locating screws with the fingers to prevent it from sliding from its original offset position.\\

\subsection{p-w}
\label{subsec:JSCTpw}

\lipsum[3]

\section{Shear Cell Simulation}
\label{sec:scsimulation}
%************************************************

\lipsum[1]


\section{Coefficient of Restitution}
\label{sec:COR}
%************************************************

\lipsum[1]

\subsection{p-p}
\label{subsec:CORpp}

\lipsum[2]

\subsection{p-w}
\label{subsec:CORpw}

\lipsum[3]

\section{Coefficient of Restitution Simulation - Estimation Matlab}
\label{sec:corsimulation}
%************************************************

\lipsum[1]

%% !TEX encoding = UTF-8
% !TEX TS-program = pdflatex
% !TEX root = ../Articolo.tex
% !TEX spellcheck = it-IT

%************************************************
\section{DEM Characterization Workflow Coarse Particles}
\label{section:Demcharacterizationworkflowcoarseparticles}
%************************************************

\subsection{Jenike's  Shear Cell tester}
\label{subsection:jenikeshearcell}

The Jenike's  Shear Cell tester (JSCT) is up and running, with also bigger rings for coarser particles, just minor details need to be done:
\begin{itemize}
\item{paint it with normal orange anti-rust painting;}
\item{establish a non-standard procedure to use the vertical laser displacement measurement and to analyze its results;}
\end{itemize}

\subsubsection{SCT simulation}
\label{subsubsection:sctsimulation}

I will correct the results of the internal friction plateau average manually, because not in all cases they were correct (I started the plateau averaging too soon in time).
I will also counter-check the data in future, unless I will find a reliable automatic evaluation tool.\\

\subsection{Large Scale Angle of Repose}
\label{subsection:largescaleaor}

I went to Leoben VAS facility to use their new rotating double chute: they performed 9 large scale dynamic angle of repose tests. I am now studying how to use these data as comparison for the small scale and the simulations.\\

\subsection{Granulometric curves}
\label{subsection:granulometric curves}

I have completed sieving of the fine materials (from 0 to 3.15 mm) and I have for them:
\begin{itemize}
\item{the granulometric curve;}
\item{the mean radius (R);}
\end{itemize}
I will ASAP (weather allowing) perform the sieving of the coarse materials.\\

From now on I will define:
\begin{itemize}
\item{$\alpha$, a simulation performed with monodisperse spheres, with the radius equal to the mean radius of the material studied or idelized;}
\item{$\beta$, with bi-disperse spheres, with 2 radii, trying to linearize the granulometric curve of the material studied or idelized;}
\item{$\gamma$, with polydisperse spheres, with all the radii of the sieving \textit{except the last one} of the material studied or idelized, to optimize the number of particles;}
\item{$\delta$, with polydisperse spheres, with all the radii of the sieving of the material studied or idelized;}
\end{itemize}



%% !TEX encoding = UTF-8
% !TEX TS-program = pdflatex
% !TEX root = ../Articolo.tex
% !TEX spellcheck = it-IT

%************************************************
\section{CFDEM Characterization Workflow Coarse Particles}
\label{section:CFDemcharacterizationworkflowcoarseparticles}
%************************************************

\subsection{Pressure drop tester}
\label{subsection:pressuredroptester}

I should finish soon (I must confess a delay) the realization of a second pressure drop tester, with a diameter 50 \% larger than the old one.
I will also replicate the awesome work done by Niklas Hofer concerning the insurance of a very well defined uniform boundary condition for the temperature.
In opposition to what I saw in Trondheim, that would drive to a really realiable series of experiments.\\

\subsubsection{Pressure drop tester simulation}
\label{subsubsection:pressuredroptestersimulation}

I have not managed to reach the requested volume fraction of 0.6, but only 0.5 working on the input script. The code is currently under modification by Christoph Goniva and Daniel Q. is writing a GUI for the simulation.
I will restart working on it as soon as I will receive both.\\

% % !TEX encoding = UTF-8
% !TEX TS-program = pdflatex
% !TEX root = ../Articolo.tex
% !TEX spellcheck = it-IT

%************************************************
\section{Behavior Investigation}
\label{section:behaviorinvestigation}
%************************************************

This is the second draft of what should be the main topic of my PhD Thesis. The final version will enter and modify accordingly the ToC.

\subsection{Investigation topcis}
\label{subsection:investigationtopics}

In order to define the scientific core of my PhD Thesis the following themses could be investigated:
\begin{enumerate}
\item{the micro-macro transition,}
\item{the influence of variations (distributions) of input parameters and poly-dispersity,}
\item{the behavior of the different properties in real life (e.g. segregation before doing the shear cell experiment),}
\item{the possbility to extrapolate (e.g. given 3 different fraction distributions, with known behaviors, extrapolate the behavior of a fourth fraction distribution).}
\end{enumerate}
Design of experiments and Oberkampf's guidelines will be used to perform the investigations.
The selection and number of simulations necessary to accomplish the tasks have yet to be decided.

\subsection{Micro - DEM parameters}
\label{subsection:microparameters}

The main micro parameters that characterize a single sphere in a simulation are:

\begin{enumerate}[label=(\Alph*)]
\item{the particle diameter distribution ($radii$ (R),\%) (0.00025 - 0.05 $[m[$),}
\item{the particle density ($\rho_p$) (2000-4000 $[kg/m^3]$),}
\item{the Young modulus ($E$) (5-10 $[MPa]$),}
\item{the Poisson ratio ($\nu$) (0-0.5 $[-]$),}
\item{the coefficient of restitution ($e$) (0.1-1.0 $[-]$),}
\item{the sliding friction ($sf$) (0.1-1.0 $[-]$),}
\item{the rolling friction ($rf$) (0.1-1.0 $[-]$),}
\item{the domain edge dimension ($D_{cyl}$), proportional to R (76, 100, 124 times bigger).}
\end{enumerate}

The parameter that drives the simulation time is $D_{cyl}$. The number of particles is cubically proportional to its size. An estimation is given in Tab. \ref{tab:timestimation} with 32 cores on gollum.

\begin{table}[htbp]
  \centering
  \caption{Time estimation in hours for each shear cell simulation, see Draft 1}
    \begin{tabular}{c|ccc}
    \toprule
    $D_{cyl}$ & 76    & 100   & 124 \\
    \midrule
    $\alpha$ & 0.5   & 2     & 24 \\
    $\beta$ & 2     & 24    & 168 \\
    $\gamma$ & 168   & 336   & ? \\
    $\delta$ & 720   & ?     & ? \\
    \bottomrule
    \end{tabular}%
  \label{tab:timestimation}%
\end{table}%


\subsection{Macro - bulk parameters}
\label{subsection:macroparameters}

The main macro parameters we experimentally determine for a bulk material are:

\begin{enumerate}[label=(\alph*)]
\item{the particle diameter distribution ($radii$) (0.00001 - 0.05 $[m[$),(\%),}
\item{the bulk density ($\rho_b$) (1000-3000 $[kg/m^3]$),}
\item{the Young modulus ($E$) (5-10 $[MPa]$),}
\item{the Poisson ratio ($\nu$) (0-0.5 $[-]$),}
\item{the coefficient of restitution ($e$) (0.1-1.0 $[-]$),}
\item{the cohesion in different loading conditions ($c'$) (0-100 $[kPa]$),}
\item{the internal friction angle in different loading conditions ($\phi$) (25-50 $[^\circ]$).}
\end{enumerate}

Angle of Repose simulation, individual contacts micro-DEM parameters:
Sliding friction: from 0.05 to 1, 0.05 interval $=$ 20 possibilities;
Rolling friction: from 0.05 to 1, 0.05 interval $=$ 20 possibilities;
Coefficient  of restitution: from 0.5 to 1, 0.1 interval $=$ 5 possibilities;
Particle density: e.g. from 2500 to 3500 $kg/m3$, 100 interval $=$ 11 possibilities;
(Particle to geometry ratio: 50, 100, 150 $=$ 3 possibilities;)
20 x 20 x 5 x 11 x 3 $=$ 22000 possible combinations of parameter
Each combination would require a simulation, each simulation requires 9 hours over 32 cores $=$ 8250 days!
All for only one material with a defined size distribution (e.g. from 3 to 10 mm)!
We only consider p2g $=$ 50, because otherwise with e.g. 100 the number of particles is 8 times higher, and so the time 8 times longer

Shear cell simulation, individual contacts micro-DEM parameters:
Sliding friction: from 0.05 to 1, 0.05 interval $=$ 20 possibilities;
Rolling friction: from 0.05 to 1, 0.05 interval $=$ 20 possibilities;
Coefficient  of restitution: from 0.5 to 1, 0.1 interval $=$ 5 possibilities;
Particle density: e.g. from 2500 to 3500 kg/m3, 100 interval $=$ 11 possibilities;
Normal stress: 1, 2, 5, 10 kPa $=$ 4 possibilities;
Shear %: 40, 60, 80, 100% $=$ 4 possibilities;
(Particle to geometry ratio: 20, 36, 38, 40, 100, 150 $=$ 6 possibilities;)
20 x 20 x 5 x 11 x 4 x 4 $=$ 352000 possible combinations of parameter
Each combination would require a simulation, each simulation requires 64 minutes over 32 cores $=$ 15644 days!
All for only one material with a defined size distribution (e.g. from 3 to 10 mm)!


\subsection{Artificial neural network}
\label{subsection:artificialneuralnetwork}

The representation of a multi-sphere experiment through a mathematical model presents numerous complexities.
Do we really need all these simulations?
How to identify the effect (weight) each parameter?
A different approach involves an artificial neural network, that can be realized to understand the relationship between the input and the output parameters.
An artificial neural network is composed of many artificial neurons that are linked together according to a specific network architecture. The objective of the neural network is to transform the inputs into meaningful outputs.
The inputs are:
\begin{enumerate}
\item{\textit{A}, from real data \textit{a} but simplified,}
\item{\textit{B,H},}
\item{\textit{C,D}, from literature,}
\item{\textit{E}, from real data \textit{e},}
\item{\textit{F,G}, the main calibration parameters,}
\end{enumerate}

The outputs are \textit{f} and \textit{g}.\\

The idea would be to use a \textit{backpropagation supervised learning}. The already performed simulations could be used as follows: given e.g. 5000 simulations, I will use 3000 of them as training set, and 1000 of the remaining ones could be used to validate the neural system from the $simulation$ point of view.
In this way I could study \textit{the influence of variations of input parameters and poly-dispersity}.\\

Between the 1000 simulations remained, as hypothesis 100 of them are experimentally validated. I could use 80 of them to further refine the weights of the neural network functions, and the remaining 20 to validate this second step. At this point I should be able to \textit{extrapolate the behavior of bulk with a volume too large to be simulated} with this complete neural network and to compare the results with the real scale experiments (Leoben).\\

The reliability of this work is deeply based on the already performed simulations: further numerical investigations as suggested in draft $1$ could improve it.

By reserving a portion of the simulations, we can use them to establish the most effective number of neurons inside the hidden layer, for each bulk parameter, through maximization of R2.\\

Now that the networks have been trained, we can feed the networks with all the combination we need and receive reliable macro-BULK parameters as response.
Especially, we can increase the parameter that account for the geometry dimension, without paying completely the price for it.
E.g.: for the shear cell only $~100$ simulations can be done with $g2p >20$, but thanks to them the network can be trained to understand how that parameter relates with the others $=$ We can expand this consideration and evaluate the effect of the micro-DEM parameters over the large scale AOR test!


\subsubsection{Test run with Matlab neural fitting tool}
\label{subsubsection:testrunmatlab}

I tried a run with Matlab neural fitting tool.
773 simulations of the Jenike shear cell tester have been analyzed, with monodispersed spheres with the same radius, Young's modulus, $\nu$, $\rho_p$, $e$ and domain dimensions.
The inputs values are F, G, the normal load until steady-state flow and the \% of it during the second phase.
Especially, these \% are 40\%, 60\% and 80\%.
All other DEM values are identical for all the 773 simulations.
The target value for each simulation is the coefficient of internal friction in the the second phase.
I use 10 hidden neurons in the hidden layer and 1 neuron in the output layer.
Matlab "eats" 541 simulations to train the network, 116 to validate it and 116 to test it.\\
Now I use the network. 
The inputs values are F, G, the normal load until steady-state flow and the \% of it during the second phase.
Especially, this is only 100\%.
I have a total of 256 combinations.
I compare the coefficient of internal friction in the second phase of the simulations and of the network in these 256 cases.

 \begin{equation}
\langle{ \frac{\mu_{sh,sim}-\mu_{sh,neural}}{\mu_{sh,sim}}}\rangle = -1.16 \%
\end{equation}

The average value is really promising for this approach.

  % C_{kl} = 
 % \begin{cases}
% 1 & \text{if } (\lvert{{\mu_{psh,exp}}}\rvert < 5\% ~\text{and}~ \lvert{1-\frac{\mu_{sh,sim}}{\mu_{sh,exp}}}\rvert < 5\% ) ,\\
% 0 & \text{else} .\\ 
% \end{cases}
 % \label{eq:check}
% !TEX encoding = UTF-8
% !TEX TS-program = pdflatex
% !TEX root = ../Articolo.tex
% !TEX spellcheck = it-IT

%************************************************
\section{Model Identification}
\label{sec:modelidentification}
%************************************************

Specific operating ranges  in N/alpha domain:
\begin{itemize}
	\item{Identification data: JKU DOE 2, augmented Ferrari cycle ($data_acq_tuesday_augFcycle001$).}
	\item{Validation data: JKU DOE 1 ($data_acq_tuesday_jkucycle001$).}
\end{itemize}

Please consider: number of poles (np), number of zeros (nz),polynomial degree (dPl), order of target y (oY), fit value with down sampling of 50  500 ms (ft50), MAX ERROR with down sampling of 50  (mxDf50), mean square error with down sampling of 50 (mse50), etc with higher down sampling.

\subsection{Left turbo revolutions (giri tsx)}
\begin{itemize}
	\item{inputs: Turbocharger Left Way Out Air Pressure (P2 Sx), Intercooler left way out air pressure(P5 SX)}
	\item{output: Left turbo revolutions (giri tsx)}
\end{itemize}	

\begin{landscape} 
 \begin{center} 
 \footnotesize 
 \begin{longtable}{ll|cccc|ccc|ccc|ccc|ccc} 
\caption[inputs P2 SX P5 SX   outputs GIRI TSX]{inputs P2 SX P5 SX   outputs GIRI TSX.} 
\label{tab:inputs_P2_SX_P5_SX___outputs_GIRI_TSX} 
\hline 
  mdl & type & np & nz & dPl & oY & ft50 & mxDf50 & mse50 & ft100 & mxDf100 & mse100 & ft250 & mxDf250 & mse250 & ft500 & mxDf500 & mse500 \\ 
 \hline 
tf  & iden & 1 & 1 & 0 & 0 & 76.7 & 0.35 & 0.00 & 75.1 & 0.40 & 0.00 & 66.7 & 0.51 & 0.00 & 52.5 & 0.60 & 0.00 \\ 
tf  & sim  & 1 & 1 & 0 & 0 & 68.7 & 0.37 & 0.00 & 66.3 & 0.51 & 0.00 & 56.7 & 0.50 & 0.00 & 38.5 & 0.61 & 0.00 \\ 
 \hline 
tf  & iden & 1 & 2 & 0 & 0 & 77.3 & 0.35 & 0.00 & 75.4 & 0.43 & 0.00 & 66.7 & 0.53 & 0.00 & 53.4 & 0.59 & 0.00 \\ 
tf  & sim  & 1 & 2 & 0 & 0 & 75.8 & 0.34 & 0.00 & 73.1 & 0.49 & 0.00 & 61.9 & 0.50 & 0.00 & 40.9 & 0.58 & 0.00 \\ 
 \hline 
tf  & iden & 2 & 1 & 0 & 0 & 76.7 & 0.34 & 0.00 & 75.1 & 0.40 & 0.00 & 66.8 & 0.51 & 0.00 & 7.7 & 0.63 & 0.00 \\ 
tf  & sim  & 2 & 1 & 0 & 0 & 68.7 & 0.39 & 0.00 & 66.5 & 0.50 & 0.00 & 57.2 & 0.48 & 0.00 & -6.1 & 0.46 & 0.00 \\ 
 \hline 
tf  & iden & 2 & 2 & 0 & 0 & 77.2 & 0.35 & 0.00 & 75.6 & 0.41 & 0.00 & 67.2 & 0.50 & 0.00 & 51.5 & 0.57 & 0.00 \\ 
tf  & sim  & 2 & 2 & 0 & 0 & 71.2 & 0.36 & 0.00 & 73.1 & 0.47 & 0.00 & 63.9 & 0.44 & 0.00 & 42.4 & 0.58 & 0.00 \\ 
 \hline 
narx & iden & 0 & 1 & 2 & 1 & 97.7 & 0.08 & 0.00 & 96.6 & 0.11 & 0.00 & 94.5 & 0.16 & 0.00 & 93.6 & 0.08 & 0.00 \\ 
narx & pred & 0 & 1 & 2 & 1 & 69.3 & 0.15 & 0.07 & 67.7 & 0.16 & 0.08 & 66.5 & 0.17 & 0.08 & 66.4 & 0.18 & 0.08 \\ 
narx & sim  & 0 & 1 & 2 & 1 & 72.2 & 0.21 & 0.07 & 71.9 & 0.21 & 0.07 & 71.0 & 0.18 & 0.07 & 70.7 & 0.18 & 0.07 \\ 
 \hline 
narx & iden & 0 & 1 & 3 & 1 & 98.6 & 0.04 & 0.00 & 98.2 & 0.07 & 0.00 & 97.6 & 0.07 & 0.00 & 97.0 & 0.04 & 0.00 \\ 
narx & pred & 0 & 1 & 3 & 1 & 79.3 & 0.17 & 0.05 & 78.0 & 0.18 & 0.05 & 77.7 & 0.19 & 0.05 & 77.5 & 0.19 & 0.05 \\ 
narx & sim  & 0 & 1 & 3 & 1 & 73.8 & 0.22 & 0.06 & 74.5 & 0.22 & 0.06 & 75.3 & 0.20 & 0.06 & 75.9 & 0.20 & 0.06 \\ 
 \hline 
narx & iden & 0 & 1 & 4 & 1 & 98.8 & 0.03 & 0.00 & 98.6 & 0.04 & 0.00 & 98.3 & 0.03 & 0.00 & 98.0 & 0.03 & 0.00 \\ 
narx & pred & 0 & 1 & 4 & 1 & 73.0 & 0.21 & 0.07 & 69.0 & 0.22 & 0.07 & 68.1 & 0.20 & 0.08 & 64.5 & 0.20 & 0.09 \\ 
narx & sim  & 0 & 1 & 4 & 1 & 72.4 & 0.22 & 0.07 & 72.2 & 0.22 & 0.07 & 71.6 & 0.20 & 0.07 & 69.0 & 0.20 & 0.08 \\ 
 \hline 
narx & iden & 0 & 1 & 2 & 2 & 98.1 & 0.06 & 0.00 & 97.0 & 0.08 & 0.00 & 94.7 & 0.17 & 0.00 & 93.9 & 0.08 & 0.00 \\ 
narx & pred & 0 & 1 & 2 & 2 & 69.9 & 0.16 & 0.07 & 68.7 & 0.17 & 0.08 & 66.6 & 0.17 & 0.08 & 66.6 & 0.18 & 0.08 \\ 
narx & sim  & 0 & 1 & 2 & 2 & 72.4 & 0.21 & 0.07 & 72.1 & 0.21 & 0.07 & 71.7 & 0.19 & 0.07 & 71.3 & 0.17 & 0.07 \\ 
 \hline 
narx & iden & 0 & 1 & 3 & 2 & 98.8 & 0.03 & 0.00 & 98.4 & 0.05 & 0.00 & 97.8 & 0.05 & 0.00 & 97.2 & 0.04 & 0.00 \\ 
narx & pred & 0 & 1 & 3 & 2 & 80.1 & 0.15 & 0.05 & 78.3 & 0.17 & 0.05 & 78.0 & 0.19 & 0.05 & 77.8 & 0.19 & 0.05 \\ 
narx & sim  & 0 & 1 & 3 & 2 & 73.5 & 0.22 & 0.06 & 74.2 & 0.22 & 0.06 & 74.9 & 0.20 & 0.06 & 75.9 & 0.20 & 0.06 \\ 
 \hline 
narx & iden & 0 & 1 & 4 & 2 & 98.9 & 0.02 & 0.00 & 98.7 & 0.04 & 0.00 & 98.4 & 0.03 & 0.00 & 98.0 & 0.03 & 0.00 \\ 
narx & pred & 0 & 1 & 4 & 2 & 77.0 & 0.21 & 0.06 & 71.5 & 0.21 & 0.07 & 69.1 & 0.20 & 0.07 & 64.2 & 0.20 & 0.09 \\ 
narx & sim  & 0 & 1 & 4 & 2 & 72.7 & 0.22 & 0.07 & 72.4 & 0.22 & 0.07 & 71.8 & 0.20 & 0.07 & 69.3 & 0.20 & 0.07 \\ 
 \hline 
narx & iden & 0 & 1 & 2 & 3 & 98.1 & 0.06 & 0.00 & 97.1 & 0.07 & 0.00 & 94.7 & 0.17 & 0.00 & 93.8 & 0.08 & 0.00 \\ 
narx & pred & 0 & 1 & 2 & 3 & 70.0 & 0.16 & 0.07 & 68.4 & 0.21 & 0.08 & 66.6 & 0.17 & 0.08 & 66.4 & 0.18 & 0.08 \\ 
narx & sim  & 0 & 1 & 2 & 3 & 71.8 & 0.21 & 0.07 & 72.0 & 0.21 & 0.07 & 71.7 & 0.19 & 0.07 & 71.6 & 0.18 & 0.07 \\ 
 \hline 
narx & iden & 0 & 1 & 3 & 3 & 98.8 & 0.03 & 0.00 & 98.4 & 0.05 & 0.00 & 97.8 & 0.06 & 0.00 & 97.2 & 0.04 & 0.00 \\ 
narx & pred & 0 & 1 & 3 & 3 & 80.1 & 0.15 & 0.05 & 78.2 & 0.18 & 0.05 & 77.9 & 0.19 & 0.05 & 77.5 & 0.19 & 0.05 \\ 
narx & sim  & 0 & 1 & 3 & 3 & 73.3 & 0.22 & 0.06 & 73.9 & 0.22 & 0.06 & 74.8 & 0.20 & 0.06 & 75.7 & 0.19 & 0.06 \\ 
 \hline 
narx & iden & 0 & 2 & 1 & 1 & 97.4 & 0.18 & 0.00 & 95.7 & 0.18 & 0.00 & 91.9 & 0.26 & 0.00 & 89.3 & 0.23 & 0.00 \\ 
narx & pred & 0 & 2 & 1 & 1 & 96.7 & 0.20 & 0.01 & 94.6 & 0.21 & 0.01 & 90.3 & 0.21 & 0.02 & 87.4 & 0.20 & 0.03 \\ 
narx & sim  & 0 & 2 & 1 & 1 & 64.7 & 0.23 & 0.09 & 68.0 & 0.21 & 0.08 & 69.2 & 0.21 & 0.07 & 72.3 & 0.20 & 0.07 \\ 
 \hline 
narx & iden & 0 & 2 & 2 & 1 & 98.8 & 0.05 & 0.00 & 98.3 & 0.05 & 0.00 & 97.2 & 0.05 & 0.00 & 96.1 & 0.06 & 0.00 \\ 
narx & pred & 0 & 2 & 2 & 1 & 83.3 & 0.15 & 0.04 & 76.3 & 0.17 & 0.06 & 78.2 & 0.20 & 0.05 & 67.0 & 0.21 & 0.08 \\ 
narx & sim  & 0 & 2 & 2 & 1 & 72.1 & 0.22 & 0.07 & 72.1 & 0.22 & 0.07 & 71.2 & 0.20 & 0.07 & 71.0 & 0.20 & 0.07 \\ 
 \hline 
narx & iden & 0 & 2 & 1 & 2 & 97.5 & 0.17 & 0.00 & 95.8 & 0.18 & 0.00 & 92.2 & 0.26 & 0.00 & 89.6 & 0.23 & 0.00 \\ 
narx & pred & 0 & 2 & 1 & 2 & 96.7 & 0.19 & 0.01 & 94.7 & 0.21 & 0.01 & 90.3 & 0.21 & 0.02 & 87.3 & 0.20 & 0.03 \\ 
narx & sim  & 0 & 2 & 1 & 2 & 63.1 & 0.24 & 0.09 & 65.9 & 0.23 & 0.08 & 67.7 & 0.21 & 0.08 & 70.9 & 0.20 & 0.07 \\ 
 \hline 
narx & iden & 0 & 2 & 2 & 2 & 98.9 & 0.05 & 0.00 & 98.4 & 0.05 & 0.00 & 97.4 & 0.05 & 0.00 & 96.4 & 0.05 & 0.00 \\ 
narx & pred & 0 & 2 & 2 & 2 & 84.3 & 0.14 & 0.04 & 78.2 & 0.16 & 0.05 & 78.5 & 0.19 & 0.05 & 68.4 & 0.20 & 0.08 \\ 
narx & sim  & 0 & 2 & 2 & 2 & 72.1 & 0.22 & 0.07 & 72.1 & 0.22 & 0.07 & 71.3 & 0.20 & 0.07 & 71.4 & 0.19 & 0.07 \\ 
 \hline 
narx & iden & 0 & 2 & 1 & 3 & 97.5 & 0.17 & 0.00 & 95.8 & 0.18 & 0.00 & 92.2 & 0.26 & 0.00 & 89.6 & 0.23 & 0.00 \\ 
narx & pred & 0 & 2 & 1 & 3 & 96.7 & 0.19 & 0.01 & 94.7 & 0.21 & 0.01 & 90.3 & 0.21 & 0.02 & 87.3 & 0.20 & 0.03 \\ 
narx & sim  & 0 & 2 & 1 & 3 & 60.4 & 0.24 & 0.10 & 65.9 & 0.23 & 0.08 & 67.7 & 0.21 & 0.08 & 70.9 & 0.20 & 0.07 \\ 
 \hline 
narx & iden & 0 & 2 & 2 & 3 & 98.9 & 0.05 & 0.00 & 98.4 & 0.05 & 0.00 & 97.4 & 0.06 & 0.00 & 96.4 & 0.05 & 0.00 \\ 
narx & pred & 0 & 2 & 2 & 3 & 84.6 & 0.14 & 0.04 & 78.2 & 0.16 & 0.05 & 78.8 & 0.19 & 0.05 & 69.0 & 0.19 & 0.08 \\ 
narx & sim  & 0 & 2 & 2 & 3 & 72.1 & 0.22 & 0.07 & 72.0 & 0.22 & 0.07 & 71.3 & 0.20 & 0.07 & 71.4 & 0.19 & 0.07 \\ 
 \hline 
\end{longtable} 
\normalsize \end{center} 
 \end{landscape}

\subsection{Left turbo revolutions (giri tsx)}
\begin{itemize}
	\item{inputs: engine revolutions (nmot w), actual engine torque (trqCLth), rail pressure real value left side (PRIST W), air flow left side (MSHFM1), Turbocharger Left Way Out Air Pressure (P2 Sx)}
	\item{output: Left turbo revolutions (giri tsx)}
\end{itemize}	

\begin{center} 
\begin{longtable}{ll|ccc|cc|cc} 
\caption[inputs nmot w trqCLth PRIST W MSHFM1 P2 SX   outputs GIRI TSX]{inputs nmot w trqCLth PRIST W MSHFM1 P2 SX   outputs GIRI TSX.} 
\label{tab:inputs_nmot_w_trqCLth_PRIST_W_MSHFM1_P2_SX___outputs_GIRI_TSX} 
\hline 
  model & type & oU & dPl & oY & ft50 & mxDf50 & ft100 & mxDf100 \\ 
 \hline 
narx & iden & 1 & 1 & 1 & 94.5 & 0.17 & 92.7 & 0.13 \\ 
narx & pred & 1 & 1 & 1 & 92.2 & 0.21 & 89.7 & 0.16 \\ 
narx & sim  & 1 & 1 & 1 & 82.5 & 0.23 & 85.3 & 0.16 
 \hline 
narx & iden & 1 & 2 & 1 & 98.4 & 0.04 & 98.2 & 0.04 \\ 
narx & pred & 1 & 2 & 1 & 72.6 & 0.15 & 71.0 & 0.13 \\ 
narx & sim  & 1 & 2 & 1 & 75.7 & 0.15 & 73.4 & 0.13 
 \hline 
narx & iden & 1 & 2 & 2 & 98.5 & 0.04 & 98.2 & 0.04 \\ 
narx & pred & 1 & 2 & 2 & 72.9 & 0.13 & 71.3 & 0.14 \\ 
narx & sim  & 1 & 2 & 2 & 75.8 & 0.13 & 73.5 & 0.13 
 \hline 
narx & iden & 2 & 1 & 1 & 97.9 & 0.12 & 96.9 & 0.11 \\ 
narx & pred & 2 & 1 & 1 & 97.4 & 0.12 & 96.0 & 0.10 \\ 
narx & sim  & 2 & 1 & 1 & 87.4 & 0.14 & 87.3 & 0.15 
 \hline 
narx & iden & 2 & 1 & 2 & 97.9 & 0.12 & 96.9 & 0.11 \\ 
narx & pred & 2 & 1 & 2 & 97.4 & 0.12 & 96.0 & 0.10 \\ 
narx & sim  & 2 & 1 & 2 & 87.4 & 0.14 & 87.3 & 0.15 
 \hline 
narx & iden & 2 & 2 & 1 & 99.1 & 0.04 & 99.0 & 0.03 \\ 
narx & pred & 2 & 2 & 1 & 88.2 & 0.09 & 83.1 & 0.13 \\ 
narx & sim  & 2 & 2 & 1 & 76.2 & 0.10 & 74.3 & 0.13 
 \hline 
narx & iden & 2 & 2 & 2 & 99.1 & 0.04 & 99.0 & 0.02 \\ 
narx & pred & 2 & 2 & 2 & 89.0 & 0.08 & 85.1 & 0.13 \\ 
narx & sim  & 2 & 2 & 2 & 75.9 & 0.13 & 75.8 & 0.12 
 \hline 
\end{longtable} 
\end{center}

\subsection{Left turbo revolutions (giri tsx)}
\begin{itemize}
	\item{inputs: engine revolutions (nmot w), actual engine torque (trqCLth), rail pressure real value left side (PRIST W), air flow left side (MSHFM1), Turbocharger Left Way Out Air Pressure (P2 Sx), Intercooler left way out air pressure(P5 SX)}
	\item{output: Left turbo revolutions (giri tsx)}
\end{itemize}	

\begin{center} 
\begin{longtable}{ll|ccc|cc|cc} 
\caption[inputs nmot w trqCLth PRIST W MSHFM1 P2 SX P5 SX   outputs GIRI TSX]{inputs nmot w trqCLth PRIST W MSHFM1 P2 SX P5 SX   outputs GIRI TSX.} 
\label{tab:inputs_nmot_w_trqCLth_PRIST_W_MSHFM1_P2_SX_P5_SX___outputs_GIRI_TSX} 
\hline 
  model & type & oU & dPl & oY & ft50 & mxDf50 & ft100 & mxDf100 \\ 
 \hline 
narx & iden & 1 & 1 & 1 & 94.8 & 0.14 & 93.1 & 0.11 \\ 
narx & pred & 1 & 1 & 1 & 92.2 & 0.23 & 89.7 & 0.19 \\ 
narx & sim  & 1 & 1 & 1 & 83.0 & 0.25 & 85.4 & 0.20 
 \hline 
narx & iden & 1 & 1 & 2 & 95.3 & 0.15 & 93.5 & 0.13 \\ 
narx & pred & 1 & 1 & 2 & 92.9 & 0.24 & 90.3 & 0.21 \\ 
narx & sim  & 1 & 1 & 2 & 83.8 & 0.24 & 85.2 & 0.22 
 \hline 
narx & iden & 1 & 2 & 1 & 98.6 & 0.04 & 98.3 & 0.04 \\ 
narx & pred & 1 & 2 & 1 & 73.6 & 0.12 & 74.3 & 0.12 \\ 
narx & sim  & 1 & 2 & 1 & 76.1 & 0.11 & 76.3 & 0.10 
 \hline 
narx & iden & 1 & 2 & 2 & 98.7 & 0.04 & 98.4 & 0.03 \\ 
narx & pred & 1 & 2 & 2 & 73.8 & 0.12 & 73.9 & 0.12 \\ 
narx & sim  & 1 & 2 & 2 & 76.1 & 0.12 & 76.0 & 0.10 
 \hline 
narx & iden & 2 & 1 & 1 & 97.9 & 0.12 & 96.9 & 0.10 \\ 
narx & pred & 2 & 1 & 1 & 97.3 & 0.12 & 96.0 & 0.09 \\ 
narx & sim  & 2 & 1 & 1 & 87.7 & 0.16 & 87.8 & 0.15 
 \hline 
narx & iden & 2 & 1 & 2 & 97.9 & 0.12 & 96.9 & 0.10 \\ 
narx & pred & 2 & 1 & 2 & 97.3 & 0.12 & 96.0 & 0.09 \\ 
narx & sim  & 2 & 1 & 2 & 87.7 & 0.16 & 87.8 & 0.15 
 \hline 
narx & iden & 2 & 2 & 1 & 99.3 & 0.02 & 99.1 & 0.02 \\ 
narx & pred & 2 & 2 & 1 & 88.9 & 0.09 & 85.4 & 0.12 \\ 
narx & sim  & 2 & 2 & 1 & 75.2 & 0.12 & 75.0 & 0.12 
 \hline 
narx & iden & 2 & 2 & 2 & 99.3 & 0.01 & 99.2 & 0.02 \\ 
narx & pred & 2 & 2 & 2 & 89.7 & 0.08 & 85.0 & 0.11 \\ 
narx & sim  & 2 & 2 & 2 & 75.7 & 0.12 & 74.8 & 0.12 
 \hline 
\end{longtable} 
\end{center}

\subsection{Left turbo revolutions (giri tsx)}
\begin{itemize}
	\item{inputs: engine revolutions (nmot w), boost pressure (pvd w), actual engine torque (trqCLth), rail pressure real value left side (PRIST W), air flow left side (MSHFM1), Turbocharger Left Way Out Air Pressure (P2 Sx), Intercooler left way out air pressure(P5 SX)}
	\item{output: Left turbo revolutions (giri tsx)}
\end{itemize}	

\begin{center} 
\begin{longtable}{ll|ccc|cc|cc} 
\caption[inputs nmot w pvd w trqCLth PRIST W MSHFM1 P2 SX P5 SX   outputs GIRI TSX]{inputs nmot w pvd w trqCLth PRIST W MSHFM1 P2 SX P5 SX   outputs GIRI TSX.} 
\label{tab:inputs_nmot_w_pvd_w_trqCLth_PRIST_W_MSHFM1_P2_SX_P5_SX___outputs_GIRI_TSX} 
\hline 
  model & type & oU & dPl & oY & ft50 & mxDf50 & ft100 & mxDf100 \\ 
 \hline 
narx & iden & 1 & 2 & 1 & 98.7 & 0.04 & 98.4 & 0.03 \\ 
narx & pred & 1 & 2 & 1 & 75.1 & 0.17 & 74.8 & 0.12 \\ 
narx & sim  & 1 & 2 & 1 & 77.4 & 0.13 & 77.1 & 0.10 
 \hline 
narx & iden & 1 & 2 & 2 & 98.8 & 0.04 & 98.4 & 0.03 \\ 
narx & pred & 1 & 2 & 2 & 76.2 & 0.14 & 75.1 & 0.13 \\ 
narx & sim  & 1 & 2 & 2 & 76.3 & 0.14 & 77.1 & 0.11 
 \hline 
narx & iden & 2 & 1 & 2 & 98.1 & 0.08 & 97.0 & 0.09 \\ 
narx & pred & 2 & 1 & 2 & 97.4 & 0.09 & 96.1 & 0.09 \\ 
narx & sim  & 2 & 1 & 2 & 76.5 & 0.13 & 87.6 & 0.13 
 \hline 
narx & iden & 2 & 2 & 1 & 99.3 & 0.01 & 99.2 & 0.02 \\ 
narx & pred & 2 & 2 & 1 & 90.8 & 0.09 & 88.7 & 0.12 \\ 
narx & sim  & 2 & 2 & 1 & 73.3 & 0.13 & 71.8 & 0.13 
 \hline 
narx & iden & 2 & 2 & 2 & 99.3 & 0.01 & 99.2 & 0.02 \\ 
narx & pred & 2 & 2 & 2 & 91.1 & 0.09 & 89.1 & 0.12 \\ 
narx & sim  & 2 & 2 & 2 & 75.1 & 0.13 & 74.5 & 0.15 
 \hline 
\end{longtable} 
\end{center}


\subsection{Intercooler left way out air pressure(P5 SX)}
\begin{itemize}
	\item{inputs: Turbocharger Left Way Out Air Pressure (P2 Sx), Left turbo revolutions (giri tsx)}
	\item{output: Intercooler left way out air pressure(P5 SX)}
\end{itemize}	

\begin{center} 
\begin{longtable}{ll|ccc|cc|cc} 
\caption[inputs GIRI TSX P2 SX   outputs P5 SX]{inputs GIRI TSX P2 SX   outputs P5 SX.} 
\label{tab:inputs_GIRI_TSX_P2_SX___outputs_P5_SX} 
\hline 
  model & type & oU & dPl & oY & ft50 & mxDf50 & ft100 & mxDf100 \\ 
 \hline 
narx & iden & 1 & 1 & 1 & 95.8 & 0.05 & 95.4 & 0.04 \\ 
narx & pred & 1 & 1 & 1 & 94.8 & 0.05 & 94.4 & 0.04 \\ 
narx & sim  & 1 & 1 & 1 & 93.5 & 0.05 & 93.9 & 0.04 
 \hline 
narx & iden & 1 & 1 & 2 & 95.9 & 0.06 & 95.5 & 0.04 \\ 
narx & pred & 1 & 1 & 2 & 94.9 & 0.07 & 94.5 & 0.04 \\ 
narx & sim  & 1 & 1 & 2 & 93.5 & 0.07 & 93.9 & 0.04 
 \hline 
narx & iden & 1 & 2 & 1 & 99.2 & 0.01 & 99.2 & 0.01 \\ 
narx & pred & 1 & 2 & 1 & 87.5 & 0.04 & 87.5 & 0.05 \\ 
narx & sim  & 1 & 2 & 1 & 87.6 & 0.04 & 87.5 & 0.05 
 \hline 
narx & iden & 1 & 2 & 2 & 99.2 & 0.01 & 99.2 & 0.01 \\ 
narx & pred & 1 & 2 & 2 & 87.6 & 0.05 & 87.5 & 0.05 \\ 
narx & sim  & 1 & 2 & 2 & 87.6 & 0.04 & 87.5 & 0.05 
 \hline 
narx & iden & 2 & 1 & 1 & 99.3 & 0.02 & 99.0 & 0.02 \\ 
narx & pred & 2 & 1 & 1 & 99.1 & 0.01 & 98.8 & 0.02 \\ 
narx & sim  & 2 & 1 & 1 & 93.7 & 0.03 & 94.0 & 0.02 
 \hline 
narx & iden & 2 & 1 & 2 & 99.3 & 0.02 & 99.0 & 0.02 \\ 
narx & pred & 2 & 1 & 2 & 99.1 & 0.01 & 98.8 & 0.02 \\ 
narx & sim  & 2 & 1 & 2 & 93.7 & 0.03 & 94.0 & 0.02 
 \hline 
narx & iden & 2 & 2 & 1 & 99.5 & 0.01 & 99.5 & 0.01 \\ 
narx & pred & 2 & 2 & 1 & 93.8 & 0.02 & 93.1 & 0.03 \\ 
narx & sim  & 2 & 2 & 1 & 89.3 & 0.03 & 89.3 & 0.03 
 \hline 
narx & iden & 2 & 2 & 2 & 99.5 & 0.01 & 99.5 & 0.01 \\ 
narx & pred & 2 & 2 & 2 & 93.9 & 0.02 & 93.1 & 0.03 \\ 
narx & sim  & 2 & 2 & 2 & 89.3 & 0.03 & 89.3 & 0.03 
 \hline 
\end{longtable} 
\end{center}

%%%%  RIGHT

\subsection{Right turbo revolutions (giri tdx)}
\begin{itemize}
	\item{inputs: Turbocharger Right Way Out Air Pressure (P2 Dx), Intercooler right way out air pressure(P5 DX)}
	\item{output: Right turbo revolutions (giri tdx)}
\end{itemize}	

\begin{landscape} 
 \begin{center} 
 \footnotesize 
 \begin{longtable}{ll|cccc|ccc|ccc|ccc|ccc} 
\caption[inputs P2 DX P5 DX   outputs GIRI TDX]{inputs P2 DX P5 DX   outputs GIRI TDX.} 
\label{tab:inputs_P2_DX_P5_DX___outputs_GIRI_TDX} 
\hline 
  mdl & type & np & nz & dPl & oY & ft50 & mxDf50 & mse50 & ft100 & mxDf100 & mse100 & ft250 & mxDf250 & mse250 & ft500 & mxDf500 & mse500 \\ 
 \hline 
tf  & iden & 1 & 1 & 0 & 0 & 76.6 & 0.33 & 0.00 & 74.9 & 0.41 & 0.00 & 66.4 & 0.54 & 0.00 & 51.3 & 0.60 & 0.00 \\ 
tf  & sim  & 1 & 1 & 0 & 0 & 68.7 & 0.38 & 0.00 & 66.6 & 0.51 & 0.00 & 56.8 & 0.51 & 0.00 & 42.2 & 0.56 & 0.00 \\ 
 \hline 
tf  & iden & 1 & 2 & 0 & 0 & 77.2 & 0.32 & 0.00 & 75.4 & 0.42 & 0.00 & 66.4 & 0.54 & 0.00 & 53.0 & 0.58 & 0.00 \\ 
tf  & sim  & 1 & 2 & 0 & 0 & 76.3 & 0.34 & 0.00 & 73.4 & 0.48 & 0.00 & 57.1 & 0.50 & 0.00 & 42.2 & 0.57 & 0.00 \\ 
 \hline 
tf  & iden & 2 & 1 & 0 & 0 & 76.6 & 0.35 & 0.00 & 75.0 & 0.41 & 0.00 & 66.4 & 0.54 & 0.00 & 53.9 & 0.54 & 0.00 \\ 
tf  & sim  & 2 & 1 & 0 & 0 & 68.9 & 0.40 & 0.00 & 66.6 & 0.52 & 0.00 & 57.0 & 0.50 & 0.00 & 38.6 & 0.57 & 0.00 \\ 
 \hline 
tf  & iden & 2 & 2 & 0 & 0 & 77.2 & 0.32 & 0.00 & 75.5 & 0.41 & 0.00 & 66.7 & 0.52 & 0.00 & 55.1 & 0.52 & 0.00 \\ 
tf  & sim  & 2 & 2 & 0 & 0 & 76.1 & 0.35 & 0.00 & 73.4 & 0.48 & 0.00 & 62.8 & 0.47 & 0.00 & 38.6 & 0.58 & 0.00 \\ 
 \hline 
narx & iden & 0 & 1 & 2 & 1 & 96.8 & 0.09 & 0.00 & 95.8 & 0.11 & 0.00 & 94.0 & 0.17 & 0.00 & 93.1 & 0.09 & 0.00 \\ 
narx & pred & 0 & 1 & 2 & 1 & 74.1 & 0.14 & 0.06 & 69.8 & 0.16 & 0.07 & 66.1 & 0.19 & 0.08 & 66.1 & 0.20 & 0.08 \\ 
narx & sim  & 0 & 1 & 2 & 1 & 71.4 & 0.21 & 0.07 & 71.4 & 0.21 & 0.07 & 70.4 & 0.19 & 0.07 & 70.4 & 0.18 & 0.07 \\ 
 \hline 
narx & iden & 0 & 1 & 4 & 1 & 98.0 & 0.05 & 0.00 & 97.7 & 0.05 & 0.00 & 97.4 & 0.03 & 0.00 & 97.2 & 0.03 & 0.00 \\ 
narx & pred & 0 & 1 & 4 & 1 & 81.9 & 0.19 & 0.04 & 77.8 & 0.20 & 0.05 & 73.2 & 0.20 & 0.06 & 68.2 & 0.20 & 0.08 \\ 
narx & sim  & 0 & 1 & 4 & 1 & 71.9 & 0.21 & 0.07 & 71.6 & 0.21 & 0.07 & 71.2 & 0.20 & 0.07 & 69.5 & 0.20 & 0.07 \\ 
 \hline 
narx & iden & 0 & 1 & 2 & 2 & 97.1 & 0.07 & 0.00 & 96.1 & 0.08 & 0.00 & 94.3 & 0.17 & 0.00 & 93.3 & 0.09 & 0.00 \\ 
narx & pred & 0 & 1 & 2 & 2 & 76.2 & 0.12 & 0.06 & 71.7 & 0.15 & 0.07 & 66.1 & 0.18 & 0.08 & 66.0 & 0.19 & 0.08 \\ 
narx & sim  & 0 & 1 & 2 & 2 & 71.5 & 0.21 & 0.07 & 71.4 & 0.21 & 0.07 & 70.9 & 0.19 & 0.07 & 70.9 & 0.18 & 0.07 \\ 
 \hline 
narx & iden & 0 & 1 & 2 & 3 & 97.1 & 0.07 & 0.00 & 96.2 & 0.07 & 0.00 & 94.3 & 0.17 & 0.00 & 93.4 & 0.09 & 0.00 \\ 
narx & pred & 0 & 1 & 2 & 3 & 76.4 & 0.12 & 0.06 & 71.6 & 0.20 & 0.07 & 66.1 & 0.18 & 0.08 & 66.1 & 0.19 & 0.08 \\ 
narx & sim  & 0 & 1 & 2 & 3 & 71.6 & 0.21 & 0.07 & 71.4 & 0.21 & 0.07 & 70.9 & 0.19 & 0.07 & 71.0 & 0.18 & 0.07 \\ 
 \hline 
narx & iden & 0 & 2 & 1 & 1 & 97.3 & 0.17 & 0.00 & 95.4 & 0.17 & 0.00 & 91.6 & 0.25 & 0.00 & 88.9 & 0.22 & 0.00 \\ 
narx & pred & 0 & 2 & 1 & 1 & 96.6 & 0.19 & 0.01 & 94.5 & 0.20 & 0.01 & 90.0 & 0.19 & 0.02 & 87.2 & 0.19 & 0.03 \\ 
narx & sim  & 0 & 2 & 1 & 1 & 61.6 & 0.22 & 0.09 & 66.9 & 0.20 & 0.08 & 66.9 & 0.19 & 0.08 & 71.2 & 0.19 & 0.07 \\ 
 \hline 
narx & iden & 0 & 2 & 2 & 1 & 98.4 & 0.11 & 0.00 & 97.7 & 0.10 & 0.00 & 96.5 & 0.06 & 0.00 & 95.6 & 0.07 & 0.00 \\ 
narx & pred & 0 & 2 & 2 & 1 & 90.2 & 0.20 & 0.02 & 85.5 & 0.17 & 0.04 & 75.8 & 0.20 & 0.06 & 69.1 & 0.22 & 0.08 \\ 
narx & sim  & 0 & 2 & 2 & 1 & 71.9 & 0.21 & 0.07 & 71.8 & 0.21 & 0.07 & 71.0 & 0.20 & 0.07 & 71.2 & 0.21 & 0.07 \\ 
 \hline 
narx & iden & 0 & 2 & 2 & 2 & 98.5 & 0.11 & 0.00 & 97.8 & 0.10 & 0.00 & 96.7 & 0.07 & 0.00 & 95.8 & 0.06 & 0.00 \\ 
narx & pred & 0 & 2 & 2 & 2 & 91.1 & 0.19 & 0.02 & 86.8 & 0.17 & 0.03 & 79.6 & 0.17 & 0.05 & 70.0 & 0.21 & 0.07 \\ 
narx & sim  & 0 & 2 & 2 & 2 & 71.9 & 0.22 & 0.07 & 71.6 & 0.21 & 0.07 & 70.8 & 0.20 & 0.07 & 71.5 & 0.20 & 0.07 \\ 
 \hline 
narx & iden & 0 & 2 & 2 & 3 & 98.5 & 0.11 & 0.00 & 97.8 & 0.10 & 0.00 & 96.8 & 0.06 & 0.00 & 95.7 & 0.06 & 0.00 \\ 
narx & pred & 0 & 2 & 2 & 3 & 91.3 & 0.19 & 0.02 & 86.9 & 0.17 & 0.03 & 79.2 & 0.18 & 0.05 & 70.0 & 0.21 & 0.07 \\ 
narx & sim  & 0 & 2 & 2 & 3 & 71.8 & 0.22 & 0.07 & 71.6 & 0.21 & 0.07 & 70.6 & 0.21 & 0.07 & 71.5 & 0.20 & 0.07 \\ 
 \hline 
\end{longtable} 
\normalsize \end{center} 
 \end{landscape}


\subsection{Right turbo revolutions (giri tdx)}
\begin{itemize}
	\item{inputs: engine revolutions (nmot w), actual engine torque (trqCLth), rail pressure real value Right side (prist w), air flow Right side (mshfm1), Turbocharger Right Way Out Air Pressure (P2 Dx)}
	\item{output: Right turbo revolutions (giri tdx)}
\end{itemize}	

\begin{center} 
\begin{longtable}{ll|ccc|cc|cc} 
\caption[inputs nmot w trqCLth prist w mshfm1 P2 DX   outputs GIRI TDX]{inputs nmot w trqCLth prist w mshfm1 P2 DX   outputs GIRI TDX.} 
\label{tab:inputs_nmot_w_trqCLth_prist_w_mshfm1_P2_DX___outputs_GIRI_TDX} 
\hline 
  model & type & oU & dPl & oY & ft50 & mxDf50 & ft100 & mxDf100 \\ 
 \hline 
narx & iden & 1 & 1 & 1 & 94.5 & 0.15 & 92.7 & 0.12 \\ 
narx & pred & 1 & 1 & 1 & 92.4 & 0.19 & 90.1 & 0.14 \\ 
narx & sim  & 1 & 1 & 1 & 83.2 & 0.22 & 86.0 & 0.15 
 \hline 
narx & iden & 1 & 1 & 2 & 95.1 & 0.17 & 93.3 & 0.13 \\ 
narx & pred & 1 & 1 & 2 & 93.1 & 0.20 & 90.6 & 0.17 \\ 
narx & sim  & 1 & 1 & 2 & 83.9 & 0.24 & 85.4 & 0.18 
 \hline 
narx & iden & 1 & 2 & 1 & 98.3 & 0.05 & 98.1 & 0.04 \\ 
narx & pred & 1 & 2 & 1 & 71.5 & 0.13 & 70.2 & 0.14 \\ 
narx & sim  & 1 & 2 & 1 & 74.2 & 0.14 & 72.1 & 0.13 
 \hline 
narx & iden & 1 & 2 & 2 & 98.5 & 0.04 & 98.1 & 0.03 \\ 
narx & pred & 1 & 2 & 2 & 72.4 & 0.13 & 70.7 & 0.14 \\ 
narx & sim  & 1 & 2 & 2 & 74.5 & 0.14 & 72.9 & 0.12 
 \hline 
narx & iden & 2 & 1 & 1 & 97.8 & 0.12 & 96.8 & 0.11 \\ 
narx & pred & 2 & 1 & 1 & 97.2 & 0.11 & 96.1 & 0.09 \\ 
narx & sim  & 2 & 1 & 1 & 88.3 & 0.13 & 88.4 & 0.12 
 \hline 
narx & iden & 2 & 1 & 2 & 97.8 & 0.12 & 96.8 & 0.11 \\ 
narx & pred & 2 & 1 & 2 & 97.2 & 0.11 & 96.1 & 0.09 \\ 
narx & sim  & 2 & 1 & 2 & 88.3 & 0.13 & 88.4 & 0.12 
 \hline 
narx & iden & 2 & 2 & 1 & 99.0 & 0.03 & 98.9 & 0.02 \\ 
narx & pred & 2 & 2 & 1 & 86.4 & 0.09 & 85.7 & 0.12 \\ 
narx & sim  & 2 & 2 & 1 & 75.1 & 0.12 & 75.4 & 0.12 
 \hline 
narx & iden & 2 & 2 & 2 & 99.0 & 0.04 & 98.9 & 0.03 \\ 
narx & pred & 2 & 2 & 2 & 86.6 & 0.09 & 85.2 & 0.11 \\ 
narx & sim  & 2 & 2 & 2 & 74.6 & 0.13 & 74.9 & 0.12 
 \hline 
\end{longtable} 
\end{center}

\subsection{Right turbo revolutions (giri tdx)}
\begin{itemize}
	\item{inputs: engine revolutions (nmot w), actual engine torque (trqCLth), rail pressure real value Right side (prist w), air flow Right side (mshfm1), Turbocharger Right Way Out Air Pressure (P2 Dx), Intercooler Right way out air pressure(P5 DX)}
	\item{output: Right turbo revolutions (giri tdx)}
\end{itemize}	

\begin{center} 
\begin{longtable}{ll|ccc|cc|cc} 
\caption[inputs nmot w trqCLth prist w mshfm1 P2 DX P5 DX   outputs GIRI TDX]{inputs nmot w trqCLth prist w mshfm1 P2 DX P5 DX   outputs GIRI TDX.} 
\label{tab:inputs_nmot_w_trqCLth_prist_w_mshfm1_P2_DX_P5_DX___outputs_GIRI_TDX} 
\hline 
  model & type & oU & dPl & oY & ft50 & mxDf50 & ft100 & mxDf100 \\ 
 \hline 
narx & iden & 1 & 1 & 1 & 94.6 & 0.14 & 92.9 & 0.11 \\ 
narx & pred & 1 & 1 & 1 & 92.5 & 0.21 & 90.3 & 0.16 \\ 
narx & sim  & 1 & 1 & 1 & 83.9 & 0.22 & 86.5 & 0.17 
 \hline 
narx & iden & 1 & 1 & 2 & 95.2 & 0.16 & 93.4 & 0.14 \\ 
narx & pred & 1 & 1 & 2 & 93.1 & 0.21 & 90.8 & 0.18 \\ 
narx & sim  & 1 & 1 & 2 & 84.6 & 0.23 & 86.0 & 0.19 
 \hline 
narx & iden & 1 & 2 & 1 & 98.4 & 0.05 & 98.1 & 0.04 \\ 
narx & pred & 1 & 2 & 1 & 72.5 & 0.13 & 72.4 & 0.13 \\ 
narx & sim  & 1 & 2 & 1 & 75.3 & 0.12 & 74.6 & 0.11 
 \hline 
narx & iden & 1 & 2 & 2 & 98.5 & 0.04 & 98.2 & 0.03 \\ 
narx & pred & 1 & 2 & 2 & 73.2 & 0.14 & 72.1 & 0.13 \\ 
narx & sim  & 1 & 2 & 2 & 75.4 & 0.13 & 74.4 & 0.11 
 \hline 
narx & iden & 2 & 1 & 1 & 97.8 & 0.12 & 96.9 & 0.11 \\ 
narx & pred & 2 & 1 & 1 & 97.3 & 0.11 & 96.1 & 0.09 \\ 
narx & sim  & 2 & 1 & 1 & 88.5 & 0.13 & 88.9 & 0.12 
 \hline 
narx & iden & 2 & 1 & 2 & 97.8 & 0.12 & 96.9 & 0.11 \\ 
narx & pred & 2 & 1 & 2 & 97.3 & 0.11 & 96.1 & 0.09 \\ 
narx & sim  & 2 & 1 & 2 & 88.5 & 0.13 & 88.9 & 0.12 
 \hline 
narx & iden & 2 & 2 & 1 & 99.1 & 0.02 & 99.0 & 0.02 \\ 
narx & pred & 2 & 2 & 1 & 89.1 & 0.09 & 85.6 & 0.12 \\ 
narx & sim  & 2 & 2 & 1 & 76.2 & 0.11 & 73.0 & 0.12 
 \hline 
narx & iden & 2 & 2 & 2 & 99.2 & 0.02 & 99.1 & 0.02 \\ 
narx & pred & 2 & 2 & 2 & 89.2 & 0.10 & 86.0 & 0.11 \\ 
narx & sim  & 2 & 2 & 2 & 75.7 & 0.11 & 75.1 & 0.12 
 \hline 
\end{longtable} 
\end{center}

\subsection{Right turbo revolutions (giri tdx)}
\begin{itemize}
	\item{inputs: engine revolutions (nmot w), boost pressure (pvd w), actual engine torque (trqCLth), rail pressure real value Right side (prist w), air flow Right side (mshfm1), Turbocharger Right Way Out Air Pressure (P2 Dx), Intercooler Right way out air pressure(P5 DX)}
	\item{output: Right turbo revolutions (giri tdx)}
\end{itemize}	

\begin{center} 
\begin{longtable}{ll|ccc|cc|cc} 
\caption[inputs nmot w pvd w trqCLth prist w mshfm1 P2 DX P5 DX   outputs GIRI TDX]{inputs nmot w pvd w trqCLth prist w mshfm1 P2 DX P5 DX   outputs GIRI TDX.} 
\label{tab:inputs_nmot_w_pvd_w_trqCLth_prist_w_mshfm1_P2_DX_P5_DX___outputs_GIRI_TDX} 
\hline 
  model & type & oU & dPl & oY & ft50 & mxDf50 & ft100 & mxDf100 \\ 
 \hline 
narx & iden & 1 & 2 & 1 & 98.5 & 0.05 & 98.2 & 0.04 \\ 
narx & pred & 1 & 2 & 1 & 74.8 & 0.12 & 73.6 & 0.13 \\ 
narx & sim  & 1 & 2 & 1 & 76.9 & 0.10 & 76.0 & 0.11 
 \hline 
narx & iden & 1 & 2 & 2 & 98.7 & 0.04 & 98.3 & 0.04 \\ 
narx & pred & 1 & 2 & 2 & 74.9 & 0.12 & 73.8 & 0.13 \\ 
narx & sim  & 1 & 2 & 2 & 73.9 & 0.12 & 76.1 & 0.11 
 \hline 
narx & iden & 2 & 1 & 2 & 98.0 & 0.09 & 96.9 & 0.10 \\ 
narx & pred & 2 & 1 & 2 & 97.4 & 0.10 & 96.1 & 0.09 \\ 
narx & sim  & 2 & 1 & 2 & 82.9 & 0.11 & 84.5 & 0.12 
 \hline 
narx & iden & 2 & 2 & 1 & 99.2 & 0.02 & 99.1 & 0.02 \\ 
narx & pred & 2 & 2 & 1 & 90.7 & 0.09 & 87.6 & 0.11 \\ 
narx & sim  & 2 & 2 & 1 & 76.3 & 0.11 & 71.8 & 0.14 
 \hline 
narx & iden & 2 & 2 & 2 & 99.2 & 0.02 & 99.1 & 0.02 \\ 
narx & pred & 2 & 2 & 2 & 90.2 & 0.08 & 87.5 & 0.11 \\ 
narx & sim  & 2 & 2 & 2 & 75.7 & 0.11 & 73.4 & 0.12 
 \hline 
\end{longtable} 
\end{center}

\subsection{Intercooler right way out air pressure(P5 DX)}
\begin{itemize}
	\item{inputs: Turbocharger Right Way Out Air Pressure (P2 Dx), Right turbo revolutions (giri tdx)}
	\item{output: Intercooler right way out air pressure(P5 DX)}
\end{itemize}	

\begin{center} 
\begin{longtable}{ll|ccc|cc|cc} 
\caption[inputs GIRI TDX P2 DX   outputs P5 DX]{inputs GIRI TDX P2 DX   outputs P5 DX.} 
\label{tab:inputs_GIRI_TDX_P2_DX___outputs_P5_DX} 
\hline 
  model & type & oU & dPl & oY & ft50 & mxDf50 & ft100 & mxDf100 \\ 
 \hline 
narx & iden & 1 & 1 & 1 & 95.7 & 0.05 & 95.3 & 0.04 \\ 
narx & pred & 1 & 1 & 1 & 94.7 & 0.06 & 94.3 & 0.04 \\ 
narx & sim  & 1 & 1 & 1 & 93.4 & 0.06 & 93.8 & 0.04 
 \hline 
narx & iden & 1 & 1 & 2 & 95.8 & 0.06 & 95.4 & 0.04 \\ 
narx & pred & 1 & 1 & 2 & 94.9 & 0.07 & 94.4 & 0.04 \\ 
narx & sim  & 1 & 1 & 2 & 93.4 & 0.07 & 93.7 & 0.04 
 \hline 
narx & iden & 1 & 2 & 1 & 98.5 & 0.01 & 98.5 & 0.01 \\ 
narx & pred & 1 & 2 & 1 & 87.0 & 0.05 & 86.7 & 0.05 \\ 
narx & sim  & 1 & 2 & 1 & 86.6 & 0.05 & 86.5 & 0.05 
 \hline 
narx & iden & 1 & 2 & 2 & 98.5 & 0.01 & 98.5 & 0.01 \\ 
narx & pred & 1 & 2 & 2 & 87.1 & 0.05 & 86.8 & 0.05 \\ 
narx & sim  & 1 & 2 & 2 & 86.6 & 0.05 & 86.5 & 0.05 
 \hline 
narx & iden & 2 & 1 & 1 & 98.2 & 0.02 & 98.2 & 0.02 \\ 
narx & pred & 2 & 1 & 1 & 98.0 & 0.02 & 97.9 & 0.02 \\ 
narx & sim  & 2 & 1 & 1 & 93.8 & 0.03 & 94.0 & 0.03 
 \hline 
narx & iden & 2 & 1 & 2 & 98.2 & 0.02 & 98.2 & 0.02 \\ 
narx & pred & 2 & 1 & 2 & 98.0 & 0.02 & 97.9 & 0.02 \\ 
narx & sim  & 2 & 1 & 2 & 93.8 & 0.03 & 94.0 & 0.03 
 \hline 
\end{longtable} 
\end{center}
%% !TEX encoding = UTF-8
% !TEX TS-program = pdflatex
% !TEX root = ../Articolo.tex
% !TEX spellcheck = it-IT

%************************************************
\section{Application 1: Sinter Chute}
\label{sec:application1sinterchute}
%************************************************

I started to work on the first application, the Cooler Charging Chute of the
Sinter Plant (CCCSP) for one of the industrial partner: PRIMETALS
(ex-Siemens-VAI).
A first version, purely
based on LIGGGHTS, has been developed, with focus over the outflow mass rate 
and the particle distribution in the chute.
PRIMETALS representatives suggested that The simulation’s domain of the CCCSP
should be expanded. It should account for the complete filling of a container 
below the chute and a partial filling of the chute itself.
They also provided me an updated design model.
The focus should be in the particle distribution over vertical sections in these
filled sections. The aim is to demonstrate the deposition of the larger
particles on the bottom.\\
Now I am trying to reach the required filling: there is a large number of
particles involved and I need to optimize the simulation.




%% !TEX encoding = UTF-8
% !TEX TS-program = pdflatex
% !TEX root = ../Articolo.tex
% !TEX spellcheck = it-IT

%************************************************
\section{Application 2: Raceway}
\label{sec:application2raceway}
%************************************************

I started to work on the raceway, using the old simulation setup by Hager.
The first results show a relevant effect of the sliding friction over the
raceway area size.\\
I have started the geometry for the new one.




%% !TEX encoding = UTF-8
% !TEX TS-program = pdflatex
% !TEX root = ../Articolo.tex
% !TEX spellcheck = it-IT

%************************************************
\section{Thesis}
\label{sec:thesis}
%************************************************

A new draft will be available by Monday the $21^{st}$ of September.




%% !TEX encoding = UTF-8
% !TEX TS-program = pdflatex
% !TEX root = ../Articolo.tex
% !TEX spellcheck = it-IT

%************************************************
\section{CFD Conference}
\label{sec:cfdconference}
%************************************************

I found very useful and interesting the partecipation to the CFD 2014 Conference in Trondheim.\\
After inputs from the audience, I know now that future presentations of my paper, \textit{Establishing the predictive capabilities of DEM simulations: sliding and rolling friction coefficients of non-spherical particles}, should underline that: 
\begin{itemize}
\item{the real material is polydisperse, eventually showing a granulometric curve;}
\item{the simulations were performed with monodisperse spheres;}
\item{cohesion was not studied;}
\item{which diagrams belong to the experiments and which to the simulations;}
\item{the simulations were 3D;}
\item{the experiment time VS the simulation time;}
\item{the Coefficient of Restitution is an extimation (I am working on it);}
\item{and I should also show some photos of the raw materials.}
\end{itemize}

On the other hand, I attended many lectures, surely the most interesting for my work was held by Prof. Oberkampf, \textit{Concepts and Practice of Verification, Validation, and Uncertainty Quantification}. \\
A lot of people presented about packed and fluidized bed. Unfortunately the large majority were only simulation works performed with FLUENT. The few with experimental work were not extremely strict about uniform boundary conditions.
Furthermore a post-doc (Dr. Pereira) of Prof. Cleary team presented a work concerning dynamic AOR with DEM. It involved mainly segregation in bi-disperse bulk, but I will study it more carefully study to see if I can find some suggestions.\\
%\input{Paragrafi/trota}
\appendix
%\input{Paragrafi/Dolor}
% *****************************************************************
% Materiale finale
%% !TEX encoding = UTF-8
% !TEX TS-program = pdflatex
% !TEX root = ../Articolo.tex
% !TEX spellcheck = it-IT

%************************************************
\section{Acronyms list}
\label{sec:acro}
%************************************************
%*******************************************************
% Elenco degli acronimi
%*******************************************************

		
\begin{acronym}[TDMA]
%\acro{CDMA}{Code Division Multiple Access}
%\acro{GSM}{Global System for Mobile communication}
%\acro{NA}[\ensuremath{N_{\mathrm A}}]{Number of Avogadro\acroextra{ (see \S\ref{Chem})}}
%\acro{NAD+}[NAD\textsuperscript{+}]{Nicotinamide Adenine Dinucleotide}
%\acro{NUA}{Not Used Acronym}
%\acro{TDMA}{Time Division Multiple Access}
%\acro{UA}{Used Acronym}
%\acro{lox}[\ensuremath{LOX}]{Liquid Oxygen}%
%\acro{lh2}[\ensuremath{LH_2}]{Liquid Hydrogen}%
%\acro{IC}{Integrated Circuit}%
%\acro{BUT}{Block Under Test}%
%\acrodefplural{BUT}{Blocks Under Test}%

\acro{phis}[$\phi_s$]{angle of sliding friction}
\acro{aor}[$AOR$]{Angle of repose}
\acro{omega1}[$\omega_1$]{angular speed before first impact}
\acro{omega2}[$\omega_2$]{angular speed after first impact}
\acro{omega3}[$\omega_3$]{angular speed before second impact}
\acro{omega4}[$\omega_4$]{angular speed after second impact}
\acro{dem}[$DEM$]{Discrete Element Method}
\acro{euno}[$e_1$]{coefficient of first restitution}
\acro{cor}[$COR$]{coefficient of restitution}
\acro{edue}[$e_2$]{coefficient of second restitution}
\acro{mu}[$\mu_s$]{coefficient of sliding friction}
\acro{mur}[$\mu_r$]{coefficient of static rolling friction}
\acro{jsct}[$JSCT$]{Jenike Shear Cell tester}
\acro{pmsct}[$PMSCT$]{``Poor Man'' Shear Cell tester}
\acro{liggghts}[$LIGGGHTS$]{LAMMPS improved for general granular and granular heat transfer simulations}
\acro{r}[$r$]{radius of the particle}
\acro{ra}[$R$]{external radius of the sphere}
\acro{sasct}[$SASCT$]{Schulze Annular Shear Cell tester}

\end{acronym}

%% !TEX encoding = UTF-8
% !TEX TS-program = pdflatex
% !TEX root = ../Tesi.tex
% !TEX spellcheck = it-IT

%*******************************************************
% Bibliografia
%*******************************************************
%\cleardoublepage

%************************************************
%\section{Bibliografia}
%\label{sec:bibliografia}
%************************************************
%*******************************************************
% Elenco degli acronimi
%*******************************************************
%\nocite{*}




\printbibliography
\label{section:bibliography}
%******************************************************************
\end{document}


