Particles in various forms – ranging from raw materials to food grains and pharmaceutical powders – play a major role in a variety of industries, including process industry and metallurgy. In his book, Holdich (2002) stated that "between 1 and 10% of all the energy is used in comminution, i.e. the processes of crushing, grinding, milling, micronising". However, a universal method for particle characterization has so far not been established (, since the
results obtained through experiments could be biased – can you leave this out? The following sentence explains it anyway.). From the experimental point of view, the main issues are the difficult (/complex?) setups and the general reliability and reproducibility of tests. From the numerical (/mathematical) point of view, no general procedure is available, and the existence of a mathematically unique solution describing macro/micro particle contact has yet  to be proved. In a recent study, Krantz et al. (2009) implied "that the dynamic properties of a powder cannot be applied to universally predict the static properties of a powder, and, likewise, the static properties cannot be used to predict dynamic properties".


Since the consolidation phase is not completely reliable, the simplified JSCT produced different \mpsh results than the SRSCT. Although we compared these results with the numerical values, they cannot be considered valid before the unconsolidated state is accurately defined for all materials (Is this what you wanted to say?).