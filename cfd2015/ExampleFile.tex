\documentclass{CFD2015}
\usepackage{CFD2015}

%%%%%%%%%%%%%%%%%%%%%%%%%%%%%%%%%%%%%%%%%%%%%%%%%%%%%%%%%%%%%%%%%%%%%%%%%
%This template is created for complying with the author instructions for the 
%11th International Conference on CFD in Mineral and Process
% Industries
%2015
%


%Required files:
%   ExampleFile.tex
%   ExampleFile.nls
%   CFD2015.bst
%   CFD2015.cls
%   CFD2015.sty
%   References.bib

%Example-specific files:
%   figure.eps
%   Table.tex
%%%%%%%%%%%%%%%%%%%%%%%%%%%%%%%%%%%%%%%%%%%%%%%%%%%%%%%%%%%%%%%%%%%%%%%%%




\title{DEM parameter identification by means of artificial neural network for iron ore sintering}
\paperID{CFD 2015}
\author{Luca}{Benvenuti} %{forename}{surname}
\presenting  %the previous author is presenting the paper (name becomes underlined)
\address{JKU Department of Particulate Flow Modelling, 4040 Linz, Austria} %affiliation of the previous author
\email{luca.benvenuti@jku.at}%e-mail address of the previous author
\author{Christoph}{Kloss}
\address{DCS Computing, 4040 Linz, Austria}

\author{Stefan}{Pirker}
\sameaddressas{1}
%\email{stefan.pirker@jku.at}



\begin{document}
\maketitle  %create the title page
\headers   %create the page headers and footers


\abstract{
Particle-particle contact laws and particles size distributions determine the macroscopic simulation results in 
Discrete Element Method (DEM). Commonly, contact laws depend on semi-empirical parameters which are difficult to 
obtain by direct microscopic measurements. Consequently, macroscopic experiments are performed, 
and the relationship between their results and microscopic DEM simulation is
investigated.\\
We present a method for the identification of DEM simulation parameters though macroscopic 
experiments and dedicated artificial neural networks. First, a feed-forward artificial neural network 
is trained by backward propagation reinforcement thanks to the macroscopic results of a series of DEM 
simulations, each with a different set of particle-based simulation parameters and individual particle distributions. 
We subsequently utilize this artificial neural network to forecast the macroscopic ensemble behaviour in relation to 
additional sets of particle-based simulation parameters and particle distributions. 
By this method, a comprehensive database was established, relating particle-based simulation parameters to macroscopic 
ensemble output.
If compared to an experiment of a specific granular material, this database identifies valid sets of DEM parameters 
which lead to the same macroscopic results as observed in the experiments.
Finally, we applied this method of DEM parameter identification to an industrial scale process of iron ore sintering.
}
\keywords{
Discrete Element Method ($DEM$) Simulations, Parameter Identification,
Artificial Neural Networks }
\normalfont\normalsize

%\section{Nomenclature}
%A complete list of symbols used, with dimensions, is required.
%the nomenclature needs to be entered into the file ExampleFile.nls
\printnomenclature[0.7cm]  
\vskip .1em

\section{Introduction}
\label{sec:introduction}
%************************************************

Particles in various forms - ranging from raw materials to food grains and pharmaceutical powders - 
play a major role in a variety of industries. 
Discrete Element Methods ($DEMs$) are widely used to simulate
particle behaviour in these granular processes \cite{RefWorks:130}.\\
In their original formulation of $DEM$, \citet{RefWorks:172} allowed two 
particles to slightly overlap upon contact, and consequently they proposed
repulsive forces in relation to this overlap distance.
Their fundamental modelling concept has since been widely accepted in the
literature and their soft-sphere contact law has been developed further by
numerous researchers (\citet{RefWorks:148} and Di Renzo and Di Maio
\citet{RefWorks:145}).
With increasing computational resources, $DEM$ simulation have become very
popular giving rise to the development of commercial (e.g., $PFC3D$, used by
\citet{RefWorks:87}) and open-source software (e.g.,
$LIGGGHTS$, \citet{RefWorks:136}, \citet{RefWorks:139}).
Soft-sphere $DEM$ simulations of thousands of particles have been proven to 
faithfully model particle bulk behaviour \cite{RefWorks:136}. \\
In these macroscopic $DEM$ simulations, the contact law kernel between a 
pair of particles determines the global bulk behaviour of the granular material \cite{RefWorks:131}. 
As a consequence, defining a correct contact law is of crucial importance for the predictive 
capability of $DEM$ simulations. 
Since $DEM$ contact laws are based 
on a set of semi-empirical parameters, correct contact law 
parameters must be defined for a given granular material
or $DEM$ simulations will fail \cite{RefWorks:177}. \\
Identifying $DEM$ contact law parameters is not a trivial task. 
Due to the huge number of particles in a granular material, it
may be impractical to identify valid parameter sets by performing bilateral 
particle collision experiments. 
Furthermore, some contact law parameters such as the coefficient of rolling
friction are purely empirical and cannot be determined by direct 
particle-to-particle measurements \cite{RefWorks:87}.
Therefore, $DEM$ contact law parameters are
commonly determined by comparing the macroscopic outcome of large-scale $DEM$ simulations with 
bulk experiments \cite{RefWorks:91}. 
If $DEM$ simulation results disagree with bulk measurements, the set of contact
law parameters must be adjusted until reasonable agreement is achieved.\\
However, this purely forward methodology of parameter identification is limited by 
the multi-dimensionality of the parameter space and the associated computational costs of the required 
$DEM$ test simulations. 
Moreover, one parameter set which is valid for one bulk behaviour (e.g., angle
of repose) might fail for another (e.g., shear tester). \\
Clearly, there is a need for an efficient method for identifying
$DEM$ contact law parameters.
In our study, we harnessed Artificial Neural Networks ($ANNs$) in order to
reduce the number of $DEM$ test simulations required. 
$ANNs$ have proven to be a versatile tool in analysing complex, non-linear
systems of multi-dimensional input streams 
\cite{RefWorks:150}.
%(Vaferi et al. \cite{RefWorks:150}, Witten et
%al. \cite{RefWorks:174} and Haykin \cite{RefWorks:158}).
In our case, we fed an $ANN$ with $DEM$ contact law parameters as input
and compared the output with the bulk behaviour 
predicted by a corresponding $DEM$ simulation. 
The difference between $ANN$ prediction and $DEM$ prediction is used to train our 
specific $ANN$ with a backward-propagation algorithm (described further below). 
After a training phase comprising a limited number of $DEM$ test simulations,
the $ANN$ can then be used as a stand-alone prediction tool for the bulk behaviour of a 
granular material in relation to $DEM$ contact law parameters. \\
In this study, we applied this parameter identification method to two different
granular bulk behaviours, namely the angle of repose ($AoR$) test and the
Schulze shear cell ($SSC$) test.
In both cases, we first trained a specific $ANN$ using a number of $DEM$ test
simulations before we identified valid sets of $DEM$ contact law parameters by
comparing the stand-alone $ANN$ predictions with corresponding bulk experiments. 
For both cases we obtained valid sets of contact law parameters, 
which we then compared to formulate a reliable contact law for a given
granular material.
We further show that these data can be used to model an industrial scale process
of iron ore sintering. \\
In the next section we define some prerequisites including $DEM$ contact law
definitions, a general description of the $ANN$ functionality, and the proposed
method of $DEM$ contact law parameter identification.
We then describe how it is applied to characterize the $DEM$ contact law
parameters of sinter fines.
%************************************************

\section{DEM Parameter Identification}
\label{sec:methodology}

Fig. \ref{fig:19methodology} illustrates the methodology used.

%************************************************
\begin{figure}%[!htb] 
\centering 
\includegraphics[width=.96\columnwidth]{19methodology} 
\caption[Method]{Method. 
In the training phase (dashed lines)
$DEM$ simulations are performed
with random initial input parameters.
The behaviours obtained are used to train the
Artificial Neural Networks ($ANNs$) in a loop that continues until the
difference between the outputs of each $ANN$ and its simulations is below the
limit ($\Delta$) (see Section \ref{subsec:ann}).
In the parameters identification phase (solid
lines) we identify valid input parameters by comparing (\textbf{=}) $ANNs$ and
experimental behaviours.
}
\label{fig:19methodology} 
\end{figure}
%************************************************
\subsection{Discrete element method}
\label{subsec:dem}

We decided to utilize a single
contact law for all the simulations performed, for details see
\citet{RefWorks:180}.
The $DEM$ parameters for the Young's modulus ($E$) and the Poisson's coefficient
($\nu$) were taken from the literature, see \citet{RefWorks:175}; 
however we reduced the former to increase the time step
($\Delta t$), following the recommendations of \citet{RefWorks:131}.
The time step was between $1.29 \%$ and $1.53 \%$ of the Rayleigh time, which
also depends on the particle density ($\rho_p$).
Furthermore, we locked the size distribution, which was obtained by experimental
sieving, see Table \ref{tab:09DEMFixedinputvalues}.
In the contact law we used, 
the tangential component of the contact force between two generic particles
($F_t$) is truncated to fulfil:
%************************************************
\begin{equation}
F_{t} \leq \mu_s F_{n},
 \label{eq:force_t}
\end{equation}
%************************************************
where $F_n$ is the normal component and $\mu_s$ is the coefficient of sliding
friction, one of the particle-based $DEM$ parameter we investigated, 
another being the coefficient of rolling friction ($\mu_r$). 
For coarse non-spherical particles, this is a critical parameter and describes
inter-particle friction in medium to dense granular flow simulations. It is proportional to the 
torque counteracting the rotation of the particle. The $\mu_r$ parameter enters the 
equations according to the elasto-rolling resistance model presented by
\citet{RefWorks:87} and \citet{RefWorks:131}. 
The model is called $EPSD2$ in $LIGGGHTS$ and is appropriate for both one-way and cyclical rolling cases.
The maximum magnitude of rolling resistance torque is (Eq. \ref{eq:trmax}):
%************************************************
\begin{equation}
T_{r~max} = \mu_r R_r |\tilde{F_n}| ~,
 \label{eq:trmax}
\end{equation}
%************************************************
where $R_r$ is the equivalent radius and $F_n$ the normal force.
The last two particle-based $DEM$ parameters we investigated were $\rho_p$
and the coefficient of restitution ($COR$) as defined by \citet{RefWorks:131}.
These coefficients, $COR$, $\mu_s$, $\mu_r$,
$\rho_p$ and $dCylDp$ (the cylinder dimension, proportional to the mean
particle diameter), as indicated in Table \ref{tab:10DEMVariableinputvalues}, 
were constant in each simulation, but their combination differed between
simulations.
Further, $dCylDp$ was used to evaluate the wall effect, but only $~10\%$ of the
simulations had a $dCylDp$ larger than $20$ (additional information can be found
in \citet{RefWorks:180}).
The normal stress $\sigma_n$ and its
percentage during the incipient flow condition $\tau_{\%}$
varied to replicate twelve shear-cell load conditions. 
The complete description of the shear-cell 
and the $AoR$ simulations
can be found in \citet{RefWorks:180}.
A Matlab script allowed us to extract from the simulation output the numerical
values representative of bulk behaviour (hereafter called $bulk ~ values$)
for each DEM simulation parameter combination, which consists of
bulk density ($\rho_b$),
coefficient of internal friction in the pre-shear phase ($\mu_{psh}$),
coefficient of internal friction in the shear phase ($\mu_{sh}$),
and angle of repose ($AoR$).
The first bulk value ($\rho_b$) was provided directly. 
First, the $\sigma_n$ was kept constant while the coefficient of internal
friction ($\mu_{ie}$) initially increased and then reached a plateau.
The second bulk value ($\mu_{psh}$) was calculated as the average of the
$\mu_{ie}$ in this plateau.
The $\sigma_n$ was then automatically reduced, in our example to $80 \%$ of
its initial value.
Subsequently, a second plateau developed.
We obtained the third
value ($\mu_{sh}$) as the average of $\mu_{ie}$ in this second plateau.
The stress path accords with the experimental one, especially the plateaux.\\
In the $AoR$ tests the average of the repose angles provided us with the fourth
bulk value, allowing us to define the numerical bulk behaviour.
%************************************************
\begin{table}%[h]
\centering
\begin{tabular}{ccccc}
\hline
    Mean & Std.dev.  & Young's & Poisson's & $\Delta t$\\
    $R$ & $R$ & modulus & ratio & \\
    (mm)  & (mm)  & (MPa) & (-) & (s)\\
    \hline
    $0.732$ & $0.41$ & $10$    & $0.40$ & $10^{-6}$\\
\hline
\end{tabular}
\caption{DEM fixed input values}
\label{tab:09DEMFixedinputvalues}
\end{table}
%************************************************
\begin{table*}%[h]
\centering
\begin{tabular}{ccccc}
\hline
    $\mu_s$ & $\mu_r$ & $COR$ & $\rho_p$ & $dCylDp$ \\
    	(-)  & (-)   & (-)   & (kg/m3) & (-) \\
    \hline
    0.4 / 0.6 / 0.8 & 0.4 / 0.6 / 0.8 & 0.5 / 0.7 / 0.9 & 2500 / 3000 / 3500 & 20 / 36 / 38 / 40 \\

\hline
\end{tabular}
\caption[DEM variable input values]{DEM variable input values for training the
Artificial Neural Networks}
\label{tab:10DEMVariableinputvalues}
\end{table*}
%************************************************
\begin{table*}%[h]
\centering
\begin{tabular}{lcccc}
\hline
 &  $\mu_s$ & $\mu_r$ & $COR$ & $\rho_p$  \\
   &	(-)  & (-)   & (-)   & (kg/m3) \\
          \hline
    range & $[0.1 \ldots 1.0]$ & $[0.1 \ldots 1.0]$ & $[0.5 \ldots 0.9]$ &
    $[2000 \ldots 3500]$     \\
    number of values & 100   & 100   & 25    & 25    \\

\hline
\end{tabular}
\caption[DEM random input values]{DEM random input values. Within each range the
indicated number of random values was chosen according to a standard uniform
distribution.}
\label{tab:12DEMRandominputvalues}
\end{table*}
%************************************************

\subsection{Artificial Neural Networks}
\label{subsec:ann}
We first defined the typology of Artificial Neural Networks ($ANNs$) we used and
the input we fed them, see \citet{RefWorks:180}.
Our $ANNs$ have three different layers: the input layer has a number of neurons
equal to the number of different inputs of the network, see Fig. \ref{fig:18nnscheme}.
The hidden (or central) layer's number of neurons was to be investigated. 
The output layer contains one neuron for the output.
The transfer functions for the central layer are the tangential sigmoid.\\
Thus, we were able to use the $DEM$ parameter combinations and their
corresponding bulk values to train the $ANNs$.
Note that 15\% of the simulations ($test ~ simulations$) were
randomly picked and excluded from the training processes.
We started with all the $DEM$ parameter combinations and their corresponding
numerical $\mu_{psh}$ to create 36 $ANNs$ that differed in their numbers of
neurons in the hidden layer (between five to forty neurons).
We then determined the coefficient of determination ($R^2$) between the
$bulk-macro$ behaviours in the output of the $ANN$ and the 15\% $test ~ simulations$, 
which were not correlated with the remaining 85\% used for the training. 
Thus, we could select for $\mu_{psh}$ the $ANN$ with the maximum $R^2$, 
again as suggested by \citet{RefWorks:150}, and we noted its number
of neurons.
We repeated the same $ANN$ creation steps for $\mu_{sh}$, $\rho_b$
and $AoR$, obtaining one trained $ANN$ for each bulk value. \\
Since $\mu_{psh}$, $\mu_{sh}$ and $\rho_b$ belonged to the shear-cell
simulations, their $ANNs$ were handled together: we had one cluster with three 
$ANNs$ for the shear cell and one with only one $ANN$ for the $AoR$.
We could then proceed in identifying valid input parameters.
\citet{RefWorks:160} suggested using a Design of Experiments
($DoE$) method to determine the parameter combinations to be simulated.
They stated that this approach allows optimization of computation time
with an acceptable loss of precision.
The speed of the trained $ANNs$ enabled us to follow a different approach to
maximizing the precision of the characterization.
We created random values
in the range and numbers defined in Table \ref{tab:12DEMRandominputvalues}
according to a standard uniform distribution.
The total number of combinations of these random values was 6,250,000.
These combinations were then fed to and processed by the selected
$ANNs$, and thus three bulk values for the shear
cell and one for the $AoR$ were obtained.
%************************************************
\begin{figure}%[!htb] 
\centering 
\includegraphics[width=.96\columnwidth]{18nnscheme} 
\caption[ANN Scheme]{Artificial Neural Network ($ANN$) Scheme
of how the Multilayer Perceptron $ANN$ ($MLPNN$) derives one
bulk-behaviour-dependent variable from the mutually independent simulation variables.}
\label{fig:18nnscheme} 
\end{figure}
%************************************************

\subsection{Macroscopic Experiments and Parameter Identification}
\label{subsec:macroscopicexperimentsparameteridentification}
The experimental characterization was performed as described in
\citet{RefWorks:180}. 
We obtained for each of the twelve load conditions of the $SSC$ three bulk
values ($\mu_{psh}$, $\mu_{sh}$ and $\rho_b$).
The fourth bulk value was the result of two angle of repose ($AoR$) tests that
recreated the repose angle observed in a pile of the
real material.
Subsequently, we compared the $ANN$ and experimental bulk behaviours for the
twelve shear-cell load conditions.
If in a DEM-parameter combination all the three bulk values differed by less 
than 5\% from those of the corresponding experiments, i.e.:
%************************************************
\begin{equation}
 \begin{cases}
\text{if } & \lvert{1-\frac{\mu_{psh,num}}{\mu_{psh,exp}}}\rvert < 5\%  ,\\
\text{and if } & \lvert{1-\frac{\mu_{sh,num}}{\mu_{sh,exp}}}\rvert < 5\% , \\ 
\text{and if } & \lvert{1-\frac{\rho_{p,num}}{\rho_{p,exp}}}\rvert < 5\% ,\\ 
\end{cases}
 \label{eq:check2}
\end{equation}
%************************************************
the combination was marked. The marked combinations were processed by the
$AoR$ $ANN$, and then compared with the experiment.
We considered those valid those that differed by less than $5\%$ also in this
comparison (Eq. \ref{eq:checkaor}):
%************************************************
\begin{equation}
\text{if} ~~~~~~ \lvert{1-\frac{AoR_{num}}{AoR_{exp}}}\rvert < 5\% .
\label{eq:checkaor}
\end{equation}
%************************************************
Further details to prove the validity of the method can be found in
\citet{RefWorks:180}.
%************************************************

\subsection{Application}
\label{subsec:application}

The method previously explained allowed to collect data regarding the contact
law. We took the average value for each of the $DEM$ parameter and we used them
in an industrial scale $DEM$ simulation, with the same software, of an iron ore
sintering process.
Especially, e examined the behaviour of these particles in a sinter chute
cooler. The particles moved from the top of a specifically designed chute and
were collected by moving boxes at the bottom.
These boxes were holed at the base, allowing cool air to lower the temperature
of the sinter.
It was critical to ascertain if the larger particles were moving to the bottom
of the boxes, while the smaller to the top, allowing a more effective
distribution of the cool air.
The simulation was performed with a maximum of 500,000 particles.

%************************************************

\section{Results and discussion}
\label{sec:results}
%************************************************

\subsection{DEM Simulations}
\label{subsec:simulations}

For sinter fine, 546 shear cell and 81 static $AoR$ simulations were run with
the parameter combinations described in Table
\ref{tab:10DEMVariableinputvalues}.
The computational time amounted to 1 hour with 32 AMD cores for a benchmark
shear-cell simulation and to 9 hours for a benchmark $AoR$ simulation, both with
50,000 particles.
Simulations with larger $dCylDp$ required more time (e.g., about 12 hours for
the shear cell with 400,000 particles ). \\


\subsection{ANN model development}
\label{subsec:annmodeldev}

First, we determined the regression of the bulk behaviour parameters, for
instance the $\mu_{psh}$; see Fig. \ref{fig:22regression}, where the
corresponding plot for the $ANN$ with the maximum $R^2$ is shown. Each circle represents one of the 546
simulations.
The plot shows a consistent agreement between the 
$DEM$ and the $ANN$ values and an almost linear regression ($R^2
= 0.94$).
The clearest connections were between $\mu_s$ and $\mu_{psh}$, and
$\rho_p$ and $\rho_b$.
In contrast, for $\mu_{sh}$ and $AoR$, the $\mu_r$ balanced the influence of the 
$\mu_s$. \\
We then investigated how the $R^2$ changed with the number of neurons
for the $\mu_{psh}$.
In this case, we achieved a $R^2 = 0.96$ for an $ANN$ with fifteen neurons. 
Increasing the number of neurons did not improve the $R^2$; it even started to
oscillate with higher numbers of neurons.
We subsequently obtained the optimal number of neurons for all $ANNs$.
Further, we processed the random combinations (Table
\ref{tab:10DEMVariableinputvalues}) with the $ANN$.
The $ANN$ evaluation was significantly faster than the $DEM$ simulations. The
individuation of the numerical bulk behaviours for all the $DEM$ combinations
did not take more than a few seconds on a single core.
%************************************************
\begin{figure}%[!h] 
\centering 
\includegraphics[width=.96\columnwidth]{22regression}
\caption[Comparison between prediction of the trained ANN and full DEM
simulation]{Comparison between prediction of the trained Artificial Neural
Network ($ANN$) and 546 full DEM simulations of the coefficient of pre-shear
($\mu_{psh}$). In this case the regression line is nearly linear (0.94), and
demonstrates the accurate predictive power of the $ANN$.}
\label{fig:22regression} 
\end{figure}

% %************************************************

\subsection{Experiments and Parameter Identification}
\label{subsec:experimentsparameteridentification}

Experimental values identifying the bulk behavior, $\mu_{psh}$, $\mu_{sh}$ and $\rho_{b}$, 
of sinter fine were acquired through $SSC$ tests. 
The $\rho_b$ has an average of 1,760 kg/m3 with a 42 
kg/m3 deviation.
Two $AoR$ tests were performed that gave an average angle of
38.85$^\circ$.
We obtained the radius ($R$) mean and standard
deviations, as shown in Table
\ref{tab:09DEMFixedinputvalues}, from sieving experiments.
The comparison between numerical and experimental behaviours led to a first
series of marked combinations ($MC1$) for one load condition of
the shear cell ($\sigma_n=10,070$ Pa, P=1.0), as plotted in Fig.
\ref{fig:24radarpirker1schulze10070}, where 
the minimum and maximum values are shown, together with the mean. 
Each axis of the parameter space plot represents one simulation parameter.
The shaded area indicates valid parameter combinations, and dark shaded
values indicate the confidence range.
Note that the confidence interval is large, 
especially for the $COR$, which highlights its insignificant influence on the
characterization.
Both the $\rho_p$  and the $\mu_s$, however, show a narrow confidence interval, 
which demonstrates their influence and the ability of this procedure to find
valid $DEM$ parameters.
These results agree with our examination of the ratio of the standard deviation
to the range, see Table \ref{tab:13DEMvalidvalues}.
Further, we observed that various $DEM$ parameter
combinations could reproduce the experimental behaviour, and thus evaluated
their mutual dependencies.
This is shown more clearly in a density plot (see Fig. 
\ref{fig:25cloudpirker1schulze10070} for $MC1$) 
of the particles' coefficient of restitution ($COR$) in relation to
the coefficients of sliding friction ($\mu_s$) and rolling friction ($\mu_r$); 
in the white area, no valid sets of simulation parameters can be found.
In each cell the valid sets are grouped according to the 4 different COR
ranges.
Each cell is coloured according to the group with the most members.
Multiple
combinations (250,407 or 4\% of the total) of $\mu_s$ and $\mu_r$ reproduced
the experimental behaviour with varying $COR$.
This underlines once more their correlation, as already stated by
\citet{RefWorks:87}.\\
We then processed the random combinations with the $AoR$ $ANN$. In Fig.
\ref{fig:31radarpirker1aor} the parameter space plot for the same criteria as
before can be seen.
In accordance with theory \cite{RefWorks:87}, in a simulation dominated
by rolling particles, the coefficient of rolling friction has the maximum
influence. \\
Finally, we extracted from the $MC1$ values the $AoR$ $ANN$ behaviour
and compared it with the experimental one.
As can be seen in the parameter space plot in Fig.
\ref{fig:33radarpirker1schulze10070aor}, the confidence interval is very small,
indicating that all the parameters but the $COR$ played an important role, 
and demonstrating the reliability of these parameter
combinations in representing the bulk behaviour.
From the initial 6,250,000 combinations, only 3,884 were valid (0.0621
\%), see Table \ref{tab:13DEMvalidvalues}.
%************************************************
\begin{table*}%[h]
\centering
\begin{tabular}{llccc}
\hline

          & type  & SSC & AoR   & SSC \& AoR \\
          \hline

    $\mu_s$ & mean  & 0.831 & 0.177 & 0.664 \\
    $(-)$   & std. dev. (SD) & 0.097 & 0.095 & 0.029 \\
          & range ($R$) & 0.9   & 0.9   & 0.9 \\
          & SD / R & 0.108 & 0.106 & 0.032 \\
          \hline
    $\mu_r$ & mean  & 0.692 & 0.830 & 0.916 \\
    $(-)$   & std. dev. (SD) & 0.215 & 0.193 & 0.042 \\
          & range ($R$) & 0.9   & 0.9   & 0.9 \\
          & SD / R & 0.239 & 0.214 & 0.046 \\
          \hline
              COR   & mean  & 0.708 & 0.590 & 0.590 \\
    $(-)$   & std. dev. (SD) & 0.104 & 0.073 & 0.065 \\
          & range ($R$) & 0.4   & 0.4   & 0.4 \\
          & SD / R & 0.259 & 0.183 & 0.161 \\
          \hline
    $\rho_p$ & mean  & 2245.7 & 3192.8 & 2283.9 \\
    $(kg/m3)$ & std. dev. (SD) & 80.5  & 277.4 & 67.1 \\
          & range ($R$) & 1500  & 1500  & 1500 \\
          & SD / R & 0.054 & 0.185 & 0.045 \\
          \hline
    valid & number & 290203 & 816552 & 3884 \\
    combinations & (\%) & 4.64  & 13.06 & 0.06 \\  

\hline
\end{tabular}
\caption[Valid DEM values]{Valid DEM values. For each parameter we show the
valid parameter statistics in the two tests and in their intersection.
Finally, we show the number of valid parameter combinations over the total
(6250000).}
\label{tab:13DEMvalidvalues}
\end{table*}
%************************************************
\begin{figure}%[!htb] 
\centering 
\includegraphics[width=.96\columnwidth]{24radarpirker1schulze10070} 
\caption{Parameter space plot, $SSC$, $\sigma_n=10070$ Pa, P=1.0}
\label{fig:24radarpirker1schulze10070} 
\end{figure}
%************************************************
\begin{figure}%[!htb] 
\centering 
\includegraphics[width=.96\columnwidth]{25cloudpirker1schulze10070} 
\caption{Density plot, $SSC$, $\sigma_n=10070$ Pa, P=1.0}
\label{fig:25cloudpirker1schulze10070} 
\end{figure}
%************************************************
\begin{figure}%[!htb] 
\centering 
\includegraphics[width=.96\columnwidth]{31radarpirker1aor} 
\caption{Parameter space plot, $AoR_{exp} = 38.85 ^\circ$}
\label{fig:31radarpirker1aor} 
\end{figure}
%************************************************
\begin{figure}%[!htb] 
\centering 
\includegraphics[width=.96\columnwidth]{33radarpirker1schulze10070aor} 
\caption{Parameter space plot, $AoR_{exp} = 38.85
        ^\circ$ \& $SSC$: $\sigma_n=10070$ Pa}
\label{fig:33radarpirker1schulze10070aor} 
\end{figure}

%************************************************

\subsection{Macroscopic application}
\label{subsec:macroapplication}

We divided the first of the boxes filled by the chute in three (equally high)
layers.
The top layer was not considered, because it was continuously supplied of
particles from the chute.
In Fig. \ref{fig:42particleDistributionChute} the total volume of the particles
available in each layer, grouped by radius, is shown.
We could clearly see how the larger particles disposed mostly in the bottom
layer, validating the realized design.

%42particleDistributionChute
%************************************************
\begin{figure*}%[!htb] 
\centering 
\includegraphics[width=.96\textwidth]{42particleDistributionChute} 
\caption{Distribution of particles in the box}
\label{fig:42particleDistributionChute} 
\end{figure*}

%************************************************

\section{Conclusions}
\label{sec:conclusions}
%************************************************
We have presented a two-step method for $DEM$ simulation parameter
identification. In the first step, an artificial neural network is 
trained using dedicated $DEM$ simulations in order to predict bulk 
behaviours as function of a set of $DEM$ simulation parameters. 
In the second step, this artificial neural network is then used 
to predict the bulk behaviour of a huge number of additional $DEM$ parameter
sets.
The main findings of this study can be summarized as follows:
\begin{itemize}
  \item{An artificial neural network can be trained by a limited number of
  dedicated $DEM$ simulations.
  		The trained artificial neural network is then able to predict
  		granular bulk behaviour.}
  \item{This prediction of granular bulk behaviour is much more efficient
  		than computationally expensive $DEM$ simulations.
  		Thus, the macroscopic output associated with a huge number of parameter sets
  		can be studied.}
  \item{If the predictions of the artificial neural network are compared to a bulk experiment, 
  		valid sets of $DEM$ simulation parameters can be readily deduced for a
  		specific granular material.}
  \item{This $DEM$ parameter identification method can be applied to
  arbitrary bulk experiments.
  		Combining two artificial neural networks which predict two different bulk
  		behaviours leads to winnowing the set of valid $DEM$ simulation parameters.}
  \item{The parameters collected with this method can be trustworthy used for
  large scale simulations.}
\end{itemize}
As part of future work, we will develop this method further by considering
different fractions of granular materials, which will lead to size-dependent sets of $DEM$
simulation parameters.

%************************************************


\section{Aknowledgements}
This study was funded by Christian Doppler Forschungsgesellschaft, Siemens VAI Metals Technologies and Voestalpine Stahl. The authors gratefully aknowledge their support.\\

%% %---------------------------
%% %BIBLIOGRAPHY
%% %---------------------------
%The bibliography is created using BiBTeX
\bibliographystyle{CFD2015}
\bibliography{References}   

\end{document}