%************************************************
\section{Introduction}
\label{sec:introduction}
%************************************************

Particles in various forms - ranging from raw materials to food grains and pharmaceutical powders - 
play a major role in a variety of industries. 
Discrete Element Method ($DEM$) simulations are widely used to picture particle
behaviour in these granular processes (Cleary and Sawley \cite{RefWorks:130}).\\
In their original formulation of $DEM$, Cundall and Strack \cite{RefWorks:172} allowed two 
particles to slightly overlap upon contact and consequently they proposed repulsive forces in dependence on this overlap distance. 
Since then their fundamental modelling concept has been widely accepted in literature 
and their soft sphere contact law has been further developed by numerous researchers 
(Vu-Quoc and Zhang \cite{RefWorks:148} and Di Renzo and Di Maio \cite{RefWorks:145}). 
With increasing computational resources $DEM$ simulation have become very prominent 
giving rise to the development of commercial (e.g. $PFC3D$, used by Wensrich and
Katterfeld \cite{RefWorks:87}) and open-source software (e.g. $LIGGGHTS$, Kloss
et al. \cite{RefWorks:136}, Aigner et al. \cite{RefWorks:139}).
Soft sphere $DEM$ simulations of thousands of particles have proven to 
picture particles bulk behaviour (Hohner et al. \cite{RefWorks:86}). \\
In these macroscopic $DEM$ simulations the contact law kernel between a 
pair of particles determines the global bulk behaviour of the granular material (Ai et al. \cite{RefWorks:131}). 
As a consequence, defining a correct contact law is of crucial importance for the predictive 
capability of $DEM$ simulations of granular material. 
Since $DEM$ contact laws, in turn, are based 
on a set of semi-empirical parameters, this further requires defining correct contact law 
parameters for a given granular material. 
Without material dependent contact law parameter identification $DEM$ simulations are prone to fail (Combarros et al. \cite{RefWorks:177}). \\
Identifying $DEM$ contact law parameters is not at all a trivial task. 
It may be impracticable to identify valid parameter sets by performing bilateral 
particle collision experiments due to the sheer number of different particles within in a granular material. 
Furthermore, some contact law parameters like the coefficient of rolling
friction are purely empirical in nature and cannot be determined by direct 
particle to particle measurements (Wensrich and Katterfeld \cite{RefWorks:87}).
Therefore, $DEM$ contact law parameters (table \ref{tab:08DEMparameters}) are
commonly determined by comparing the macroscopic outcome of large scale $DEM$ simulations with bulk experiments (Alenzi et al. \cite{RefWorks:91}). 
If $DEM$ simulation results disagree with bulk measurements the set of contact law parameters 
has to be adapted until reasonable agreement id achieved.\\
However, this purely forward methodology of parameter identification is limited by 
the multi-dimensionality of the parameter space and the associated computational costs of the required $DEM$ test simulations. 
To make things worse, one parameter set which is valid for one bulk behaviour (e.g. angle of repose) 
might fail in case of another bulk behaviour (e.g. shear tester). \\
Obviously, there is a need for an efficient methodology for the identification of $DEM$ contact law parameters. 
In our study we harnessed Artificial Neural Networks ($ANN$) in order to reduce the number of required $DEM$ test simulations. 
$ANN$ have proven to be a versatile tool in analysing complex, non-linear systems of 
multi-dimensional input-streams (Vaferi et al. \cite{RefWorks:150}, Witten et
al. \cite{RefWorks:174} and Haykin \cite{RefWorks:158}).
In our case we fed an $ANN$ with $DEM$ contact law parameters as input parameters and 
subsequently compared the output of the $ANN$ with the predicted bulk behaviour of a corresponding $DEM$ simulation. 
The difference between $ANN$ prediction and $DEM$ prediction is utilized to train our 
specific $ANN$ by a backward propagation algorithm (described further below). 
After a training phase by a limited number of $DEM$ test simulations, the $ANN$ can then be 
used as a stand-alone prediction tool for the bulk behaviour of a granular material in dependence of $DEM$ contact law parameters. \\
In this study we applied this parameter identification methodology to two different granular bulk behaviours, 
namely the angle of repose ($AoR$) test and Schulze shear cell ($SSC$) test. 
In both cases we first trained a specific $ANN$ by a number of $DEM$ test simulations, 
before we identified valid sets $DEM$ contact law parameters by comparing those 
stand-alone $ANN$ predictions with corresponding bulk experiments. 
For both cases we obtained valid sets of contact law parameters 
which we then intersected to get a reliable contact law formulation for a given granular material. 
We further show that the same $ANN$ can be utilized to characterize different granular materials. \\
In the following section we define some pre-requisites including $DEM$ contact law definitions, 
a general description of the $ANN$ functionality and sketches of the experimental set-ups under consideration. 
Next, we present the proposed methodology of $DEM$ contact law parameter identification. 
Subsequently, we apply this methodology to characterize $DEM$ contact law
parameters of sinter fines.
%\begin{table}[h]
\centering
\begin{tabular}{l}
\hline 
    Radius \ac{R} (m)   \\ [5pt]

	Size distribution (-) \\ [5pt]

    Young's modulus \ac{E} (Pa)  \\ [5pt]

    Poisson's ratio \ac{nu} (-) \\ 
     Time step \ac{deltat} (s) \\ [5pt]
        \hline
     Coefficient of sliding friction \ac{mus} (-)\\  [5pt]
    Coefficient of rolling friction \ac{mur} (-) \\ [5pt]
    Coefficient of restitution \ac{CoR} (-)   \\ [5pt]
     Particle density $\ac{rhop} = \frac{mass ~ of ~ one ~ particle}{volume ~ of
     ~ one ~ particle}$ ($kg/m^3$)  \\ [5pt]
     Geometry factor \ac{dCylDp} (-)  \\ [5pt]
   
\hline
\end{tabular}
\caption[DEM parameters]{DEM parameters. The upper parameters were
identical in all simulations. The lower parameters were constant in each
simulation, but were varied between simulations.}
\label{tab:08DEMparameters}
\end{table}


%************************************************
