%&pdflatex
\documentclass[review]{elsarticle}
%\usepackage[utf8]{inputenc}
\usepackage{subcaption}
\usepackage{amsmath}
\usepackage{listings}
\usepackage{courier}
\usepackage{graphicx}
% \usepackage{picins}


\usepackage[nomarkers]{endfloat}

\newcommand{\listoflstlistings}{\lstlistoflistings}
\DeclareDelayedFloat{lstlisting}[flol]{\textbf{List of Listings}}

\lstset{tabsize=2,language=C++,basicstyle=\footnotesize\tt,keywordstyle=\color{blue}\bfseries}

\journal{Powder Technology}

%%%%%%%%%%%%%%%%%%%%%%%%%%%%%%%%%%%%%%%%%%%%%%%%%%%%%%%%%%%%%%%%%%%%%%%%%%%%%%%%%%%%%%%%%%%%%%%%%%%%%%%%%%%%%%%%%%%%%%%%%%
\begin{document}

%%%%%%%%%%%%%%%%%%%%%%%%%%%%%%%%%%%%%%%%%%%%%%%%%%%%%%%%%%%%%%%%%%%%%%%%%%%%%%%%%%%%%%%%%%%%%%%%%%%%%%%%%%%%%%%%%%%%%%%%%%
\begin{frontmatter}

\title{Identification of DEM Simulation Parameters by Artificial Neural Networks
and Bulk Experiments}

\author[jku]{L.~Benvenuti\corref{cor1}}
\ead{luca.benvenuti@jku.at}
%\ead{Tel. +43 73224686483}

\author[dcs]{C.~Kloss}
%\ead{christoph.kloss@dcs-computing.com}

\author[jku]{S.~Pirker}
%\ead{stefan.pirker@jku.at}

\cortext[cor1]{Corresponding author}

\address[jku]{Johannes Kepler University Linz, Department on Particulate Flow
Modelling, Altenbergerstrasse 69, 4040, Linz, Austria}

\address[dcs]{DCS Computing GmbH, Altenbergerstr. 66a - Science Park, 4040 Linz,
Austria}

%%%%%%%%%%%%%%%%%%%%%%%%%%%%%%%%%%%%%%%%%%%%%%%%%%%%%%%%%%%%%%%%%%%%%%%%%%%%%%%%%%%%%%%%%%%%%%%%%%%%%%%%%%%%%%%%%%%%%%%%%%
\begin{abstract}
In Discrete Element Method ($DEM$) simulations, particle-particle contact laws
determine the macroscopic simulation results. Particle based contact laws, in
turn, commonly rely on semi-empirical parameters, which can be hardly obtained
by direct microscopic measurements.
In this study we present a methodology for the identification of
$DEM$ simulation parameters by linking the macroscopic experimental results to the
microscopic numerical parameters by artificial neural networks.
In a first step, a series
of $DEM$ simulations with varying simulation parameters are used to train a feed
forward artificial neural network by backward propagation reinforcement. In a
second step, this artificial neural network is utilized to predict the
macroscopic ensemble behaviour in dependence of additional sets of particle
based simulation parameters.
As a result, a comprehensive database is obtained,
which links particle based simulation parameters to a specific macroscopic
bulk behaviour of the ensemble.
The trained artificial neural network is able to predict the behaviour of
additional sets of input parameters accurately and highly efficient.
Furthermore, this methodology can be applied to
identify $DEM$ material parameters in a generic way.
For each set of calibration experiments, the training of the neural network has
to be performed just once. 
After the training, the neural network provides a generic link between the macroscopic 
experimental results and the microscopic $DEM$ simulation parameters.
By the help of these experiments, the $DEM$ simulation parameters of a specific
non-cohesive granular material can be identified.

\end{abstract}

\begin{keyword}
Discrete Element Method ($DEM$) Simulations \sep Parameter Identification \sep Artificial Neural Networks
\end{keyword}
\end{frontmatter}
%%%%%%%%%%%%%%%%%%%%%%%%%%%%%%%%%%%%%%%%%%%%%%%%%%%%%%%%%%%%%%%%%%%%%%%%%%%%%%%%%%%%%%%%%%%%%%%%%%%%%%%%%%%%%%%%%%%%%%%%%%

%************************************************
\section{Highlights}
\label{sec:highlights}
%************************************************
\begin{itemize}
  \item{We trained an Artificial Neural Network by DEM simulations with varying
  parameters}
  \item{The Artificial Neural Network then predicts granular bulk behaviour}
  \item{By comparison with bulk experiments DEM simulation parameters are
  identified}
  \item{This DEM parameter identification can be applied to different bulk
  behaviours}
  \item{This DEM parameter identification can be applied to different granular
  materials}
\end{itemize}
%************************************************

%************************************************
\section{Introduction}
\label{sec:introduction}
%************************************************

Particles in various forms - ranging from raw materials to food grains and pharmaceutical powders - 
play a major role in a variety of industries. 
Discrete Element Method ($DEM$) simulations are widely used to picture particle
behaviour in these granular processes (Cleary and Sawley \cite{RefWorks:130}).\\
In their original formulation of $DEM$, Cundall and Strack \cite{RefWorks:172} allowed two 
particles to slightly overlap upon contact and consequently they proposed repulsive forces in dependence on this overlap distance. 
Since then their fundamental modelling concept has been widely accepted in literature 
and their soft sphere contact law has been further developed by numerous researchers 
(Vu-Quoc and Zhang \cite{RefWorks:148} and Di Renzo and Di Maio \cite{RefWorks:145}). 
With increasing computational resources $DEM$ simulation have become very prominent 
giving rise to the development of commercial (e.g. $PFC3D$, used by Wensrich and
Katterfeld \cite{RefWorks:87}) and open-source software (e.g. $LIGGGHTS$, Kloss
et al. \cite{RefWorks:136}, Aigner et al. \cite{RefWorks:139}).
Soft sphere $DEM$ simulations of thousands of particles have proven to 
picture particles bulk behaviour (Hohner et al. \cite{RefWorks:86}). \\
In these macroscopic $DEM$ simulations the contact law kernel between a 
pair of particles determines the global bulk behaviour of the granular material (Ai et al. \cite{RefWorks:131}). 
As a consequence, defining a correct contact law is of crucial importance for the predictive 
capability of $DEM$ simulations of granular material. 
Since $DEM$ contact laws, in turn, are based 
on a set of semi-empirical parameters, this further requires defining correct contact law 
parameters for a given granular material. 
Without material dependent contact law parameter identification $DEM$ simulations are prone to fail (Combarros et al. \cite{RefWorks:177}). \\
Identifying $DEM$ contact law parameters is not at all a trivial task. 
It may be impracticable to identify valid parameter sets by performing bilateral 
particle collision experiments due to the sheer number of different particles within in a granular material. 
Furthermore, some contact law parameters like the coefficient of rolling
friction are purely empirical in nature and cannot be determined by direct 
particle to particle measurements (Wensrich and Katterfeld \cite{RefWorks:87}).
Therefore, $DEM$ contact law parameters (Table \ref{tab:08DEMparameters}) are
commonly determined by comparing the macroscopic outcome of large scale $DEM$ simulations with bulk experiments (Alenzi et al. \cite{RefWorks:91}). 
If $DEM$ simulation results disagree with bulk measurements the set of contact law parameters 
has to be adapted until reasonable agreement id achieved.\\
However, this purely forward methodology of parameter identification is limited by 
the multi-dimensionality of the parameter space and the associated computational costs of the required $DEM$ test simulations. 
To make things worse, one parameter set which is valid for one bulk behaviour (e.g. angle of repose) 
might fail in case of another bulk behaviour (e.g. shear tester). \\
Obviously, there is a need for an efficient methodology for the identification of $DEM$ contact law parameters. 
In our study we harnessed Artificial Neural Networks ($ANN$) in order to reduce the number of required $DEM$ test simulations. 
$ANN$ have proven to be a versatile tool in analysing complex, non-linear systems of 
multi-dimensional input-streams (Vaferi et al. \cite{RefWorks:150}, Witten et
al. \cite{RefWorks:174} and Haykin \cite{RefWorks:158}).
In our case we fed an $ANN$ with $DEM$ contact law parameters as input parameters and 
subsequently compared the output of the $ANN$ with the predicted bulk behaviour of a corresponding $DEM$ simulation. 
The difference between $ANN$ prediction and $DEM$ prediction is utilized to train our 
specific $ANN$ by a backward propagation algorithm (described further below). 
After a training phase by a limited number of $DEM$ test simulations, the $ANN$ can then be 
used as a stand-alone prediction tool for the bulk behaviour of a granular material in dependence of $DEM$ contact law parameters. \\
In this study we applied this parameter identification methodology to two different granular bulk behaviours, 
namely the angle of repose ($AoR$) test and Schulze shear cell ($SSC$) test. 
In both cases we first trained a specific $ANN$ by a number of $DEM$ test simulations, 
before we identified valid sets $DEM$ contact law parameters by comparing those 
stand-alone $ANN$ predictions with corresponding bulk experiments. 
For both cases we obtained valid sets of contact law parameters 
which we then intersected to get a reliable contact law formulation for a given granular material. 
We further show that the same $ANN$ can be utilized to characterize different granular materials. \\
In the following section we define some pre-requisites including $DEM$ contact law definitions, 
a general description of the $ANN$ functionality and sketches of the experimental set-ups under consideration. 
Next, we present the proposed methodology of $DEM$ contact law parameter identification. 
Subsequently, we apply this methodology to characterize $DEM$ contact law
parameters of sinter fines.
%************************************************
%\begin{table}[h]
\centering
\begin{tabular}{l}
\toprule
    Radius \acs{R} (m)   \\ [5pt]

	Size distribution (-) \\ [5pt]

    Young's modulus \acs{E} (Pa)  \\ [5pt]

    Poisson's ratio \acs{nu} (-) \\ 
     Time step \acs{deltat} (s) \\ [5pt]
\midrule
     Coefficient of sliding friction \acs{mus} (-)\\  [5pt]
    Coefficient of rolling friction \acs{mur} (-) \\ [5pt]
    Coefficient of restitution \acs{CoR} (-)   \\ [5pt]
     Particle density $\acs{rhop} = \frac{mass ~ of ~ one ~ particle}{volume ~ of
     ~ one ~ particle}$ ($kg/m^3$)  \\ [5pt]
     Geometry factor $\acs{dCylDp} = \frac{diameter ~ of ~ the ~
     cylinder}{diameter ~ of ~ one ~ particle}$ (-)  \\ [5pt]
   
\bottomrule
\end{tabular}
\caption[DEM parameters]{DEM parameters. The upper parameters were
identical in all simulations.
%  The lower parameters were constant in each
% simulation, but were varied between simulations.
}
\label{tab:08DEMparameters}
\end{table}


\begin{table}[h]
\centering
\begin{tabular}{l}
\hline 
    Particles radius $R$ (m)   \\ [5pt]

	Size distribution (-) \\ [5pt]

    Young's modulus $E$ (Pa) \\ [5pt]

    Poisson's ratio $\nu$ (-) \\ 
     Time step $\Delta t$ (-) \\ [5pt]
        \hline
     Coefficient of sliding friction $\mu_s$ (-) \\  [5pt]
    Coefficient of rolling friction $\mu_r$ (-) \\ [5pt]
    Coefficient of restitution $COR$ (-) \\ [5pt]
     Particle density $\rho_p$ (kg/m^3) \\ [5pt]
    Geometry factor $dCylDp$ (-) \\ [5pt]
   
\hline
\end{tabular}
\caption[DEM parameters]{DEM parameters. The upper parameters are
identical in all the simulations. The lower parameters are constant in each
simulation, but are varied between simulations.}
\label{tab:08DEMparameters}
\end{table}
%************************************************

\section{Methodology of DEM Parameter Identification}
\label{sec:methodology}

We now illustrate the methodology used, also shown in Fig.
\ref{fig:19methodology}.
%************************************************
\begin{figure}[!htb] 
\centering 
\includegraphics[width=.96\textwidth]{19methodology} 
\caption[Methodology]{Methodology. 
In the training phase (dashed lines) from the initial random input parameters
$DEM$ simulations are performed. The behaviours provided are used to train the
Neural Networks ($NN$), in a loop that continues until the difference is within
the limit ($\Delta$).
Then in the parameters' identification phase (continous
lines) we identify the valid input parameters by comparing (\textbf{=}) $NN$ and
experimental behaviours.
Further explanations in section \ref{sec:methodology}.
}
\label{fig:19methodology} 
\end{figure}
%************************************************
\subsection{Discrete element method}
\label{subsec:dem}

We decided to fix one single
contact law for all the simulations performed, see
\ref{subsec:srsctsimulation}.
The $DEM$ parameters for the Young's modulus ($E$) and the coefficient of Poisson ($\nu$) 
have been taken from the literature, see \cite{RefWorks:175} and \cite{RefWorks:176}, 
although we reduced the former to increase the time step ($\Delta t$), following
the indications of Ai et al. \cite{RefWorks:131}.
The last was between $1.29 \%$ and $1.53 \%$ of the Rayleigh time, that also
depends to the particle density ($\rho_p$).
Furthermore, we locked the size distribution, provided by experimental sieving,
see Table \ref{tab:09DEMFixedinputvalues}.
In the contact law we used, 
the tangential component of the contact force between two generic particles
($F_t$) is truncated to fulfil:
%************************************************
\begin{equation}
F_{t} \leq \mu_s F_{n},
 \label{eq:force_t}
\end{equation}
%************************************************
where $F_n$ is the normal component and $\mu_s$ is the coefficient of sliding
friction, one of the particle based $DEM$ parameter we investigated. 
Another parameter was the coefficient of rolling friction ($\mu_r$). 
For coarse non-spherical particles this is a critical parameter and describes
inter-particle friction in medium to dense granular flows simulations. It is proportional to the 
torque counteracting the rotation of the particle. The $\mu_r$ parameter enters the 
equations according to the elasto-rolling resistance model presented by Wensrich and 
Katterfeld \cite{RefWorks:87} and Ai et al. \cite{RefWorks:131}, 
based on the work of Jiang et al. \cite{RefWorks:143}. 
The model is called $EPSD2$ in $LIGGGHTS$. This is appropriate for the one way
rolling cases as well as the cycling rolling ones.
The maximum magnitude of rolling resistance torque is (Eq. \ref{eq:trmax}):
%************************************************
\begin{equation}
T_{r~max} = \mu_r R_r |\tilde{F_n}| ~,
 \label{eq:trmax}
\end{equation}
%************************************************
where $R_r$ is the equivalent radius and $F_n$ the normal force.
The last two particle based $DEM$ parameter we investigated were the $\rho_p$
and the coefficient of restitution ($COR$), as defined by Ai. et al. \cite{RefWorks:131}.
The coefficients, $COR$, $\mu_s$, $\mu_r$,
$\rho_p$ and $dCylDp$ (the cylinder dimension, as proportion to the mean
particle diameter), as indicated in Table \ref{tab:10DEMVariableinputvalues}, were constant in each simulation, but their combination differed between
simulations.
Further, $dCylDp$ was used to evaluate the wall effect, but only $~10\%$ of the
all simulations had $dCylDp$ larger than $20$ (additional information can be found in \ref{subsec:srsctsimulation}). 
The normal stress $\sigma_n$ and its
percentage during the incipient flow condition $\tau_{\%}$
varied to replicate twelve shear cell load conditions. 
The complete description of the shear cell simulations can be found in \ref{subsec:srsctsimulation}, 
while the $AoR$ simulation is presented in \ref{subsec:aorsimulation}.
A Matlab script allowed us to extract from the simulations output the numerical
bulk representative values for
each $simulation-DEM$ parameter combination:
bulk density ($\rho_b$),
coefficient of internal friction in the pre-shear phase ($\mu_{psh}$),
coefficient of internal friction in the shear phase ($\mu_{sh}$),
and angle of repose ($AoR$).
The first bulk behaviour representative value ($\rho_b$) was directly provided. 
If a simulation was performed correctly, see \ref{subsec:srsctsimulation}, we
observed a stress path as in Fig. \ref{fig:21simexample}.
First, the $\sigma_n$ was kept constant. 
Meanwhile, the coefficient of internal friction ($\mu_{ie}$) initially increased
and then reached a plateau.
The second bulk behaviour representative value ($\mu_{psh}$) was calculated as average of the $\mu_{ie}$ in this plateau.
Automatically, the $\sigma_n$ was reduced, in this example to $80 \%$ of its
initial value.
After, a second plateau started.
As average of $\mu_{ie}$ in this second plateau we obtained the third value
($\mu_{sh}$).
The stress path is
comparable to the experimental one, especially the plateaux.\\
In the $AoR$ tests the average of the repose angles provided us the fourth bulk
value, allowing us to define the numerical bulk behaviour.
%************************************************
\begin{figure}[htp] \centering
    \begin{subfigure}[b]{0.96\columnwidth}
        \includegraphics[width=\textwidth]{20experimental}
        \caption{Experimental shear cell tester stress path - $\sigma_n = 10000
        ~Pa$}
        \label{fig:20experimental} 
    \end{subfigure}\\
        \begin{subfigure}[b]{0.96\columnwidth}
        \includegraphics[width=\textwidth]{21simexample}
        \caption{Numerical shear cell tester stress path - $\sigma_n = 10000
        ~Pa$}
        \label{fig:21simexample} 
    \end{subfigure}
    \caption[Stress path]{Sample of the stress path for
	the Schulze ring shear cell tester, experimental and numerical.
	Time is normalized: $\tilde{t} = t/t_{change}$, where $t_{change}$ is the
	time when the normal stress ($\sigma_n$) is modified during the tests.
	Until $\tilde{t}=1$ the $\sigma_n = 10000 ~Pa$ is kept constant. 
	In Fig. \ref{fig:20experimental} at $\tilde{t}~=0.91$
 	a plateau is reached.
	The coefficient of pre-shear ($\mu_{psh}$) is calculated as average of the
	coefficient of internal friction ($\mu_{ie}$) in this first plateau.
	Later, at $\tilde{t}=1$, the $\sigma_n$ is reduced to $80 \%$ of its initial
	value.
	Soon, a second plateau starts.
	As average of $\mu_{ie}$ in this second plateau we obtain coefficient of
	shear ($ \mu_{sh}$).
	The stress path is in the numerical simulation is comparable to the
	experimental one, especially the plateaux.
	They were clearly relevant because there we collected the numerical bulk
	behaviour representative values. }
    \label{fig:40experimentalsimulation}
\end{figure}
%************************************************
\begin{table}[h]
\centering
\begin{tabular}{ccccc}
\hline
    Mean & Std.dev.  & Young's & Poisson's & $\Delta t$\\
    $R$ & $R$ & modulus & ratio & \\
    (mm)  & (mm)  & (MPa) & (-) & (s)\\
    \hline
    $0.732$ & $0.41$ & $10$    & $0.40$ & $10^{-6}$\\
\hline
\end{tabular}
\caption{DEM fixed input values}
\label{tab:09DEMFixedinputvalues}
\end{table}
%************************************************
\begin{table}[h]
\centering
\begin{tabular}{ccccc}
\hline
    $\mu_s$ & $\mu_r$ & $COR$ & $\rho_p$ & $dCylDp$ \\
    	(-)  & (-)   & (-)   & (kg/m3) & (-) \\
    \hline
    0.4 / 0.6 / 0.8 & 0.4 / 0.6 / 0.8 & 0.5 / 0.7 / 0.9 & 2500 / 3000 / 3500 & 20 / 36 / 38 / 40 \\

\hline
\end{tabular}
\caption[DEM variable input values]{DEM variable input values for training the
Neural Networks}
\label{tab:10DEMVariableinputvalues}
\end{table}
%************************************************
\begin{table}[h]
\centering
\begin{tabular}{lcccc}
\hline
 &  $\mu_s$ & $\mu_r$ & $COR$ & $\rho_p$  \\
  &	(-)  & (-)   & (-)   & (kg/m3) \\
          \hline
    range & $[0.1 \ldots 1.0]$ & $[0.1 \ldots 1.0]$ & $[0.5 \ldots 0.9]$ &
    $[2000 \ldots 3500]$     \\
    number of values & 100   & 100   & 25    & 25    \\

\hline
\end{tabular}
\caption[DEM random input values]{DEM random input values. Within each range the
indicated number of random values is chosen according to a standard uniform
distribution.}
\label{tab:12DEMRandominputvalues}
\end{table}
%************************************************

\subsection{Artificial Neural Networks}
\label{subsec:ann}
We first defined the typology of Artificial Neural Network ($ANN$) we used and
the input we imposed to them, see \ref{sec:appann}.
Our $ANN$ are built with three different layers. 
The input layer has a number of neurons equal to the number of different inputs
of the network, see Fig. \ref{fig:18nnscheme}.
The hidden (or central) layer's number of neurons must be investigated. 
The output layer contains one neuron for the output.
The transfer functions between the first two layers are the tangential sigmoid, 
while between the hidden and central layers are linear.\\
So, we could use the $DEM$ parameter combinations and their corresponding bulk
values to train the $ANN$,
dashed line in Fig. \ref{fig:19methodology}.
Notably, we excluded 15\% of the simulations ($test ~ simulations$),
randomly picked, from the training processes.
We started with all the $DEM$ parameter combinations and their corresponding numerical $\mu_{psh}$ to create 36 $ANN$. 
They differed because they included from five to forty neurons in the hidden
layer.
Later, we controlled the coefficient of determination ($R^2$) between the
$bulk-macro$ behaviours in the output of the $ANN$ and the 15\% $test ~ simulations$, 
which are not correlated with the remaining 85\% used for the training. 
So, we could select for $\mu_{psh}$ the $ANN$ with the maximum $R^2$, 
again as suggested by Vaferi et al. \cite{RefWorks:150}, and we noted its number
of neurons.
We repeated the same steps from the $ANN$ creations for $\mu_{sh}$, $\rho_b$ and $AoR$, 
obtaining one trained $ANN$ for each bulk representative value. \\
Notably, $\mu_{psh}$, $\mu_{sh}$ and $\rho_b$ belonged to the shear cell
simulations, so their $ANN$ were handled together. 
We had one cluster with three $ANN$ for the shear cell and one with only one $ANN$
for the $AoR$.
We could then proceed in identifying the valid input parameters, straight line
in Fig. \ref{fig:19methodology}.
Oberkampf et al. \cite{RefWorks:160} suggest to use a Design of Experiments
($DoE$) method to determine the parameters' combinations to be simulated.
They affirm that this approach allows optimizing the computational time with
only an acceptable loss of precision.
Thanks to the speed of the trained $ANN$, we followed a different approach to
maximize the precision of the characterization.
We created random values
in the range and number defined in Table \ref{tab:12DEMRandominputvalues} according to a standard uniform
distribution.
Together, the total number of combinations of these random values was $6250000$.
These combinations were then imposed as input and processed by the selected
$ANN$, granting for each three bulk representative parameters for the shear cell and one for the $AoR$. 
%************************************************
\begin{figure}[!htb] 
\centering 
\includegraphics[width=.96\textwidth]{18nnscheme} 
\caption[NN Scheme]{Neural Network ($NN$) Scheme. This is the schematic of how
the Multi Layer Perceptron $NN$ ($MLPNN$) derives one bulk behaviour
dependent variable from the mutually independent simulation variables.}
\label{fig:18nnscheme} 
\end{figure}
%************************************************

\subsection{Macroscopic Experiments and Parameter Identification}
\label{subsec:macroscopicexperimentsparameteridentification}
The experimental characterization has been performed as described in
\ref{subsec:srsctexperiment} and \ref{subsec:aorexperiment}. 
We obtained for each of the twelve load conditions of the $SSC$ three values
representative of the bulk behaviour ($\mu_{psh}$, $\mu_{sh}$ and $\rho_b$).
Furthermore, to recreate the repose angle observed in a pile of the real material, 
we performed two angle of repose ($AoR$) tests, as the $AoR$ was the fourth
behaviour value. 

Later, we confronted the $ANN$ and experimental bulk behaviours for the twelve shear cell load conditions. 
If in a $DEM-parameter$ combination all the three bulk representative parameters differed less 
than 5\% from the corresponding experiments, see Eq. \ref{eq:check2}:
%************************************************
\begin{equation}
 \begin{cases}
\text{if } & \lvert{1-\frac{\mu_{psh,num}}{\mu_{psh,exp}}}\rvert < 5\%  ,\\
\text{and if } & \lvert{1-\frac{\mu_{sh,num}}{\mu_{sh,exp}}}\rvert < 5\% , \\ 
\text{and if } & \lvert{1-\frac{\rho_{p,num}}{\rho_{p,exp}}}\rvert < 5\% ,\\ 
\end{cases}
 \label{eq:check2}
\end{equation}
%************************************************
then the combination was marked. The marked combinations were handled by the
$AoR$ $ANN$, and then confronted with the experiment.
Were branded as valid only those that differed less than $5\%$ also in this
comparison (Eq. \ref{eq:checkaor}):
%************************************************
\begin{equation}
\text{if} ~~~~~~ \lvert{1-\frac{AoR_{num}}{AoR_{exp}}}\rvert < 5\% .
\label{eq:checkaor}
\end{equation}
%************************************************
Further, to prove the system validity, we tested the marked combinations by
modifying the experimental bulk behaviour representative values of the shear cell. 
We artificially decreased or increased $\mu_{psh}$ and $\mu_{sh}$ by a product
coefficient ($P$), e.g. Eq. \ref{eq:pcoeff}:
%************************************************
\begin{equation}
\label{eq:pcoeff}
\mu_{psh, new} = \mu_{psh, old} \cdot P .
\end{equation}
%************************************************

\section{Results and discussion}
\label{sec:results}
%************************************************

\subsection{DEM Simulations}
\label{subsec:simulations}

For sinter fine 546 shear cell and 81 static angle of repose simulations have
been realized with the variations described in Table
\ref{tab:10DEMVariableinputvalues}.
The computational time resulted in 1 hour with 32 AMD cores for a benchmark
shear cell simulation and 9 hours for a benchmark $AoR$ simulation, both with
50000 particles.
Simulations with larger $dCylDp$ required a greater time amount (e.g. with
400000 particles about 12 hours for the shear cell). \\


\subsection{ANN model development}
\label{subsec:annmodeldev}

First, we controlled the regression of the bulk behaviour parameters, e.g. the
$\mu_{psh}$, see Fig. \ref{fig:22regression}, where the corresponding plot for
the $ANN$ with the maximum $R^2$ is shown. Each circle represents one of the 546
simulations.
The plot presents a consistent agreement between the results distribution
and the regression (or fitting) line.
The linear relationship between the
training values have been evaluated in Table \ref{tab:06inputRelationshipTable}.
The clearest connections were between $\mu_s$ and $\mu_{psh}$, and
$\rho_p$ and $\rho_b$.
Instead, for $\mu_{sh}$ and $AoR$ the $\mu_r$ balanced the influence of the 
$\mu_s$, and further parameters were worthly correlated. \\
Then we observed how the $R^2$ changed with the different number of neurons for the $\mu_{psh}$. 
In this case we reached a $R^2 = 0.96$ for a $ANN$ with fifteen neurons. 
Increasing the number of neurons did not improve the $R^2$, that even started to oscillate with the neuron number. 
We later obtained the correct number of neurons for all the $ANN$.
Further, we processed the random combinations (Table
\ref{tab:10DEMVariableinputvalues}) with the $ANN$.
The $ANN$ evaluation was incredibly faster compared to the $DEM$ simulations. The
individuation of the numerical bulk behaviours for all the $DEM$ combinations
did not take more than a few seconds on a single core.
%************************************************
\begin{figure}[!h] 
\centering 
\includegraphics[width=.96\textwidth]{22regression}
%[width=.96\textwidth]
\caption[Comparison between prediction of the trained NN and full DEM
simulation]{Comparison between prediction of the trained Artificial Neural
Network ($ANN$) and 546 full DEM simulations of the coefficient of pre-shear
($\mu_{psh}$). In this case the regression line is nearly linear (0.94), and
demonstrates the accurate prediction of the $ANN$.}
\label{fig:22regression} 
\end{figure}
%************************************************
\begin{table}[h]
\centering
\scalebox{1.0}{
\begin{tabular}{lcccccccc}
\hline
         & $\mu_s$ & $\mu_r$ & $COR$ & $\rho_p$ & $\mu_{sh}$ &
        $\mu_{psh}$ & $\rho_{b}$ & $AoR$ \\
          \hline
    $\mu_s$ & 100.00 & 0.55  & 0.04  & 0.00  & 3.84  & 87.26 & 8.39  & 49.48 \\
    $\mu_r$ & 0.55  & 100.00 & 0.15  & 0.00  & 58.92 & 33.70 & 3.10  & 60.20 \\
    $COR$ & 0.04  & 0.15  & 100.00 & 0.00  & 15.52 & 0.57  & 1.71  & 21.35 \\
    $\rho_p$ & 0.00  & 0.00  & 0.00  & 100.00 & 4.98  & 5.71  & 99.00 & 0.00 \\
    $\mu_{sh}$ & 3.84  & 58.92 & 15.52 & 4.98  & 100.00 & 26.03 & 9.52  & 0.00 \\
    $\mu_{psh}$ & \textbf{87.26} & 33.70 & 0.57  & 5.71  & 26.03 & 100.00 & 4.33 
    & 0.00
    \\
    $\rho_{b}$ & 8.39  & 3.10  & 1.71  & \textbf{99.00} & 9.52  & 4.33  & 100.00
    & 0.00 \\
    $AoR$ \hspace{5ex} & 49.48 & \textbf{60.20} & 21.35  & 0.00  & 0.00  & 0.00 
    & 0.00  & 100.00 \\
    
\hline
\end{tabular}}
\caption[Values of linear relationship between considered variables]{Values of
linear relationship between considered variables multiplied by 100. Sliding
friction ($\mu_s$), rolling friction ($\mu_r$) and particle density ($\rho_p$)
influence the most, respectively, the coefficient of pre-shear ($\mu_{psh}$),
the angle of repose  ($AoR$) and the bulk density ($\rho_b$). Notably, $\rho_p$
is not used as training parameter for the $AoR$ bulk behaviour.}
\label{tab:06inputRelationshipTable}
\end{table}
%************************************************


\subsection{Experiments and Parameter Identification}
\label{subsec:experimentsparameteridentification}

Experimental values identifying the bulk behavior, $\mu_{psh}$, $\mu_{sh}$ and $\rho_{b}$, 
for sinter fine have been acquired through the $SSC$, e.g. in Table
\ref{tab:05sinterTableExperimental}
these values for three load conditions are presented.
The $\mu_{psh}$ is clearly decreasing. 
Instead, the $\mu_{sh}$ is oscillating.
The $\rho_b$ presents a clear average of 1760 kg/m3 with a 42 
kg/m3 deviation.
The stress path for the second load condition of Table
\ref{tab:05sinterTableExperimental} is shown in Fig.
\ref{fig:20experimental}.
Later, two $AoR$ test have been performed, thus identifying an average angle of
38.85$^\circ$.
We also realized the sieving, obtaining the radius ($R$) mean and standard
deviation, already shown in Table \ref{tab:09DEMFixedinputvalues}.
The confrontation between numerical and experimental behaviours led to a first
series of marked combinations ($MC1$) for one load condition of
the shear cell ($\sigma_n=10070$ Pa, P=1.0), represented in Fig.
\ref{fig:24radarpirker1schulze10070}.
Here, the minimum and maximum values, together with the mean are shown. 
Furthermore, the shaded area represents valid parameter combinations.
Dark shaded values stand for the confidence interval, provided by the square
deviation.
Notably, the confidence interval is large, 
especially for the $COR$, highlighting its scarce influence over the characterization. 
Instead, both the $\rho_p$  and the $\mu_s$ show a narrow confidence interval, 
displaying at the same time their influence and the validity of this procedure to find valid $DEM$ parameters. 
That agrees with the examination of the ratio of the standard deviation to the
range, see Table \ref{tab:13DEMvalidvalues}.
Further, we could see how different $DEM$ parameters
combinations could reproduce the experimental behaviour and evaluate their mutual dependencies. 
This is clearer in a density plot, as in Fig. 
\ref{fig:25cloudpirker1schulze10070} for $MC1$, 
of the particles' coefficient of restitution (COR) in dependence
of coefficient of sliding friction ($\mu_s$) and coefficient of rolling
friction ($\mu_r$); in the white area no valid sets of simulation parameter can
be found.
In each cell the valid sets are grouped accordingly to the four different $COR$
ranges.
Each cell is coloured accordingly to the group with the most members. 
While the $COR$ varied, multiple
combinations (250407 or 4\%  of the total) of $\mu_s$ and $\mu_r$ reproduced
the experimental behaviour.
This underlines once more their correlation, as already stated by Wensrich and 
Katterfeld \cite{RefWorks:87}.
To further demonstrate the validity of the procedure, we modified the product
coefficient. In the first attempt we set it to $P=0.8$ and we obtained another
series of marked combinations ($MC2$).
We can see in the parameter space plot in Fig.
\ref{fig:26radarpirker08schulze10070} that the confidence interval is narrower
compared to $P=1.0$, while in the density plot in Fig. 
\ref{fig:27cloudpirker08schulze10070} the area
appears larger, although slightly less densely populated. Finally, for $P=1.2$
and its marked combinations ($MC3$) the parameter space plot in Fig.
\ref{fig:28radarpirker12schulze10070} shows a largely different confidence
range, while the density plot in Fig. \ref{fig:30cloudpirker12schulze10070} 
illustrates a smaller area. As expected, the procedure was highly sensible to the variations of the experimental data. 
Thus, it could be effectively handled for a wide range of bulk materials.\\
We then processed the random combinations with the $AoR$ $ANN$. In Fig.
\ref{fig:31radarpirker1aor} the parameter space plot realized with the same criteria as
before can be seen.
In accordance with the theory (Wensrich and Katterfeld \cite{RefWorks:87}), in a simulation dominated
by the particles rolling the coefficient of rolling friction has the maximum
influence. \\
Finally, we extracted from the $MC1$ values the $AoR$ $ANN$ behaviour
and compared it with the experimental one.
As can be seen in the parameter space plot in Fig.
\ref{fig:33radarpirker1schulze10070aor}, the confidence interval is meager, indicating that all the parameters but the $COR$ 
had an important role and demonstrating the reliability of these parameters
combinations to represent the bulk behaviour.
From the initial 6250000 combinations, only 3884 of them were valid (0.0621 \%),
see Table \ref{tab:13DEMvalidvalues}.
%************************************************
\begin{table}[h]
\centering
\begin{tabular}{cccccc}
\hline
$\sigma_n$ (Pa) & $\tau$ (Pa) & $\mu_{psh}$ (-) & $\tau_{\%}$ ($\%$) &
$\mu_{sh}$ (-) & $\rho_b$ (kg/m3) \\
\hline
    1068  & 1059  & 0.9916 & 80 & 1.2333 & 1718 \\
    2069  & 1818  & 0.8787 & 80 & 0.9994 & 1759 \\
    10070 & 8232  & 0.8175 & 80 & 1.1712 & 1802 \\

\hline
\end{tabular}
\caption[Experimental results]{Experimental results. Values for three
load conditions}
\label{tab:05sinterTableExperimental}
\end{table}
%************************************************
\begin{table}[h]
\centering
\begin{tabular}{llccc}
\hline

          & type  & SSC & AoR   & SSC \& AoR \\
          \hline

    $\mu_s$ & mean  & 0.831 & 0.177 & 0.664 \\
    $(-)$   & std. dev. (SD) & 0.097 & 0.095 & 0.029 \\
          & range ($R$) & 0.9   & 0.9   & 0.9 \\
          & SD / R & 0.108 & 0.106 & 0.032 \\
          \hline
    $\mu_r$ & mean  & 0.692 & 0.830 & 0.916 \\
    $(-)$   & std. dev. (SD) & 0.215 & 0.193 & 0.042 \\
          & range ($R$) & 0.9   & 0.9   & 0.9 \\
          & SD / R & 0.239 & 0.214 & 0.046 \\
          \hline
              COR   & mean  & 0.708 & 0.590 & 0.590 \\
   $ (-)$   & std. dev. (SD) & 0.104 & 0.073 & 0.065 \\
          & range ($R$) & 0.4   & 0.4   & 0.4 \\
          & SD / R & 0.259 & 0.183 & 0.161 \\
          \hline
    $\rho_p$ & mean  & 2245.7 & 3192.8 & 2283.9 \\
    $(kg/m3)$ & std. dev. (SD) & 80.5  & 277.4 & 67.1 \\
          & range ($R$) & 1500  & 1500  & 1500 \\
          & SD / R & 0.054 & 0.185 & 0.045 \\
          \hline
    valid & number & 290203 & 816552 & 3884 \\
    combinations & ($\%$) & 4.64  & 13.06 & 0.06 \\  

\hline
\end{tabular}
\caption[DEM valid values]{DEM valid values. For each parameter we show the
valid parameter statistics in the two tests and in their intersection.
Finally, we show the number of valid parameter combinations over the total
(6250000).}
\label{tab:13DEMvalidvalues}
\end{table}
%************************************************
\begin{figure}[htp] \centering
        \begin{subfigure}[b]{0.5\columnwidth}
        \includegraphics[width=\textwidth]{26radarpirker08schulze10070}
        \caption{Parameter space plot, $SSC$, $\sigma_n=10070 ~Pa$, $P=0.8$}
        \label{fig:26radarpirker08schulze10070} 
    \end{subfigure}\\
     \begin{subfigure}[b]{0.5\columnwidth}
        \includegraphics[width=\textwidth]{24radarpirker1schulze10070}
        \caption{Parameter space plot, $SSC$, $\sigma_n=10070 ~Pa$, $P=1.0$}
        \label{fig:24radarpirker1schulze10070}
    \end{subfigure} \\
        \begin{subfigure}[b]{0.5\columnwidth}
        \includegraphics[width=\textwidth]{28radarpirker12schulze10070}
        \caption{Parameter space plot, $SSC$, $\sigma_n=10070 ~Pa$, $P=1.2$}
        \label{fig:28radarpirker12schulze10070} 
    \end{subfigure}
    \caption[Parameter space plot of valid simulations parameters for three different
    bulk behaviours measured by SCT]{Parameter space plot of valid simulations
    parameters for three different bulk behaviours measured by shear cell tester ($SSC$).
    Each axis of the parameter space plot represents one simulation parameter.
    Furthermore, the shaded area represents valid parameter combinations.
    Dark shaded values stand for the confidence interval.
    The marked combinations for $\sigma_n=10070 ~Pa$ are here presented.
    Further explanation in
    section \ref{subsec:experimentsparameteridentification}.
   }
    \label{fig:29schulzeradarandcloud}
\end{figure}
%************************************************
\begin{figure}[htp] \centering

    \begin{subfigure}[b]{0.96\columnwidth}
        \includegraphics[width=\textwidth]{27cloudpirker08schulze10070}
        \caption{Density plot, $SSC$, $\sigma_n=10070 ~Pa$, $P=0.8$}
        \label{fig:27cloudpirker08schulze10070} 
    \end{subfigure}\\
    \begin{subfigure}[b]{0.96\columnwidth}
        \includegraphics[width=\textwidth]{25cloudpirker1schulze10070}
        \caption{Density plot, $SSC$, $\sigma_n=10070 ~Pa$, $P=1.0$}
        \label{fig:25cloudpirker1schulze10070}
    \end{subfigure}\\

    \begin{subfigure}[b]{0.96\columnwidth}
        \includegraphics[width=\textwidth]{30cloudpirker12schulze10070}
        \caption{Density plot, $SSC$, $\sigma_n=10070 ~Pa$, $P=1.2$}
        \label{fig:30cloudpirker12schulze10070} 
    \end{subfigure}
    \caption[Density plot comparison of SCT results]{Density plot comparison of
    shear cell tester ($SSC$) results. The marked combinations for $\sigma_n=10070 ~Pa$ are here presented. 
    Density plot of the particles' coefficient of restitution (COR) in dependence
	of coefficient of sliding friction ($\mu_s$) and coefficient of rolling
	friction $\mu_r$; in the white area no valid sets of simulation parameter can
	be found.
	In each cell the valid sets are grouped accordingly to the 4 different COR
	ranges.
	Each cell is colored accordingly to the group with the most members. 
    Here, the values plotted are selected between the numerical
    values from the Neural Network with initially the original experimental
    results for the $SSC$, with a product coefficient $P=1.0$ (Fig.
    \ref{fig:25cloudpirker1schulze10070}). 
        Later, they have been chosen with  
    a virtually lower shear stress ($P=0.8$)
    (\ref{fig:27cloudpirker08schulze10070}).
    The last image (Fig. \ref{fig:30cloudpirker12schulze10070}) represents
    instead the selection with a virtually higher shear stress ($P=1.2$).    }
    \label{fig:29schulzeradarandcloud}
\end{figure}
%************************************************
\begin{figure}[htp] \centering
    \begin{subfigure}[b]{0.96\columnwidth}
        \includegraphics[width=\textwidth]{31radarpirker1aor}
        \caption{Parameter space plot, $AoR_{exp} = 38.85 ^\circ$}
        \label{fig:31radarpirker1aor} 
    \end{subfigure}\\
        \begin{subfigure}[b]{0.96\columnwidth}
        \includegraphics[width=\textwidth]{33radarpirker1schulze10070aor}
        \caption{Parameter space plot, $AoR_{exp} = 38.85
        ^\circ$ \& $SSC$: $\sigma_n=10070 ~Pa$}
        \label{fig:33radarpirker1schulze10070aor} 
    \end{subfigure}
    \caption[Parameter space plot of valid simulations parameters for the AoR and
    the merge between AoR and SCT valid parameters]{Parameter space plot of
    valid simulations parameters for the angle of repose tester ($AoR$) and the merge
    between AoR and shear cell tester ($SSC$).
    Each axis of the parameter space plot represents one simulation parameter.
    Furthermore, the shaded area represents valid parameter combinations.
    Dark shaded values stand for the confidence interval.
    Further explanation in
    section \ref{subsec:experimentsparameteridentification}. }
    \label{fig:35schulze10070aorradarandcloud}
\end{figure}
%************************************************
%************************************************
\section{Conclusions}
\label{sec:conclusions}
%************************************************
We presented a two phase methodology for DEM simulation parameter
identification. In a first phase an artificial neural network is 
trained by dedicated DEM simulation in order to predict bulk 
behaviour in dependence on a set of DEM simulation parameters. 
In a second phase this artificial neural network is then utilized 
to predict the bulk behaviour of a huge number of additional DEM parameter sets. 
The main findings of this study can be summarized as:
\begin{itemize}
  \item{An artificial neural network can be trained by a limited number of dedicated DEM simulations. 
  		Subsequently, the trained artificial neural network is able to predict
  		granular bulk behaviour.}
  \item{This prediction of granular bulk behaviour is highly efficient if
  		compared to computationally expensive DEM simulations.
  		Thus, a huge number of parameter sets can be studied with respect to their
  		macroscopic output.}
  \item{If the predictions of the artificial neural network are compared to a bulk experiment, 
  		valid sets of DEM simulation parameters can be readily deduced for a
  		specific granular material.}
  \item{This DEM parameter identification methodology can be applied to arbitrary bulk experiments. 
  		Combining two artificial neural networks, which predict two different bulk
  		behaviours, leads to focusing the set of valid DEM simulation parameters.}
\end{itemize}
In future we will further develop this methodology by considering different
fractions of granular materials, which will lead to size dependent sets of DEM simulation parameters.
%************************************************

\section{Acknowledgements}
This study was funded by Christian Doppler Forschungsgesellschaft, Siemens VAI Metals Technologies and Voestalpine Stahl. The authors gratefully aknowledge their support.
 
%%%%%%%%%%%%%%%%%%%%%%%%%%%%%%%%%%%%%%%%%%%%%%%%%%%%%%%%%%%%%%%%%%%%%%%%%%%%%%%%%%%%%%%%%%%%%%%%%%%%%%%%%%%%%%%%%%%%%%%%%%
\bibliographystyle{elsarticle-num}
%\bibliography{Bibliografia}

\begin{thebibliography}{10}
\expandafter\ifx\csname url\endcsname\relax
  \def\url#1{\texttt{#1}}\fi
\expandafter\ifx\csname urlprefix\endcsname\relax\def\urlprefix{URL }\fi
\expandafter\ifx\csname href\endcsname\relax
  \def\href#1#2{#2} \def\path#1{#1}\fi

\bibitem{RefWorks:130}
P.~W. Cleary, M.~L. Sawley, DEM modelling of industrial granular flows: 3d case
  studies and the effect of particle shape on hopper discharge, Applied
  Mathematical Modelling 26~(2) (2002) 89--111.

\bibitem{RefWorks:172}
P.~A. Cundall, O.~D.~L. Strack, A discrete numerical model for granular
  assemblies, Geotechnique 29~(Volume 29, Issue 1) (1979) 47--65(18).

\bibitem{RefWorks:148}
L.~Vu-Quoc, X.~Zhang, An accurate and efficient tangential force-displacement
  model for elastic frictional contact in particle-flow simulations, Mechanics
  of Materials 31~(4) (1999) 235--269.

\bibitem{RefWorks:145}
A.~D. Renzo, F.~P.~D. Maio, Comparison of contactforce models for the
  simulation of collisions in DEM-based granular flow codes, Chemical
  Engineering Science 59~(3) (2004) 525--541.

\bibitem{RefWorks:87}
C.~M. Wensrich, A.~Katterfeld, Rolling friction as a technique for modelling
  particle shape in DEM, Powder Technology 217~(0) (2012) 409--417.

\bibitem{RefWorks:136}
C.~Kloss, C.~Goniva, A.~Hager, S.~Amberger, S.~Pirker, Models, algorithms and
  validation for opensource DEM and CFDDEM, Progress in Computational Fluid
  Dynamics, an International Journal 12~(2) (2012) 140--152.

\bibitem{RefWorks:139}
A.~Aigner, S.~Schneiderbauer, C.~Kloss, S.~Pirker, Determining the coefficient
  of friction by shear tester simulation, 3rd International Conference on
  Particle-Based Methods (2013) 335--342.

\bibitem{RefWorks:86}
D.~Hohner, S.~Wirtz, V.~Scherer, A numerical study on the influence of particle
  shape on hopper discharge within the polyhedral and multi-sphere discrete
  element method, Powder Technology 226~(0) (2012) 16--28.

\bibitem{RefWorks:131}
J.~Ai, J.-F. Chen, J.~M. Rotter, J.~Y. Ooi, Assessment of rolling resistance
  models in discrete element simulations, Powder Technology 206~(3) (2011)
  269--282.

\bibitem{RefWorks:177}
M.~Combarros, H.~J. Feise, H.~Zetzener, A.~Kwade, Segregation of particulate
  solids: Experiments and DEM simulations, Particuology 12~(0) (2014) 25--32.

\bibitem{RefWorks:91}
A.~Alenzi, M.~Marinack, C.~F. Higgs, J.~J. McCarthy, DEM validation using an
  annular shear cell, Powder Technology 248~(0) (2013) 131--142.

\bibitem{RefWorks:150}
B.~Vaferi, F.~Samimi, E.~Pakgohar, D.~Mowla, Artificial neural network approach
  for prediction of thermal behavior of nanofluids flowing through circular
  tubes, Powder Technology 267~(0) (2014) 1--10.

\bibitem{RefWorks:174}
I.~Witten, E.~Frank, M.~Hall, Data Mining: Practical machine learning tools and
  techniques, Morgan Kaufmann, 2005.

\bibitem{RefWorks:158}
S.~Haykin, Neural Networks and Learning Machines, no. v. 10, Prentice Hall,
  2009, 2008034079.

\bibitem{RefWorks:175}
N.~Tsafnat, N.~Amanat, A.~S. Jones, Analysis of coke under compressive loading:
  A combined approach using microcomputed tomography, finite element analysis,
  and empirical models of porous structures, Fuel 90~(1) (2011) 384--388.

\bibitem{RefWorks:176}
J.~Kovacik, Correlation between young modulus and porosity in porous materials,
  Journal of Material Science 18 (1999) 1007--1010.

\bibitem{RefWorks:143}
M.~J. Jiang, H.~S. Yu, D.~Harris, A novel discrete model for granular material
  incorporating rolling resistance, Computers and Geotechnics 32~(5) (2005)
  340--357.

\bibitem{RefWorks:160}
W.~L. Oberkampf, C.~J. Roy, Verification and Validation in Scientific
  Computing, Cambridge University Press, 2010.

\bibitem{RefWorks:118}
D.~Schulze, Powders and bulk solids: behavior, characterization, storage and
  flow, Springer, 2008.

\bibitem{RefWorks:173}
L.~Benvenuti, A.~Aigner, D.~Queteschiner, S.~Pirker, C.~Kloss, Establishing the
  predictive capabilities of DEM simulations: sliding and rolling friction
  coefficients of non-spherical particles, in: CFD 2014 Proceedings, 2014.

\bibitem{RefWorks:161}
S.~J. Antony, C.~H. Zhou, X.~Wang, An integrated mechanistic-neural network
  modelling for granular systems, Applied Mathematical Modelling 30~(1) (2006)
  116--128.

\bibitem{RefWorks:166}
A.-K. Seghouane, Y.~Moudden, G.~Fleury, Regularizing the effect of input noise
  injection in feedforward neural networks training, Neural Computing and
  Applications 13~(3) (2004) 248--254, j2: Neural Comput and Applic.

\bibitem{RefWorks:167}
D.~Barrasso, A.~Tamrakar, R.~Ramachandran, A reduced order pbm-ann model of a
  multi-scale PBM-DEM description of a wet granulation process, Chemical
  Engineering Science~(0).

\bibitem{RefWorks:168}
T.~Varol, A.~Canakci, S.~Ozsahin, Artificial neural network modeling to effect
  of reinforcement properties on the physical and mechanical properties of
  al2024-b4c composites produced by powder metallurgy, Composites Part B:
  Engineering 54~(0) (2013) 224--233.

\bibitem{RefWorks:169}
A.~Canakci, S.~Ozsahin, T.~Varol, Modeling the influence of a process control
  agent on the properties of metal matrix composite powders using artificial
  neural networks, Powder Technology 228~(0) (2012) 26--35.

\bibitem{RefWorks:170}
M.~Lashkarbolooki, B.~Vaferi, M.~R. Rahimpour, Comparison the capability of
  artificial neural network (ann) and eos for prediction of solid solubilities
  in supercritical carbon dioxide, Fluid Phase Equilibria 308~(1-2) (2011)
  35--43.

\bibitem{RefWorks:178}
B.~Das, U.~P. Ganguly, N.~Bar, S.~K. Das, Holdup prediction in inverse
  fluidization using non newtonian pseudoplastic liquids: Empirical correlation
  and ann modeling, Powder Technology 273~(0) (2015) 83--90.

\bibitem{RefWorks:179}
S.~M. Turp, B.~Eren, A.~Ates, Prediction of adsorption efficiency for the
  removal of nickel(ii) ions by zeolite using artificial neural network(ann)
  approach, Fresenius Environmental Bulletin 20~(12) (2011) 3158--3165.

\end{thebibliography}

%%%%%%%%%%%%%%%%%%%%%%%%%%%%%%%%%%%%%%%%%%%%%%%%%%%%%%%%%%%%%%%%%%%%%%%%%%%%%%%%%%%%%%%%%%%%%%%%%%%%%%%%%%%%%%%%%%%%%%%%%%

\newpage
%%%%%%%%%%%%%%%%%%%%%%%%%%%%%%%%%%%%%%%%%%%%%%%%%%%%%%%%%%%%%%%%%%%%%%%%%%%%%%%%%%%%%%%%%%%%%%%%%%%%%%%%%%%%%%%%%%%%%%%%%%
\begin{appendix}
\label{appendix}

%************************************************
\section{Simulations}
\label{sec:appsimulations}

\subsection{SRSCT simulation}
\label{subsec:srsctsimulation}
For each particle i inside the domain a Discrete Element Method ($DEM$) code
follows the trajectory and calculates the force that particle i exerts on particle j.
The main forces involved are: gravity, contact forces due to collisions and further interactions such as electrostatic, 
Van der Waals, cohesive forces and fluid-solid interactions in multiphase flows. For the raw material used in this work 
Di Renzo and Di Maio \cite{RefWorks:145} suggested using the non-linear Hertzian model without cohesion for 
the particle-particle and particle-wall contacts. 
This granular model uses the following formula for the contact force between two granular particles (Eq. \ref{eq:forceij}):
%************************************************
\begin{equation}
 F_{ij} = 
\begin{cases}
F_{n,ij} + F_{t,ij} = \left( k_n \delta_{n,ij} + \gamma_n v_{n,ij} \right) + \left( k_t \delta_{t,ij} + \gamma_t v_{t,ij} \right) & \text{if } r < d ,\\
0    & \text{if } r > d ,\\
\end{cases}
 \label{eq:forceij}
\end{equation}
%************************************************
where the subscript $n$ stands for normal and $t$ for tangential. 
Here, $k$ and $\gamma$ are respectively the stiffness and damping coefficients, 
while $\delta$ and $v$ are the displacement and the velocity, $r$ is the
distance between two particles of radii $R_i$ and $R_j$ and $d = R_i + R_j $ is the
contact distance.
Both the normal and the tangential
force comprise two terms, a spring force and a damping force. 
The tangential (or shear) force is a "history" effect that accounts for the
tangential displacement ("tangential overlap") between the particles for the duration of contact. 
In the work Wensrich and Katterfeld \cite{RefWorks:87} further details on the
method can be found.\\
$LIGGGHTS$, the simulation toolbox we used, meets all the requirements of
modelling the shear tester described in the \ref{subsec:srsctexperiment}. 
First, it is capable of importing triangulated meshes of the two rings and a top lid. 
Since the real setup had a wall thickness, contact forces acting on a mesh are summed and can be saved, 
and thus shear force calculation is available out of the box. Moreover, the code can move a mesh with constant 
velocity as required for the measurement. To determine the shear stresses, the bulk solid had to be stressed with 
user-defined normal stresses. Therefore, a stress-controlled wall ($servo-wall$ in $LIGGGHTS$) was applied to the lid. \\
Although the geometry differs, the $SSC$ was designed to obtain the same values for the shear stresses as the 
Jenike shear cell tester ($JSCT$), but with improved automation and reliability,
see Schulze \cite{RefWorks:118}. 
For this reason, the simulation setup has been
based over the $JSCT$.
As suggested by Aigner et al. \cite{RefWorks:139} and Benvenuti et al. \cite{RefWorks:173}, 
the diameter and the height of the rings operated in the simulations had to be sufficiently large to avoid relevant wall effects. 
Nevertheless, a larger domain increases the number of particles, and thus the simulation time. 
For this reason, we considered the cylinder dimension, as proportion to the mean particle diameter ($dCylDp$), 
an additional $DEM$ parameter to be investigated. \\   
A simulation run comprised four phases. 
First, the shear cell was filled with the granulate material, and it was allowed
settling.
Then, the top lid was lowered and applied the first normal stress to the bulk solid. 
As in the experiment, the servo-wall allows calculating the position of the lid
while the first particle is touched. 
The distance between the lid and the bottom of the domain, multiplied by the 
simulation area, gave the total volume.
Since the software already provided the total mass, we were able to calculate:
%************************************************
\begin{equation}
\rho_b = \frac{mass}{volume}.
 \label{eq:rhob}
\end{equation}
%************************************************
Next, the ring moved for a distance $l=0.1875 \cdot radius ~of ~the ~ring$, and the required pre-shear force was measured. 
Finally, the normal load was reduced to a fraction of the initial load, 
the ring was moved again by a distance $l$, and the shear force was recorded. 
Unlike in the original experiment, the bottom ring was moved to facilitate the numerical simulation. 
The velocity of the ring displacement, and consequently the total simulation time, are determined 
through a challenging equilibrium between the downgrading of normal load oscillation and the computational time containment. 
The former is obtained by low (relatively) velocity, the latter by high speed. We imposed a constant velocity 
of $3*(mean-particle-radius)/seconds$, the most suitable to satisfy the two competing requests. \\
The normal stresses (pre-shear and shear phases) applied in each simulation were
the same as in the experiments, see \ref{subsec:srsctexperiment}.
The corresponding $\tau_{psh}$ and $\tau_{sh}$ were calculated - as in the experiments - from the mean of the plateau.\\

%************************************************
\subsection{AoR simulation}
\label{subsec:aorsimulation}
In $AoR$ simulations we tried to replicate meticulously the experimental setup, 
considering both the plate and the lift-able boundary, with the same domain size consideration as before. 
The particles had the same properties as in the shear cell simulation. The first phase was identical to that of the shear cell simulation. 
After lifting the boundary, the particles formed a heap.
An image post-processing software was used to obtain the average slope.


%************************************************
\section{Artificial Neural Networks}
\label{sec:appann}

An Artificial Neural Network ($ANN$) is a powerful modellization technique, 
based on non-linear functions (Haykin \cite{RefWorks:158}). 
In this paper, we first use the $ANN$ to fit the $DEM$ numerical simulation data, 
and then to process vast amount of parameters combinations. 
They map combinations of input data into convenient outputs (fitting). 
There is a variety of types of $ANN$, remarkably the Feedforward ($FF$) 
and the Radial basis function ($RBF$). For $FF-NN$, considerable amount 
of training algorithms are available. The most common are based on
backpropagation:
e.g. Levenberg-Marquardt, Bayesian regulation and scaled conjugate gradient. 
To recognize not linearly separable data the standard linear perceptron $ANN$ 
has been modified into \textit{FF Multilayer Perceptron Neural Networks (MLPNN)}. 
Here, each processing unit or node (neuron) possesses a nonlinear activation function. 
Together, they are interconnected into layers, also linked together. 
The validity of the $MLPNN$, with a backpropagation reinforcement learning 
training algorithm (scaled conjugate gradient), has been widely demonstrated in the 
literature, see Haykin \cite{RefWorks:158}. Several scientists 
\cite{RefWorks:161, RefWorks:166, RefWorks:167, RefWorks:168, RefWorks:169,
RefWorks:170, RefWorks:178, RefWorks:179} have used $ANN$ to model materials'
mechanical properties.
Following the best practice suggested by Vaferi et al. \cite{RefWorks:150} $MLPNN$ have been handled.
Further, we should question the quality of the $ANN$ data. Haykin \cite{RefWorks:158} 
suggests questioning both the $ANN$ training process and the following data
generation from provided inputs.
The former is usually challenged when dealing with experimental training data, and frequently 
managed by noise-corrupted patterns calibration. Nevertheless, our training pool
was numerical.
The particles in each of our simulations were inserted through a random
seed value, and the training pool was extensive.
For massive training data the effect of noise-corrupted patterns is negligible, see Haykin \cite{RefWorks:158}. 
Instead the latter was a challenging aspect of our work. Once trained, as input for the $ANN$ we imposed 
combinations of $DEM$ parameters. 
We tried different methods to generate these combinations. 
Our first attempt was assigning to the investigated variables parameters in even increments 
from the minimum to the maximum values. 
E.g. the $COR$ ranges from 0.5 to 0.9, the first value would be 0.5, the second 0.508163 and so on. 
To increase the generalization, we decided to follow a different approach. 
Random values generators created values in the defined ranges and in the requested 
number for each of the investigated parameter. Then, they were combined and imposed as input.\\

%************************************************
\section{Experiments}
\label{sec:appexperiments}

\subsection{SRSCT experiment}
\label{subsec:srsctexperiment}
A representative sample of bulk solid was placed in a shear cell of specified
dimensions ($external ~ radius = 100 ~ mm$, $internal ~ radius = 50 ~ mm$).
A normal load was applied to the cover. As soon as the lid touches the sample, its position is calculated.
Together with the area of the ring, the total volume can be calculated, and subsequently the $bulk ~ density ~ (\rho_b)$ 
of the sample is obtained, the first value representative of the bulk behaviour.
Then the specimen was pre-sheared until a steady-state shear value was reached.
The steady-state flow horizontal stress
is called $steady-state-flow/pre-shear$ stress.
Knowing the normal stress, it provides (Eq. \ref{eq:phi_ps}) the angle of
internal friction of the pre-shear phase ($\phi_{e-psh}$), and consequently the
$pre-shear-coefficient-of-internal-friction $ $ (\mu_{psh})$, the second
behaviour value, see Schulze \cite{RefWorks:118}:
%************************************************
\begin{equation}
\begin{aligned}
\phi_{e-psh} &= \arctan \left(\frac{\tau_{psh}}{\sigma_{n,psh}} \right) ,\\
\mu_{psh} &=\tan(\phi_{e-psh}) .
\end{aligned}
 \label{eq:phi_ps}
\end{equation}
%************************************************
The normal stress and the angular velocity were then immediately reduced to zero. 
Subsequently, the specimen was sheared under a fraction ($shear-perc$) of the first normal load until the shear force 
reached a maximum and began to decrease. 
Both the pre-shear and shear phases were executed at constant velocity. 
We define the horizontal stress during the shear force peak as maximum shear stress, 
thus obtaining the $incipient-flow/shear ~ coefficient-of-internal-friction $ $
(\mu_{sh})$, third behaviour value (Eq. \ref{eq:phi_s})\cite{RefWorks:118}:
%************************************************
\begin{equation}
\begin{aligned}
\phi_{e-sh} &= \arctan \left(\frac{\tau_{sh}}{\sigma_{n,sh}} \right) ,\\
\mu_{sh} &= \tan(\phi_{e-sh}) .
\end{aligned}
 \label{eq:phi_s}
\end{equation}
%************************************************
Three different pre-shear normal loads were applied in the experiment
(1000, 2000, and 10000 Pa).
For each we used a normal load proportional to the initial one ($shear-perc$) increasing from stage one (40\%) 
to stage four (100\%) with two escalating intermediate stages (60\% and 80\%),
for a total of twelve load conditions.
Each experiment was performed on a fresh material sample. \\

%************************************************
\subsection{AoR experiment}
\label{subsec:aorexperiment}
A sample was deposited on a 20 cm diameter plate with lift-able boundary, called
static angle of repose ($AoR$) tester.
Once the particles were in position, the boundary was lifted, allowing some particles to drop. 
Once stabilized, the $AoR$ was measured eight times using a digital protractor at different positions of the heap. 
The result is produced as the average of the measurements, granting the fourth
behaviour value.
Notably, the experiments were performed only for larger size bulk solids. 
So, the compaction condition in the initial state was not critical to the final result.

\end{appendix}
%%%%%%%%%%%%%%%%%%%%%%%%%%%%%%%%%%%%%%%%%%%%%%%%%%%%%%%%%%%%%%%%%%%%%%%%%%%%%%%%%%%%%%%%%%%%%%%%%%%%%%%%%%%%%%%%%%%%%%%%%%

\renewcommand\thefigure{\arabic{figure}}

\end{document}
