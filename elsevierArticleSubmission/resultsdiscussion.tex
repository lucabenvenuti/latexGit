\section{Results and discussion}
\label{sec:results}
%************************************************



\subsection{DEM Simulations}
\label{subsec:simulations}

For sinter fine 546 shear cell and 81 static angle of repose simulations have
been realized with the variations described in table
\ref{tab:10DEMVariableinputvalues}.
The computational time resulted in 1 hour with 32 AMD cores for a benchmark
shear cell simulation and 9 hours for a benchmark $AoR$ simulation, both with 50K particles. 
Simulations with larger $dCylDp$ required a greater time amount (e.g. with 400K
particles about 12 hours for the shear cell). \\


\subsection{ANN model development}
\label{subsec:annmodeldev}

First, we controlled the regression of the bulk behaviour parameters, e.g. the
$\mu_{psh}$, see Fig. \ref{fig:22regression}, where the corresponding plot for
the $ANN$ with the maximum $R^2$ is shown. Each circle represents one of the 546
simulations.
The plot presents a consistent agreement between the $DEM$ results distribution
(T in the legend) and the $ANN$ regression (or fitting) line.
The linear relationship between the
training values have been evaluated in Table \ref{tab:06inputRelationshipTable}.
The clearest connections were between $\mu_s$ and $\mu_{psh}$, and
$\rho_p$ and $\rho_b$.
Instead, for $\mu_{sh}$ and $AoR$ the $\mu_r$ balanced the influence of the 
$\mu_s$, and further parameters were worthly correlated. \\
Then we observed how the $R^2$ changed with the different number of neurons for the $\mu_{psh}$. 
In this case we reached a $R^2 = 0.96$ for a $ANN$ with fifteen neurons. 
Increasing the number of neurons did not improve the $R^2$, that even started to oscillate with the neuron number. 
We later obtained the correct number of neurons for all the $ANN$.
Further, we processed the random combinations (table
\ref{tab:10DEMVariableinputvalues}) with the $ANN$.
The $ANN$ evaluation was incredibly faster compared to the $DEM$ simulations. The
individuation of the numerical bulk behaviours for all the $DEM$ combinations
did not take more than a few seconds on a single core.
\begin{figure}[!h] 
\centering 
\includegraphics[width=.96\textwidth]{22regression}
%[width=.96\textwidth]
\caption[Comparison between prediction of the trained NN and full DEM
simulation]{Comparison between prediction of the trained Neural Network ($NN$)
and 546 full DEM simulations of the coefficient of pre-shear ($\mu_{psh}$). In
this case the regression line is nearly linear, and demonstrates the accurate
prediction of the $NN$.}
\label{fig:22regression} 
\end{figure}

% Regression plot - $\mu_{psh}$. In this case the
% $Output = 0.96 \cdot Target + 0.046$. With 546 simulations the $R^2 = 0.98044$. The plot
% presents a consistent agreement between the $DEM$ results distribution and the $NN$ regression line.
% \begin{figure}[htp]
%     \centering
%     \includegraphics[width=.2\textwidth]{images/vitae/lbenvenuti}
%     \caption{OpenMP, MPI, MPI/OpenMP Hybrid runs of Box in a box testcase on 32
%     cores. The OpenMP-only run suffers from limited memory bandwidth in
%     memory-bound algorithms inside of the Modify section of the code. MPI-only has
%     low averaged runtimes for each section, but a very large Other timing, which
%     hints for a large amount of load-imbalance. Hybrid timings are a bit worse
%     on average, but because of better balancing, processes have lower wait times
%     inside of Other timing.}
% 	\label{fig:boxInBoxComparison}

\begin{table}[H!]                                                                                                                                                          
\centering                                                                                                                                                                 
\begin{tabular}{|c|c|c|c|c|c|c|c|c|c|c|c|}                                                                                                                                 
\hline                                                                                                                                                                     
 & sf & rf & rest & dt & dCylDp & ctrlStress & shearperc & dens & mush & mupsh & rhob \\                                                                                   
\hline                                                                                                                                                                     
sf & 1 & 5.549787e-03 & -3.818461e-04 & -1.268763e-15 & -1.628657e-02 & 1.282025e-15 & 4.517397e-03 & 0 & 3.838826e-02 & 8.725701e-01 & -8.393464e-02 \\                   
\hline                                                                                                                                                                     
rf & 5.549787e-03 & 1 & -1.523330e-03 & -2.349289e-15 & -5.968531e-02 & 2.322503e-15 & 1.802162e-02 & 3.348007e-18 & 5.891756e-01 & 3.370233e-01 & -3.101856e-02 \\        
\hline                                                                                                                                                                     
rest & -3.818461e-04 & -1.523330e-03 & 1 & -1.555718e-15 & -2.758674e-01 & 1.568359e-15 & 8.090707e-02 & 6.680307e-18 & 1.551852e-01 & -5.671687e-03 & -1.712429e-02 \\    
\hline                                                                                                                                                                     
dt & -1.268763e-15 & -2.349289e-15 & -1.555718e-15 & 1 & -1.026312e-16 & -1.000000e+00 & -2.681936e-17 & 0 & 6.168810e-16 & -4.320958e-15 & 1.243669e-14 \\                
\hline                                                                                                                                                                     
dCylDp & -1.628657e-02 & -5.968531e-02 & -2.758674e-01 & -1.026312e-16 & 1 & 7.853515e-17 & -2.939311e-01 & 2.688281e-17 & -2.879551e-01 & -1.916393e-01 & 9.603603e-02 \\ 
\hline                                                                                                                                                                     
ctrlStress & 1.282025e-15 & 2.322503e-15 & 1.568359e-15 & -1 & 7.853515e-17 & 1 & -3.731389e-17 & 0 & -6.100950e-16 & 4.292811e-15 & -1.234126e-14 \\                      
\hline                                                                                                                                                                     
shearperc & 4.517397e-03 & 1.802162e-02 & 8.090707e-02 & -2.681936e-17 & -2.939311e-01 & -3.731389e-17 & 1 & -3.512479e-17 & 5.730199e-02 & 5.380657e-02 & -5.095294e-03 \\
\hline                                                                                                                                                                     
dens & 0 & 3.348007e-18 & 6.680307e-18 & 0 & 2.688281e-17 & 0 & -3.512479e-17 & 1 & -4.980664e-02 & 5.709445e-02 & 9.900341e-01 \\                                         
\hline                                                                                                                                                                     
mush & 3.838826e-02 & 5.891756e-01 & 1.551852e-01 & 6.168810e-16 & -2.879551e-01 & -6.100950e-16 & 5.730199e-02 & -4.980664e-02 & 1 & 2.603411e-01 & -9.516313e-02 \\      
\hline                                                                                                                                                                     
mupsh & 8.725701e-01 & 3.370233e-01 & -5.671687e-03 & -4.320958e-15 & -1.916393e-01 & 4.292811e-15 & 5.380657e-02 & 5.709445e-02 & 2.603411e-01 & 1 & -4.329071e-02 \\     
\hline                                                                                                                                                                     
rhob & -8.393464e-02 & -3.101856e-02 & -1.712429e-02 & 1.243669e-14 & 9.603603e-02 & -1.234126e-14 & -5.095294e-03 & 9.900341e-01 & -9.516313e-02 & -4.329071e-02 & 1 \\   
\hline                                                                                                                                                                     
\end{tabular}                                                                                                                                                              
\caption{MyTableCaption}                                                                                                                                                   
\label{table:MyTableLabel}                                                                                                                                                 
\end{table}               


\subsection{Experiments and Parameter Identification}
\label{subsec:experimentsparameteridentification}

Experimental values identifying the bulk behavior, $\mu_{psh}$, $\mu_{sh}$ and $\rho_{b}$, 
for sinter fine have been acquired through the $SSC$, e.g. in Table
\ref{tab:05sinterTableExperimental}
these values for three load conditions are presented.
The $\mu_{psh}$ is clearly decreasing. 
Instead, the $\mu_{sh}$ is oscillating.
The $\rho_b$ presents a clear average of $1760 ~ [kg/m^3]$ with a $42 ~
[kg/m^3]$ deviation.
The stress path for the second load condition of Table
\ref{tab:05sinterTableExperimental} is shown in Fig.
\ref{fig:20experimental}.

Later, two $AoR$ test have been performed, thus identifying an average angle of
$38.85 ^\circ$.
We also realized the sieving, obtaining the radius ($R$) mean and standard
deviation, already shown in Table \ref{tab:09DEMFixedinputvalues}.

The confrontation between numerical and experimental behaviours led to a first
series of marked combinations ($MC1$) for one load condition of
the shear cell ($\sigma_n=10070 ~[Pa]$, $P=1.0$), represented in Fig.
\ref{fig:24radarpirker1schulze10070}.
Here, the minimum and maximum values, together with the mean are shown. 
Furthermore, the shaded area represents valid parameters combinations.
Dark shaded values stand for the confidence range, provided by the square
deviation.
Notably, the confidence range is large, 
especially for the $COR$, highlighting its scarce influence over the characterization. 
Instead, both the $\rho_p$  and the $\mu_s$ show a narrow confidence range, 
displaying at the same time their influence and the validity of this procedure to find valid $DEM$ parameters. 
That agrees with the examination of the ratio of the standard deviation to the
range, see table \ref{tab:13DEMvalidvalues}.
Further, we could see how different $DEM$ parameters
combinations could reproduce the experimental behaviour and evaluate their mutual dependencies. 
This is clearer in a density plot, as in Fig. 
\ref{fig:25cloudpirker1schulze10070} for $MC1$, 
of the particles' coefficient of restitution (COR) in dependence
of coefficient of sliding friction and coefficient of rolling friction; in the
white area no valid sets of simulation parameter can be found.
In each cell the valid sets are grouped accordingly to the four different $COR$
ranges.
Each cell is coloured accordingly to the group with the most members. 
While the $COR$ varied, multiple
combinations ($250407 --> 4\% $ of the total) of $\mu_s$ and $\mu_r$ reproduced
the experimental behaviour.
This underlines once more their correlation, as already stated by Wensrich and 
Katterfeld \cite{RefWorks:87}.
To further demonstrate the validity of the procedure, we modified the product
coefficient. In the first attempt we set it to $P=0.8$ and we obtained another
series of marked combinations ($MC2$).
We can see in the parameter space plot in Fig.
\ref{fig:26radarpirker08schulze10070} that the confidence range is narrower
compared to $P=1.0$, while in the density plot in Fig. 
\ref{fig:27cloudpirker08schulze10070} the area
appears larger, although slightly less densely populated. Finally, for $P=1.2$
and its marked combinations ($MC3$) the parameter space plot in Fig.
\ref{fig:28radarpirker12schulze10070} shows a largely different confidence
range, while the density plot in Fig. \ref{fig:30cloudpirker12schulze10070} 
illustrates a smaller area. As expected, the procedure was highly sensible to the variations of the experimental data. 
Thus, it could be effectively handled for a wide range of bulk materials.\\
We then processed the random combinations with the $AoR$ $ANN$. In Fig.
\ref{fig:31radarpirker1aor} the parameter space plot realized with the same criteria as
before can be seen.
In accordance with the theory (Wensrich and Katterfeld \cite{RefWorks:87}), in a simulation dominated
by the particles rolling the coefficient of rolling friction has the maximum
influence. \\
Finally, we extracted from the $MC1$ values the $AoR$ $ANN$ behaviour
and compared it with the experimental one.
As can be seen in the parameter space plot in Fig.
\ref{fig:33radarpirker1schulze10070aor}, the confidence range is meager, indicating that all the parameters but the $COR$ 
had an important role and demonstrating the reliability of these parameters
combinations to represent the bulk behaviour.
From the initial 6250000 combinations, only 3884 of them were valid (0.0621 \%),
see table \ref{tab:13DEMvalidvalues}.
\begin{table}[h]
\centering
\begin{tabular}{cccccc}
$\sigma_n$ [Pa] & $\tau$ [Pa] & $\mu_{psh}$ [-] & $\mu_{sh}$ [-] &
$\rho_b$ [kg/m3] & AOR $\circ$ \\
\hline
    1068  & 1059  & 0.9916 & 0.9916 & 1718  & 38.85 \\
    2069  & 1818  & 0.8787 & 0.8787 & 1759  & 38.85 \\
    10070 & 8232  & 0.8175 & 0.8175 & 1802  & 38.85 \\

\hline
\end{tabular}
\caption{Experimental values for sinter fine}
\label{tab:05sinterTableExperimental}
\end{table}
\begin{table}[h]
\centering
\scalebox{0.7}{
\begin{tabular}{llccc}
\hline

          & type  & SCT & AOR   & SCT \& AOR \\
          \hline

    $\mu_s$ & mean  & 0.831 & 0.177 & 0.664 \\
    $[-]$   & std. dev. (SD) & 0.097 & 0.095 & 0.029 \\
          & range ($R$) & 0.9   & 0.9   & 0.9 \\
          & SD / R & 0.108 & 0.106 & 0.032 \\
          \hline
    $\mu_r$ & mean  & 0.692 & 0.830 & 0.916 \\
    $[-]$   & std. dev. (SD) & 0.215 & 0.193 & 0.042 \\
          & range ($R$) & 0.9   & 0.9   & 0.9 \\
          & SD / R & 0.239 & 0.214 & 0.046 \\
          \hline
              COR   & mean  & 0.708 & 0.590 & 0.590 \\
   $ [-]$   & std. dev. (SD) & 0.104 & 0.073 & 0.065 \\
          & range ($R$) & 0.4   & 0.4   & 0.4 \\
          & SD / R & 0.259 & 0.183 & 0.161 \\
          \hline
    $\rho_p$ & mean  & 2245.7 & 3192.8 & 2283.9 \\
    $[kg/m3]$ & std. dev. (SD) & 80.5  & 277.4 & 67.1 \\
          & range ($R$) & 1500  & 1500  & 1500 \\
          & SD / R & 0.054 & 0.185 & 0.045 \\
          \hline
    valid & number & 290203 & 816552 & 3884 \\
    combinations & [$\%$] & 4.64  & 13.06 & 0.06 \\
    

\hline
\end{tabular}}
\caption{DEM valid values.}
\label{tab:13DEMvalidvalues}
\end{table}
\begin{figure}[htp] \centering
        \begin{subfigure}[b]{0.5\columnwidth}
        \includegraphics[width=\textwidth]{26radarpirker08schulze10070}
        \caption{Parameter space plot, $SSC$, $\sigma_n=10070 ~[Pa]$, $P=0.8$}
        \label{fig:26radarpirker08schulze10070} 
    \end{subfigure}\\
     \begin{subfigure}[b]{0.5\columnwidth}
        \includegraphics[width=\textwidth]{24radarpirker1schulze10070}
        \caption{Parameter space plot, $SSC$, $\sigma_n=10070 ~[Pa]$, $P=1.0$}
        \label{fig:24radarpirker1schulze10070}
    \end{subfigure} \\
        \begin{subfigure}[b]{0.5\columnwidth}
        \includegraphics[width=\textwidth]{28radarpirker12schulze10070}
        \caption{Parameter space plot, $SSC$, $\sigma_n=10070 ~[Pa]$, $P=1.2$}
        \label{fig:28radarpirker12schulze10070} 
    \end{subfigure}
    \caption[Parameter space plot of valid simulations parameters for three different
    bulk behaviours measured by SCT]{Parameter space plot of valid simulations
    parameters for three different bulk behaviours measured by shear cell tester ($SSC$).
    Each axis of the parameter space plot represents one simulation parameters.
    Furthermore, the shaded area represents valid parameters combinations.
    Dark shaded values stand for the confidence range.
    We represent the marked combinations for one load condition of the shear
    cell.
    Further explanation in the text.
   }
    \label{fig:29schulzeradarandcloud}
\end{figure}
\begin{figure}[htp] \centering

    \begin{subfigure}[b]{0.96\columnwidth}
        \includegraphics[width=\textwidth]{images/original/27cloudpirker08schulze10070}
        \caption{Cloud plot, $SSC$, $\sigma_n=10070 ~[Pa]$, $P=0.8$}
        \label{fig:27cloudpirker08schulze10070} 
    \end{subfigure}\\
    \begin{subfigure}[b]{0.96\columnwidth}
        \includegraphics[width=\textwidth]{images/original/25cloudpirker1schulze10070}
        \caption{Cloud plot, $SSC$, $\sigma_n=10070 ~[Pa]$, $P=1.0$}
        \label{fig:25cloudpirker1schulze10070}
    \end{subfigure}\\

    \begin{subfigure}[b]{0.96\columnwidth}
        \includegraphics[width=\textwidth]{images/original/30cloudpirker12schulze10070}
        \caption{Cloud plot, $SSC$, $\sigma_n=10070 ~[Pa]$, $P=1.2$}
        \label{fig:30cloudpirker12schulze10070} 
    \end{subfigure}
    \caption[Density plot comparison of SCT results]{Density plot comparison of
    shear cell tester ($SSC$) results. We represent the marked combinations for
    one load condition of the shear cell. 
    Density plot of the particles' coefficient of restitution (COR) in dependence
	of coefficient of sliding friction and coefficient of rolling friction; in the
	white area no valid sets of simulation parameter can be found.
	In each cell the valid sets are grouped accordingly to the 4 different COR
	ranges.
	Each cell is colored accordingly to the group with the most members. 
    Here, the values plotted are selected between the numerical
    values from the Neural Network with initially the original experimental
    results for the $SSC$, with a product coefficient $P=1.0$ (Fig.
    \ref{fig:25cloudpirker1schulze10070}). 
        Later, they have been chosen with  
    the virtual decreased results $P=0.8$
    (\ref{fig:27cloudpirker08schulze10070}).
    The last image (Fig. \ref{fig:30cloudpirker12schulze10070}) represents
    instead the selection with the the virtual increased results $P=1.2$.    }
    \label{fig:29schulzeradarandcloud}
\end{figure}
\begin{figure}[htp] \centering
    \begin{subfigure}[b]{0.96\columnwidth}
        \includegraphics[width=\textwidth]{images/original/31radarpirker1aor}
        \caption{Radar plot, $AoR_{exp} = 38.85 ^\circ$}
        \label{fig:31radarpirker1aor} 
    \end{subfigure}\\
        \begin{subfigure}[b]{0.96\columnwidth}
        \includegraphics[width=\textwidth]{images/original/33radarpirker1schulze10070aor}
        \caption{Radar plot, $AoR_{exp} = 38.85
        ^\circ$ \& $SSC$: $\sigma_n=10070 ~[Pa]$}
        \label{fig:33radarpirker1schulze10070aor} 
    \end{subfigure}
    \caption[Radar plot of valid simulations parameters for the AOR and
    the merge between AOR and SCT valid parameters]{Radar plot of valid
    simulations parameters for the angle of repose tester ($AoR$) and the merge
    between AOR and shear cell tester ($SSC$).
    Each axes of the radar plot represents one simulation parameters.
    Furthermore, the shaded area represents valid parameters combinations.
    Dark shaded values stand for the confidence range.
    We represent the marked combinations for one load condition of the shear
    cell.
    Further explanation in the text. }
    \label{fig:35schulze10070aorradarandcloud}
\end{figure}




% \subsection{}
% \label{subsec:}