%************************************************
\section{Conclusions}
\label{sec:conclusions}
%************************************************

We presented a two phase methodology for DEM simulation parameter
identification. In a first phase an artificial neural network is 
trained by dedicated DEM simulation in order to predict bulk 
behaviour in dependence on a set of DEM simulation parameters. 
In a second phase this artificial neural network is then utilized 
to predict the bulk behaviour of a huge number of additional DEM parameter sets. 
The main findings of this study can be summarized as:
\begin{itemize}
  \item{An artificial neural network can be trained by a limited number of dedicated DEM simulations. 
  		Subsequently, the trained artificial neural network is able to predict
  		granular bulk behaviour.}
  \item{This prediction of granular bulk behaviour is highly efficient if
  		compared to computationally expensive DEM simulations.
  		Thus, a huge number of parameter sets can be studied with respect to their
  		macroscopic output.}
  \item{If the predictions of the artificial neural network are compared to a bulk experiment, 
  		valid sets of DEM simulation parameters can be readily deduced for a
  		specific granular material.}
  \item{This DEM parameter identification methodology can be applied to arbitrary bulk experiments. 
  		Combining two artificial neural networks, which predict two different bulk
  		behaviours, leads to focusing the set of valid DEM simulation parameters.}
\end{itemize}
In future we will further develop this methodology by considering different
fractions of granular materials, which will lead to size dependent sets of DEM simulation parameters.
