% !TEX encoding = UTF-8
% !TEX TS-program = pdflatex
% !TEX root = ../Articolo.tex
% !TEX spellcheck = it-IT

%************************************************
\section{Introduction}
\label{section:introduction}
%************************************************

This report describes the numerical modellization of heat transfer in a porous
media inside a vertical pipe, while hot air flows, and its comparison with available laboratory
experiments. The simulations were performed with OpenFOAM\textregistered ($OF$).

\subsection{Laboratory experiment}
\label{subsection:laboratoryexperiment}

The experimental test has been performed with the system available in the
laboratory of the Department of Particulate Flow Modelling in the JKU.
It can be seen in Fig. \ref{fig:02tester}, while the heating element is shown in
Fig. \ref{fig:03heater}. 
The porous media was a densely packed bed of glass silibeads, 
hence its porosity properties were considered isotropic.
The total height was
1.15 m, with a diameter of 0.15 m. \\
The experimental data have been collected from the report of
\textcite{ReportHofer}.
He used two thermocouples, one positioned 0.10 m above the bed bottom, one 0.10
m below.

\begin{figure}[!h]
\centering
\subfloat[Experimental setup]
{\label{fig:02tester}%
\includegraphics[height=6cm]{02tester}} \quad  %,width=.48\columnwidth
\subfloat[Heating element]
{\label{fig:03heater}%
\includegraphics[height=6cm]{03heater}} \\  
\caption[Experiment]{Experiment}
\label{fig:experiments}
\end{figure}

\subsection{Porosity theory and parent tutorial}
\label{subsection:porositytheorytutorial}

The theorical foundation of the porosity model used in this work is the
Darcy's law.
The solvers that handle porosity in $OF$ substitute the
permeability tensor with the tensor $\bar{\bar{d}}$, where $d= \frac {1}{k}$.
This porosity model is widely investigated in the reports of
\textcite{ReportReveillon} and \textcite{Soulaine}, together with its usage in
$OF$.
The settings are defined in the fvOptions file \cite{OForgFv},
while the porosity values were provided by \textcite{Permeability}.


\subsection{Heat transfer theory and parent tutorial}
\label{subsection:heattransfertheorytutorial}

The investigated experiment belongs to the conjugate heat transfer
problems class.
The model is presented in the report of \textcite{ReportTempel},
and further information can be found in \textcite{OFWikicht}.
