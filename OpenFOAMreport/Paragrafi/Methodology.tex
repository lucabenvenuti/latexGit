% !TEX encoding = UTF-8
% !TEX TS-program = pdflatex
% !TEX root = ../Articolo.tex
% !TEX spellcheck = it-IT

%************************************************
\section{Methodology}
\label{section:methodology}
%************************************************

Two regions were created: porous and air.
The porous region was a parallelepiped with a quadratic base and an height of
1.15 m in the z axis. The base edge was 0.14 m, making the base area
equivalent to the experimental one.
The air region had the same base of the porous region, but had an height of
1.35 m.
From $z=0.1 ~m$ to $z=1.25 ~m$ the two regions shared the same volume.\\
The solver used is \textbf{chtMultiRegionSimpleFoam}, since we had a steady hot
air flow as lower boundary condition, hence a steady state solver.
The solver consider the fluid region as compressible flow.\\
A series of probes, as indicated by \textcite{RegionProbe}, were positioned to
register the temperature.

\subsection{Permeability coefficient}
\label{subsection:permeabilitycoefficient}

The value of the $d$ coefficient has been varied, by order of magnitudes, to
evaluate its effect on the temperature.

\subsection{Flow velocity}
\label{subsection:flowvelocity}

The value of the flow velocity $U$ has been varied to
evaluate its effect on the temperature.

\subsection{Time interval}
\label{subsection:timeinterval}

The total simulation time has been corrected to match the experimental time.