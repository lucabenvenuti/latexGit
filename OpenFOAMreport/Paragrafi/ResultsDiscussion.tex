% !TEX encoding = UTF-8
% !TEX TS-program = pdflatex
% !TEX root = ../Articolo.tex
% !TEX spellcheck = it-IT

%************************************************
\section{Results and Discussion}
\label{section:resultsdiscussion}
%************************************************



\subsection{Comparison}
\label{subsection:comparison}

First, the effect of different $\bar{\bar{d}}$ coefficients has been evaluated, see Fig.
\ref{fig:04dvariation}.
It can be seen that the effect for low speed flow regimes ($U = 0.1 ~m/s$) is
negligible.\\
Later, we selected $\bar{\bar{d}} = 10 ~m^{-2}$, as suggested by
\textcite{Permeability}, and we considered the variation of the flow speed $U$.
Since the flow velocity deeply influences the temperature, see Fig.
\ref{fig:05uvariation}, all the simulations in Fig. \ref{fig:04dvariation} were
not further considered.\\
Finally, we performed simulations with the correct duration and temperature, to
evaluate the flow speed.
It is clear from Fig. \ref{fig:06uvariation2} that the flow speed $U = 1.5 ~m/s$
is the most accurate prediction between those evaluated.

\begin{figure}[!h]
\centering
\subfloat[$\bar{\bar{d}}$ variation, $U = 0.1 ~m/s$]
{\label{fig:04dvariation}%
\includegraphics[width=0.8\columnwidth]{04dvariation}} \\ 
\subfloat[$U$ variation, $\bar{\bar{d}} = 10 ~m^{-2}$]
{\label{fig:05uvariation}%
\includegraphics[width=0.8\columnwidth]{05uvariation}} \\  
\subfloat[$U$ variation, $\bar{\bar{d}} = 10 ~m^{-2}$, experimental comparison]
{\label{fig:06uvariation2}%
\includegraphics[width=0.8\columnwidth]{06uvariation2}} \\  
\caption[Simulations]{Simulations}
\label{fig:simulations}
\end{figure}

\subsection{Temperature variation}
\label{subsection:temperaturevariation}

\begin{figure}[!h]
\centering
\subfloat[t = 25 s]
{\label{fig:0701}%
\includegraphics[height=8cm]{0701}} \quad 
\subfloat[t = 150 s]
{\label{fig:0806}%
\includegraphics[height=8cm]{0806}} \quad  
\subfloat[t = 300 s]
{\label{fig:0912}%
\includegraphics[height=8cm]{0912}} \\  
\subfloat[t = 450 s]
{\label{fig:1018}%
\includegraphics[height=8cm]{1018}} \quad  
\subfloat[t = 600 s]
{\label{fig:1124}%
\includegraphics[height=8cm]{1124}} \\  
\caption[Temperature variations with time]{Temperature variations with time}
\label{fig:simulations2}
\end{figure}

In Fig. \ref{fig:simulations2} we show the variation of temperature with time
during the most accurate simulation ($\bar{\bar{d}} = 10 ~m^{-2}$, $U = 1.5
~m/s$).
In these images the domain is sliced along the z axis, so we can see how the
temperature boundary condition propagates through the domain.
