%************************************************
\section{Conclusions}
\label{sec:conclusions}
%************************************************
After the experimental characterization with a $SSC$ and an $AoR$ tester we performed a 
limited number of $DEM$ simulations designed to be representative of the experiments. 
By varying a selection of the $DEM$ parameters combinations we obtained different numerical bulk behaviours. 
We used these numerical results to train $ANN$. Once trained, they processed huge 
numbers of $DEM$ parameters combinations to fast collect their numerical bulk behaviours. 
The latters were compared with the experimental values to gather valid combinations. 
The validity of the procedure was tested against a different artificial series of experimental 
values and also by using more than one bulk characterization experiment. 
We could then claim that is a stringent methodology of refined $DEM$ parameters identification. 
Further, we demonstrated that an entire series of non-cohesive bulk materials could be tested with this procedure.
Since the poly-dispersity had been considered in the simulations, but not varied, a future aim would 
be to characterize with these procedure at least two bulk of the same material, with different size distributions. 
Then, provided the size distribution of the mixture of the two, we could evaluate if the $ANN$ could predict 
the bulk behaviour of the mixture without any further training. 
Finally, we could expand the characterization to more materials, like coke, iron
ore and limestone.
