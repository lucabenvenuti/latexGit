\section{Results and discussion}
\label{sec:results}
%************************************************

\subsection{Experiments}
\label{subsec:experiments}

Initially, experimental values identifying the bulk behavior, $\mu_{e-psh}$, $\mu_{e-sh}$ and $\rho_{b}$, 
for sinter fine have been acquired though the SRSCT, see Table \ref{tab:05sinterTableExperimental}. 
A representative stress path can be seen in Fig. \ref{fig:20experimental}.
\begin{table}[h]
\centering
\begin{tabular}{cccccc}
$\sigma_n$ [Pa] & $\tau$ [Pa] & $\mu_{psh}$ [-] & $\mu_{sh}$ [-] &
$\rho_b$ [kg/m3] & AOR $\circ$ \\
\hline
    1068  & 1059  & 0.9916 & 0.9916 & 1718  & 38.85 \\
    2069  & 1818  & 0.8787 & 0.8787 & 1759  & 38.85 \\
    10070 & 8232  & 0.8175 & 0.8175 & 1802  & 38.85 \\

\hline
\end{tabular}
\caption{Experimental values for sinter fine}
\label{tab:05sinterTableExperimental}
\end{table}
\begin{figure}[!htb] 
\centering 
\includegraphics[width=.96\textwidth]{images/original/20experimental} 
\caption[Experimental stress path]{Sample of the experimental stress path for
the Schulze ring shear cell tester.
In the first 300 seconds the $\sigma_n = 2000 ~[Pa]$ is kept constant. After 250
seconds a plateau is reached.
The $\mu_{psh}$ is calculated as average of the $\mu_{ie}$ in this plateau.
Later, the $\sigma_n$ is reduced to $80 \%$ of its initial value.
After approximately 30 seconds, a second plateau starts.
As average of $\mu_{ie}$ in this second plateau we obtain $\mu_{sh}$.
}
\label{fig:20experimental}
\end{figure}

%SCT - experimental - sn = 2000 [Pa]
% \begin{figure}[htp]
%     \centering
%     \includegraphics[width=.2\textwidth]{images/vitae/lbenvenuti}
%     \caption{OpenMP, MPI, MPI/OpenMP Hybrid runs of Box in a box testcase on 32
%     cores. The OpenMP-only run suffers from limited memory bandwidth in
%     memory-bound algorithms inside of the Modify section of the code. MPI-only has
%     low averaged runtimes for each section, but a very large Other timing, which
%     hints for a large amount of load-imbalance. Hybrid timings are a bit worse
%     on average, but because of better balancing, processes have lower wait times
%     inside of Other timing.}
% 	\label{fig:boxInBoxComparison}

Later, two AOR test have been performed, given an average angle of $38.85
^\circ$.
We also realized the sieving.

\subsection{DEM Simulations}
\label{subsec:simulations}

For sinter fine 546 shear cell and 81 static angle of repose simulations have
been realized with the variation described in Tab.
\ref{tab:10DEMVariableinputvalues}.
A representative stress path can be seen in Fig. \ref{fig:21simexample}.
\begin{figure}[!htb] 
\centering 
\includegraphics[width=.96\textwidth]{images/original/21simexample} 
\caption[Numerical stress path]{NEW SHORTER GRAPH!!!!! Sample of the numerical
stress path for a shear cell tester simulation, $\sigma_n = 10000 ~[Pa]$.
Although the duration is smaller by two order of magnitude, the stress path is comparable to the experimental one, especially the plateaux.
They were clearly relevant because there we collected the numerical bulk
behaviour representative values.}
\label{fig:21simexample} 
\end{figure}


% \begin{figure}[htp]
%     \centering
%     \includegraphics[width=.2\textwidth]{images/vitae/lbenvenuti}
%     \caption{OpenMP, MPI, MPI/OpenMP Hybrid runs of Box in a box testcase on 32
%     cores. The OpenMP-only run suffers from limited memory bandwidth in
%     memory-bound algorithms inside of the Modify section of the code. MPI-only has
%     low averaged runtimes for each section, but a very large Other timing, which
%     hints for a large amount of load-imbalance. Hybrid timings are a bit worse
%     on average, but because of better balancing, processes have lower wait times
%     inside of Other timing.}
% 	\label{fig:boxInBoxComparison}

The computational time resulted in 1 hour with 32 AMD cores for a benchmark
shear cell simulation and 9 hours for a benchmark $AOR$ simulation, both with 50K particles. 
Simulations with large $dCylDp$ required a greater time amount (e.g. with 400K
particles $\~ 12 hours$ for the shear cell). \\




\subsection{ANN model development}
\label{subsec:annmodeldev}

First, we present the regression plot of a bulk behaviour parameter, e.g. the
$\mu_{e-psh}$, see Fig. \ref{fig:22regression}, where the regression plot for
the $NN$ with the maximum $R^2$ in shown. Each circle represents a simulation. 
The plot presents a consistent agreement between the DEM results distribution and the NN regression line.
\begin{figure}[!h] 
\centering 
\includegraphics[width=.96\textwidth]{22regression}
%[width=.96\textwidth]
\caption[Comparison between prediction of the trained NN and full DEM
simulation]{Comparison between prediction of the trained Neural Network ($NN$)
and 546 full DEM simulations of the coefficient of pre-shear ($\mu_{psh}$). In
this case the regression line is nearly linear, and demonstrates the accurate
prediction of the $NN$.}
\label{fig:22regression} 
\end{figure}

% Regression plot - $\mu_{psh}$. In this case the
% $Output = 0.96 \cdot Target + 0.046$. With 546 simulations the $R^2 = 0.98044$. The plot
% presents a consistent agreement between the $DEM$ results distribution and the $NN$ regression line.
% \begin{figure}[htp]
%     \centering
%     \includegraphics[width=.2\textwidth]{images/vitae/lbenvenuti}
%     \caption{OpenMP, MPI, MPI/OpenMP Hybrid runs of Box in a box testcase on 32
%     cores. The OpenMP-only run suffers from limited memory bandwidth in
%     memory-bound algorithms inside of the Modify section of the code. MPI-only has
%     low averaged runtimes for each section, but a very large Other timing, which
%     hints for a large amount of load-imbalance. Hybrid timings are a bit worse
%     on average, but because of better balancing, processes have lower wait times
%     inside of Other timing.}
% 	\label{fig:boxInBoxComparison}

The linear relationship between the
training values have been evaluated in Table \ref{tab:06inputRelationshipTable}.
\begin{table}[H!]                                                                                                                                                          
\centering                                                                                                                                                                 
\begin{tabular}{|c|c|c|c|c|c|c|c|c|c|c|c|}                                                                                                                                 
\hline                                                                                                                                                                     
 & sf & rf & rest & dt & dCylDp & ctrlStress & shearperc & dens & mush & mupsh & rhob \\                                                                                   
\hline                                                                                                                                                                     
sf & 1 & 5.549787e-03 & -3.818461e-04 & -1.268763e-15 & -1.628657e-02 & 1.282025e-15 & 4.517397e-03 & 0 & 3.838826e-02 & 8.725701e-01 & -8.393464e-02 \\                   
\hline                                                                                                                                                                     
rf & 5.549787e-03 & 1 & -1.523330e-03 & -2.349289e-15 & -5.968531e-02 & 2.322503e-15 & 1.802162e-02 & 3.348007e-18 & 5.891756e-01 & 3.370233e-01 & -3.101856e-02 \\        
\hline                                                                                                                                                                     
rest & -3.818461e-04 & -1.523330e-03 & 1 & -1.555718e-15 & -2.758674e-01 & 1.568359e-15 & 8.090707e-02 & 6.680307e-18 & 1.551852e-01 & -5.671687e-03 & -1.712429e-02 \\    
\hline                                                                                                                                                                     
dt & -1.268763e-15 & -2.349289e-15 & -1.555718e-15 & 1 & -1.026312e-16 & -1.000000e+00 & -2.681936e-17 & 0 & 6.168810e-16 & -4.320958e-15 & 1.243669e-14 \\                
\hline                                                                                                                                                                     
dCylDp & -1.628657e-02 & -5.968531e-02 & -2.758674e-01 & -1.026312e-16 & 1 & 7.853515e-17 & -2.939311e-01 & 2.688281e-17 & -2.879551e-01 & -1.916393e-01 & 9.603603e-02 \\ 
\hline                                                                                                                                                                     
ctrlStress & 1.282025e-15 & 2.322503e-15 & 1.568359e-15 & -1 & 7.853515e-17 & 1 & -3.731389e-17 & 0 & -6.100950e-16 & 4.292811e-15 & -1.234126e-14 \\                      
\hline                                                                                                                                                                     
shearperc & 4.517397e-03 & 1.802162e-02 & 8.090707e-02 & -2.681936e-17 & -2.939311e-01 & -3.731389e-17 & 1 & -3.512479e-17 & 5.730199e-02 & 5.380657e-02 & -5.095294e-03 \\
\hline                                                                                                                                                                     
dens & 0 & 3.348007e-18 & 6.680307e-18 & 0 & 2.688281e-17 & 0 & -3.512479e-17 & 1 & -4.980664e-02 & 5.709445e-02 & 9.900341e-01 \\                                         
\hline                                                                                                                                                                     
mush & 3.838826e-02 & 5.891756e-01 & 1.551852e-01 & 6.168810e-16 & -2.879551e-01 & -6.100950e-16 & 5.730199e-02 & -4.980664e-02 & 1 & 2.603411e-01 & -9.516313e-02 \\      
\hline                                                                                                                                                                     
mupsh & 8.725701e-01 & 3.370233e-01 & -5.671687e-03 & -4.320958e-15 & -1.916393e-01 & 4.292811e-15 & 5.380657e-02 & 5.709445e-02 & 2.603411e-01 & 1 & -4.329071e-02 \\     
\hline                                                                                                                                                                     
rhob & -8.393464e-02 & -3.101856e-02 & -1.712429e-02 & 1.243669e-14 & 9.603603e-02 & -1.234126e-14 & -5.095294e-03 & 9.900341e-01 & -9.516313e-02 & -4.329071e-02 & 1 \\   
\hline                                                                                                                                                                     
\end{tabular}                                                                                                                                                              
\caption{MyTableCaption}                                                                                                                                                   
\label{table:MyTableLabel}                                                                                                                                                 
\end{table}               
Then we show how the $R^2$ changed with the different number of neurons. The
straight line, referred only to the test simulations, is more sensible to
variations of number of neurons, see Fig. \ref{fig:23regressiongraph}.
\begin{figure}[!h] 
\centering 
\includegraphics[width=.96\textwidth]{images/original/23regressiongraph}
%[width=.96\textwidth]
\caption[Regression graph]{Regression graph. The variation of $R^2$ for
$\mu_{psh}$ with the number of neurons in the NN is shown. The
straight line, referred only to the test simulations, is more sensible to
variations of number of neurons, compared
to the total line. In the latter, also the correlated simulations used for
training participate in the $R^2$ evaluation.}
\label{fig:23regressiongraph} 
\end{figure}

%SCT: sn = 10070 [Pa], coeff. P. = 1
% \begin{figure}[htp]
%     \centering
%     \includegraphics[width=.2\textwidth]{images/vitae/lbenvenuti}
%     \caption{OpenMP, MPI, MPI/OpenMP Hybrid runs of Box in a box testcase on 32
%     cores. The OpenMP-only run suffers from limited memory bandwidth in
%     memory-bound algorithms inside of the Modify section of the code. MPI-only has
%     low averaged runtimes for each section, but a very large Other timing, which
%     hints for a large amount of load-imbalance. Hybrid timings are a bit worse
%     on average, but because of better balancing, processes have lower wait times
%     inside of Other timing.}
% 	\label{fig:boxInBoxComparison}

Thus, we selected the $NN$ with the maximum $R^2$ in the test line, as stated in the methodology.
Later, we processed the random combinations (Tab.
\ref{tab:10DEMVariableinputvalues}) with the $NN$.
The $NN$ evaluation was incredibly faster compared to the $DEM$ simulations. The
individuation of all the tabbed $DEM$ combinations for the shear cell did not take more than a few seconds on a single core. 
We represented the tabbed combinations for one load condition of the shear cell in Fig.
\ref{fig:24radarpirker1schulze10070}.
\input{images/texCaller/24radarpirker1schulze10070}
Here, the minimum and maximum values, together with the mean and the confidence
range, provided by the square deviation, are shown. Notably, the confidence range is large, 
especially for the $COR$, highlighting its scarce influence over the characterization. 
Instead, both the $\rho_p$  and the $\mu_s$ show a narrow confidence range, 
displaying at the same time their influence and the validity of this procedure to find valid $DEM$ parameters. 
Especially, we could see how different $DEM$ parameters combinations could reproduce the experimental 
behaviour and evaluate their mutual dependencies. 
This is clearer in a cloud plot, as in Fig. 
\ref{fig:25cloudpirker1schulze10070}. While the $COR$ varied, multiple
combinations ($250407 --> 4\% $ of the total) of $\mu_s$ and $\mu_r$ reproduced
the experimental behaviour.
\input{images/texCaller/25cloudpirker1schulze10070}
To further demonstrate the validity of the procedure, we modified the product
coefficient. In the first attempt we set it to $P=0.8$. We can see in the radar plot in Fig. 
\ref{fig:26radarpirker08schulze10070} that the confidence range is narrower
compared to $P=1.0$.
\input{images/texCaller/26radarpirker08schulze10070}











% \ref{eq:rsquare}.
% \begin{equation}
R^2 = \frac {SSR}{SST} = 1 - \frac {SSE}{SST} ,
 \label{eq:rsquare}
\end{equation}

% 
% \ref{eq:rootMeanSquareError}.
% \begin{equation}
RMSE = \sqrt{\frac{\sum _{i=1}^{n} (x_{i}-\widehat{x}_{i})^{2}}{n}} ,
\label{eq:rootMeanSquareError}
\end{equation}



% \lipsum[1]
% \begin{equation}
m \ddot{x}_{ij} + c \dot{x}_{ij} + k x_{ij} =  F_{i} .
\label{equ:newtonlaw}
\end{equation}

% \subsection{ANN identification}
% \label{subsec:annmodeliden}
% \subsection{ANN application}
% \label{subsec:annapplication}
% 
% Later, each of these three trained $NN$ received as insertion $100M$ different
% combinations $DEM-micro$ parameters.
% So, we gained the numerical bulk behavior for each of this combination. 
% We then compared the values of these behaviors against the experimental bulk
% values, $SRSCT$ and $AOR$, obtaining a narrow range of valid DEM-micro
% combinations (about 80K).
% These results have been showed through radar plots (Figs. \ref{fig:15Schulze}
% and \ref{fig:16aorSchulzeIntersectionWorking}).
% To highlight an eventual $clumping$ behavior, we also plot the results in a
% cloud shape, see Fig. \ref{fig:17aorSchulzeIntersectionCloudSFRFCOR}.
% 
% %\input{images/texCaller/14aorSchulzeIntersection}
% \input{images/texCaller/15Schulze}
% % \lipsum[1]
% \input{images/texCaller/16aorSchulzeIntersectionWorking}
% % \begin{equation}
\begin{aligned}
\phi_{e-psh} &= \arctan \left(\frac{\tau_{psh}}{\sigma_{n,psh}} \right) ,\\
\mu_{psh} &=\tan(\phi_{e-psh}) .
\end{aligned}
 \label{eq:phi_ps}
\end{equation}

% \input{images/texCaller/17aorSchulzeIntersectionCloudSFRFCOR}
