\section{Modelling Pre-requisites}
\label{sec:modellingprerequisites}
%************************************************
%\section{Method}
%\label{sec:method}
%************************************************

%%%\lipsum[1]
%%\begin{equation}
\label{eq:emc}
e = mc^2
\end{equation}


\subsection{Macroscopic Experiments}
\label{subsec:Macroscopicexperiments}

The first step of the procedure was using a SRSCT (see \cite{RefWorks:142}) to characterize particle flow properties, 
especially the complete yield locus.
With the procedure described in appendix A1 we obtained for each of the twelve load conditions 
three values representative of the bulk behaviour: bulk density ($\rho_b$),
coefficient of internal friction in the pre-shear phase ($ (\mu_{psh})$) and
coefficient of internal friction in the shear phase  ($ (\mu_{sh})$).\\
Furthermore, in order to recreate the repose angle observed in a pile of the real material, we performed angle of repose (AOR) tests, as the AOR was the fourth behaviour value. The complete description can be found in appendix B1.
Moreover, we sieved the materials samples to obtain the size distribution of the particles. 6 different sifters have been used.



\subsection{Discrete element method}
\label{subsec:dem}
The $DEM$ is based on a relatively elementary idea. For each particle i inside
the domain it follows the course and calculate the force that particle i exerts on particle j
by integrating Newton' second law:
\begin{equation}
m \ddot{x}_{ij} + c \dot{x}_{ij} + k x_{ij} =  F_{i} .
\label{equ:newtonlaw}
\end{equation}

\begin{equation}
\label{eq:cor}
e^n = exp \left(-\frac{\gamma^n}{2\cdot m_{eff}} \cdot \pi \cdot
\left(\frac{k^n}{m_{eff}}- \left(\frac{\gamma^n}{2\cdot m_{eff}} \right)^2
\right)^{-\frac{1}{2}} \right)
\end{equation}
 
and the position and orientation equations.
The main forces interested are: gravitation, contact forces due to collisions,
solid-solid interactions such as electrostatic, Van der Waals, cohesive forces
and fluid-solid interactions in multiphase flows.
For the raw materials used in this work Di Renzo and Di Maio \cite{RefWorks:145}
suggested using the non-linear Hertzian model without cohesion for the particle-particle and particle-wall contacts.
Although, its complexity increased the computational effort to fathom the
micro-macro transition relationship.\\
This granular model uses the following formula for the force between two granular particles (Eq. \ref{eq:forceij}):
\begin{equation}
 F_{ij} = 
\begin{cases}
F_{n,ij} + F_{t,ij} = \left( k_n \delta_{n,ij} + \gamma_n v_{n,ij} \right) + \left( k_t \delta_{t,ij} + \gamma_t v_{t,ij} \right) & \text{if } r < d ,\\
0    & \text{if } r > d ,\\
\end{cases}
 \label{eq:forceij}
\end{equation}

while the tangential force component is truncated to fulfill
\begin{equation}
F_{t} \leq \mu_s F_{n},
 \label{eq:force_t}
\end{equation}

Both the normal and the tangential force comprise two terms, a spring force and a damping force. The shear force is a "history" effect that accounts for the tangential displacement 
("tangential overlap") between the particles for the duration of contact. \\

The $k_n$, $k_t$, $\gamma_n$, and $\gamma_t$ coefficients are calculated from the material properties as follows:
\begin{equation}
\begin{aligned}
	k_n &= \frac{4}{3} E_{eq} \sqrt{R_{eq} \delta_n} ,\\
	\gamma_n &= 2 \sqrt{\frac{5}{6}} \beta \sqrt{S_n m_{eq}} ,\\
	k_t &= 8 G_{eq} \sqrt{R_{eq}} \delta_n ,\\
	\gamma_t &= 2 \sqrt{\frac{5}{6}} \beta \sqrt{S_t m_{eq}} .
\end{aligned}
\label{eq:hertz}
\end{equation}

In addition to the equations \ref{eq:hertz} the following relations (Eqns. \ref{eq:equivProp2}) are required:
\begin{equation}
\begin{aligned}
 \frac{1}{E_{eq}} & = \frac{1-\nu_i^2}{E_i} + \frac{1-\nu_j^2}{E_j} ,\\
 \frac{1}{G_{eq}} & = \frac{2(2+\nu_i)(1-\nu_i)}{E_i} + \frac{2(2+\nu_j)(1-\nu_j)}{E_j} ,\\
 \frac{1}{R_{eq}} &= \frac{1}{R_i} + \frac{1}{R_j} ,\\
 \frac{1}{m_{eq}} &= \frac{1}{m_i} + \frac{1}{m_j} ,\\
 \beta & = \frac{\ln(e)}{\sqrt{ln^2(e)+\pi^2}} ,\\
 S_n & = 2 E_{eq} \sqrt{R_{eq} \delta_n} ,\\
 S_t & = 8 G_{eq} \sqrt{R_{eq} \delta_n} ,\\
 k_r & = k_t R_{eq}^2 .\\
\end{aligned}
\label{eq:equivProp2}
\end{equation}


The coefficient of rolling friction for coarse round particles is a critical
parameter describing inter-particle friction in medium to dense granular flows simulations.
It is proportional to the torque counteracting the rotation of the particle and defined as (Eq. \ref{equ:mur}):
\begin{equation}
 \mu_r =  \tan(\iota) .
\label{equ:mur}
\end{equation}

The $\mu_r$ parameter enters the equations according to the elasto-rolling
resistance model presented by \cite{RefWorks:87} and \cite{RefWorks:131}, also
used by \cite{RefWorks:173}, based on the work of
\cite{RefWorks:143}(and in contrast to \cite{RefWorks:144}). The model is called EPSD2 in LIGGGHTS.
This is appropriate for the one way rolling cases as well as the cycling rolling ones.
The total rolling resistance torque is (Eq. \ref{eq:trmax}):
\begin{equation}
T_{r~max} = \mu_r R_r |\tilde{F_n}| .
 \label{eq:trmax}
\end{equation}


Given these equations, completely defining a dry material for DEM simulations requires these data:
\begin{itemize}
\item{the radius of the particles ($R$);}
\item{the Young's modulus ($E$) and the Poisson's coefficient ($\nu$);}
\item{the particle density ($\rho_p$) and the coefficient of restitution ($e$);}
\item{the coefficients of sliding ($\mu_s$) and rolling ($\mu_r$) friction.}
\end{itemize}


Further details on the method can be found in \cite{RefWorks:133}.\\

%%%%%%%%%%%%%%%%%%%%%%%%%%%%%%%%%%%%%%%%%%%%%%%%%%%%%%%%%%%%%%%%%%%%%%%%%%%%%%%%%%%%%%%%%%%%%%%%%%%%%%%%%%%%%%%%%%%%%%%%%%%%%%%%%%%%
%%Simulations
%%%%%%%%%%%%%%%%%%%%%%%%%%%%%%%%%%%%%%%%%%%%%%%%%%%%%%%%%%%%%%%%%%%%%%%%%%%%%%%%%%%%%%%%%%%%%%%%%%%%%%%%%%%%%%%%%%%%%%%%%%%%%%%%%%%%

$LIGGGHTS$, the simulation toolbox we used, meets most of the requirements of modelling the shear tester described above. 
First, it is capable of importing triangulated meshes of the two rings and a top lid. Since the real set-up had a wall thickness, 
contact forces acting on a mesh are summed and can be saved, and thus shear force calculation is available out of the box. Moreover, 
the code can move a mesh with constant velocity as required for the measurement. 
To determine the shear stresses, the bulk solid had to be stressed with user-defined normal stresses.
Therefore, a stress-controlled wall ($servo-wall$ in $LIGGGHTS$) was applied to the lid. \\

Although the geometry differs, the $SRSCT$ was designed to obtain the same
values for the shear stresses as the Jenike shear cell tester ($JSCT$), but with
improved automation and reliability \cite{RefWorks:118}.
For this reason, the simulation setup has been based over the $JSCT$.
As suggested by \cite{RefWorks:139}, the diameter and the height of the rings operated in the simulations were respectively 50 and 13 times the diameter of the particles involved.
The layout of the simulation geometry can be seen in Figure \ref{fig:09simGeometry}. \\
\input{images/texCaller/09simGeometry} 

A simulation run comprised four phases. 
First, the shear cell was filled with the granulate material, and it was allowed to settle. 
Then, the top lid was lowered and applied the first normal stress to the bulk solid. 
As in the experiment, the servo-wall allows to calculate the position of the lid while the first particle is touched. Together with the simulation area, we calculated $\rho_b$.
Next, the ring moved for a distance $l=0.1875 \cdot radius ~of ~the ~ring$, and the required pre-shear force was measured. 
Finally, the normal load was reduced to a fraction of the initial load, the ring was moved again by a distance $l$, and the shear force was recorded. 
Unlike in the original experiment, the bottom ring was moved to facilitate the numerical simulation.\\
The normal stresses (pre-shear and shear phases) applied in each simulation were
the same as in the experiments. The corresponding $\tau_{psh}$ and $\tau_{sh}$
were calculated - as in the experiments - from the mean of the plateau.\\

Further, we also performed $AOR$ simulations. Here we tried to replicate meticulously the experimental set-up, considering both the plate and the lift-able boundary, 
respectively 50 and 20 times the diameter of the particles involved.
The particles had the same properties as in the shear cell simulation.
The first phase was identical to that of the shear cell simulation. 
After lifting the boundary, the particles formed a heap (Fig. \ref{fig:08aorsim}). 
\input{images/texCaller/08aorsim}
An image post-processing software was used to obtain the average slope.
%%%%%%%%%%%%%%%%%%%Furthermore, in both simulation set-ups the $E$ value was not realistic, but .
This led to a reduction in computational time, resulting in 1 hour with 32 cores
for each shear cell simulation and 9 hours for each $AOR$ simulation.\\

\subsection{Artificial Neural Networks}
\label{subsec:ann}

An Artificial Neural Network (NN) is a powerful modellization technique, based
on non-linear functions (Haykin \cite{RefWorks:158}). In this paper, we first use them to fit the DEM
numerical simulation data, and then to process vast amount of parameters combinations.
NN original idea is directly borrowed from human brain design, with neurons and
synapses. They map combinations of input data into convenient outputs (fitting).
There are divers types of NN, remarkably the Feedforward (FF) and the Radial
basis function (RBF). For FF-NN, considerable
amount of traning algorithms are available. The most common are based on
backpropagation: e.g. Levenberg-Marquardt, Bayesian regulation and scaled
conjugate gradient. 
To recognize not linearly separable data the standard 
linear perceptron $NN$ has been modified into \textit{FF Multilayer Perceptron
Neural Networks (MLPNN)}. Here, each processing units or node (neuron) possesses a nonlinear activation
function. Together, they are interconnected into layers, also linked together.
The trustworthiness of the $MLPNN$, with a backpropagation
reinforcement learning training algorithm (scaled conjugate gradient), has been widely
demonstrated in the literature, see Haykin \cite{RefWorks:158}.
Several scientists \cite{RefWorks:161, RefWorks:166, RefWorks:167, RefWorks:168, RefWorks:169,
RefWorks:170} have operated $NN$ to model materials mechanical properties.
In fact, $MLPNN$ are built with three different layers.
The input layer has a number of neurons equal to the number of different inputs
of the network. 
Following the best practice suggested by Vaferi et al. \cite{RefWorks:150}
$MLPNN$ have been handled. 
Similarly, the best practice also demands to establish the
most appropriate number of neurons inside the hidden layer of each $NN$.


%\lipsum[1]
%\begin{equation}
\begin{aligned}
M_r &= M_r^k ,\\
M_{r,ti+\Delta ti}^k &= M_{r,ti}^k - k_r \Delta \theta_r ,\\
\lvert{M_{r,ti+\Delta ti}^k}\rvert & \leq M_r^m = \mu_r R_{eq} F_n .\\
\end{aligned}
 \label{eq:mrtm}
\end{equation}



IMAGE OF MY NEURAL NETWORK WITH INPUT AND OUTPUT PARAMETERS. \\
