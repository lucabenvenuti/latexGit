\section{Methodology of DEM Parameter Identification}
\label{sec:methodology}

We now illustrate the methodology used, also shown in Fig.
\ref{fig:19methodology}.
The experimental characterization has been performed as described in
\ref{subsec:srsctexperiment} and \ref{subsec:aorexperiment}. We performed
three tests with the $SSC$ for the sinter fine bulk, for a total of twelve
load conditions. 
The first bulk behaviour representative value ($\rho_b$) was directly provided. 
If the test was performed correctly, see \ref{subsec:srsctexperiment}, we
observed a linear increase in the coefficient of internal friction.
Later, the first plateau was reached. 
The second bulk behaviour representative value ($\mu_{psh}$) was calculated by averaging the coefficient in this plateau. 
Further, the normal load ($\sigma_n$) was modified, and then a second plateau
was reached.
The third value ($\mu_{sh}$) was determined by averaging the coefficient in this plateau. 
Next, we performed two $AoR$ tests. 
The average of the repose angles provided us the fourth bulk value, allowing us
to define the experimental bulk behaviour.
For simulations purposes, we also sieved the bulk to know the size distribution.
We could then focus on the simulations. 

\subsection{Discrete element method}
\label{subsec:dem}

We decided to fix one single
contact law for all the simulations performed.
The tangential force component is truncated to fulfil 
\begin{equation}
F_{t} \leq \mu_s F_{n},
 \label{eq:force_t}
\end{equation}

where $\mu_s$ is the coefficient of sliding friction, one of the particle based
$DEM$ parameter we investigated. 
Another parameter was the coefficient of rolling friction ($\mu_r$). 
For coarse not round particles is a critical parameter and describes inter-particle 
friction in medium to dense granular flows simulations. It is proportional to the 
torque counteracting the rotation of the particle. The $\mu_r$ parameter enters the 
equations according to the elasto-rolling resistance model presented by Wensrich and 
Katterfeld \cite{RefWorks:87} and Ai et al. \cite{RefWorks:131}, 
based on the work of Jiang et al. \cite{RefWorks:143}. 
The model is called $EPSD2$ in $LIGGGHTS$. This is appropriate for the one way
rolling cases as well as the cycling rolling ones.
The maximum magnitude of rolling resistance torque is (Eq. \ref{eq:trmax}):
\begin{equation}
T_{r~max} = \mu_r R_r |\tilde{F_n}| .
 \label{eq:trmax}
\end{equation}

where $R_r$ is the equivalent radius and $F_n$ the normal force.
The $DEM$ parameters for the Young's modulus ($E$) and the coefficient of Poisson ($\nu$) 
have been taken from the literature, see \cite{RefWorks:175} and \cite{RefWorks:176}, 
although we reduced the former to increase the time step, following the indications of Ai et al. \cite{RefWorks:131}. 
The particles radii have been approximated from the experimental measurements. 
The last two particle based $DEM$ parameter we investigated were the particle density 
($\rho_p$) and the coefficient of restitution ($COR$), as defined by Ai. et al. \cite{RefWorks:131}. 
Furthermore, we locked the size distribution, as provided by the sieving, the
elastic coefficients and the time step, see table
\ref{tab:09DEMFixedinputvalues}.
The last was between $1.29 \%$ and $1.53 \%$ of the Rayleigh time, that depends
to the particle density.
Other coefficients, $COR$, $\mu_s$, $\mu_r$,
$\rho_p$ and $dCylDp$, as indicated in table \ref{tab:10DEMVariableinputvalues},
were constant in each simulation, but their combination differed between
simulations.
Further, $dCylDp$ was used to evaluate the wall effect, but only $~10\%$ of the
all simulations had $dCylDp$ larger than $20$ (additional information can be found in \ref{subsec:srsctsimulation}). 
The normal stress $\sigma_n$ and its
percentage during the incipient flow condition $\tau_{\%}$
varied to replicate the twelve shear cell load conditions. 
The complete description of the shear cell simulations can be found in \ref{subsec:srsctsimulation}, 
while the AOR simulation is presented in \ref{subsec:aorsimulation}.
In total, we realized $546$ shear cell and $81$ angle of repose simulations.
A Matlab script allowed us to extract from the simulations output the numerical
bulk representative values ($\mu_{psh}$, $\mu_{sh}$, $\rho_b$ and $AoR$) for each $simulation-DEM$ parameter combination. 

\subsection{Artificial Neural Networks}
\label{subsec:ann}
We first defined the typology of Artificial Neural Network ($ANN$) we used and
the input we imposed to them, see \ref{sec:appann}
In fact, $MLPNN$ are built with three different layers. 
The input layer has a number of neurons equal to the number of different inputs
of the network, see Fig. \ref{fig:18nnscheme}.
The hidden (or central) layer's number of neurons must be defined. 
In our case this number has been established through mean square maximization ($R^2$), 
again as suggested by Vaferi et al. \cite{RefWorks:150}. 
For each investigated output we chose the number of neurons with the greater $R^2$.
Finally, the output layer contains one neuron for the output.
The transfer functions between the first two layers are the tangential sigmoid, 
while between the hidden and central layers are linear.\\
Random values generators created values in the defined ranges and in the requested 
number for each of the investigated parameter. Then, they were combined and imposed as input.
\begin{figure}[!htb]
\centering
\includegraphics[width=.96\columnwidth]{images/18nnscheme}
\caption[ANN Scheme]{Artificial Neural Network ($ANN$) Scheme
of how the Multilayer Perceptron $ANN$ ($MLPNN$) derives one
bulk-behaviour-dependent variable from the mutually independent simulation variables.}
\label{fig:18nnscheme}
\end{figure}

So, we could use the $DEM$ parameter combinations and their corresponding bulk
values to train the $ANN$,
dashed line in Fig. \ref{fig:19methodology}.
Notably, we excluded 15\% of the simulations ($test ~ simulations$),
randomly picked, from the training processes.
First, we started with all the $DEM$ parameter combinations and their corresponding numerical $\mu_{psh}$ to create 36 $ANN$. 
They differed because they included from five to forty neurons in the hidden
layer.
Later, we controlled the square regression error between the $bulk-macro$ behaviours in the output of 
the $ANN$ and the 15\% $test ~ simulations$, granted uncorrelated. 
So, we could select for $\mu_{psh}$ the $ANN$ with the maximum $R^2$, and we noted its number of neurons. 
We repeated the same steps from the $ANN$ creations for $\mu_{sh}$, $\rho_b$ and $AoR$, 
obtaining one trained $ANN$ for each bulk representative value. \\
Notably, $\mu_{psh}$, $\mu_{sh}$ and $\rho_b$ belonged to the shear cell
simulations, so their $ANN$ were handled together. 
We had one cluster with three $ANN$ for the shear cell and one with only one $ANN$
for the $AoR$.
We could then proceed in identifying the valid input parameters, straight line
in Fig. \ref{fig:19methodology}.
Oberkampf et al. \cite{RefWorks:160} suggest to use a Design of Experiments
($DOE$) method to determine the parameters' combinations to be simulated.
They affirm that this approach allows optimizing the computational time with
only an acceptable loss of precision.
Thanks to the speed of the trained $ANN$, we followed a different approach to
maximize the precision of the characterization.
We created random values
in the range and number defined in table \ref{tab:12DEMRandominputvalues}.
The total number of combinations of these random values was $6250000$. These
combinations were then processed by the selected $ANN$, granting for each three bulk representative parameters for the shear cell and one for the $AoR$. 

\subsection{Macroscopic Experiments}
\label{subsec:macroscopicexperiments}
With the procedure described in \ref{subsec:srsctexperiment} we
obtained for each of the twelve load conditions three values representative of the bulk behaviour: bulk density ($\rho_b$),
coefficient of internal friction in the pre-shear phase $ (\mu_{psh})$ and
coefficient of internal friction in the shear phase  $ (\mu_{sh})$.\\
Furthermore, to recreate the repose angle observed in a pile of the real material, 
we performed angle of repose ($AoR$) tests, as the $AoR$ was the fourth
behaviour value. The complete description can be found in
\ref{subsec:aorexperiment}.

Later, we confronted the $ANN$ and experimental bulk behaviours for the twelve shear cell load conditions. 
If in a $DEM-parameter$ combination all the three bulk representative parameters differed less 
than 5\% from the corresponding experiments, see Eq. \ref{eq:check2}:
 \begin{equation}
 \begin{cases}
\text{if } & \lvert{1-\frac{\mu_{psh,num}}{\mu_{psh,exp}}}\rvert < 5\%  ,\\
\text{and if } & \lvert{1-\frac{\mu_{sh,num}}{\mu_{sh,exp}}}\rvert < 5\% , \\ 
\text{and if } & \lvert{1-\frac{\rho_{p,num}}{\rho_{p,exp}}}\rvert < 5\% ,\\ 
\end{cases}
 \label{eq:check2}
\end{equation}

then the combination was marked. The marked combinations were handled by the
$AoR$ $ANN$, and then confronted with the experiment.
Were branded as valid only those that differed less than $5\%$ also in this
comparison (Eq. \ref{eq:checkaor}):
\begin{equation}
\text{if} ~~~~~~ \lvert{1-\frac{AoR_{num}}{AoR_{exp}}}\rvert < 5\% .
\label{eq:checkaor}
\end{equation}
%************************************************
Further, to prove the system validity, we tested the marked combinations by
modifying the experimental bulk behaviour representative values of the shear cell. 
We artificially decreased or increased $\mu_{psh}$ and $\mu_{sh}$ by a product
coefficient ($P$), e.g. Eq. \ref{eq:pcoeff}:
\begin{equation}
\label{eq:pcoeff}
\mu_{psh, new} = \mu_{psh, old} \cdot P .
\end{equation}
%************************************************
\begin{figure}[!htb] 
\centering 
\includegraphics[width=.96\textwidth]{images/19methodology} 
\caption[Method]{Method. 
In the training phase (dashed lines)
$DEM$ simulations are performed
with random initial input parameters.
The behaviours obtained are used to train the
Artificial Neural Networks ($ANNs$) in a loop that continues until the
difference between the outputs of each $ANN$ and its simulations is below the
limit ($\Delta$) (see Section \ref{subsec:ann}).
In the parameters identification phase (solid
lines) we identify valid input parameters by comparing (\textbf{=}) $ANNs$ and
experimental behaviours.
Further explanations can be found in Section \ref{sec:methodology}.
}
\label{fig:19methodology} 
\end{figure}
\begin{table}[h]
\centering
\begin{tabular}{c|c|c|c|c}
\hline
average & std dev & constant & DEM   & DEM \\
    particle & particle & ring  & Young's & Poisson's \\
    radius & radius & velocity & modulus & ratio \\
    [mm]  & [mm]  & [mm/s] & [Gpa] & [-] \\
    \hline
    0.732 & 0.41  & 2.196 & 10    & 0.40 \\


\hline
\end{tabular}
\caption{DEM fixed input values}
\label{tab:09DEMFixedinputvalues}
\end{table}
\begin{table}[h]
\centering
\begin{tabular}{c|c|c|c|c}
\hline
	DEM   & DEM   & DEM   & average & simulation \\
    sliding & rolling & coefficient & particle & domain diameter \\
    friction & friction & restitution & density & to particle mean \\
    	$[-]$  & $[-]$   & $[-]$   & $[kg/m3]$ & diameter ratio \\
    \hline
    0.4 - 0.6 - 0.8 & 0.4 - 0.6 - 0.8 & 0.5 - 0.7 - 0.9 & 2500 - 3000 - 3500 & 20 - 36 - 38 - 40 \\

\hline
\end{tabular}
\caption{DEM variable input values}
\label{tab:10DEMVariableinputvalues}
\end{table}
\begin{table}[h]
\centering
\begin{tabular}{lcccc}
\hline
 &  \ac{mus} & \ac{mur} & \ac{CoR} & \ac{rhop}  \\
  &	$[-]$  & $[-]$   & $[-]$   & $[kg/m3]$ \\
          \hline
    range & $[0.1 \ldots 1.0]$ & $[0.1 \ldots 1.0]$ & $[0.5 \ldots 0.9]$ &
    $[2000 \ldots 3500]$     \\
    \# rnd & 100   & 100   & 25    & 25    \\

\hline
\end{tabular}
\caption[DEM random input values]{DEM random input values. Within each range \#
random values are chosen.}
\label{tab:12DEMRandominputvalues}
\end{table}