\begin{abstract}
In Discrete Element Method ($DEM$) simulations, particle-particle contact laws
determine the macroscopic simulation results. Particle based contact laws, in
turn, commonly rely on semi-empirical parameters, which can be hardly obtained
by direct microscopic measurements.
In this study we present a methodology for the identification of
$DEM$ simulation parameters by linking the macroscopic experimental results to the
microscopic numerical parameters by artificial neural networks.
In a first step, a series
of $DEM$ simulations with varying simulation parameters are used to train a feed
forward artificial neural network by backward propagation reinforcement. In a
second step, this artificial neural network is utilized to predict the
macroscopic ensemble behaviour in dependence of additional sets of particle
based simulation parameters.
As a result, a comprehensive database is obtained,
which links particle based simulation parameters to a specific macroscopic
bulk behaviour of the ensemble.
The trained artificial neural network is able to predict the behaviour of
additional sets of input parameters accurately and highly efficient.
Furthermore, this methodology can be applied to
identify $DEM$ material parameters in a generic way.
For each set of calibration experiments, the training of the neural network has
to be performed just once. 
After the training, the neural network provides a generic link between the macroscopic 
experimental results and the microscopic $DEM$ simulation parameters.
By the help of these experiments, the $DEM$ simulation parameters of a specific
non-cohesive granular material can be identified.

\end{abstract}


\begin{keyword}
%% keywords here, in the form: keyword \sep keyword
Discrete Element Method ($DEM$) Simulations \sep Parameter Identification \sep Artificial Neural Networks
%% PACS codes here, in the form: \PACS code \sep code

%% MSC codes here, in the form: \MSC code \sep code
%% or \MSC[2008] code \sep code (2000 is the default)

\end{keyword}