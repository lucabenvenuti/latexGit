%************************************************
\section{Introduction}
\label{sec:introduction}
%************************************************

Particles in various forms - ranging from raw materials to food grains and pharmaceutical powders - 
play a major role in a variety of industries, including process industry and metallurgy. 
In his book, Holdich \cite{RefWorks:117} stated that "between 1 and 10\% of all the energy 
is used in comminution, i.e. the processes of crushing, grinding, milling, micronising". 
However, a univocal method to characterize these particles has so far not been established. 
Discrete Element Method (DEM) simulations are widely used to understand particle behaviour 
and in the simulation of granular processes, such as bulk handling.\\
$LIGGGHTS$ (LAMMPS improved for general granular and granular heat transfer simulations) 
is one of the most powerful open source $DEM$ simulation software packages available. 
The models it can analyze are described in detail in the literature, see Kloss
et al. \cite{RefWorks:136}, while a useful example is provided by the shear cell tester 
simulation developed by Aigner et al. \cite{RefWorks:139}.\\
From the experimental point of view, we are focusing on the bulk behaviour of the materials analysed. 
It was out of scope to investigate the properties of each particle separately, like shape, 
Young modulus ($E$), coefficient of Poisson ($\nu$) of each sample.\\
Nevertheless, the $DEM$ requires defining these and other attributes for each
single particle.
In our study, we assigned the same particles parameters, except for the radius, to all the 
particles in each simulation. 
In other words, the DEM attributes may vary comparing different 
numerical setups, but are kept constant for each of simulation. 
Further, we tried to obtain simulations ideally perform, with the correct parameters, 
the same macroscopic behaviour of the experiments, the static angle of repose
($AOR$) and Schulze ring shear cell tester ($SRSCT$). That allowed us to later
compare numerical and experimental results, as suggested by Ai et al.
\cite{RefWorks:131}.\\
We could identify the most suitable value for each DEM parameter 
investigated performing large numbers of simulations. Each will have a different value. 
For instance, for the coefficient of rolling friction ($\mu_r$), ranging from
$0.01$ to $1.00$, a reasonable approach would involve $100$ simulations.
\begin{table}[h]
\centering
\begin{tabular}{l}
\hline 
    Radius \ac{R} (m)   \\ [5pt]

	Size distribution (-) \\ [5pt]

    Young's modulus \ac{E} (Pa)  \\ [5pt]

    Poisson's ratio \ac{nu} (-) \\ 
     Time step \ac{deltat} (s) \\ [5pt]
        \hline
     Coefficient of sliding friction \ac{mus} (-)\\  [5pt]
    Coefficient of rolling friction \ac{mur} (-) \\ [5pt]
    Coefficient of restitution \ac{CoR} (-)   \\ [5pt]
     Particle density $\ac{rhop} = \frac{mass ~ of ~ one ~ particle}{volume ~ of
     ~ one ~ particle}$ ($kg/m^3$)  \\ [5pt]
     Geometry factor \ac{dCylDp} (-)  \\ [5pt]
   
\hline
\end{tabular}
\caption[DEM parameters]{DEM parameters. The upper parameters were
identical in all simulations. The lower parameters were constant in each
simulation, but were varied between simulations.}
\label{tab:08DEMparameters}
\end{table}


Regrettably, as stated by Hoehner et al. \cite{RefWorks:86}, all the parameters listed in table \ref{tab:08DEMparameters}
contribute to define the numerical bulk behaviour. Performing and investigate the 
more than $10^8$ simulations required was out of the scope of this paper.
Instead, we executed a limited number of simulations and we investigated them. 
Statistical method as the Polynomial Chaos Expansion ($PCE$) or Bayesian
networks could be used to correlate numerical and experimental behaviour. 
Still, these methods are challenged by the non-linearity intrinsic to our system. 
Rather, as suggested by Vaferi et al. \cite{RefWorks:150} and Witten et al.
\cite{RefWorks:174}, we harnessed Artificial Neural Networks ($NN$) for their
stability and reliability.
Furthermore, the main aim of this work was to improve the characterization 
of several $DEM$ parameters for non-spherical particles. 
From the performed simulations bulk representative parameters were extracted. 
They provide the information for the output layer of the NN. Instead the DEM 
parameters of the same simulations provided the information for the input layer. 
Once trained, these NN were fed with random combinations of DEM parameters (~6M), 
and they provided numerical bulk representative parameters for each of these combinations. 
Those were compared with the experimental results. A portion of the combinations
(ca. 0.1\%)
had parameters matching with the experiments, i.e. the ratio of the NN behaviour
values over the experimental values differed less than $5\%$, as suggested by
Oberkampf et al. \cite{RefWorks:160}.
We kept these combinations as working solutions.
Since this study was supported by the metallurgical industry, the material examined was  sinter fine (0-3.15 mm).
Eventually, all the un-cohesive materials that can be tested through the $AOR$
and $SRSCT$ could be numerically characterized through this workflow, because no
further hypothesis have been imposed.
This flexibility would allow extending the procedure to different granular materials, further than the ones examined.
In the next section we will present the pre-requisites enforced. 
First, the setups we operated to experimentally define the materials. 
Later, the portion of the $DEM$ theory we focused onto. 
Then, the typology of $NN$ we harnessed for this work. 
Furthermore, a detailed methodology is presented. Finally, we show our results.
