%************************************************
\section{Introduction}
\label{sec:introduction}
%************************************************

Particles in various forms - ranging from raw materials to food grains and pharmaceutical powders - play a major role in a variety of industries, 
including process industry and metallurgy. In his book, Holdich \cite{RefWorks:117}
stated that "between 1 and 10\% of all the energy is used in comminution, i.e. the processes of crushing, grinding, milling, micronising". 
However, a univocal method to characterize these particles has so far not been established.
From the experimental point of view, the main issues are the difficult setups and the general reliability and reproducibility of the tests. 
From the numerical point of view, no general procedure is available, and the existence of a mathematically unique solution describing macro/micro particle contact has yet to be proved.
Moreover, in a recent study, Krantz et al. \cite{RefWorks:56} implied "that the dynamic properties of a powder cannot be applied to universally predict the static properties of a powder, and,
likewise, the static properties cannot be used to predict dynamic properties".\\
Discrete Element Method (DEM) simulations are widely used to understand particle behavior.
Cundall and Strack \cite{RefWorks:172} defined the $DEM$ as "a numerical model capable of describing the mechanical behaviour of assemblies of discs and spheres".
$LIGGGHTS$ (LAMMPS improved for general granular and granular heat transfer simulations) is one of the most powerful open source $DEM$ simulation software packages available.
The models it can analyze are described in detail in the literature, see Kloss
et al. \cite{RefWorks:136}.\label{par:overviewdemliggghts} In combination with
the shear cell tester simulation developed by Aigner et al. \cite{RefWorks:139},
$LIGGGHTS$ has correctly defined the coefficient of sliding friction for coarse round particles.\\
Since the bulk solid is represented by perfect spheres, the only parameter the software uses to describe its shape is the radius of the particle ($R$).
However, since the shape is one of the most relevant aspects defining particle behavior, we consider the coefficient of rolling friction ($\mu_r$) as an additional $DEM$ shape parameter.
The ulterior $DEM-micro$ parameters we investigated are the
$coefficient-of-restitution$ ($COR$) and the $particle-density$ ($\rho_p$),
given their centrality in the whole model.
$DEM$ simulations have recently been used to reduce the bias of the experiments, and more precise devices such as the Schulze ring shear cell tester (SRSCT)(see Schulze 
et al. \cite{RefWorks:104})
have been built.
A dedicated workflow that combines experiments and simulations must now be devised following the Design of Experiments method, as illustrated by Grossman and Del Vecchio \cite{RefWorks:116}.\\
The main goal of this new procedure should be the characterization of
non-spherical particles, through $DEM$ coefficients,
following standardizable steps.
With this objective in mind, we profited from the shear cell experimental and numerical setup in combination with $LIGGGHTS$ simulation to improve the accuracy and the range of
applicability of particle characterization.
Nevertheless, $DEM$ simulations require tens of thousand of particles to achieve the necessary reliability for a straight-forward trial-and-error calibration procedure.
The calibration compels to identify the $DEM-micro$ combination of parameters that numerically grants the same $bulk-macro$ behavior experimentally registered, measured as
$steady-state-flow/pre-shear ~ coefficient-of-internal-friction $ $ (\mu_{ie-ps})$, $incipient-flow/shear ~ coefficient-of-internal-friction $ $ (\mu_{ie-s})$ and $bulk ~ density ~
(\rho_b)$.
Thus, the time necessary to perform all the possible $DEM-micro$ parameters combinations became boundless.
In order to overcome this $doomed$ situation we decided to operate artificial
neural networks, as suggested by Antony et al. \cite{RefWorks:161}.
A limited number of combinations have been simulated, designed to maxime the representativity.
Following the indications of Vaferi et al. \cite{RefWorks:150}, \textit{feed forward Multilayer Perceptron Neural Networks (MLPNN)} have been handled.
Their trustworthiness, together with a backpropagation reinforcement learning training algorithm(scaled conjugate gradient), has been widely demonstrated in the literature, see
Haykin \cite{RefWorks:158}.
The DEM-micro parameters of the simulations have been used as inputs of the
Neural Networks ($NN$), while the bulk values and behavior as targets for them.
We then compared their outputs against the values provided by the shear cell
experiments (within a 5\% error), gaining the $DEM-micro$ coefficients range.
Oberkampf and Roy \cite{RefWorks:160} prescribe an ulterior validation step.
To accomplish his demand, we realized a $static-angle-of-repose$ ($SAOR$) experiment and $SAOR$ simulation.
Furthermore, we performed the $SAOR$ simulation, again with the same limited number of combinations.
These allowed to determine one ulterior $bulk-macro$ behavior property, the
$angle-of-repose$ ($AOR$).
Likewise, we then trained $NN$ and optimized the number of its neurons.
Later, this new trained $NN$ received as insertion the $DEM-micro$ coefficients
range previously determined.
We then compared their outputs against the values provided by the $SAOR$ experiment (again within a 5\% error), gaining a narrower $DEM-micro$ coefficients range.
Since this study was supported by the metallurgical industry, the materials
examined were: silibeads (2 mm), coke, iron ore, limestone, sinterfine (all
0-3.15 mm).
For the same reason, cohesive materials have been excluded from this study.\\ \label{par:materials}