%************************************************
\section{Introduction}
\label{sec:introduction}
%************************************************

Particles in various forms - ranging from raw materials to food grains and pharmaceutical powders - 
play a major role in a variety of industries, including process industry and metallurgy. 
Discrete Element Method ($DEM$) simulations are widely used to picture particle
behaviour in these granular processes.
From the original formulation by Cundall and Strack \cite{RefWorks:172}
the method has been further refined in the hard-spheres and soft-spheres
approaches, explained in details, respectively, by Vu-Quoc and Zhang
\cite{RefWorks:148} and Di Renzo and Di Maio \cite{RefWorks:145}.
Di Renzo and Di Maio \cite{RefWorks:145} confirm that the soft-spheres approach
allows an easier implementation of the inter-particle forces.
the approach has been implemented in the open source code $LIGGGHTS$ by
Kloss et al. \cite{RefWorks:136}.
Further, Ai et al. \cite{RefWorks:131} demonstrated that a series of DEM
models can reproduce the behaviours of non-spherical materials.
These models depict the effect of the particles' shape with the coefficient
of rolling friction.
The reliability of this modelization has been further granted by Wensrich and 
Katterfeld \cite{RefWorks:87}.
Like us, both Alenzi et al. \cite{RefWorks:91} and Aigner et al.
\cite{RefWorks:139} dealt with the issue of identifying the parameters for the
inter-particle forces evaluated with these methods.
The parameters, empirically by nature, that depict this phenomenon are challenging
to obtain experimentally.
So, they used shear cell testers to perform calibration experiments, 
focusing on the bulk behaviour of the materials analysed.
The calibration consisted in using this macro behaviour to 
investigate the relationship with the particle-particle $DEM$
contact parameters, a methodology that we also followed.
It was out of scope to investigate the properties of each
particle separately, like shape, Young modulus ($E$), coefficient of Poisson ($\nu$) of each sample.\\
Nevertheless, the $DEM$ requires defining these and other attributes for each
single particle.
In our study, we assigned the same particles parameters, except for the radius, to all the 
particles in each simulation. 
In other words, the $DEM$ attributes may vary comparing different 
numerical setups, but are kept constant for each of simulation. 
Further, we tried to obtain simulations ideally perform, with the correct parameters, 
the same macroscopic behaviour of the experiments, the static angle of repose
($AOR$) and Schulze ring shear cell tester ($SRSCT$). That allowed us to later
compare numerical and experimental results, as suggested by Ai et al.
\cite{RefWorks:131}.\\
Regrettably, as stated by Hoehner et al. \cite{RefWorks:86}, all the parameters
listed in table \ref{tab:08DEMparameters}, defined in details in
\ref{subsec:dem} and \ref{subsec:srsctsimulation}, contribute to define the
numerical bulk behaviour.
This increase exponentially the complexity of calibration, in the meaning of 
identify the most suitable value for each $DEM$ parameter.
Ideally, we could perform huge numbers of simulations and then investigate them.
Each will have a different value for each $DEM$ parameter.
For instance, for the coefficient of rolling friction ($\mu_r$), ranging from
$0.01$ to $1.00$, a reasonable approach would involve $100$ simulations.
Performing and investigate the 
more than $10^8$ simulations required was out of the scope of this paper.
Instead, as suggested by Vaferi et al. \cite{RefWorks:150} and Witten et al.
\cite{RefWorks:174}, we harnessed Artificial Neural Networks ($NN$) for their
stability and reliability with non-linear systems like ours.
Furthermore, the main aim of this work was to improve the characterization 
of several $DEM$ parameters for non-spherical particles. 
From the performed simulations bulk representative parameters were extracted. 
They provide the information for the output layer of the $NN$. Instead the $DEM$ 
parameters of the same simulations provided the information for the input layer. 
Once trained, these $NN$ were fed with random combinations of $DEM$ parameters
(about 6M), and they provided numerical bulk representative parameters for each
of these combinations.
Those were compared with the experimental results. A portion of the combinations
(ca. 0.1\%) had parameters matching with the experiments, i.e. the ratio of the $NN$ behaviour
values over the experimental values differed less than $5\%$, as suggested by
Oberkampf et al. \cite{RefWorks:160}.
We kept these combinations as working solutions. \\
The choice of our raw materials is broad, and covers e.g. coke, sinter, and iron
ore in the metallurgical industry, polypropylene pellets in the plastic industry,
and powders in the pharmaceutical industry.
Since this study was supported by the metallurgical industry, we decided to
focus on non-cohesive materials, and the material examined was sinter fine
(0-3.15 mm).
Eventually, all the un-cohesive materials that can be tested through the $AOR$
and $SRSCT$ could be numerically characterized through this workflow, because no
further hypothesis have been imposed.
This flexibility would allow extending the procedure to different granular materials, further than the ones examined.
In the next section we will present the pre-requisites enforced. 
First, the setups we operated to experimentally define the materials. 
Later, the portion of the $DEM$ theory we focused onto. 
Then, the typology of $NN$ we harnessed for this work. 
Furthermore, a detailed methodology is presented. Finally, we show our results.
\begin{table}[h]
\centering
\begin{tabular}{l}
\hline 
    Radius \ac{R} (m)   \\ [5pt]

	Size distribution (-) \\ [5pt]

    Young's modulus \ac{E} (Pa)  \\ [5pt]

    Poisson's ratio \ac{nu} (-) \\ 
     Time step \ac{deltat} (s) \\ [5pt]
        \hline
     Coefficient of sliding friction \ac{mus} (-)\\  [5pt]
    Coefficient of rolling friction \ac{mur} (-) \\ [5pt]
    Coefficient of restitution \ac{CoR} (-)   \\ [5pt]
     Particle density $\ac{rhop} = \frac{mass ~ of ~ one ~ particle}{volume ~ of
     ~ one ~ particle}$ ($kg/m^3$)  \\ [5pt]
     Geometry factor \ac{dCylDp} (-)  \\ [5pt]
   
\hline
\end{tabular}
\caption[DEM parameters]{DEM parameters. The upper parameters were
identical in all simulations. The lower parameters were constant in each
simulation, but were varied between simulations.}
\label{tab:08DEMparameters}
\end{table}


