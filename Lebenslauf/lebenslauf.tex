\documentclass[11pt,a4paper,sans]{moderncv}
\moderncvstyle{banking}
\moderncvcolor{blue}
\usepackage{lipsum}
\usepackage[utf8]{inputenc}
\usepackage[scale=0.75]{geometry}

\patchcmd{\maketitle}
  {\hfil}
  {\hspace*{0.05\textwidth}}
  {}
  {}
\patchcmd{\maketitle}
  {\setlength{\maketitlewidth}{0.8\textwidth}}
  {\setlength{\maketitlewidth}{0.67\textwidth}}
  {}
  {}
\patchcmd{\maketitle}
  {\\[2.5em]}
  {\hfil\raisebox{-2.7cm}{\framebox{\includegraphics[width=\@photowidth]{\@photo}}}\\[2.5em]}
  {}
  {}

% personal data
\name{Luca}{Benvenuti, D.I.}
  % \extrainfo{additional information}
%\title{Resume' title}
\address{Aubrunnerweg 25}{4040 Linz}{Austria}
\phone[mobile]{+43~680~220~2490}
% \phone[fixed]{+2~(345)~678~901}
% \phone[fax]{+3~(456)~789~012}
\email{lucabenvenuti@gmail.com}
\homepage{http://goo.gl/ZNy4hX}
\social[linkedin]{lucabenvenuti}
% \social[twitter]{jdoe}
% \social[github]{jdoe}
% \extrainfo{additional information}

\photo[3.5cm]{luca}

\quote{Ich liebe Herausforderungen: Im Jahr 2013 war ich in Libyen. 
Ich bin lernwillig und an pers\"{o}nlicher Weiterbildung interessiert: Deshalb
absolviere ich ein Doktoratsstudium.
Als Ingenieur bin ich ein logisch denkender Mensch.}

\begin{document}

\makecvtitle

\section{Pers\"{o}nliche Angaben}
\cventry{Nationalit\"{a}t}{Italien}{}{}{}{}{}     
\cventry{Geburt}{12. Mai 1988}{Desenzano del Garda}{}{}{}{}
% \cventry{Familienstand}{verheiratet, 2 Kinder}{}{}{}{}
% \cventry{Ehemann}{Max Mustermann}{Musterberuf}{}{}{}

%\cventry{}{}{}{}{}{}

\section{Berufserfahrung}
% \subsection{Schule}
% \cventry{1971--1975}{Grundschule}{Musterschule, Berlin}{}{}{} % arguments 3 to 6 are optional
% \cventry{1975--1984}{Gymnasium}{Mustergymnasium}{Berlin}{}{}
% \cventry{07/1984}{allgemeine Hochschulreife}{}{}{}{}
% 
% \subsection{Akademische Ausbildung und beruflicher Werdegang}
\cventry{04/2013--jetzt}{Johannes Kepler Universit\"{a}t}{Particulate Flow
Modelling}{}{}{Wissenschaftlicher Mitarbeiter}

% {Wissenschaftlicher
% Mitarbeiter}{Johannes Kepler Universit\"{a}t}{}{}{Particulate Flow Modelling.}

\cventry{01/2013--04/2013}{LISCO expansion}{Ferretti
International}{}{}{Bauleiter und Quantity Surveyor}

\cvline{}{}{}

\section{Ausbildung}

\cventry{04/2013--jetzt}{Johannes Kepler Universit\"{a}t}{Particulate Flow
Modelling}{}{}{Doktoratsstudium Mechatronik}

\cventry{09/2010--12/2012}{Politecnico di
Milano}{Bauingenieurwesen}{}{}{Masterstudium}

\cventry{09/2007--12/2010}{Politecnico di
Milano}{Bauingenieurwesen}{}{}{Bacherlorstudium}

\cventry{09/2002--07/2007}{Liceo Bagatta}{Gymnasium}{}{}{}

\cvline{}{}{}

\section{Projekte}

\cventry{04/2013--jetzt}{Kooperationen:}{Particulate Flow
Modelling}{}{}{Voestalpine, Primetals, TU Braunschweig}

\cventry{01/2013--04/2013}{LISCO expansion}{Ferretti
International}{}{}{Gro{\upshape{\ss}}projekt zur Erh\"{o}rung der Stahlproduktion in der
LISCO Produktionsanlage um \\
das Dreifache von 1,5 auf 4,5 Mio. Tonnen pro Jahr.
Gro{\upshape{\ss}}es, internationales Team \\
(200 Mitarbeiter/innen aus Bangladesch, 
\"{A}gypten, Indien, Italien, Libyen, Marokko, \\
Tansania, Tunesien und der Ukraine).}

\cvline{}{}{}

\section{Qualifikationen}
%\subsection{EDV Anwendungen}
\cventry{}{Ausgezeichnet}{}{}{}{Excel\\
Matlab\\
MS Office\\
Numerical Analysis\\
PowerPoint\\
Simulations}

\cventry{}{Fortgeschritten}{}{}{}{Bauleitung\\
Liggghts\\
Projekt-Sch\"{a}tzung\\
Projektplanung\\
Projektsteuerung\\
Simulations}

\cventry{}{Fortgeschritten}{}{}{}{C++\\
Fotran\\
Java\\
Python}


\cvline{}{}{}

\section{Sprachkenntnisse}
\cvlanguage{Italienisch}{Muttersprache}{}
\cvlanguage{Englisch}{Flie{\upshape{\ss}}end}{}
\cvlanguage{Deutsch}{Gut}{}
\cvlanguage{Franz\"{o}sisch}{Gut}{}
\cvlanguage{Spanisch}{Grundkenntnisse}{}

\cvline{}{}{}

% \section{Interessen}
% \cventry{Hobbys}{Angeln}{Angelschein seit 1979}{}{}{}
% \cventry{}{Tischtennis}{Tischtennisverein Nordlichtchen}{}{}{}
% \cventry{}{Tanzsportverein Holdrio}{Ehrenvorsitzende}{}{}{\\
% Dissertationsthema: Bezeichnung DEM Simulationsparameter durch K\"{u}nstliche
% Neuronale Netze und Bulk-Experimente. \\
% Interessengebiete: Engineering (Zivil-, geotechnische, Materialien,
% mechanische), neuronale Netze. \\
% Kooperationen: Voestalpine, Primetals, TU Braunschweig. }
% \cvline{}{}{}

\section{Motivation}

% \cventry{}{}{}{}{}{}
% 
% \cvline{}{}{}
% \cvline{}{}{}
% \cvline{}{}{}

Ich kann auf Erfolge in kleinen und gro{\upshape{\ss}}en
Teams in vielen L\"{a}ndern arbeiten. Mit Innovationsgeist \\
und meinen
analytischen F\"{a}higkeiten m\"{o}chte ch zu den Erfolgen Ihres Teams
beitragen.\\

% \cvline{}{}{}
% \cvline{}{}{}
% \cvline{}{}{}
 \cvline{}{}{}

Linz, am 3. Dezember 2015

\end{document} 