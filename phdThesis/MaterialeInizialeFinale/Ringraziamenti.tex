% !TEX encoding = UTF-8
% !TEX TS-program = pdflatex
% !TEX root = ../Tesi.tex
% !TEX spellcheck = it-IT

%*******************************************************
% Ringraziamenti
%*******************************************************
\cleardoublepage
\phantomsection
\pdfbookmark{Ringraziamenti}{ringraziamenti}

\begin{flushright}{\slshape    
	One equal temper of heroic hearts,\\
	Made weak by time and fate, but strong in will\\
	To strive, to seek, to find, and not to yield.} \\ \medskip
    --- Alfred Tennyson
\end{flushright}


\bigskip

\begingroup

\renewcommand{\chaptermark}[1]{\markboth{#1}{}}

\fancyhf{}
\fancyhead[LE,RO]{\thepage}
\fancyhead[RE,LO]{Ringraziamenti}
%\fancyhead[LO]{\nouppercase{\rightmark}}
\renewcommand{\headrulewidth}{0.5pt}

\renewcommand{\footrulewidth}{0pt}
\fancyheadoffset{0\columnwidth}



\let\clearpage\relax
\let\cleardoublepage\relax
\let\cleardoublepage\relax

\chapter*{Ringraziamenti}

Il primo ringraziamento va sicuramente ai componenti della mia famiglia, Daniela, Fabrizio e Nicol�, che mi hanno permesso di raggiungere l'obiettivo della laurea magistrale in ingegneria, sostenendomi sia moralmente che materialmente, per questi cinque lunghi anni (e sopportandomi anche!).\\

Voglio quindi ringraziare i relatori: il Prof. Claudio di Prisco, le sue approfondite lezioni e le lunghe discussioni davanti ai risultati del software mi hanno permesso di esaminare con spirito critico i problemi da affrontare e i percorsi da seguire, e contemporaneamente le sue spiegazioni sostenevano le mie interpretazioni pi� deboli o rimettevano sulla retta via un sentiero ingegneristicamente tortuoso; il Prof. Castellanza, che seppure fra mille impegni, � riuscito sempre a considerare i miei risultati e a suggerirmi nuove ipotesi di modellazione, oltre a spiegarmi il funzionamento del software Midas GTS, adorato profondamente da tutto il Dipartimento di Ingegneria Strutturale del Politecnico, e mi ha anche prestato la sua workstation per eseguirlo; quindi l'Ing. Frigerio, che oltre a collaborare con me per lo sviluppo del modello, e a permettermi di usare la parte pi� grafica del software, ha analizzato anche i risultati dei primi modelli completi antecedentemente alla visione da parte dei Professori; infine, anche se non compreso nel novero dei relatori, il Prof. Berry dell'universit� di Bologna, che ha fornito un'ulteriore punto di vista.\\

Ringrazio quindi tutti coloro che mi hanno fornito i dati sperimentali per la determinazione delle propriet� dei materiali, Flessati, Frassinella, Castelletti (con cui ho svolto la tesi triennale), Breviario, Ossola e Bedani, ultimo, ma non ultimo, che ringrazio anche per avermi passato i primi modelli di tentativo.\\

Ringrazio anche il geologo Gianmarco Orlandi, che ha valutato i rischi idrogeologici, progettato gli interventi di salvaguardia, e ha collaborato per indirizzare la modellazione su risultati che potessero essere concretamente utilizzati, il suo collega Spada, che dall'alto della sua esperienza pluriennale mi ha dato preziosi suggerimenti e indicazioni, e il geologo Fabrizio Giorgini della Subsoil, che ha materialmente effettuato i carotaggi.\\

Va quindi ringraziato l'Ing. Bianchini della Tencate s.p.a., che mi ha fornito con brevissimo preavviso le schede tecniche necessarie per studiare l'intervento e sviluppare l'analisi, l'Ing. Spini che ha dimensionato le cerchiature e l'Ing. Giussani che mi ha aiutato ad idealizzarle.\\

Luca Flessati e Francesco Frassinella verranno ulteriormente ringraziati per avermi sopportato l'ultimo anno in appartamento con loro, e di questo ringrazio anche Elisabetta Frassinella, aiutato con la revisione finale e dato una grande mano sia nella parte iniziale della tesi, sia nella preparazione dei progetti per gli ultimi esami.\\

Ringrazio nuovamente anche mio fratello Nicol� e mia mamma Daniela, che hanno corretto gli errori d'ortografia e battitura sparsi per la tesi.\\

Un ringraziamento va anche a Stefano Sandrone, che, oltre ad aver revisionato i miei abstract, � stato un caro amico e compagno d'avventure in questi cinque anni.\\

Un ringraziamento finale va anche alla workstation, fusa per il troppo uso, ma non alla Fujitsu, che poteva fare un modello pi� affidabile.

\bigskip
 
\noindent\textit{\myLocation, \myTime}
\hfill L.~B.

\endgroup

