% !TEX encoding = UTF-8
% !TEX TS-program = pdflatex
% !TEX root = ../Tesi.tex
% !TEX spellcheck = it-IT

%*******************************************************
% Sommario+Abstract
%*******************************************************
\cleardoublepage
\phantomsection
\pdfbookmark{Abstract}{Abstract}
\begingroup
\let\clearpage\relax
\let\cleardoublepage\relax
\let\cleardoublepage\relax


\renewcommand{\chaptermark}[1]{\markboth{#1}{}}

\fancyhf{}
\fancyhead[LE,RO]{\thepage}
\fancyhead[RE]{\nouppercase{\leftmark}}
\fancyhead[LO]{\nouppercase{\rightmark}}
\renewcommand{\headrulewidth}{0.5pt}

\renewcommand{\footrulewidth}{0pt}
\fancyheadoffset{0\columnwidth}



% \chapter*{Abstract}
% 
% L'ex cava Prete Santo di San Lazzaro di Savena, complesso estrattivo ormai abbandonato, � da tempo oggetto di attenzione da parte delle autorit�, in quanto si � osservato che il materiale gesso che costituisce la miniera � soggetto a degradazione progressiva.\\
% Infatti ci� che le autorit� temono � la propagazione del processo sino a coinvolgere, seppure marginalmente, le unit� abitative poste in superficie.\\
% In questa tesi sono contestualizzate e discusse le evidenze sperimentali ottenute esaminando il materiale prelevato dall'ex cava. Esse hanno permesso di determinare i valori di resistenza del materiale in sito e di effettuare analisi numeriche 3D agli Elementi Finiti, cos� da definire i possibili scenari di collasso.\\
% Sono stati anche considerati gli interventi di mitigazione del rischio, basati sull'utilizzo di cerchiature in calcestruzzo e geosintetici ad alta resistenza.
% %\vfill

\selectlanguage{english}
%\pdfbookmark{Abstract in english}{Abstract in english}
\pdfbookmark{Abstract}{Abstract}
\chapter*{Abstract}

An abandoned gypsum cave, Prete Santo in the town of San Lazzaro di Savena (BO), has been under observation by local authorities for many years, because the gypsum, which is the mine's constituent, is subjected to gradual degradation.\\
The authorities believe that the propagation could extend until the involvement, even if marginally, of the the residential area on its surface.\\
Here we discussed the experimental evidence gained through focused investigation of the directly collected cave materials, and calculated
the in situ strengh values allowing to perform a 3D Finite Element analysis to model the possible scenarios.\\
Moreover, we examined the main risk reduction action plans, based on the application of circular structures in concrete and
high-performance geocomposites.

\selectlanguage{italian}

\endgroup			

\vfill

%, whose contributions were then further modeled in case of collapse, adopting a reduction-of-external-subsidence perspective
%, a seguito della quale i pilastri che sostengono il sistema perdono rigidezza e resistenza, con il rischio di propagare cedimenti in %superficie.\\
%
%constitutes a significant hydrogeological risk because of the presence of residential areas on its top. Gypsum degradation leads to loss %of stiffness and strengh in the mine pillars, thus potentially propagating subsidence on surface.\\