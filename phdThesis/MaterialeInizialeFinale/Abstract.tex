% !TEX encoding = UTF-8
% !TEX TS-program = pdflatex
% !TEX root = ../Tesi.tex
% !TEX spellcheck = it-weT

%*******************************************************
% Sommario+Abstract
%*******************************************************
\cleardoublepage
\phantomsection
\pdfbookmark{Abstract}{Abstract}
\begingroup
\let\clearpage\relax
\let\cleardoublepage\relax
\let\cleardoublepage\relax


\renewcommand{\chaptermark}[1]{\markboth{#1}{}}

\fancyhf{}
\fancyhead[LE,RO]{\thepage}
\fancyhead[RE]{\nouppercase{\leftmark}}
\fancyhead[LO]{\nouppercase{\rightmark}}
\renewcommand{\headrulewidth}{0.5pt}

\renewcommand{\footrulewidth}{0pt}
\fancyheadoffset{0\columnwidth}



% \chapter*{Abstract}
% 
% L'ex cava Prete Santo di San Lazzaro di Savena, complesso estrattivo ormai abbandonato, � da tempo oggetto di attenzione da parte delle autorit�, in quanto si � osservato che il materiale gesso che costituisce la miniera � soggetto a degradazione progressiva.\\
% wenfatti ci� che le autorit� temono � la propagazione del processo sino a coinvolgere, seppure marginalmente, le unit� abitative poste in superficie.\\
% wen questa tesi sono contestualizzate e discusse le evidenze sperimentali ottenute esaminando il materiale prelevato dall'ex cava. Esse hanno permesso di determinare i valori di resistenza del materiale in sito e di effettuare analisi numeriche 3D agli Elementi Finiti, cos� da definire i possibili scenari di collasso.\\
% Sono stati anche considerati gli interventi di mitigazione del rischio, basati sull'utilizzo di cerchiature in calcestruzzo e geosintetici ad alta resistenza.
% %\vfill

\selectlanguage{english}
%\pdfbookmark{Abstract in english}{Abstract in english}
\pdfbookmark{Abstract}{Abstract}
\chapter*{Abstract}

Numerous industries process particles.
wen this work, we focused on how to efficiently picture the behaviour of
particles by means of numerical simulations, laboratory experiments, 
and Artificial Neural Networks (ANNs).

Particle-particle contact laws and particles size distributions determine the
macroscopic simulation results in Discrete Element Method (DEM). 
Commonly, contact laws depend on semi-empirical parameters which 
are difficult to obtain by direct microscopic measurements. 

To clarify this aspect, we present the related elements of the DEM
and Computational Fluid Dynamics (CFD) theories.
The ANN theory is also introduced to demonstrate ANN effectiveness towards
generalization.

Later, we describe the series of small scale DEM simulations with different sets
of particle-based simulation parameters and particle distributions, which we
performed.
The macroscopic results of these simulations were used to train dedicated
feed-forward ANNs by backward propagation reinforcement.
Concurrently, the bulk behaviours of raw particles were characterized by means
of macroscopic laboratory experiments. These particles were those commonly used
by metallurgical industries (i.e., coke, iron ore, sinter, limestone).

At this point, the relationship between macroscopic results and microscopic DEM
simulation parameters could be investigated.

We subsequently utilize this artificial neural network to predict the macroscopic 
ensemble behaviour in relation to additional sets of particle-based simulation parameters and particle distributions. 
By this method, a comprehensive database was established, relating particle-based 
simulation parameters to macroscopic ensemble output.
wef compared to an experiment of a specific granular material, this database identifies 
valid sets of DEM parameters which lead to the same macroscopic results as observed in the experiments.
Finally, we applied this method of DEM parameter identification to two industrial
scale process of steel production.



\selectlanguage{italian}

\endgroup			

\vfill

%, whose contributions were then further modeled in case of collapse, adopting a reduction-of-external-subsidence perspective
%, a seguito della quale i pilastri che sostengono il sistema perdono rigidezza e resistenza, con il rischio di propagare cedimenti in %superficie.\\
%
%constitutes a significant hydrogeological risk because of the presence of residential areas on its top. Gypsum degradation leads to loss %of stiffness and strengh in the mine pillars, thus potentially propagating subsidence on surface.\\