% !TEX encoding = UTF-8
% !TEX TS-program = pdflatex
% !TEX root = ../Tesi.tex
% !TEX spellcheck = it-IT

%************************************************
\chapter{DEM Parameters}
\label{cap:demparameters}
%************************************************

\lipsum[1]

\section{Literature Values}
\label{sec:literaturevalues}

\lipsum[1]

\section{Particle Distribution}
\label{sec:particledistribution}

\lipsum[1]

\subsection{coke}
\label{subsec:coke}

\lipsum[2]

\section{Bulk Density}
\label{sec:bulkdensity}


\lipsum[1]


\section{Angle of Repose (p-p) - Small Scale}
\label{sec:aor}


\lipsum[1]

\section{Angle of Repose (p-p) - Large Scale}
\label{sec:aorlargescale}


\lipsum[1]

\section{Angle of Repose Simulation}
\label{sec:aorsimulation}


\lipsum[1]

\section{Maximum Static Angle (p-w)}
\label{sec:msa}
%************************************************

\lipsum[1]

\section{Maximum Static Angle Simulation}
\label{sec:msasimulation}
%************************************************

\lipsum[1]

\section{Schulze Ring Shear Cell tester (p-p)}
\label{sec:SRSCT}
%************************************************

\lipsum[1]

\section{Jenike Shear Cell tester}
\label{sec:jsct}
%************************************************

\lipsum[1]

\subsection{p-p}
\label{subsec:JSCTpp}

\lipsum[2]

\subsubsection{Instructions}
\label{subsubsec:instructions}

1. Scope*
1.1 This method covers the apparatus and procedures for measuring the cohesive strength of bulk solids during both continuous flow and after storage at rest. In addition, measurements of internal friction, bulk density, and wall friction on various wall surfaces are included.\\
1.2 This standard is not applicable to testing bulk solids that do not reach the steady state requirement within the travel limit of the shear cell. It is impossible to classify ahead of time which bulk solids cannot be tested, but one example may be those consisting of highly elastic particles. \\
1.3 The values stated in SI units are to be regarded as standard.\\
1.4 The most common use of this information is in the design of storage bins and hoppers to prevent flow stoppages due to arching and ratholing, including the slope and smoothness of hopper walls to provide mass flow. Parameters for structural design of such equipment also may be derived from this data.\\
3. Terminology \\
3.1 Definitions: \\
3.1.1 Definitions of terms used in this test method are in accordance with Terminology D653. \\
3.1.2 adhesion test, a static wall friction test with time consolidation. \\
3.1.3 angle of internal friction, $\phi_e$, the angle between the axis of normal stress (abscissa) and the tangent to the yield locus. \\
3.1.4 angle of wall friction, $\phi_w$, the arctan of the ratio of the wall shear stress to the wall normal stress. \\
3.1.5 bin, a container or vessel for holding a bulk solid, frequently consisting of a vertical cylinder with a converging hopper. Sometimes referred to as silo, bunker, or elevator. \\
3.1.6 bulk density,  $\rho_b$, the mass of a quantity of a bulk solid divided by its total volume. \\
3.1.7 bulk solid, an assembly of solid particles handled in sufficient quantities that its characteristics can be described by the properties of the mass of particles rather than the characteristics of each individual particle. May also be referred to as granular material, particulate solid, or powder. Examples are sugar, flour, ore, and coal. \\
3.1.8 bunker, synonym for bin, but sometimes understood as being a bin without any or only a small vertical part at the top of the hopper. \\
3.1.9 cohesive strength, synonym for unconfined yield strength. \\
3.1.10 consolidation, the process of increasing the strength of a bulk solid. \\
3.1.11 critical state, a state of stress in which the bulk density of a bulk solid and the shear stress in the shear zone remain constant. \\
3.1.12 effective angle of friction, $\delta$, the inclination of the effective yield locus (EYL). \\
3.1.13 effective yield locus (EYL), straight line passing through the origin of the $\sigma, \tau$-plane and tangential to the steady state Mohr circle, corresponding to steady state flow conditions of a bulk solid of given bulk density. \\
3.1.14 elevator, synonym for bin, commonly used in the grain industry. \\
3.1.15 failure (of a bulk solid), plastic deformation of an overconsolidated bulk solid subject to shear, causing dilation and a decrease in strength. \\
3.1.16 flow, steady state, continuous plastic deformation of a bulk solid at critical state.  \\
3.1.17 flow function, FF, the plot of unconfined yield strength versus major consolidation stress for one specific bulk solid. \\
3.1.18 granular material, synonym for bulk solid. \\
3.1.19 hopper, the converging portion of a bin. \\
3.1.20 major consolidation stress, $\sigma_1$, the major principal stress given by the Mohr stress circle of steady state flow. This Mohr stress circle is tangential to the effective yield locus. \\
3.1.21 Mohr stress circle, the graphical representation of a state of stress in coordinates of normal and shear stress, that is, in the $\sigma, \tau$-plane. \\
3.1.22 normal stress, $\sigma$, the stress acting normally to the considered plane. \\
3.1.23 overconsolidated specimen, a condition in which the shear force passes through a maximum and then decreases during preshear. \\
3.1.24 particulate solid, synonym for bulk solid. \\
3.1.25 powder, synonym for bulk solid, particularly when the particles of the bulk solid are fine. \\
3.1.26 silo, synonym for bin. \\
3.1.27 shear test, an experiment to determine the flow properties of a bulk solid by applying different states of stress and strain to it. \\
3.1.28 shear tester, an apparatus for performing shear tests. \\
%##3.1.29 time angle of internal friction, ft, inclination of the time yield locus of the tangency point with the Mohr stress circle passing through the origin. \\
%##3.1.30 time yield locus, the yield locus of a bulk solid which has remained at rest under a given normal stress for a certain time. \\ 
3.1.31 unconfined yield strength, $f_c$, the major principal stress of the Mohr stress circle being tangential to the yield locus with the minor principal stress being zero.A synonym for compressive strength. \\
3.1.32 underconsolidated specimen, a condition in which the shear force increases continually during preshear. \\
3.1.33 wall normal stress, $\sigma_w$, the normal stress present at a confining wall. \\
3.1.34 wall shear stress, $\tau_w$, the shear stress present at a confining wall. \\
3.1.35 wall yield locus,  a plot of the wall shear stress versus wall normal stress. The angle of wall friction is obtained from the wall yield locus as the arctan of the ratio of the wall shear stress to wall normal stress. \\
3.1.36 yield locus, plot of shear stress versus normal stress at failure. The yield locus (YL) is sometimes called the instantaneous yield locus to differentiate it from the time yield locus. \\
 
4. Summary of Test Method \\
4.1 A representative sample of bulk solid is placed in a shear cell of specific dimensions. This specimen is preconsolidated by twisting the shear cell cover while applying a compressive load normal to the cover.
4.2 When running an instantaneous
% or time shear test
, a normal load is applied to the cover, and the specimen is presheared until a steady state shear value has been reached. \\
4.3 An instantaneous test is run by shearing the specimen under a reduced normal load until the shear force goes through a maximum value and then begins to decrease. \\
%4.4 A time shear test is run similarly to an instantaneous
%shear test, except that the specimen is placed in a consolidation
%bench between preshear and shear.
4.5 A wall friction test is run by sliding the specimen over a coupon of wall material and measuring the frictional resistance as a function of normal, compressive load. \\
%4.6 A wall friction time test involves sliding the specimen
%over the coupon of wall material, leaving the load on the
%specimen for a predetermined period of time, then sliding it
%again to see if the shearing force has increased.

5. Significance and Use \\
5.1 Reliable, controlled flow of bulk solids from bins and hoppers is essential in almost every industrial facility. Unfortunately, flow stoppages due to arching and ratholing are common. Additional problems include uncontrolled flow (flooding) of powders, segregation of particle mixtures, useable capacity which is significantly less than design capacity, caking and spoilage of bulk solids in stagnant zones, and structural failures. \\
5.2 By measuring the flow properties of bulk solids, and designing bins and hoppers based on these flow properties, most flow problems can be prevented or eliminated. \\
5.3 For bulk solids with a significant percentage of particles (typically, one third or more) finer than about 6 mm (1/4 in.), the cohesive strength is governed by the fines (6mm fraction). For such bulk solids, cohesive strength and wall friction tests may be performed on the fine fraction only. \\
NOTE 1: The quality of the result produced by this test method is dependent on the competence of the personnel performing it, and the suitability of the equipment and facilities used. \\


6. Apparatus
6.1 The Jenike shear cell is shown in \textbf{Fig. 1}. It consists of a base (1), shear ring (2), and shear lid (3), the latter having a bracket (4) and pin (5). Before shear, the ring is placed in an offset position as shown in \textbf{Fig. 1}, and a vertical force $F_v$ is applied to the lid, and hence, to the particulate solid within the cell by means of a weight hanger (6) and weights (7). A horizontal force is applied to the bracket by a mechanically driven measuring stem (8). \\
6.2 It is especially important that the shear force measuring stem acts on the bracket in the shear plane (plane between base and shear ring) and not above or below this plane. \\
6.3 The dimensions of the Jenike shear cells supplied by Jenike and Johanson, Inc. are given in the first two columns of the table in \textbf{Fig. 4}. These dimensions have been derived from English units. The standard size Jenike shear cell is made from aluminum or stainless steel, and a smaller 63mm diameter cell made from stainless steel is also available. Since the actual dimensions are not believed to be critical, the same results could be obtained with a shear cell of the dimensions listed in the third column of the table in \textbf{Fig. 4} or with other shear cells of different sizes provided that proportions of these dimensions are maintained approximately. In addition, \textbf{the shear cell diameter must be at least 20 times the maximum particle size of the bulk solid being tested}. Besides the shear cell, the complete shear tester includes a force transducer which measures the shear force $F_s$, an amplifier and a recorder, a motor driving the force measuring stem, a twisting wrench, a weight hanger, 
%a time consolidation bench, 
an accessory for mounting wall material sample plates, and a calibrating device. A spatula having a blade at least 50 % longer than the diameter of the shear cell, 
%and at least a 10-mm width, 
is needed. The force  transducer should be capable of measuring a force up to 300 N with a precision of 0.1 % of full scale. The signal from the force transducer is conditioned by an amplifier and shown on
a recorder. The motor driving the force measuring stem advances the stem at a constant speed in the range from 1 to 3 mm/min. \\

7. Specimen Preparation \\
7.1 Filling the Cell \textbf{Fig. 8}: \\
7.1.1 Place the shear ring on the base in the offset position shown in \textbf{Fig. 1} and gently press the ring with the fingers against the locating screws (10) as shown in Fig. 3 and Fig. 9. Set these  screws to give an overlap of approximately 3 mm for standard cell sizes and to ensure that the axis of the cell is aligned with the force measuring stem. Then place the mould ring (11) on the shear ring. \\
7.1.2 Fill the assembled cell uniformly in small horizontal layers by a spoon or spatula without applying force to the surface of the material until the material is somewhat over the top of the mould ring. The filling should be conducted in such a way as to ensure that there are no voids within the cell, particularly at ??????? \textbf{Fig. 8} where the ring and the base overlap. Remove excess material in small quantities by scraping off with a blade (1). The blade should be scraped across the ring in a zig-zag motion. Take care not to disturb the position of the ring on the base. For scraping, a rigid sharp  straight blade should be used, and, during scraping, the blade should be tilted as shown in \textbf{Fig. 8}. \\
7.2 Preconsolidation:\\
7.2.1 Place the twisting or consolidation lid (12) shown in \textbf{Fig. 9} on the leveled surface of the material in the mould, then place the hanger (6) on the twisting lid with weights (7) of mass $m_{Wtw}$ being hung from the hanger. See \textbf{Fig. 1}. Lower the lid, hanger, and weights as slowly as possible to minimize aerated material being ejected from the cell. \\
7.2.2 Visually observe the vertical movement of the lid as the material of the cell is compressed.Wait until this movement appears to stop. \\
7.2.3 Remove the weights, hanger, and twisting lid. Fill and level the space above the compressed material as during filling. \\
NOTE 3: As will be mentioned later, this refilling procedure may not be necessary at all or may need to be performed several times, depending on the compressibility of the powder being tested. This operation determines what height of compacted material will have to be scraped off the ring after twisting. \\
7.3 Twisting: \\
7.3.1 Place the twisting lid (12) with a smooth bottom surface on the leveled surface of material in the mould after filling or refilling. Place the hanger with weights of $m_{Wtw}$ on the twisting lid. The weights on the hanger should correspond to a pressure of $\sigma_{tw}$, approximately equal to $\sigma_{p}$. \\
7.3.2 Empty the cell and repeat the filling operation if the surface of material in the cell does not appear to the naked eye to be level. \\
7.3.3 Having filled the cell, the twisting lid is usually twisted through 20 cycles by means of the twisting wrench (spanner) (13) or twisting device. Each twisting cycle consists of a $90 ~ degrees$ rotation of the lid which is then reversed. Care must be taken not to apply vertical forces to the lid during twisting. While twisting, press the ring against the locating screws with the fingers to prevent it from sliding from its original offset position.\\

\subsection{p-w}
\label{subsec:JSCTpw}

\lipsum[3]

\section{Shear Cell Simulation}
\label{sec:scsimulation}
%************************************************

\lipsum[1]


\section{Coefficient of Restitution}
\label{sec:COR}
%************************************************

\lipsum[1]

\subsection{p-p}
\label{subsec:CORpp}

\lipsum[2]

\subsection{p-w}
\label{subsec:CORpw}

\lipsum[3]

\section{Coefficient of Restitution Simulation - Estimation Matlab}
\label{sec:corsimulation}
%************************************************

\lipsum[1]
