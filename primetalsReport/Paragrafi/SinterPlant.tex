% !TEX encoding = UTF-8
% !TEX TS-program = pdflatex
% !TEX root = ../Tesi.tex
% !TEX spellcheck = en-EN

% %************************************************
% \chapter{Sinter Segregator}
% \label{cap:sintersegregator}
% %************************************************

\section{Design}
\label{sec:design}

\begin{figure}[!htb]
\centering
\includegraphics[width=.80\columnwidth]{images/057sinterChuteOrtoLayout}
\caption[Sinter chute orto layout]{Sinter chute orto layout.}
\label{fig:057sinterChuteOrtoLayout}
\end{figure}

The method presented in the article \textit{Characterization of Particle Based Simulation Parameters
by means of Artificial Neural Networks} allowed to collect data
regarding the contact law. We took the average value for each of the DEM parameter and we used them 
in the industrial scale DEM simulation, with the same software, 
of an iron ore sintering process, see Fig. \ref{fig:057sinterChuteOrtoLayout}.
Especially, we examined the behaviour of these particles in a sinter chute cooler.

\begin{figure}[!htb]
\centering
\includegraphics[width=.40\columnwidth]{images/056sinterChuteBox}
\caption[Sinter chute box]{Sinter chute box.}
\label{fig:056sinterChuteBox}
\end{figure}

The particles moved from the top of a specifically designed chute and were 
collected by moving boxes at the bottom. These boxes were holed at the base, 
allowing cool air to lower the temperature of the sinter. 
It was critical to ascertain if the larger particles segregated, 
by moving to the bottom of the boxes (see Fig. \ref{fig:056sinterChuteBox}), while the smaller to the top,
allowing a more effective distribution of the cool air. 
Especially, the regions
highlighted in Fig. \ref{fig:055sinterChuteVerticalLayout} and
Fig. \ref{fig:067sinterchute} for the simulation were investigated.
The simulation was performed with a maximum of 500,000 particles.
With this method is a huge number, obtained with a coarse graining factor of
five, meaning that all the particles have radii five time larger than reality.
Avoiding coarse graining would have meant more than 10,000,000 particles,
totally unfeasibile for us.

\begin{figure}[!htb]
\centering
\includegraphics[width=.80\columnwidth]{055sinterChuteVerticalLayout}
\caption[Sinter chute vertical layout]{Sinter chute vertical layout.}
\label{fig:055sinterChuteVerticalLayout}
\end{figure}
\begin{figure}[!htb]
\centering
\includegraphics[width=.80\columnwidth]{images/067sinterchute}
\caption[Sinter chute simulation]{Sinter chute simulation.}
\label{fig:067sinterchute}
\end{figure}

\section{Results}
\label{sec:results}


There are two main phases:
\begin{itemize}
  \item{Initial filling: during this phase many small particles deposited inside
  the chute}
  \item{Moving bottom box: also in this phase particles were
continuously inserted from the top. After the filling was completed, the
scraper was moved and all the particles further a defined surface were
deleted. Inside the sinter, all the places that small particles could
occupy already were. At this point the bottom box was filled again, but
with a more realistic size distribution, and it reached steady state, as can
be seen in figures \ref{fig:091rS120150824174530}, \ref{fig:092rS220150824174530}, \ref{fig:093rS420150824174530}, and
\ref{fig:094rS620150824174530}.}
\end{itemize}
As expected, when the steady state was reached, the particles actually
segregated, as it was intended to demonstrate.
In fact, there were more particles with larger radii, from 8 to 25 mm, in the
bottom layers.
Instead, the smallest particles, radius = 2.5 mm, was rather on the top layer. 


As explained in the Application section, we divided the first of the boxes
filled by the chute in three (equally high) layers, from 6 on top to 8 
on bottom, see Fig. \ref{fig:056sinterChuteBox}. 
The volume over these layers was not considered, because it was continuously 
supplied of particles from the chute. In Fig. \ref{fig:066sinterbarplot} the
percentage of the total volume of the particles available in each layer at steady-state, 
grouped by radius, is shown. 
We could clearly see how the larger particles disposed mostly 
in the bottom layer, validating the realized design.
\begin{figure}[htbp]
\centering 
  \subfloat[Ratio over the volume occupied by the particles of each radius.]
  {
	  \includegraphics[width=.96\columnwidth]{121SinterBarPlot20151111150702}
	  \label{fig:121SinterBarPlot20151111150702}
  }
  \\
    \subfloat[Ratio over the number of the particles of each radius.]
    {
	  \includegraphics[width=.96\columnwidth]{122SinterBarPlot20151123180606}
	  \label{fig:122SinterBarPlot20151123180606}
  }
  \\
  \caption{Sinter bar plot.}
  \label{fig:066sinterbarplot}
\end{figure}
\begin{sidewaysfigure}[htbp]
\centering 
  \subfloat%[Jenike shear cell tester.]
  {
	  \includegraphics[width=.48\columnwidth]{images/113rS120151111150702}
	  \label{fig:113rS120151111150702}
  }
  \quad
    \subfloat
    %[Simplified Jenike shear cell tester.]
    {
	  \includegraphics[width=.48\columnwidth]{images/115rS220151111150702}
	  \label{fig:115rS220151111150702}
  }
  \\
  \subfloat%[Jenike shear cell tester layout.]
  {
	  \includegraphics[width=.48\columnwidth]{images/117rS420151111150702}
	  \label{fig:117rS420151111150702}
  }
  \quad
    \subfloat%[Jenike shear cell tester diagram.]
    {
	  \includegraphics[width=.48\columnwidth]{images/119rS620151111150702}
	  \label{fig:119rS620151111150702}  }
  \\
  \caption{Particle size distribution during the simulation, in $m^3$ of
  material. Once steady state is reached, large particles are more likely in the
  inferior layers, small particles in the superior ones.}
  \label{fig:096sinterplots}
\end{sidewaysfigure}


