% This is "sig-alternate.tex" V2.1 April 2013
% This file should be compiled with V2.5 of "sig-alternate.cls" May 2012
%
% This example file demonstrates the use of the 'sig-alternate.cls'
% V2.5 LaTeX2e document class file. It is for those submitting
% articles to ACM Conference Proceedings WHO DO NOT WISH TO
% STRICTLY ADHERE TO THE SIGS (PUBS-BOARD-ENDORSED) STYLE.
% The 'sig-alternate.cls' file will produce a similar-looking,
% albeit, 'tighter' paper resulting in, invariably, fewer pages.
%
% ----------------------------------------------------------------------------------------------------------------
% This .tex file (and associated .cls V2.5) produces:
%       1) The Permission Statement
%       2) The Conference (location) Info information
%       3) The Copyright Line with ACM data
%       4) NO page numbers
%
% as against the acm_proc_article-sp.cls file which
% DOES NOT produce 1) thru' 3) above.
%
% Using 'sig-alternate.cls' you have control, however, from within
% the source .tex file, over both the CopyrightYear
% (defaulted to 200X) and the ACM Copyright Data
% (defaulted to X-XXXXX-XX-X/XX/XX).
% e.g.
% \CopyrightYear{2007} will cause 2007 to appear in the copyright line.
% \crdata{0-12345-67-8/90/12} will cause 0-12345-67-8/90/12 to appear in the copyright line.
%
% ---------------------------------------------------------------------------------------------------------------
% This .tex source is an example which *does* use
% the .bib file (from which the .bbl file % is produced).
% REMEMBER HOWEVER: After having produced the .bbl file,
% and prior to final submission, you *NEED* to 'insert'
% your .bbl file into your source .tex file so as to provide
% ONE 'self-contained' source file.
%
% ================= IF YOU HAVE QUESTIONS =======================
% Questions regarding the SIGS styles, SIGS policies and
% procedures, Conferences etc. should be sent to
% Adrienne Griscti (griscti@acm.org)
%
% Technical questions _only_ to
% Gerald Murray (murray@hq.acm.org)
% ===============================================================
%
% For tracking purposes - this is V2.0 - May 2012

\documentclass{sig-alternate-05-2015}

\usepackage{epstopdf}

\begin{document}

% Copyright
%%%%\setcopyright{acmcopyright}
%\setcopyright{acmlicensed}
%\setcopyright{rightsretained}
%\setcopyright{usgov}
%\setcopyright{usgovmixed}
%\setcopyright{cagov}
%\setcopyright{cagovmixed}


% DOI
\doi{??}

% ISBN
\isbn{??}

%Conference
\conferenceinfo{EANN 2015}{September 25--28, 2015, Rhodes, Greece}

\acmPrice{\$xx.00}

%
% --- Author Metadata here ---
\conferenceinfo{RHODES}{2015 Rhodes, Greece}
%\CopyrightYear{2007} % Allows default copyright year (20XX) to be over-ridden - IF NEED BE.
%\crdata{0-12345-67-8/90/01}  % Allows default copyright data (0-89791-88-6/97/05) to be over-ridden - IF NEED BE.
% --- End of Author Metadata ---

\title{Characterization of DEM particles by means of artificial neural networks
and macroscopic experiments}
%  Alternate {\ttlit ACM} SIG Proceedings Paper in LaTeX
% Format\titlenote{(Produces the permission block, and
% copyright information). For use with
% SIG-ALTERNATE.CLS. Supported by ACM.}}
% \subtitle{[Extended Abstract]
% \titlenote{A full version of this paper is available as
% \textit{Author's Guide to Preparing ACM SIG Proceedings Using
% \LaTeX$2_\epsilon$\ and BibTeX} at
% \texttt{www.acm.org/eaddress.htm}}}
%
% You need the command \numberofauthors to handle the 'placement
% and alignment' of the authors beneath the title.
%
% For aesthetic reasons, we recommend 'three authors at a time'
% i.e. three 'name/affiliation blocks' be placed beneath the title.
%
% NOTE: You are NOT restricted in how many 'rows' of
% "name/affiliations" may appear. We just ask that you restrict
% the number of 'columns' to three.
%
% Because of the available 'opening page real-estate'
% we ask you to refrain from putting more than six authors
% (two rows with three columns) beneath the article title.
% More than six makes the first-page appear very cluttered indeed.
%
% Use the \alignauthor commands to handle the names
% and affiliations for an 'aesthetic maximum' of six authors.
% Add names, affiliations, addresses for
% the seventh etc. author(s) as the argument for the
% \additionalauthors command.
% These 'additional authors' will be output/set for you
% without further effort on your part as the last section in
% the body of your article BEFORE References or any Appendices.

\numberofauthors{3} %  in this sample file, there are a *total*
% of EIGHT authors. SIX appear on the 'first-page' (for formatting
% reasons) and the remaining two appear in the \additionalauthors section.
%
\author{
% You can go ahead and credit any number of authors here,
% e.g. one 'row of three' or two rows (consisting of one row of three
% and a second row of one, two or three).
%
% The command \alignauthor (no curly braces needed) should
% precede each author name, affiliation/snail-mail address and
% e-mail address. Additionally, tag each line of
% affiliation/address with \affaddr, and tag the
% e-mail address with \email.
%
% 1st. author
\alignauthor
Luca Benvenuti\titlenote{Corresponding author.}\\
       \affaddr{Johannes Kepler University Linz}\\
       \affaddr{Department on Particulate Flow Modelling}\\
       \affaddr{Altenbergerstrasse 69}\\
       \affaddr{4040, Linz, Austria}\\
       \email{luca.benvenuti@jku.at}
% 2nd. author
\alignauthor
Christoph Kloss \\%\titlenote{}\\
       \affaddr{DCS Computing GmbH}\\
       \affaddr{Altenbergerstrasse 66a - Science Park}\\
       \affaddr{4040, Linz, Austria}\\
       \email{christoph.kloss@dcs-computing.com}
% 3rd. author
\alignauthor 
Stefan Pirker \\ %\titlenote{}\\
       \affaddr{Johannes Kepler University Linz}\\
       \affaddr{Department on Particulate Flow Modelling}\\
       \affaddr{Altenbergerstrasse 69}\\
       \affaddr{4040, Linz, Austria}\\
       \email{stefan.pirker@jku.at}
% \and  % use '\and' if you need 'another row' of author names
% % 4th. author
% \alignauthor Lawrence P. Leipuner\\
%        \affaddr{Brookhaven Laboratories}\\
%        \affaddr{Brookhaven National Lab}\\
%        \affaddr{P.O. Box 5000}\\
%        \email{lleipuner@researchlabs.org}
% % 5th. author
% \alignauthor Sean Fogarty\\
%        \affaddr{NASA Ames Research Center}\\
%        \affaddr{Moffett Field}\\
%        \affaddr{California 94035}\\
%        \email{fogartys@amesres.org}
% % 6th. author
% \alignauthor Charles Palmer\\
%        \affaddr{Palmer Research Laboratories}\\
%        \affaddr{8600 Datapoint Drive}\\
%        \affaddr{San Antonio, Texas 78229}\\
%        \email{cpalmer@prl.com}
}
% There's nothing stopping you putting the seventh, eighth, etc.
% author on the opening page (as the 'third row') but we ask,
% for aesthetic reasons that you place these 'additional authors'
% in the \additional authors block, viz.
% \additionalauthors{Additional authors: John Smith (The Th{\o}rv{\"a}ld Group,
% email: {\texttt{jsmith@affiliation.org}}) and Julius P.~Kumquat
% (The Kumquat Consortium, email: {\texttt{jpkumquat@consortium.net}}).}
\date{30 July 1999}
% Just remember to make sure that the TOTAL number of authors
% is the number that will appear on the first page PLUS the
% number that will appear in the \additionalauthors section.

\maketitle
\begin{abstract}
In Discrete Element Method ($DEM$) simulations, particle-particle contact laws
determine the macroscopic simulation results. \\
Particle-based contact laws, in
turn, commonly rely on semi-empirical parameters which are difficult to obtain
by direct microscopic measurements.
In this study, we present a method for the identification of
$DEM$ simulation parameters 
that uses artificial neural networks to link
macroscopic experimental results to
microscopic numerical parameters.
In the first step, a series
of $DEM$ simulations with varying simulation parameters is used to train a
feed-forward artificial neural network by backward-propagation reinforcement. 
In the second step, this artificial neural network is used to predict the
macroscopic ensemble behaviour in relation to additional sets of particle-based
simulation parameters.
Thus, a comprehensive database is obtained
which links particle-based simulation parameters to specific macroscopic
bulk behaviours of the ensemble.
The trained artificial neural network is able to predict the behaviours of
additional sets of input parameters accurately and highly efficiently.
Furthermore, this method can be used generically to
identify $DEM$ material parameters.
For each set of calibration experiments, the neural network 
needs to be trained only once.
After the training, the neural network provides a generic link between the macroscopic 
experimental results and the microscopic $DEM$ simulation parameters.
Based on these experiments, the $DEM$ simulation parameters of any given
non-cohesive granular material can be identified.
\end{abstract}


%
% The code below should be generated by the tool at
% http://dl.acm.org/ccs.cfm
% Please copy and paste the code instead of the example below. 
%
% \begin{CCSXML}
% <ccs2012>
%  <concept>
%   <concept_id>10010520.10010553.10010562</concept_id>
%   <concept_desc>Computer systems organization~Embedded systems</concept_desc>
%   <concept_significance>500</concept_significance>
%  </concept>
%  <concept>
%   <concept_id>10010520.10010575.10010755</concept_id>
%   <concept_desc>Computer systems organization~Redundancy</concept_desc>
%   <concept_significance>300</concept_significance>
%  </concept>
%  <concept>
%   <concept_id>10010520.10010553.10010554</concept_id>
%   <concept_desc>Computer systems organization~Robotics</concept_desc>
%   <concept_significance>100</concept_significance>
%  </concept>
%  <concept>
%   <concept_id>10003033.10003083.10003095</concept_id>
%   <concept_desc>Networks~Network reliability</concept_desc>
%   <concept_significance>100</concept_significance>
%  </concept>
% </ccs2012>  
% \end{CCSXML}
% 
% \ccsdesc[500]{Computer systems organization~Embedded systems}
% \ccsdesc[300]{Computer systems organization~Redundancy}
% \ccsdesc{Computer systems organization~Robotics}
% \ccsdesc[100]{Networks~Network reliability}
% 
% 
% %
% % End generated code
% %
% 
% %
% %  Use this command to print the description
% %
% \printccsdesc

% We no longer use \terms command
%\terms{Theory}

\keywords{Discrete Element Method ($DEM$) Simulations; Parameter Identification; Artificial Neural Networks}

\section{Introduction}
\label{sec:introduction}
%************************************************

Particles in various forms - ranging from raw materials to food grains and pharmaceutical powders - 
play a major role in a variety of industries. 
Discrete Element Methods ($DEMs$) are widely used to simulate
particle behaviour in these granular processes (Cleary and Sawley \cite{RefWorks:130}).\\
In their original formulation of $DEM$, Cundall and Strack \cite{RefWorks:172} allowed two 
particles to slightly overlap upon contact, and consequently they proposed
repulsive forces in relation to this overlap distance.
Their fundamental modelling concept has since been widely accepted in the
literature and their soft-sphere contact law has been developed further by
numerous researchers (Vu-Quoc and Zhang \cite{RefWorks:148} and Di Renzo and Di Maio \cite{RefWorks:145}). 
With increasing computational resources, $DEM$ simulation have become very
popular giving rise to the development of commercial (e.g., $PFC3D$, used by
Wensrich and Katterfeld \cite{RefWorks:87}) and open-source software (e.g.,
$LIGGGHTS$, Kloss et al. \cite{RefWorks:136}, Aigner et al. \cite{RefWorks:139}).
Soft-sphere $DEM$ simulations of thousands of particles have been proven to 
faithfully model particle bulk behaviour (Hohner et al. \cite{RefWorks:86}). \\
In these macroscopic $DEM$ simulations, the contact law kernel between a 
pair of particles determines the global bulk behaviour of the granular material (Ai et al. \cite{RefWorks:131}). 
As a consequence, defining a correct contact law is of crucial importance for the predictive 
capability of $DEM$ simulations. 
Since $DEM$ contact laws are based 
on a set of semi-empirical parameters, correct contact law 
parameters must be defined for a given granular material
or $DEM$ simulations will fail (Combarros et al. \cite{RefWorks:177}). \\
Identifying $DEM$ contact law parameters is not a trivial task. 
Due to the huge number of particles in a granular material, it
may be impractical to identify valid parameter sets by performing bilateral 
particle collision experiments. 
Furthermore, some contact law parameters such as the coefficient of rolling
friction are purely empirical and cannot be determined by direct 
particle-to-particle measurements (Wensrich and Katterfeld \cite{RefWorks:87}).
Therefore, $DEM$ contact law parameters (Table \ref{tab:08DEMparameters}) are
commonly determined by comparing the macroscopic outcome of large-scale $DEM$ simulations with 
bulk experiments (Alenzi et al. \cite{RefWorks:91}). 
If $DEM$ simulation results disagree with bulk measurements, the set of contact
law parameters must be adjusted until reasonable agreement is achieved.\\
However, this purely forward methodology of parameter identification is limited by 
the multi-dimensionality of the parameter space and the associated computational costs of the required 
$DEM$ test simulations. 
Moreover, one parameter set which is valid for one bulk behaviour (e.g., angle
of repose) might fail for another (e.g., shear tester). \\
Clearly, there is a need for an efficient method for identifying
$DEM$ contact law parameters.
In our study, we harnessed Artificial Neural Networks ($ANNs$) in order to
reduce the number of $DEM$ test simulations required. 
$ANNs$ have proven to be a versatile tool in analysing complex, non-linear
systems of multi-dimensional input streams (Vaferi et al. \cite{RefWorks:150}, Witten et
al. \cite{RefWorks:174} and Haykin \cite{RefWorks:158}).
In our case, we fed an $ANN$ with $DEM$ contact law parameters as input
and compared the output with the bulk behaviour 
predicted by a corresponding $DEM$ simulation. 
The difference between $ANN$ prediction and $DEM$ prediction is used to train our 
specific $ANN$ with a backward-propagation algorithm (described further below). 
After a training phase comprising a limited number of $DEM$ test simulations,
the $ANN$ can then be used as a stand-alone prediction tool for the bulk behaviour of a 
granular material in relation to $DEM$ contact law parameters. \\
In this study, we applied this parameter identification method to two different
granular bulk behaviours, namely the angle of repose ($AoR$) test and the
Schulze shear cell ($SSC$) test.
In both cases, we first trained a specific $ANN$ using a number of $DEM$ test
simulations before we identified valid sets of $DEM$ contact law parameters by
comparing the stand-alone $ANN$ predictions with corresponding bulk experiments. 
For both cases we obtained valid sets of contact law parameters, 
which we then compared to formulate a reliable contact law for a given
granular material.
We further show that the same $ANN$ can be used to characterize different granular materials. \\
In the next section we define some prerequisites including $DEM$ contact law
definitions, a general description of the $ANN$ functionality, and the proposed
method of $DEM$ contact law parameter identification.
We then describe how it is applied to characterize the $DEM$ contact law
parameters of sinter fines.
%************************************************
%\begin{table}[h]
\centering
\begin{tabular}{l}
\toprule
    Radius \acs{R} (m)   \\ [5pt]

	Size distribution (-) \\ [5pt]

    Young's modulus \acs{E} (Pa)  \\ [5pt]

    Poisson's ratio \acs{nu} (-) \\ 
     Time step \acs{deltat} (s) \\ [5pt]
\midrule
     Coefficient of sliding friction \acs{mus} (-)\\  [5pt]
    Coefficient of rolling friction \acs{mur} (-) \\ [5pt]
    Coefficient of restitution \acs{CoR} (-)   \\ [5pt]
     Particle density $\acs{rhop} = \frac{mass ~ of ~ one ~ particle}{volume ~ of
     ~ one ~ particle}$ ($kg/m^3$)  \\ [5pt]
     Geometry factor $\acs{dCylDp} = \frac{diameter ~ of ~ the ~
     cylinder}{diameter ~ of ~ one ~ particle}$ (-)  \\ [5pt]
   
\bottomrule
\end{tabular}
\caption[DEM parameters]{DEM parameters. The upper parameters were
identical in all simulations.
%  The lower parameters were constant in each
% simulation, but were varied between simulations.
}
\label{tab:08DEMparameters}
\end{table}


\begin{table}[h]
\centering
\begin{tabular}{l}
\hline 
     Particle radius $R$ (m)   \\ [5pt]

	Size distribution (-) \\ [5pt]

    Young's modulus $E$ (Pa)  \\ [5pt]

    Poisson's ratio $\nu$ (-) \\ 
     Time step $\Delta t$ (-) \\ [5pt]
        \hline
     Coefficient of sliding friction $\mu_s$ (-)\\  [5pt]
    Coefficient of rolling friction $\mu_r$ (-) \\ [5pt]
    Coefficient of restitution $COR$ (-)   \\ [5pt]
     Particle density $\rho_p$ (kg/m3)  \\ [5pt]
    Geometry factor $dCylDp$ (-)  \\ [5pt]
   
\hline
\end{tabular}
\caption[DEM parameters]{DEM parameters. The upper parameters were
identical in all simulations. The lower parameters were constant in each
simulation, but were varied between simulations.}
\label{tab:08DEMparameters}
\end{table}
%************************************************

\section{DEM Parameter Identification}
\label{sec:methodology}

Fig. \ref{fig:19methodology} illustrates the methodology used.

%************************************************
\begin{figure}%[!htb] 
\centering 
\includegraphics[width=.96\columnwidth]{19methodology} 
\caption[Method]{Method. 
In the training phase (dashed lines)
$DEM$ simulations are performed
with random initial input parameters.
The behaviours obtained are used to train the
Artificial Neural Networks ($ANNs$) in a loop that continues until the
difference between the outputs of each $ANN$ and its simulations is below the
limit ($\Delta$) (see Section \ref{subsec:ann}).
In the parameters identification phase (solid
lines) we identify valid input parameters by comparing (\textbf{=}) $ANNs$ and
experimental behaviours.
Further explanations can be found in Section \ref{sec:methodology}.
}
\label{fig:19methodology} 
\end{figure}
%************************************************
\subsection{Discrete element method}
\label{subsec:dem}

We decided to utilize a single
contact law for all the simulations performed, for details see
Benvenuti et al. \cite{RefWorks:180}.
The $DEM$ parameters for the Young's modulus ($E$) and the Poisson's coefficient
($\nu$) were taken from the literature, see \cite{RefWorks:175} 
and \cite{RefWorks:176}; however we reduced the former to increase the time step
($\Delta t$), following the recommendations of Ai et al. \cite{RefWorks:131}.
The time step was between $1.29 \%$ and $1.53 \%$ of the Rayleigh time, which
also depends on the particle density ($\rho_p$).
Furthermore, we locked the size distribution, which was obtained by experimental
sieving, see Table \ref{tab:09DEMFixedinputvalues}.
In the contact law we used, 
the tangential component of the contact force between two generic particles
($F_t$) is truncated to fulfil:
%************************************************
\begin{equation}
F_{t} \leq \mu_s F_{n},
 \label{eq:force_t}
\end{equation}
%************************************************
where $F_n$ is the normal component and $\mu_s$ is the coefficient of sliding
friction, one of the particle-based $DEM$ parameter we investigated, 
another being the coefficient of rolling friction ($\mu_r$). 
For coarse non-spherical particles, this is a critical parameter and describes
inter-particle friction in medium to dense granular flow simulations. It is proportional to the 
torque counteracting the rotation of the particle. The $\mu_r$ parameter enters the 
equations according to the elasto-rolling resistance model presented by Wensrich and 
Katterfeld \cite{RefWorks:87} and Ai et al. \cite{RefWorks:131} 
based on the work of Jiang et al. \cite{RefWorks:143}. 
The model is called $EPSD2$ in $LIGGGHTS$ and is appropriate for both one-way and cyclical rolling cases.
The maximum magnitude of rolling resistance torque is (Eq. \ref{eq:trmax}):
%************************************************
\begin{equation}
T_{r~max} = \mu_r R_r |\tilde{F_n}| ~,
 \label{eq:trmax}
\end{equation}
%************************************************
where $R_r$ is the equivalent radius and $F_n$ the normal force.
The last two particle-based $DEM$ parameters we investigated were $\rho_p$
and the coefficient of restitution ($COR$) as defined by Ai. et al. \cite{RefWorks:131}.
These coefficients, $COR$, $\mu_s$, $\mu_r$,
$\rho_p$ and $dCylDp$ (the cylinder dimension, proportional to the mean
particle diameter), as indicated in Table \ref{tab:10DEMVariableinputvalues}, 
were constant in each simulation, but their combination differed between
simulations.
Further, $dCylDp$ was used to evaluate the wall effect, but only $~10\%$ of the
simulations had a $dCylDp$ larger than $20$ (additional information can be found
in Benvenuti et al. \cite{RefWorks:180}).
The normal stress $\sigma_n$ and its
percentage during the incipient flow condition $\tau_{\%}$
varied to replicate twelve shear-cell load conditions. 
The complete description of the shear-cell 
and the $AoR$ simulations
can be found in Benvenuti et al. \cite{RefWorks:180}.
A Matlab script allowed us to extract from the simulation output the numerical
values representative of bulk behaviour (hereafter called $bulk ~ values$)
for each DEM simulation parameter combination, which consists of
bulk density ($\rho_b$),
coefficient of internal friction in the pre-shear phase ($\mu_{psh}$),
coefficient of internal friction in the shear phase ($\mu_{sh}$),
and angle of repose ($AoR$).
The first bulk value ($\rho_b$) was provided directly. 
% For correctly performed simulations, see Benvenuti et al. \cite{RefWorks:180}, we
% observed a stress path as in Fig. \ref{fig:21simexample}.
First, the $\sigma_n$ was kept constant while the coefficient of internal
friction ($\mu_{ie}$) initially increased and then reached a plateau.
The second bulk value ($\mu_{psh}$) was calculated as the average of the
$\mu_{ie}$ in this plateau.
The $\sigma_n$ was then automatically reduced, in our example to $80 \%$ of
its initial value.
Subsequently, a second plateau developed.
We obtained the third
value ($\mu_{sh}$) as the average of $\mu_{ie}$ in this second plateau.
The stress path accords with the experimental one, especially the plateaux.\\
In the $AoR$ tests the average of the repose angles provided us with the fourth
bulk value, allowing us to define the numerical bulk behaviour.
%************************************************
% \begin{figure}%[htp] 
% 
% \end{figure}\centering
%     \begin{subfigure}[b]{0.96\columnwidth}
%         \includegraphics[width=\textwidth]{20experimental}
%         \caption{Experimental shear-cell tester stress path - $\sigma_n = 10000
%         ~Pa$}
%         \label{fig:20experimental} 
%     \end{subfigure}\\
%         \begin{subfigure}[b]{0.96\columnwidth}
%         \includegraphics[width=\textwidth]{21simexample}
%         \caption{Numerical shear-cell tester stress path - $\sigma_n = 10000
%         ~Pa$}
%         \label{fig:21simexample} 
%     \end{subfigure}
%     \caption[Stress path]{Experimental and numerical samples of the stress path
%     for the Schulze ring shear cell tester.
% 	Time was normalized: $\tilde{t} = t/t_{change}$, where $t_{change}$ is the
% 	point in time at which the normal stress ($\sigma_n$) was modified during the
% 	tests.
% 	Until $\tilde{t}=1$, the $\sigma_n$ was kept constant at 10,000 Pa. 
% 	In Fig. \ref{fig:20experimental}, 
%  	a plateau was reached at $\tilde{t}~=0.91$.
% 	The coefficient of pre-shear ($\mu_{psh}$) was calculated as the average of the
% 	coefficient of internal friction ($\mu_{ie}$) in this first plateau.
% 	At $\tilde{t}=1$, the $\sigma_n$ was reduced to $80 \%$ of its initial
% 	value, and soon after
% 	a second plateau developed.
% 	We obtained the coefficient of
% 	shear ($ \mu_{sh}$) as the average of $\mu_{ie}$ in this second plateau.
% 	The stress paths agree well, especially the plateaux.
% 	They were clearly relevant because
% 	the values representative of the bulk behaviours 
% 	were collected there.}
%     \label{fig:40experimentalsimulation}
% \end{figure}
%************************************************
\begin{table}%[h]
\centering
\begin{tabular}{ccccc}
\hline
    Mean & Std.dev.  & Young's & Poisson's & $\Delta t$\\
    $R$ & $R$ & modulus & ratio & \\
    (mm)  & (mm)  & (MPa) & (-) & (s)\\
    \hline
    $0.732$ & $0.41$ & $10$    & $0.40$ & $10^{-6}$\\
\hline
\end{tabular}
\caption{DEM fixed input values}
\label{tab:09DEMFixedinputvalues}
\end{table}
%************************************************
\begin{table*}%[h]
\centering
\begin{tabular}{ccccc}
\hline
    $\mu_s$ & $\mu_r$ & $COR$ & $\rho_p$ & $dCylDp$ \\
    	(-)  & (-)   & (-)   & (kg/m3) & (-) \\
    \hline
    0.4 / 0.6 / 0.8 & 0.4 / 0.6 / 0.8 & 0.5 / 0.7 / 0.9 & 2500 / 3000 / 3500 & 20 / 36 / 38 / 40 \\

\hline
\end{tabular}
\caption[DEM variable input values]{DEM variable input values for training the
Artificial Neural Networks}
\label{tab:10DEMVariableinputvalues}
\end{table*}
%************************************************
\begin{table*}%[h]
\centering
\begin{tabular}{lcccc}
\hline
 &  $\mu_s$ & $\mu_r$ & $COR$ & $\rho_p$  \\
   &	(-)  & (-)   & (-)   & (kg/m3) \\
          \hline
    range & $[0.1 \ldots 1.0]$ & $[0.1 \ldots 1.0]$ & $[0.5 \ldots 0.9]$ &
    $[2000 \ldots 3500]$     \\
    number of values & 100   & 100   & 25    & 25    \\

\hline
\end{tabular}
\caption[DEM random input values]{DEM random input values. Within each range the
indicated number of random values was chosen according to a standard uniform
distribution.}
\label{tab:12DEMRandominputvalues}
\end{table*}
%************************************************

\subsection{Artificial Neural Networks}
\label{subsec:ann}
We first defined the typology of Artificial Neural Networks ($ANNs$) we used and
the input we fed them, see Benvenuti et al. \cite{RefWorks:180}.
Our $ANNs$ have three different layers: the input layer has a number of neurons
equal to the number of different inputs of the network, see Fig. \ref{fig:18nnscheme}.
The hidden (or central) layer's number of neurons was to be investigated. 
The output layer contains one neuron for the output.
The transfer functions for the central layer are the tangential sigmoid.\\
Thus, we were able to use the $DEM$ parameter combinations and their
corresponding bulk values to train the $ANNs$.
Note that 15\% of the simulations ($test ~ simulations$) were
randomly picked and excluded from the training processes.
We started with all the $DEM$ parameter combinations and their corresponding
numerical $\mu_{psh}$ to create 36 $ANNs$ that differed in their numbers of
neurons in the hidden layer (between five to forty neurons).
We then determined the coefficient of determination ($R^2$) between the
$bulk-macro$ behaviours in the output of the $ANN$ and the 15\% $test ~ simulations$, 
which were not correlated with the remaining 85\% used for the training. 
Thus, we could select for $\mu_{psh}$ the $ANN$ with the maximum $R^2$, 
again as suggested by Vaferi et al. \cite{RefWorks:150}, and we noted its number
of neurons.
We repeated the same $ANN$ creation steps for $\mu_{sh}$, $\rho_b$
and $AoR$, obtaining one trained $ANN$ for each bulk value. \\
Since $\mu_{psh}$, $\mu_{sh}$ and $\rho_b$ belonged to the shear-cell
simulations, their $ANNs$ were handled together: we had one cluster with three 
$ANNs$ for the shear cell and one with only one $ANN$ for the $AoR$.
We could then proceed in identifying valid input parameters.
Oberkampf et al. \cite{RefWorks:160} suggested using a Design of Experiments
($DoE$) method to determine the parameter combinations to be simulated.
They stated that this approach allows optimization of computation time
with an acceptable loss of precision.
The speed of the trained $ANNs$ enabled us to follow a different approach to
maximizing the precision of the characterization.
We created random values
in the range and numbers defined in Table \ref{tab:12DEMRandominputvalues}
according to a standard uniform distribution.
The total number of combinations of these random values was 6,250,000.
These combinations were then fed to and processed by the selected
$ANNs$, and thus three bulk values for the shear
cell and one for the $AoR$ were obtained.
%************************************************
\begin{figure}%[!htb] 
\centering 
\includegraphics[width=.96\columnwidth]{18nnscheme} 
\caption[ANN Scheme]{Artificial Neural Network ($ANN$) Scheme
of how the Multilayer Perceptron $ANN$ ($MLPNN$) derives one
bulk-behaviour-dependent variable from the mutually independent simulation variables.}
\label{fig:18nnscheme} 
\end{figure}
%************************************************

\subsection{Macroscopic Experiments and Parameter Identification}
\label{subsec:macroscopicexperimentsparameteridentification}
The experimental characterization was performed as described in
Benvenuti et al. \cite{RefWorks:180}. 
We obtained for each of the twelve load conditions of the $SSC$ three bulk
values ($\mu_{psh}$, $\mu_{sh}$ and $\rho_b$).
The fourth bulk value was the result of two angle of repose ($AoR$) tests that
recreated the repose angle observed in a pile of the
real material. 

Subsequently, we compared the $ANN$ and experimental bulk behaviours for the
twelve shear-cell load conditions.
If in a DEM-parameter combination all the three bulk values differed by less 
than 5\% from those of the corresponding experiments, i.e.:
%************************************************
\begin{equation}
 \begin{cases}
\text{if } & \lvert{1-\frac{\mu_{psh,num}}{\mu_{psh,exp}}}\rvert < 5\%  ,\\
\text{and if } & \lvert{1-\frac{\mu_{sh,num}}{\mu_{sh,exp}}}\rvert < 5\% , \\ 
\text{and if } & \lvert{1-\frac{\rho_{p,num}}{\rho_{p,exp}}}\rvert < 5\% ,\\ 
\end{cases}
 \label{eq:check2}
\end{equation}
%************************************************
the combination was marked. The marked combinations were processed by the
$AoR$ $ANN$, and then compared with the experiment.
Were considered valid those that differed by less than $5\%$ also in this
comparison (Eq. \ref{eq:checkaor}):
%************************************************
\begin{equation}
\text{if} ~~~~~~ \lvert{1-\frac{AoR_{num}}{AoR_{exp}}}\rvert < 5\% .
\label{eq:checkaor}
\end{equation}
%************************************************
Further, to prove the validity of the system, we tested the marked combinations
by modifying the experimental bulk values of the shear cell. 
We artificially decreased or increased the shear force, and thus $\mu_{psh}$ and
$\mu_{sh}$, by a product coefficient ($P$), e.g. Eq. \ref{eq:pcoeff}:
%************************************************
\begin{equation}
\label{eq:pcoeff}
\mu_{psh, new} = \mu_{psh, old} \cdot P .
\end{equation}
%************************************************

\section{Results and discussion}
\label{sec:results}
%************************************************

\subsection{DEM Simulations}
\label{subsec:simulations}

For sinter fine, 546 shear cell and 81 static $AoR$ simulations were run with
the parameter combinations described in Table
\ref{tab:10DEMVariableinputvalues}.
The computational time amounted to 1 hour with 32 AMD cores for a benchmark
shear-cell simulation and to 9 hours for a benchmark $AoR$ simulation, both with
50,000 particles.
Simulations with larger $dCylDp$ required more time (e.g., about 12 hours for
the shear cell with 400,000 particles ). \\


\subsection{ANN model development}
\label{subsec:annmodeldev}

First, we determined the regression of the bulk behaviour parameters, for
instance the $\mu_{psh}$; see Fig. \ref{fig:22regression}, where the
corresponding plot for the $ANN$ with the maximum $R^2$ is shown. Each circle represents one of the 546
simulations.
The plot shows a consistent agreement between the 
$DEM$ and the $ANN$ values and an almost linear regression ($R^2
= 0.94$).
The linear relationship between the
training values can be seen in Table \ref{tab:06inputRelationshipTable}.
The clearest connections were between $\mu_s$ and $\mu_{psh}$, and
$\rho_p$ and $\rho_b$.
In contrast, for $\mu_{sh}$ and $AoR$, the $\mu_r$ balanced the influence of the 
$\mu_s$. \\
We then investigated how the $R^2$ changed with the number of neurons
for the $\mu_{psh}$.
In this case, we achieved a $R^2 = 0.96$ for an $ANN$ with fifteen neurons. 
Increasing the number of neurons did not improve the $R^2$; it even started to
oscillate with higher numbers of neurons.
We subsequently obtained the optimal number of neurons for all $ANNs$.
Further, we processed the random combinations (Table
\ref{tab:10DEMVariableinputvalues}) with the $ANN$.
The $ANN$ evaluation was significantly faster than the $DEM$ simulations. The
individuation of the numerical bulk behaviours for all the $DEM$ combinations
did not take more than a few seconds on a single core.
%************************************************
\begin{figure}%[!h] 
\centering 
\includegraphics[width=.96\columnwidth]{22regression}
%[width=.96\textwidth]
\caption[Comparison between prediction of the trained ANN and full DEM
simulation]{Comparison between prediction of the trained Artificial Neural
Network ($ANN$) and 546 full DEM simulations of the coefficient of pre-shear
($\mu_{psh}$). In this case the regression line is nearly linear (0.94), and
demonstrates the accurate predictive power of the $ANN$.}
\label{fig:22regression} 
\end{figure}
%************************************************
\begin{table*}%[h]
\centering
\scalebox{1.0}{
\begin{tabular}{lcccccccc}
\hline
         & $\mu_s$ & $\mu_r$ & $COR$ & $\rho_p$ & $\mu_{sh}$ &
        $\mu_{psh}$ & $\rho_{b}$ & $AoR$ \\
          \hline
    $\mu_s$ & 100.00 & 0.55  & 0.04  & 0.00  & 3.84  & 87.26 & 8.39  & 49.48 \\
    $\mu_r$ & 0.55  & 100.00 & 0.15  & 0.00  & 58.92 & 33.70 & 3.10  & 60.20 \\
    $COR$ & 0.04  & 0.15  & 100.00 & 0.00  & 15.52 & 0.57  & 1.71  & 21.35 \\
    $\rho_p$ & 0.00  & 0.00  & 0.00  & 100.00 & 4.98  & 5.71  & 99.00 & 0.00 \\
    $\mu_{sh}$ & 3.84  & 58.92 & 15.52 & 4.98  & 100.00 & 26.03 & 9.52  & 0.00 \\
    $\mu_{psh}$ & \textbf{87.26} & 33.70 & 0.57  & 5.71  & 26.03 & 100.00 & 4.33 
    & 0.00
    \\
    $\rho_{b}$ & 8.39  & 3.10  & 1.71  & \textbf{99.00} & 9.52  & 4.33  & 100.00
    & 0.00 \\
    $AoR$ \hspace{5ex} & 49.48 & \textbf{60.20} & 21.35  & 0.00  & 0.00  & 0.00 
    & 0.00  & 100.00 \\
    
\hline
\end{tabular}}
\caption[Values of linear relationship between considered variables]{Values of
linear relationship between variables considered multiplied by 100. Sliding
friction ($\mu_s$), rolling friction ($\mu_r$) and particle density ($\rho_p$)
had the greatest influence on, respectively, the coefficient of pre-shear
($\mu_{psh}$), the angle of repose  ($AoR$) and the bulk density ($\rho_b$). Notably, $\rho_p$
was not used as a training parameter for $AoR$ bulk behaviour.}
\label{tab:06inputRelationshipTable}
\end{table*}
%************************************************


\subsection{Experiments and Parameter Identification}
\label{subsec:experimentsparameteridentification}

Experimental values identifying the bulk behavior, $\mu_{psh}$, $\mu_{sh}$ and $\rho_{b}$, 
of sinter fine were acquired through $SSC$ tests. 
Table \ref{tab:05sinterTableExperimental} presents
these values for three load conditions: clearly the $\mu_{psh}$ decreases, and 
the $\mu_{sh}$ oscillates.
The $\rho_b$ has a clear average of 1,760 kg/m3 with a 42 
kg/m3 deviation.
% The stress path for the second load condition in Table
% \ref{tab:05sinterTableExperimental} is shown in Fig.
% \ref{fig:20experimental}.
Two $AoR$ tests were performed that gave an average angle of
38.85$^\circ$.
We obtained the radius ($R$) mean and standard
deviations, as shown in Table
\ref{tab:09DEMFixedinputvalues}, from sieving experiments.
The comparison between numerical and experimental behaviours led to a first
series of marked combinations ($MC1$) for one load condition of
the shear cell ($\sigma_n=10,070$ Pa, P=1.0), as plotted in Fig.
\ref{fig:24radarpirker1schulze10070}, where 
the minimum and maximum values are shown, together with the mean. 
Each axis of the parameter space plot represents one simulation parameter.
The shaded area indicates valid parameter combinations, and dark shaded
values indicate the confidence range.
Note that the confidence interval is large, 
especially for the $COR$, which highlights its insignificant influence on the
characterization.
Both the $\rho_p$  and the $\mu_s$, however, show a narrow confidence interval, 
which demonstrates their influence and the ability of this procedure to find
valid $DEM$ parameters.
These results agree with our examination of the ratio of the standard deviation
to the range, see Table \ref{tab:13DEMvalidvalues}.
Further, we observed that various $DEM$ parameter
combinations could reproduce the experimental behaviour, and thus evaluated
their mutual dependencies.
This is shown more clearly in a density plot (see Fig. 
\ref{fig:25cloudpirker1schulze10070} for $MC1$) 
of the particles' coefficient of restitution ($COR$) in relation to
the coefficients of sliding friction ($\mu_s$) and rolling friction ($\mu_r$); 
in the white area, no valid sets of simulation parameters can be found.
In each cell the valid sets are grouped according to the 4 different COR
ranges.
Each cell is colored according to the group with the most members.
Multiple
combinations (250,407 or 4\% of the total) of $\mu_s$ and $\mu_r$ reproduced
the experimental behaviour with varying $COR$.
This underlines once more their correlation, as already stated by Wensrich and 
Katterfeld \cite{RefWorks:87}.
To further demonstrate the validity of the procedure, we modified the product
coefficient. 
First, we set it to $P=0.8$, and we obtained another
series of marked combinations ($MC2$).
It can be seen in the parameter space plot in Fig.
\ref{fig:26radarpirker08schulze10070} that the confidence range is narrower
than for $P=1.0$, while in the density plot in Fig. 
\ref{fig:27cloudpirker08schulze10070} the area
appears larger, although slightly less densely populated. Finally, for $P=1.2$
and its marked combinations ($MC3$) the parameter space plot in Fig.
\ref{fig:28radarpirker12schulze10070} shows a largely different confidence
range, while the density plot in Fig. \ref{fig:30cloudpirker12schulze10070} 
shows a smaller area. As expected, the procedure was highly sensitive to
variations in the experimental data.
Our approach could therefore be used
for a wide range of bulk materials.\\
We then processed the random combinations with the $AoR$ $ANN$. In Fig.
\ref{fig:31radarpirker1aor} the parameter space plot for the same criteria as
before can be seen.
In accordance with theory (Wensrich and Katterfeld \cite{RefWorks:87}), in a simulation dominated
by rolling particles, the coefficient of rolling friction has the maximum
influence. \\
Finally, we extracted from the $MC1$ values the $AoR$ $ANN$ behaviour
and compared it with the experimental one.
As can be seen in the parameter space plot in Fig.
\ref{fig:33radarpirker1schulze10070aor}, the confidence interval is very small,
indicating that all the parameters but the $COR$ played an important role, 
and demonstrating the reliability of these parameter
combinations in representing the bulk behaviour.
From the initial 6,250,000 combinations, only 3,884 were valid (0.0621
\%), see Table \ref{tab:13DEMvalidvalues}.
%************************************************
\begin{table}[h]
\centering
\begin{tabular}{cccccc}
\hline
$\sigma_n$ (Pa) & $\tau$ (Pa) & $\mu_{psh}$ (-) & $\tau_{\%}$ (\%) &
$\mu_{sh}$ (-) & $\rho_b$ (kg/m3) \\
\hline
    1068  & 1059  & 0.9916 & 80 & 1.2333 & 1718 \\
    2069  & 1818  & 0.8787 & 80 & 0.9994 & 1759 \\
    10070 & 8232  & 0.8175 & 80 & 1.1712 & 1802 \\

\hline
\end{tabular}
\caption[Experimental results]{Experimental results. Values for three
load conditions}
\label{tab:05sinterTableExperimental}
\end{table}
%************************************************
\begin{table*}%[h]
\centering
\begin{tabular}{llccc}
\hline

          & type  & SSC & AoR   & SSC \& AoR \\
          \hline

    $\mu_s$ & mean  & 0.831 & 0.177 & 0.664 \\
    $(-)$   & std. dev. (SD) & 0.097 & 0.095 & 0.029 \\
          & range ($R$) & 0.9   & 0.9   & 0.9 \\
          & SD / R & 0.108 & 0.106 & 0.032 \\
          \hline
    $\mu_r$ & mean  & 0.692 & 0.830 & 0.916 \\
    $(-)$   & std. dev. (SD) & 0.215 & 0.193 & 0.042 \\
          & range ($R$) & 0.9   & 0.9   & 0.9 \\
          & SD / R & 0.239 & 0.214 & 0.046 \\
          \hline
              COR   & mean  & 0.708 & 0.590 & 0.590 \\
    $(-)$   & std. dev. (SD) & 0.104 & 0.073 & 0.065 \\
          & range ($R$) & 0.4   & 0.4   & 0.4 \\
          & SD / R & 0.259 & 0.183 & 0.161 \\
          \hline
    $\rho_p$ & mean  & 2245.7 & 3192.8 & 2283.9 \\
    $(kg/m3)$ & std. dev. (SD) & 80.5  & 277.4 & 67.1 \\
          & range ($R$) & 1500  & 1500  & 1500 \\
          & SD / R & 0.054 & 0.185 & 0.045 \\
          \hline
    valid & number & 290203 & 816552 & 3884 \\
    combinations & (\%) & 4.64  & 13.06 & 0.06 \\  

\hline
\end{tabular}
\caption[Valid DEM values]{Valid DEM values. For each parameter we show the
valid parameter statistics in the two tests and in their intersection.
Finally, we show the number of valid parameter combinations over the total
(6250000).}
\label{tab:13DEMvalidvalues}
\end{table*}
%************************************************
\begin{figure}%[!htb] 
\centering 
\includegraphics[width=.96\columnwidth]{26radarpirker08schulze10070} 
\caption{Parameter space plot, $SSC$, $\sigma_n=10070$ Pa, P=0.8}
\label{fig:26radarpirker08schulze10070} 
\end{figure}
%************************************************
\begin{figure}%[!htb] 
\centering 
\includegraphics[width=.96\columnwidth]{24radarpirker1schulze10070} 
\caption{Parameter space plot, $SSC$, $\sigma_n=10070$ Pa, P=1.0}
\label{fig:24radarpirker1schulze10070} 
\end{figure}
%************************************************
\begin{figure}%[!htb] 
\centering 
\includegraphics[width=.96\columnwidth]{28radarpirker12schulze10070} 
\caption{Parameter space plot, $SSC$, $\sigma_n=10070$ Pa, P=1.2}
\label{fig:28radarpirker12schulze10070} 
\end{figure}
%************************************************
% 
% \begin{figure}%[htp] 
% \centering
% %         \begin{subfigure}%[b]{0.4\columnwidth}
% %         \includegraphics[width=0.2\columnwidth]{26radarpirker08schulze10070}
% %         \caption{Parameter space plot, $SSC$, $\sigma_n=10070$ Pa, P=0.8}
% %         \label{fig:26radarpirker08schulze10070} 
% %     \end{subfigure}\\
% %      \begin{subfigure}%[b]{0.4\columnwidth}
% %         \includegraphics[width=0.2\textwidth]{24radarpirker1schulze10070}
% %         \caption{Parameter space plot, $SSC$, $\sigma_n=10070$ Pa, P=1.0}
% %         \label{fig:24radarpirker1schulze10070}
% %     \end{subfigure} \\
%         \begin{subfigure}%[b]{0.4\columnwidth} 
%         %\includegraphics[width=.45\columnwidth]
%         \includegraphics[width=0.2\textwidth]{28radarpirker12schulze10070}
%         \caption{Parameter space plot, $SSC$, $\sigma_n=10070$ Pa, P=1.2}
%         \label{fig:28radarpirker12schulze10070} 
%     \end{subfigure}
%     \caption[Parameter space plot of valid simulations parameters for three different
%     bulk behaviours measured by SSC]{Parameter space plot of valid simulation
%     parameters for three different bulk behaviours measured by a shear cell
%     tester ($SSC$).
%     Each axis of the parameter space plot represents one simulation parameter.
%     The shaded area indicates valid parameter combinations, and dark shaded
%     values indicate the confidence range.
% 	The marked combinations for $\sigma_n=10070$ Pa are presented.
%     Further explanations can be found in
%    Section \ref{subsec:experimentsparameteridentification}.
%    }
%     \label{fig:29schulzeradarandcloud}
% \end{figure}
%************************************************
\begin{figure}%[!htb] 
\centering 
\includegraphics[width=.96\columnwidth]{27cloudpirker08schulze10070} 
\caption{Density plot, $SSC$, $\sigma_n=10070$ Pa, P=0.8}
\label{fig:27cloudpirker08schulze10070} 
\end{figure}
%************************************************
\begin{figure}%[!htb] 
\centering 
\includegraphics[width=.96\columnwidth]{25cloudpirker1schulze10070} 
\caption{Density plot, $SSC$, $\sigma_n=10070$ Pa, P=1.0}
\label{fig:25cloudpirker1schulze10070} 
\end{figure}
%************************************************
\begin{figure}%[!htb] 
\centering 
\includegraphics[width=.96\columnwidth]{30cloudpirker12schulze10070} 
\caption{Density plot, $SSC$, $\sigma_n=10070$ Pa, P=1.2}
\label{fig:30cloudpirker12schulze10070} 
\end{figure}
%************************************************
% \begin{figure}%[htp] 
% \centering
% 
% %     \begin{subfigure}[b]{0.8\columnwidth}
% %         \includegraphics[width=0.5\textwidth]{27cloudpirker08schulze10070}
% %         \caption{Density plot, $SSC$, $\sigma_n=10070$ Pa, P=0.8}
% %         \label{fig:27cloudpirker08schulze10070} 
% %     \end{subfigure}\\
% %     \begin{subfigure}[b]{0.8\columnwidth}
% %         \includegraphics[width=0.5\textwidth]{25cloudpirker1schulze10070}
% %         \caption{Density plot, $SSC$, $\sigma_n=10070$ Pa, P=1.0}
% %         \label{fig:25cloudpirker1schulze10070}
% %     \end{subfigure}\\
% 
%     \begin{subfigure}[b]{0.8\columnwidth}
%         \includegraphics[width=0.5\textwidth]{30cloudpirker12schulze10070}
%         \caption{Density plot, $SSC$, $\sigma_n=10070$ Pa, P=1.2}
%         \label{fig:30cloudpirker12schulze10070} 
%     \end{subfigure}
%     \caption[Density plot comparison of SSC results]{Density plot comparison of
%     shear cell tester ($SSC$) results. The marked combinations for
%     $\sigma_n=10070 ~Pa$ are presented.
%     Density plot of the particles' coefficient of restitution (COR) as a
%     function of the coefficient of sliding friction ($\mu_s$) and the
%     coefficient of rolling friction ($\mu_r$); 
%     in the white area, no valid sets of simulation parameters can be found.
% 	In each cell the valid sets are grouped according to the 4 different COR
% 	ranges.
% 	Each cell is colored according to the group with the most members. 
%     The values plotted here were initially
%     selected between the numerical
%     values from the Artificial Neural Network with the original
%     experimental results for the $SSC$, with a product coefficient $P=1.0$ (Fig.
%     \ref{fig:25cloudpirker1schulze10070}). 
%     Subsequently, they were chosen with  
%     a lower virtual shear stress ($P=0.8$)
%     (\ref{fig:27cloudpirker08schulze10070}).
%     The last image (Fig. \ref{fig:30cloudpirker12schulze10070}) represents
%     the selection with a higher virtual shear stress ($P=1.2$).    }
%     \label{fig:29schulzeradarandcloud}
% \end{figure}
%************************************************
\begin{figure}%[!htb] 
\centering 
\includegraphics[width=.96\columnwidth]{31radarpirker1aor} 
\caption{Parameter space plot, $AoR_{exp} = 38.85 ^\circ$}
\label{fig:31radarpirker1aor} 
\end{figure}
%************************************************
\begin{figure}%[!htb] 
\centering 
\includegraphics[width=.96\columnwidth]{33radarpirker1schulze10070aor} 
\caption{Parameter space plot, $AoR_{exp} = 38.85
        ^\circ$ \& $SSC$: $\sigma_n=10070$ Pa}
\label{fig:33radarpirker1schulze10070aor} 
\end{figure}
%************************************************
% \begin{figure}%[htp] 
% \centering
% %     \begin{subfigure}[b]{0.96\columnwidth}
% %         \includegraphics[width=\textwidth]{31radarpirker1aor}
% %         \caption{Parameter space plot, $AoR_{exp} = 38.85 ^\circ$}
% %         \label{fig:31radarpirker1aor} 
% %     \end{subfigure}\\
%         \begin{subfigure}[b]{0.96\columnwidth}
%         \includegraphics[width=\textwidth]{33radarpirker1schulze10070aor}
%         \caption{Parameter space plot, $AoR_{exp} = 38.85
%         ^\circ$ \& $SSC$: $\sigma_n=10070$ Pa}
%         \label{fig:33radarpirker1schulze10070aor} 
%     \end{subfigure}
%     \caption[Parameter space plots of valid simulation parameters for the AoR
%     and the combination of AoR and SSC valid parameters]{Parameter space plots
%     of valid simulation parameters for the angle of repose tester ($AoR$) and the
%     combination of $AoR$ and shear cell tester ($SSC$).
%     Each axis of the parameter space plot represents one simulation parameter.
%     The shaded area 
%     and dark shaded values indicate
%     valid parameters combinations and
%     the confidence interval, respectively.
%     Further explanations are given in Section
%     \ref{subsec:experimentsparameteridentification}.}
%     \label{fig:35schulze10070aorradarandcloud}
% \end{figure}
%************************************************

\section{Conclusions}
\label{sec:conclusions}
%************************************************
We have presented a two-step method for $DEM$ simulation parameter
identification. In the first step, an artificial neural network is 
trained using dedicated $DEM$ simulations in order to predict bulk 
behaviours as function of a set of $DEM$ simulation parameters. 
In the second step, this artificial neural network is then used 
to predict the bulk behaviour of a huge number of additional $DEM$ parameter
sets.
The main findings of this study can be summarized as follows:
\begin{itemize}
  \item{An artificial neural network can be trained by a limited number of
  dedicated $DEM$ simulations.
  		The trained artificial neural network is then able to predict
  		granular bulk behaviour.}
  \item{This prediction of granular bulk behaviour is much more efficient
  		than computationally expensive $DEM$ simulations.
  		Thus, the macroscopic output associated with a huge number of parameter sets
  		can be studied.}
  \item{If the predictions of the artificial neural network are compared to a bulk experiment, 
  		valid sets of $DEM$ simulation parameters can be readily deduced for a
  		specific granular material.}
  \item{This $DEM$ parameter identification method can be applied to
  arbitrary bulk experiments.
  		Combining two artificial neural networks which predict two different bulk
  		behaviours leads to winnowing the set of valid $DEM$ simulation parameters.}
\end{itemize}
As part of future work, we will develop this method further by considering
different fractions of granular materials, which will lead to size-dependent sets of $DEM$
simulation parameters.

%************************************************

%ACKNOWLEDGMENTS are optional
\section{Acknowledgments}
This study was funded by the Christian Doppler Forschungsgesellschaft, Siemens
VAI Metals Technologies, and Voestalpine Stahl. The authors gratefully
aknowledge their support.

%
% The following two commands are all you need in the
% initial runs of your .tex file to
% produce the bibliography for the citations in your paper.
\bibliographystyle{abbrv}
\bibliography{Bibliografia}  % sigproc.bib is the name of the Bibliography in
% this case You must have a proper ".bib" file
%  and remember to run:
% latex bibtex latex latex
% to resolve all references
%
% ACM needs 'a single self-contained file'!
%
%APPENDICES are optional
%\balancecolumns
\appendix
%Appendix A
% \section{Headings in Appendices}
% The rules about hierarchical headings discussed above for
% the body of the article are different in the appendices.
% In the \textbf{appendix} environment, the command
% \textbf{section} is used to
% indicate the start of each Appendix, with alphabetic order
% designation (i.e. the first is A, the second B, etc.) and
% a title (if you include one).  So, if you need
% hierarchical structure
% \textit{within} an Appendix, start with \textbf{subsection} as the
% highest level. Here is an outline of the body of this
% document in Appendix-appropriate form:
% \subsection{Introduction}
% \subsection{The Body of the Paper}
% \subsubsection{Type Changes and  Special Characters}
% \subsubsection{Math Equations}
% \paragraph{Inline (In-text) Equations}
% \paragraph{Display Equations}
% \subsubsection{Citations}
% \subsubsection{Tables}
% \subsubsection{Figures}
% \subsubsection{Theorem-like Constructs}
% \subsubsection*{A Caveat for the \TeX\ Expert}
% \subsection{Conclusions}
% \subsection{Acknowledgments}
% \subsection{Additional Authors}
% This section is inserted by \LaTeX; you do not insert it.
% You just add the names and information in the
% \texttt{{\char'134}additionalauthors} command at the start
% of the document.
% % % % % % % \subsection{References}
% % % % % \bibliography{Bibliografia}
% Generated by bibtex from your ~.bib file.  Run latex,
% then bibtex, then latex twice (to resolve references)
% to create the ~.bbl file.  Insert that ~.bbl file into
% the .tex source file and comment out
% the command \texttt{{\char'134}thebibliography}.
% This next section command marks the start of
% Appendix B, and does not continue the present hierarchy
% \section{More Help for the Hardy}
% The sig-alternate.cls file itself is chock-full of succinct
% and helpful comments.  If you consider yourself a moderately
% experienced to expert user of \LaTeX, you may find reading
% it useful but please remember not to change it.
%\balancecolumns % GM June 2007
% That's all folks!
\end{document}
