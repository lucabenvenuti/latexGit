% !TEX encoding = UTF-8
% !TEX TS-program = pdflatex
% !TEX root = ../elsarticle-template-num.tex
% !TEX spellcheck = en-EN

%************************************************
\section{Introduction}
\label{sec:introduction}
%************************************************

Particles in various forms - ranging from raw materials to food grains and pharmaceutical powders - play a major role in a variety of industries, including process industry and metallurgy. In his book, \citet{RefWorks:117} stated that "between 1 and 10\% of all the energy is used in comminution, i.e. the processes of crushing, grinding, milling, micronising". 
However, a univocal method to characterize these particles has so far not been established.
From the experimental point of view, the main issues are the difficult setups and the general reliability and reproducibility of the tests. 
From the numerical point of view, no general procedure is available, and the existence of a mathematically unique solution describing macro/micro particle contact has yet to be proved.
Moreover, in a recent study, \citet{RefWorks:56} implied "that the dynamic properties of a powder cannot be applied to universally predict the static properties of a powder, and, likewise, the static properties cannot be used to predict dynamic properties".\\
Discrete Element Method (DEM) simulations are widely used to understand particle behavior.
\citet{RefWorks:135} defined the $DEM$ as "a special class of numerical schemes for simulating the behavior of discrete, interacting bodies".
The force that particle i exerts on particle j is defined as:
\begin{equation}
m \ddot{x}_{ij} + c \dot{x}_{ij} + k x_{ij} =  F_{ij} .
\label{equ:newtonlaw}
\end{equation}

Further details on the method can be found in \citet{RefWorks:133}.
$LIGGGHTS$ (LAMMPS improved for general granular and granular heat transfer simulations) is one of the most powerful open source $DEM$ simulation software packages available. 
The models it can analyze are described in detail in the literature, see \citet{RefWorks:136}.\label{par:overviewdemliggghts}
In combination with shear cell tester simulation developed by \citet{RefWorks:139}, $LIGGGHTS$ has correctly defined the coefficient of sliding friction for coarse round particles - 
a critical parameter describing inter-particle friction in medium to dense granular flows simulations.\\
Since the bulk solid is represented by perfect spheres, the only parameter the software uses to describe its shape is the radius of the particle ($R$).
However, since the shape is one of the most relevant aspects defining particle behavior, we consider the coefficient of rolling friction ($\mu_r$) as an additional $DEM$ shape parameter. 
It is proportional to the torque counteracting the rotation of the particle and defined as (Eq. \ref{equ:mur}):
\begin{equation}
 \mu_r =  \tan(\iota) .
\label{equ:mur}
\end{equation}

The last $DEM-micro$ parameter investigated is the $coefficient-of-restitution$ ($COR$), given its centrality in the whole model.
$DEM$ simulations have recently been used to reduce the bias of the experiments, and more precise devices such as the Schulze ring shear cell tester (SRSCT)(see \citet{RefWorks:104}) have been built.
A dedicated workflow that combines experiments and simulations must now be devised following the Design of Experiments method, as illustrated by \citet{RefWorks:116}.\\
The main goal of this new procedure should be the characterization of non-spherical particles, especially the $DEM$ coefficients of friction and restitution, following standardizable steps.
With this objective in mind, we profited from the shear cell experimental and numerical setup in combination with $LIGGGHTS$ simulation to improve the accuracy and the range of applicability of particle characterization.
Nevertheless, $DEM$ simulations require tens of thousand of particles to achieve the necessary reliability for a straight-forward trial-and-error calibration procedure.
The calibration compels to identify the $DEM-micro$ combination of parameters that numerically grants the same $bulk-macro$ behavior experimentally registered, measured as $steady-state-flow/pre-shear ~ coefficient-of-internal-friction $ $ (\mu_{ie-ps})$, $incipient-flow/shear ~ coefficient-of-internal-friction $ $ (\mu_{ie-s})$ and $bulk ~ density ~ (\rho_b)$.
Furthermore, we used Hertz' Law for the particle-particle and particle-wall interactions.
Its complexity increased the computational effort to fathom the micro-macro transition relationship. 
Thus, the time necessary to perform all the possible $DEM-micro$ parameters combinations became boundless.
In order to overcome this doomed situation we decided to operate artificial neural networks, as suggested by \citet{RefWorks:161}.
A limited number of combinations have been simulated, designed to maxime the representativity.
Following the indications of \citet{RefWorks:150}, \textit{feed forward Multilayer Perceptron Neural Networks (MLPNN)} have been handled.
Their trustworthiness, together with a backpropagation reinforcement learning training algorithm(scaled conjugate gradient), has been widely demonstrated in the literature, see \citet{RefWorks:158}.
The DEM parameters of the simulations have been used as inputs of the Neural Networks ($NN$), while the bulk values and behavior as targets for them.
Furthermore, the best practice suggested by \citet{RefWorks:150} demands to establish the most appropriate number of neurons inside the hidden layer of each $NN$.
As in literature, this goal has been achieved by first excluding 15\% of the simulations from the training processes.
We then fed the $NN$ with the same $DEM-micro$ parameters of these simulations.
Subsequently, we controlled the square regression error between the $bulk-macro$ behavior in these simulations and in the output of the $NN$.
Finally, we selected for each $bulk-macro$ behavior property ($\mu_{ie-ps}$, $\mu_{ie-s}$ and $\rho_b$) the $NN$ with the number of neurons that provided the maximum $R^2$.
Later, each of these three trained $NN$ received as insertion $1M$ different combinations $DEM-micro$ parameters.
We then compared their outputs against the values provided by the shear cell experiments (within a 5\% error), gaining the $DEM-micro$ coefficients range.
\citet{RefWorks:160} prescribes an ulterior validation step.
To accomplish his demand, we realized a $static-angle-of-repose$ ($SAOR$) experiment and $SAOR$ simulation.
Furthermore, we performed the $SAOR$ simulation, again with the same limited number of combinations.
These allowed to determine two $bulk-macro$ behavior properties, the $angle-of-repose$ ($AOR$) and the $bulk ~ density ~ (\rho_b)$.
Likewise, we then trained $NN$ and optimized the number of their neurons.
Later, each of these two trained $NN$ received as insertion the $DEM-micro$ coefficients range previously determined.
We then compared their outputs against the values provided by the $SAOR$ experiment (again within a 5\% error), gaining a narrower $DEM-micro$ coefficients range.
Since this study was supported by the metallurgical industry, the materials examined were: silibeads (2 mm), coke, iron ore, limestone (all 0-3.15 mm).
For the same reason, cohesive materials have been excluded from this study.\\ \label{par:materials}
trotaculo bis \\
trotaculo gx \\
culo fx
test
test