% !TEX encoding = UTF-8
% !TEX TS-program = pdflatex
% !TEX root = ../elsarticle-template-num.tex
% !TEX spellcheck = en-EN

%************************************************
%\section{settings}
%\label{sec:introduction}
%************************************************


%\documentclass[preprint,12pt]{elsarticle}

%% Use the option review to obtain double line spacing
%% \documentclass[authoryear,preprint,review,12pt]{elsarticle}

%% Use the options 1p,twocolumn; 3p; 3p,twocolumn; 5p; or 5p,twocolumn
%% for a journal layout:
%% \documentclass[final,1p,times]{elsarticle}
%% \documentclass[final,1p,times,twocolumn]{elsarticle}
%% \documentclass[final,3p,times]{elsarticle}

%% \documentclass[final,5p,times]{elsarticle}
%% \documentclass[final,5p,times,twocolumn]{elsarticle}

%% For including figures, graphicx.sty has been loaded in
%% elsarticle.cls. If you prefer to use the old commands
%% please give \usepackage{epsfig}

%% The amssymb package provides various useful mathematical symbols
\usepackage{amssymb}
%% The amsthm package provides extended theorem environments
\usepackage{amsthm}
\usepackage{amsmath}

%% The lineno packages adds line numbers. Start line numbering with
%% \begin{linenumbers}, end it with \end{linenumbers}. Or switch it on
%% for the whole article with \linenumbers.
%% \usepackage{lineno}

\usepackage{lipsum}                        % testo fittizio
\usepackage{acronym}

\journal{Powder Technology}




%% !TEX encoding = UTF-8
% !TEX TS-program = pdflatex
% !TEX root = ../Articolo.tex
% !TEX spellcheck = it-IT

%************************************************
\section{Acronyms list}
\label{sec:acro}
%************************************************
%*******************************************************
% Elenco degli acronimi
%*******************************************************

		
\begin{acronym}[TDMA]
%\acro{CDMA}{Code Division Multiple Access}
%\acro{GSM}{Global System for Mobile communication}
%\acro{NA}[\ensuremath{N_{\mathrm A}}]{Number of Avogadro\acroextra{ (see \S\ref{Chem})}}
%\acro{NAD+}[NAD\textsuperscript{+}]{Nicotinamide Adenine Dinucleotide}
%\acro{NUA}{Not Used Acronym}
%\acro{TDMA}{Time Division Multiple Access}
%\acro{UA}{Used Acronym}
%\acro{lox}[\ensuremath{LOX}]{Liquid Oxygen}%
%\acro{lh2}[\ensuremath{LH_2}]{Liquid Hydrogen}%
%\acro{IC}{Integrated Circuit}%
%\acro{BUT}{Block Under Test}%
%\acrodefplural{BUT}{Blocks Under Test}%

\acro{phis}[$\phi_s$]{angle of sliding friction}
\acro{aor}[$AOR$]{Angle of repose}
\acro{omega1}[$\omega_1$]{angular speed before first impact}
\acro{omega2}[$\omega_2$]{angular speed after first impact}
\acro{omega3}[$\omega_3$]{angular speed before second impact}
\acro{omega4}[$\omega_4$]{angular speed after second impact}
\acro{dem}[$DEM$]{Discrete Element Method}
\acro{euno}[$e_1$]{coefficient of first restitution}
\acro{cor}[$COR$]{coefficient of restitution}
\acro{edue}[$e_2$]{coefficient of second restitution}
\acro{mu}[$\mu_s$]{coefficient of sliding friction}
\acro{mur}[$\mu_r$]{coefficient of static rolling friction}
\acro{jsct}[$JSCT$]{Jenike Shear Cell tester}
\acro{pmsct}[$PMSCT$]{``Poor Man'' Shear Cell tester}
\acro{liggghts}[$LIGGGHTS$]{LAMMPS improved for general granular and granular heat transfer simulations}
\acro{r}[$r$]{radius of the particle}
\acro{ra}[$R$]{external radius of the sphere}
\acro{sasct}[$SASCT$]{Schulze Annular Shear Cell tester}

\end{acronym}



%% The Appendices part is started with the command \appendix;
%% appendix sections are then done as normal sections
%% \appendix

%% \section{}
%% \label{}

%% If you have bibdatabase file and want bibtex to generate the
%% bibitems, please use
%%

%% else use the following coding to input the bibitems directly in the
%% TeX file.

%% \bibitem{label}
%% Text of bibliographic item

%\bibitem{}


%\endinput
%%
%% End of file `elsarticle-template-num.tex'.

%\begin{thebibliography}{00}

%\end{thebibliography}
