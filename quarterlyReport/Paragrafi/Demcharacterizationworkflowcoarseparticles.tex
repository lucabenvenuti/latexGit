% !TEX encoding = UTF-8
% !TEX TS-program = pdflatex
% !TEX root = ../Articolo.tex
% !TEX spellcheck = it-IT

%************************************************
\section{DEM Characterization Workflow Coarse Particles}
\label{section:Demcharacterizationworkflowcoarseparticles}
%************************************************

\subsection{Jenike's  Shear Cell tester}
\label{subsection:jenikeshearcell}

We have performed from 6 to 16 experiments per bulk material.\\

\subsubsection{SCT simulation}
\label{subsubsection:sctsimulation}

The script has been modified to run on 32 cores over $Mach$. As of now each shear cell experiment is simulated approx. 250 times with different combinations of parameters.\\

\subsection{Angle of Repose}
\label{subsection:aor}

For each bulk have been performed at least 2 experiments, of which the mean, median and variance values have been considered.\\

\subsubsection{AOR simulation}
\label{subsubsection:aorsimulation}

The script has been modified to run on 32 cores over $Mach$. As of now each material is simulated approx. 108 times with different combinations of parameters.\\

\subsection{Granulometric curves}
\label{subsection:granulometric curves}

I have completed sieving of all materials (except from 0 to 1.25 mm iron ore) and I have for them:
\begin{itemize}
\item{the granulometric curve;}
\item{the mean radius (R);}
\end{itemize}

\subsection{COR characterization}
\label{subsection:corcharacterization}

A new system based on frequency has been suggested. I will start working on it ASAP.

\subsection{Hollow spheres}
\label{subsection:hollowspheres}

This project will remain in stand-by until further notice.

\subsection{Rolling drum}
\label{subsection:rolling drum}

Both the experiment and the simulation manage to rotate the spheres at a given velocity.
To identify the slope in the middle of the drum have been suggested:
\begin{itemize}
\item{to put a laser measurement system in the middle of the beam, that rotates together with the drum, but at least once per turn it registers correctly the ;}
\item{the mean radius (R);}
\end{itemize}
