% !TEX encoding = UTF-8
% !TEX TS-program = pdflatex
% !TEX root = ../Articolo.tex
% !TEX spellcheck = it-IT

%************************************************
\section{DEM Characterization Workflow Coarse Particles}
\label{section:Demcharacterizationworkflowcoarseparticles}
%************************************************

\subsection{Jenike's  Shear Cell tester}
\label{subsection:jenikeshearcell}

The Jenike's  Shear Cell tester (JSCT) is up and running, with also bigger rings for coarser particles, just minor details need to be done:
\begin{itemize}
\item{paint it with normal orange anti-rust painting;}
\item{establish a non-standard procedure to use the vertical laser displacement measurement and to analyze its results;}
\end{itemize}

\subsubsection{SCT simulation}
\label{subsubsection:sctsimulation}

I will correct the results of the internal friction plateau average manually, because not in all cases they were correct (I started the plateau averaging too soon in time).
I will also counter-check the data in future, unless I will find a reliable automatic evaluation tool.\\

\subsection{Large Scale Angle of Repose}
\label{subsection:largescaleaor}

I went to Leoben VAS facility to use their new rotating double chute: they performed 9 large scale dynamic angle of repose tests. I am now studying how to use these data as comparison for the small scale and the simulations.\\

\subsection{Granulometric curves}
\label{subsection:granulometric curves}

I have completed sieving of the fine materials (from 0 to 3.15 mm) and I have for them:
\begin{itemize}
\item{the granulometric curve;}
\item{the mean radius (R);}
\end{itemize}
I will ASAP (weather allowing) perform the sieving of the coarse materials.\\

From now on I will define:
\begin{itemize}
\item{$\alpha$, a simulation performed with monodisperse spheres, with the radius equal to the mean radius of the material studied or idelized;}
\item{$\beta$, with bi-disperse spheres, with 2 radii, trying to linearize the granulometric curve of the material studied or idelized;}
\item{$\gamma$, with polydisperse spheres, with all the radii of the sieving \textit{except the last one} of the material studied or idelized, to optimize the number of particles;}
\item{$\delta$, with polydisperse spheres, with all the radii of the sieving of the material studied or idelized;}
\end{itemize}


