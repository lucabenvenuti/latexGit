% !TEX encoding = UTF-8
% !TEX TS-program = pdflatex
% !TEX root = ../Articolo.tex
% !TEX spellcheck = it-IT

%************************************************
\section{Behavior Investigation}
\label{section:behaviorinvestigation}
%************************************************

This is the third draft of what should be the main topic of my PhD Thesis. The ToC is developed accordingly.

\subsection{Investigation topcis}
\label{subsection:investigationtopics}

In order to define the scientific core of my PhD Thesis the following themses will be investigated:
\begin{enumerate}
%\item{the micro-macro transition,}
\item{the influence of variations (distributions) of input parameters and poly-dispersity,}
%\item{the behavior of the different properties in real life (e.g. segregation before doing the shear cell experiment),}
\item{the possbility to extrapolate (e.g. given 3 different fraction distributions, with known behaviors, extrapolate the behavior of a fourth fraction distribution).}
\end{enumerate}

\subsection{Micro - DEM parameters}
\label{subsection:microparameters}

The main micro parameters in a simulation are:

\begin{enumerate}[label=(\Alph*)]
\item{the particle diameter distribution ($radii$ (R),\%) (0.00025 - 0.05 $[m[$),}
\item{the particle density ($\rho_p$) (2000-4000 $[kg/m^3]$),}
\item{the Young modulus ($E$) (5-10 $[MPa]$),}
\item{the Poisson ratio ($\nu$) (0-0.5 $[-]$),}
\item{the coefficient of restitution ($e$) (0.1-1.0 $[-]$),}
\item{the sliding friction ($sf$) (0.1-1.0 $[-]$),}
\item{the rolling friction ($rf$) (0.1-1.0 $[-]$),}
\item{the domain edge dimension ($D_{cyl}$), proportional to R (76, 100, 124 times bigger).}
\end{enumerate}

The parameter that drives the simulation time is $D_{cyl}$. The number of particles is cubically proportional to its size. 

\subsection{Macro - bulk parameters}
\label{subsection:macroparameters}

The main macro parameters we experimentally determine for a bulk material are:

\begin{enumerate}[label=(\alph*)]
\item{the particle diameter distribution ($radii$) (0.00001 - 0.05 $[m[$),(\%),}
\item{the bulk density ($\rho_b$) (1000-3000 $[kg/m^3]$),}
\item{the Young modulus ($E$) (5-10 $[MPa]$),}
\item{the Poisson ratio ($\nu$) (0-0.5 $[-]$),}
\item{the coefficient of restitution ($e$) (0.1-1.0 $[-]$),}
\item{the cohesion in different loading conditions ($c'$) (0-100 $[kPa]$),}
\item{the internal friction angle in different loading conditions ($\phi$) (25-50 $[^\circ]$).}
\end{enumerate}

Angle of Repose simulation, individual contacts micro-DEM parameters:
Sliding friction: from 0.05 to 1, 0.05 interval $=$ 20 possibilities;
Rolling friction: from 0.05 to 1, 0.05 interval $=$ 20 possibilities;
Coefficient  of restitution: from 0.5 to 1, 0.1 interval $=$ 6 possibilities;
Particle density: e.g. from 2500 to 3500 $kg/m3$, 100 interval $=$ 11 possibilities;
(Particle to geometry ratio: 50, 100, 150 $=$ 3 possibilities;)
20 x 20 x 6 x 11 x 3 $=$ 26400 possible combinations of parameter
Each combination would require a simulation, each simulation requires 9 hours over 32 cores $=$ 9900 days!
All for only one material with a defined size distribution (e.g. from 3 to 10 mm)!
We only consider p2g $=$ 50, because otherwise with e.g. 100 the number of particles is 8 times higher, and so the time 8 times longer

Shear cell simulation, individual contacts micro-DEM parameters:
Sliding friction: from 0.05 to 1, 0.05 interval $=$ 20 possibilities;
Rolling friction: from 0.05 to 1, 0.05 interval $=$ 20 possibilities;
Coefficient  of restitution: from 0.5 to 1, 0.1 interval $=$ 6 possibilities;
Particle density: e.g. from 2500 to 3500 kg/m3, 100 interval $=$ 11 possibilities;
Normal stress: 1, 2, 5, 10 kPa $=$ 4 possibilities;
Shear %: 40, 60, 80, 100% $=$ 4 possibilities;
(Particle to geometry ratio: 20, 36, 38, 40, 100, 150 $=$ 6 possibilities;)
20 x 20 x 6 x 11 x 4 x 4 $=$ 422400 possible combinations of parameter
Each combination would require a simulation, each simulation requires 64 minutes over 32 cores $=$ 18774 days!
All for only one material with a defined size distribution (e.g. from 3 to 10 mm)!


\subsection{Artificial neural network}
\label{subsection:artificialneuralnetwork}

The representation of a multi-sphere experiment through a mathematical model presents numerous complexities.
Do we really need all these simulations?
How to identify the effect (weight) each parameter?
A different approach involves an artificial neural network, that can be realized to understand the relationship between the input and the output parameters.
An artificial neural network is composed of many artificial neurons that are linked together according to a specific network architecture. The objective of the neural network is to transform the inputs into meaningful outputs.
The inputs are:
\begin{enumerate}
\item{\textit{A}, from real data \textit{a} but simplified,}
\item{\textit{B,H},}
\item{\textit{C,D}, from literature,}
\item{\textit{E}, from real data \textit{e},}
\item{\textit{F,G}, the main calibration parameters,}
\end{enumerate}

The outputs are \textit{f} and \textit{g}.\\

The idea would be to use a \textit{feed forward Multi Layer Perceptron Neural Network}.
A backpropagation reinforcement learning training algorithm has been used (scaled conjugate gradient).
A Neural Network has been created for each bulk parameter investigated ($\mu_{e,ps}$, $\mu_{e,s}$, $\rho_{b}$).
15\% of the simulations have been excluded from the training processes.
They have been used to define per each NN the correct number of neurons in the hidden layer, based on an $R^2$ maximization.
Then each trained NN received as input one million different combinations.
Now that the networks have been trained, we can feed the networks with all the combination we need and receive reliable macro-BULK parameters as response.
Especially, we can increase the parameter that account for the geometry dimension, without paying completely the price for it.\\
The DEM coefficients were obtained by fitting NN outputs to experimental data (within a 5\% error).
Further, we validated the DEM parameters by means of static angle-of-repose experiments and AOR simulations-trained NN.
The validation agreement was also within reliable limits (5\% error).\\

Between the 1000 simulations remained, as hypothesis 100 of them are experimentally validated. I could use 80 of them to further refine the weights of the neural network functions, and the remaining 20 to validate this second step. At this point I should be able to \textit{extrapolate the behavior of bulk with a volume too large to be simulated} with this complete neural network and to compare the results with the real scale experiments (Leoben).\\

The reliability of this work is deeply based on the already performed simulations: further numerical investigations as suggested in draft $1$ could improve it.

By reserving a portion of the simulations, we can use them to establish the most effective number of neurons inside the hidden layer, for each bulk parameter, through maximization of R2.\\