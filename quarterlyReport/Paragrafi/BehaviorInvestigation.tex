% !TEX encoding = UTF-8
% !TEX TS-program = pdflatex
% !TEX root = ../Articolo.tex
% !TEX spellcheck = it-IT

%************************************************
\section{Behavior Investigation}
\label{section:behaviorinvestigation}
%************************************************

This is the second draft of what should be the main topic of my PhD Thesis. The final version will enter and modify accordingly the ToC.

\subsection{Investigation topcis}
\label{subsection:investigationtopics}

In order to define the scientific core of my PhD Thesis the following themses could be investigated:
\begin{enumerate}
\item{the micro-macro transition,}
\item{the influence of variations (distributions) of input parameters and poly-dispersity,}
\item{the behavior of the different properties in real life (e.g. segregation before doing the shear cell experiment),}
\item{the possbility to extrapolate (e.g. given 3 different fraction distributions, with known behaviors, extrapolate the behavior of a fourth fraction distribution).}
\end{enumerate}
Design of experiments and Oberkampf's guidelines will be used to perform the investigations.
The selection and number of simulations necessary to accomplish the tasks have yet to be decided.

\subsection{Micro - DEM parameters}
\label{subsection:microparameters}

The main micro parameters that characterize a single sphere in a simulation are:

\begin{enumerate}[label=(\Alph*)]
\item{the particle diameter distribution ($radii$ (R),\%) (0.00025 - 0.05 $[m[$),}
\item{the particle density ($\rho_p$) (2000-4000 $[kg/m^3]$),}
\item{the Young modulus ($E$) (5-10 $[MPa]$),}
\item{the Poisson ratio ($\nu$) (0-0.5 $[-]$),}
\item{the coefficient of restitution ($e$) (0.1-1.0 $[-]$),}
\item{the sliding friction ($sf$) (0.1-1.0 $[-]$),}
\item{the rolling friction ($rf$) (0.1-1.0 $[-]$),}
\item{the domain edge dimension ($D_{cyl}$), proportional to R (76, 100, 124 times bigger).}
\end{enumerate}

The parameter that drives the simulation time is $D_{cyl}$. The number of particles is cubically proportional to its size. An estimation is given in Tab. \ref{tab:timestimation} with 32 cores on gollum.

\begin{table}[htbp]
  \centering
  \caption{Time estimation in hours for each shear cell simulation, see Draft 1}
    \begin{tabular}{c|ccc}
    \toprule
    $D_{cyl}$ & 76    & 100   & 124 \\
    \midrule
    $\alpha$ & 0.5   & 2     & 24 \\
    $\beta$ & 2     & 24    & 168 \\
    $\gamma$ & 168   & 336   & ? \\
    $\delta$ & 720   & ?     & ? \\
    \bottomrule
    \end{tabular}%
  \label{tab:timestimation}%
\end{table}%


\subsection{Macro - bulk parameters}
\label{subsection:macroparameters}

The main macro parameters we experimentally determine for a bulk material are:

\begin{enumerate}[label=(\alph*)]
\item{the particle diameter distribution ($radii$) (0.00001 - 0.05 $[m[$),(\%),}
\item{the bulk density ($\rho_b$) (1000-3000 $[kg/m^3]$),}
\item{the Young modulus ($E$) (5-10 $[MPa]$),}
\item{the Poisson ratio ($\nu$) (0-0.5 $[-]$),}
\item{the coefficient of restitution ($e$) (0.1-1.0 $[-]$),}
\item{the cohesion in different loading conditions ($c'$) (0-100 $[kPa]$),}
\item{the internal friction angle in different loading conditions ($\phi$) (25-50 $[^\circ]$).}
\end{enumerate}

Angle of Repose simulation, individual contacts micro-DEM parameters:
Sliding friction: from 0.05 to 1, 0.05 interval $=$ 20 possibilities;
Rolling friction: from 0.05 to 1, 0.05 interval $=$ 20 possibilities;
Coefficient  of restitution: from 0.5 to 1, 0.1 interval $=$ 5 possibilities;
Particle density: e.g. from 2500 to 3500 $kg/m3$, 100 interval $=$ 11 possibilities;
(Particle to geometry ratio: 50, 100, 150 $=$ 3 possibilities;)
20 x 20 x 5 x 11 x 3 $=$ 22000 possible combinations of parameter
Each combination would require a simulation, each simulation requires 9 hours over 32 cores $=$ 8250 days!
All for only one material with a defined size distribution (e.g. from 3 to 10 mm)!
We only consider p2g $=$ 50, because otherwise with e.g. 100 the number of particles is 8 times higher, and so the time 8 times longer

Shear cell simulation, individual contacts micro-DEM parameters:
Sliding friction: from 0.05 to 1, 0.05 interval $=$ 20 possibilities;
Rolling friction: from 0.05 to 1, 0.05 interval $=$ 20 possibilities;
Coefficient  of restitution: from 0.5 to 1, 0.1 interval $=$ 5 possibilities;
Particle density: e.g. from 2500 to 3500 kg/m3, 100 interval $=$ 11 possibilities;
Normal stress: 1, 2, 5, 10 kPa $=$ 4 possibilities;
Shear %: 40, 60, 80, 100% $=$ 4 possibilities;
(Particle to geometry ratio: 20, 36, 38, 40, 100, 150 $=$ 6 possibilities;)
20 x 20 x 5 x 11 x 4 x 4 $=$ 352000 possible combinations of parameter
Each combination would require a simulation, each simulation requires 64 minutes over 32 cores $=$ 15644 days!
All for only one material with a defined size distribution (e.g. from 3 to 10 mm)!


\subsection{Artificial neural network}
\label{subsection:artificialneuralnetwork}

The representation of a multi-sphere experiment through a mathematical model presents numerous complexities.
Do we really need all these simulations?
How to identify the effect (weight) each parameter?
A different approach involves an artificial neural network, that can be realized to understand the relationship between the input and the output parameters.
An artificial neural network is composed of many artificial neurons that are linked together according to a specific network architecture. The objective of the neural network is to transform the inputs into meaningful outputs.
The inputs are:
\begin{enumerate}
\item{\textit{A}, from real data \textit{a} but simplified,}
\item{\textit{B,H},}
\item{\textit{C,D}, from literature,}
\item{\textit{E}, from real data \textit{e},}
\item{\textit{F,G}, the main calibration parameters,}
\end{enumerate}

The outputs are \textit{f} and \textit{g}.\\

The idea would be to use a \textit{backpropagation supervised learning}. The already performed simulations could be used as follows: given e.g. 5000 simulations, I will use 3000 of them as training set, and 1000 of the remaining ones could be used to validate the neural system from the $simulation$ point of view.
In this way I could study \textit{the influence of variations of input parameters and poly-dispersity}.\\

Between the 1000 simulations remained, as hypothesis 100 of them are experimentally validated. I could use 80 of them to further refine the weights of the neural network functions, and the remaining 20 to validate this second step. At this point I should be able to \textit{extrapolate the behavior of bulk with a volume too large to be simulated} with this complete neural network and to compare the results with the real scale experiments (Leoben).\\

The reliability of this work is deeply based on the already performed simulations: further numerical investigations as suggested in draft $1$ could improve it.

By reserving a portion of the simulations, we can use them to establish the most effective number of neurons inside the hidden layer, for each bulk parameter, through maximization of R2.\\

Now that the networks have been trained, we can feed the networks with all the combination we need and receive reliable macro-BULK parameters as response.
Especially, we can increase the parameter that account for the geometry dimension, without paying completely the price for it.
E.g.: for the shear cell only $~100$ simulations can be done with $g2p >20$, but thanks to them the network can be trained to understand how that parameter relates with the others $=$ We can expand this consideration and evaluate the effect of the micro-DEM parameters over the large scale AOR test!


\subsubsection{Test run with Matlab neural fitting tool}
\label{subsubsection:testrunmatlab}

I tried a run with Matlab neural fitting tool.
773 simulations of the Jenike shear cell tester have been analyzed, with monodispersed spheres with the same radius, Young's modulus, $\nu$, $\rho_p$, $e$ and domain dimensions.
The inputs values are F, G, the normal load until steady-state flow and the \% of it during the second phase.
Especially, these \% are 40\%, 60\% and 80\%.
All other DEM values are identical for all the 773 simulations.
The target value for each simulation is the coefficient of internal friction in the the second phase.
I use 10 hidden neurons in the hidden layer and 1 neuron in the output layer.
Matlab "eats" 541 simulations to train the network, 116 to validate it and 116 to test it.\\
Now I use the network. 
The inputs values are F, G, the normal load until steady-state flow and the \% of it during the second phase.
Especially, this is only 100\%.
I have a total of 256 combinations.
I compare the coefficient of internal friction in the second phase of the simulations and of the network in these 256 cases.

 \begin{equation}
\langle{ \frac{\mu_{sh,sim}-\mu_{sh,neural}}{\mu_{sh,sim}}}\rangle = -1.16 \%
\end{equation}

The average value is really promising for this approach.

  % C_{kl} = 
 % \begin{cases}
% 1 & \text{if } (\lvert{{\mu_{psh,exp}}}\rvert < 5\% ~\text{and}~ \lvert{1-\frac{\mu_{sh,sim}}{\mu_{sh,exp}}}\rvert < 5\% ) ,\\
% 0 & \text{else} .\\ 
% \end{cases}
 % \label{eq:check}