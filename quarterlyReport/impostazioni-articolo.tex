%*********************************************************************************
% impostazioni-articolo.tex
% di Luca Benvenuti (2013)
% file che contiene le impostazioni dell'articolo
%*********************************************************************************


%*********************************************************************************
% Comandi personali
%*******************************************************
\newcommand{\myName}{Luca \textsc{Benvenuti}}                            % autore
\newcommand{\myMatricola}{16457}
\newcommand{\myTitle}{Quarterly report} % titolo
\newcommand{\myDegree}{Tesi di laurea}                       % tipo di tesi
\newcommand{\myUni}{JKU} % universit\`a
\newcommand{\myFaculty}{Strongmuslehre}    % facolt\`a
\newcommand{\myDepartment}{Department of \\ Particulate Flow Modelling}        
% dipartimento
\newcommand{\myProf}{Christoph \textsc{Kloss}}    %DI~Dr.~
\newcommand{\myOtherProf}{Stefan \textsc{Pirker}}     %DI~Dr.~         %
% eventuale correlatore \newcommand{\myOtherProff}{Ing.~Gabriele Frigerio}              % eventuale correlatore
%\newcommand{\myCounterProf}{Chiar.mo Prof.~Mister x}    
\newcommand{\myLocation}{Linz}                         % dove
\newcommand{\myTime}{\today}                          % quando
\newcommand{\myPhd}{Materials2Simulation2Application} % titolo
\newcommand{\myemail}{luca.benvenuti@jku.at} % titolo








%*********************************************************************************
% Impostazioni di amsmath, amssymb, amsthm
%*********************************************************************************

% comandi per gli insiemi numerici (serve il pacchetto amssymb)
\newcommand{\numberset}{\mathbb} 
\newcommand{\N}{\numberset{N}} 
\newcommand{\R}{\numberset{R}} 

% un ambiente per i sistemi
\newenvironment{sistema}%
  {\left\lbrace\begin{array}{@{}l@{}}}%
  {\end{array}\right.}

% definizioni (serve il pacchetto amsthm)
\theoremstyle{definition} 
\newtheorem{definizione}{Definizione}

% teoremi, leggi e decreti (serve il pacchetto amsthm)
\theoremstyle{plain} 
\newtheorem{teorema}{Teorema}
\newtheorem{legge}{Legge}
\newtheorem{decreto}[legge]{Decreto}
\newtheorem{murphy}{Murphy}[section]

%simboli matematici vari
\newcommand{\Rot}[1]{\nabla\times\vec{#1}}
\newcommand{\Div}[1]{\nabla\cdot\vec{#1}}
\newcommand{\Grad}[1]{\nabla #1}
\newcommand{\Lap}[1]{\nabla^2#1}
\newcommand{\parder}[2]{\frac{\partial #1}{\partial #2}}
\newcommand{\braket}[3]{\langle #1\,\vert\,\hat{#2}\,\vert\,#3\rangle}
\newcommand{\ud}{\mathrm{d}}
\newcommand{\total}{\mathrm{D}}

%********************************************************************************
% Impostazioni di biblatex
%*********************************************************************************

% \renewcommand\bibname{References} s'incazza
% \addto\captionsenglish{\renewcommand\refname{References}} 
% \addto\captionsitalian{\renewcommand\refname{References}} 

\DefineBibliographyStrings{english}{%
  bibliography = {Bibliography},
  references = {References},
}

\DefineBibliographyStrings{italian}{%
  bibliography = {Bibliography},
  references = {References},
}

\defbibheading{bibliography}{%
\cleardoublepage
\phantomsection 
\addcontentsline{toc}{section}{\refname}

\section*{\refname\markboth{\refname}
{\refname}}}









%*********************************************************************************
% Impostazioni di listings
%*********************************************************************************
\lstset{language=[LaTeX]Tex,%C++,
    keywordstyle=\color{RoyalBlue},%\bfseries,
    basicstyle=\small\ttfamily,
    %identifierstyle=\color{NavyBlue},
    commentstyle=\color{Green}\ttfamily,
    stringstyle=\rmfamily,
    numbers=none,%left,%
    numberstyle=\scriptsize,%\tiny
    stepnumber=5,
    numbersep=8pt,
    showstringspaces=false,
    breaklines=true,
    frameround=ftff,
    frame=single
	tabsize=2,                      % sets default tabsize to 2 spaces
    captionpos=b,                   % sets the caption-position to bottom
} 





%*********************************************************************************
% Impostazioni di hyperref
%*********************************************************************************
\hypersetup{%
    hyperfootnotes=false,pdfpagelabels,
    %draft,	% = elimina tutti i link (utile per stampe in bianco e nero)
    colorlinks=true, linktocpage=true, pdfstartpage=1, pdfstartview=FitV,%
    % decommenta la riga seguente per avere link in nero (per esempio per la stampa in bianco e nero)
    %colorlinks=false, linktocpage=false, pdfborder={0 0 0}, pdfstartpage=1, pdfstartview=FitV,% 
    breaklinks=true, pdfpagemode=UseNone, pageanchor=true, pdfpagemode=UseOutlines,%
    plainpages=false, bookmarksnumbered, bookmarksopen=true, bookmarksopenlevel=1,%
    hypertexnames=true, pdfhighlight=/O,%nesting=true,%frenchlinks,%
    urlcolor=webbrown, linkcolor=RoyalBlue, citecolor=webgreen, %pagecolor=RoyalBlue,%
    %urlcolor=Black, linkcolor=Black, citecolor=Black, %pagecolor=Black,%
    pdftitle={\myTitle},%
    pdfauthor={\textcopyright\ \myName},%
    pdfsubject={},%
    pdfkeywords={},%
    pdfcreator={pdfLaTeX},%
    pdfproducer={LaTeX with hyperref and ClassicThesis}%
}



%*********************************************************************************
% Impostazioni di graphicx
%*********************************************************************************
\graphicspath{{Immagini/}} % cartella dove sono riposte le immagini



%*********************************************************************************
% Impostazioni di xcolor
%*********************************************************************************
\definecolor{webgreen}{rgb}{0,.5,0}
\definecolor{webbrown}{rgb}{.6,0,0}



%*********************************************************************************
% Impostazioni di caption
%*********************************************************************************
\captionsetup{tableposition=top,figureposition=bottom,font=small,format=hang,labelfont=bf}





%*********************************************************************************
% Impostazioni di fancyhdr
%*********************************************************************************
\pagestyle{fancy}
\renewcommand{\sectionmark}[1]{\markboth{\sectionname\ \thesection.\ #1}{}}

\fancyhf{}


\fancyhead[LE,RO]{\thepage}
\fancyhead[RE]{\nouppercase{\leftmark}}
\fancyhead[LO]{\nouppercase{\rightmark}}


\renewcommand{\headrulewidth}{0.5pt}

\renewcommand{\footrulewidth}{0pt}
\fancyheadoffset{0\columnwidth}	


%*********************************************************************************
% Altro
%*********************************************************************************

% [...] ;-)
\newcommand{\omissis}{[\dots\negthinspace]}

% eccezioni all'algoritmo di sillabazione
\hyphenation{Fortran ma-cro-istru-zio-ne nitro-idrossil-amminico}



\newcommand{\HRule}{\rule{\linewidth}{0.5mm}}
\newcommand{\RM}[1]{\MakeUppercase{\romannumeral #1}} 
