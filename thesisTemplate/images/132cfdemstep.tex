\begin{figure}[!htb]
\centering
	\begin{tikzpicture}[auto, node distance=2cm ]%,>=latex']
	    % We start by placing the blocks
	    \node [input, name=input] {};
 	    \node [block, right of=input] (one) {one};
 	    \node [block, below of=one] (two) {two};
 	    \node [block, below of=two] (three) {three};
 	    \node [block, below of=three] (four) {four};
 	    \node [block, right of=one] (five) {five};
% 	    \node [block, right of=sum] (controller) {Controller};
% 	    \node [block, right of=controller, pin={[pinstyle]above:Disturbances},
% 	            node distance=3cm] (system) {System};
% 	    % We draw an edge between the controller and system block to 
% 	    % calculate the coordinate u. We need it to place the measurement block. 
% 	    \draw [->] (controller) -- node[name=u] {$u$} (system);
% 	    \node [output, right of=system] (output) {};
% 	    \node [block, below of=u] (measurements) {Measurements};
% 	
% 	    % Once the nodes are placed, connecting them is easy. 
 	    \draw [draw,->] (input) -- node {$start$} (one);
 	    \draw [draw,->] (one) -- node {$bis$} (two);
 	    \draw [draw,->] (two) -- node {$start$} (three);
 	    \draw [draw,->] (three) -- node {$bis$} (four);
 	    \draw [draw,->] (four) -- node {$bis$} (five); 	    
% 	    \draw [->] (sum) -- node {$e$} (controller);
% 	    \draw [->] (system) -- node [name=y] {$y$}(output);
% 	    \draw [->] (y) |- (measurements);
% 	    \draw [->] (measurements) -| node[pos=0.99] {$-$} 
% 	        node [near end] {$y_m$} (sum);
	    
		
	    
	\end{tikzpicture}
\label{fig:132cfdemstep}
\caption[Particle in a fluid]{Particle in a fluid [Wikimedia].}
\end{figure}