% German abstract
\chapter*{Zusammenfassung}
\addcontentsline{toc}{chapter}{Zusammenfassung}
\label{cap:zusammenfassung}
Viele Industriezweige verarbeiten granulare Medien.
Diese Arbeit beschreibt wie das Verhalten von Partikeln \"{u}ber numerische Simulationen, 
experimentelle Laborversuche, und k\"{u}nstlichen neuronale Netze (KNN) abgebildet werden kann.
Die makroskopischen Simulationsergebnisse der Diskreten Elemente Methode 
werden durch die gew\"{a}hlten Kontaktmodelle und die Gr\"{o}{\ss}enverteilung der Partikel bestimmt.
\"{u}blicherweise ben\"{o}tigen die verwendeten Modelle semi-empirischen Parameter,
welche nur schwierig durch direkte mikroskopische Messungen ermittelt werden
k\"{o}nnen.\\
Um diesen Aspekt zu verdeutlichen, werden die relevanten Bereiche der DEM Theorie erl\"{a}utert.
Die Theorie hinter KNN wird eingef\"{u}hrt um deren Effektivit\"{a}t zur L\"{o}sung von inversen 
Problemen mit nicht-linearer Regression zu demonstrieren.
Anschlie{\ss}end werden kleinere DEM Simulationen beschrieben, welche mit unterschiedlichen Parametern und 
Gr\"{o}{\ss}enverteilungen durchgef\"{u}hrt wurden. \\
Die makroskopischen Ergebnisse dieser Simulationen wurden dazu verwendet um
feed-forward KNNs \"{u}ber R\"{u}ckpropagierung zu trainieren.\\
Zeitgleich wurde das Verhalten von reinen Partikeln durch makroskopische Laborexperimente charakterisiert. 
Daf\"{u}r wurden f\"{u}r die Metallindustrie \"{u}bliche Partikel verwendet.
Der Zusammenhang zwischen den makroskopischen Ergebnissen und mikroskopischen DEM 
Simulationen konnte darauf hin untersucht werden.
Die k\"{u}nstlichen neuronalen Netze wurden anschlie{\ss}end verwendet um das
makroskopische Verhalten.\\
Durch diese Methode wurde eine umfassend Datenbank erstellt, welche Parameter von 
Partikel Simulationen mit makroskopsichen Ausgaben verkn\"{u}pft.
Diese Datenbank kann nun f\"{u}r ein Experiment mit bestimmten granularen Material eine 
g\"{u}ltige Auswahl von DEM Parametern ermitteln werden.
Die Methode wurde zum Abschluss dazu verwendet DEM parameter f\"{u}r zwei industrielle 
Prozesse bei der Stahlerzeugung zu identifizieren.
