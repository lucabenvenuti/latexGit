%% Abstracts %%%%%%%%%%%%%%%%%%%%%%%%%%%%%%%%%%%%%%%%%%%%%%%%
% english and german abstract are mandatory, german in zusammenfassung
\chapter*{Abstract}
\addcontentsline{toc}{chapter}{Abstract}
\label{cap:abstract}

Numerous industries process particles.
In this work, we focused on how to efficiently picture the behaviour of
particles by means of numerical simulations, laboratory experiments, 
and Artificial Neural Networks (ANNs).

Particle-particle contact laws and particles size distributions determine the
macroscopic simulation results in Discrete Element Method (DEM). 
Commonly, contact laws depend on semi-empirical parameters which 
are difficult to obtain by direct microscopic measurements. 

To clarify this aspect, we present the related elements of the DEM
theory.
The ANN theory is also introduced to demonstrate ANN effectiveness towards
generalization.

Later, we describe the series of small scale DEM simulations with different sets
of particle-based simulation parameters and particle distributions, which we
performed.
The macroscopic results of these simulations were used to train dedicated
feed-forward ANNs by backward propagation reinforcement.
Concurrently, the bulk behaviours of raw particles were characterized by means
of macroscopic laboratory experiments. These particles were those commonly used
by metallurgical industries (i.e., coke, iron ore, sinter, and limestone).

At this point, the relationship between macroscopic results and microscopic DEM
simulation parameters could be investigated.

We subsequently utilized this artificial neural network to predict the
macroscopic ensemble behaviour in relation to additional sets of particle-based simulation parameters and particle distributions. 
By this method, a comprehensive database was established, relating particle-based 
simulation parameters to macroscopic ensemble output.
If compared to an experiment of a specific granular material, this database identifies 
valid sets of DEM parameters which lead to the same macroscopic results as observed in the experiments.
Finally, we applied this method of DEM parameter identification to two industrial
scale processes of steel production.
