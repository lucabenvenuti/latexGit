% !TEX encoding = UTF-8
% !TEX TS-program = pdflatex
% !TEX root = ../Tesi.tex
% !TEX spellcheck = en-EN

%************************************************

\chapter{DEM Simulations}
\label{cap:demsimulations}
%************************************************

All the results in this chapter are pure \acs{DEM}, so they shoould be
materials agnostic, given the same particle distribution.

\section{Influence of variations of input parameters}
\label{sec:influence}

For sinter fine, 546 shear cell and 81 static \acs{AoR} simulations 
\wrong{write down all the simulations performed at the end.}
were run with
the parameter combinations described in Table
\ref{tab:10DEMVariableinputvalues}.
The computational time amounted to 1 hour with 32 AMD cores for a benchmark
shear-cell simulation and to 9 hours for a benchmark \acs{AoR} simulation, both with
50,000 particles.
Simulations with larger \acs{dCylDp} required more time (e.g., about 12 hours for
the shear cell with 400,000 particles ). \\

\begin{table}[h]
\centering
\begin{tabular}{c|c|c|c|c}
\hline
	DEM   & DEM   & DEM   & average & simulation \\
    sliding & rolling & coefficient & particle & domain diameter \\
    friction & friction & restitution & density & to particle mean \\
    	$[-]$  & $[-]$   & $[-]$   & $[kg/m3]$ & diameter ratio \\
    \hline
    0.4 - 0.6 - 0.8 & 0.4 - 0.6 - 0.8 & 0.5 - 0.7 - 0.9 & 2500 - 3000 - 3500 & 20 - 36 - 38 - 40 \\

\hline
\end{tabular}
\caption{DEM variable input values}
\label{tab:10DEMVariableinputvalues}
\end{table}.

\info{Some examples on how the numerical bulk values change when the DEM input
values change.}
\info{or sub-chapter with PCA?}

\section{PCA analysis}
\label{sec:pcaanalysis}

The linear relationship between the
training values can be seen in Table \ref{tab:06inputRelationshipTable}.
Sliding friction (\acs{mus}), rolling friction (\acs{mur}) and particle density (\acs{rhop})
had the greatest influence on, respectively, the coefficient of pre-shear
(\acs{mupsh}), the angle of repose  (\acs{AoR}) and the bulk density (\acs{rhob}). Notably, \acs{rhop}
was not used as a training parameter for \acs{AoR} bulk behaviour. 
\begin{table}[H!]                                                                                                                                                          
\centering                                                                                                                                                                 
\begin{tabular}{|c|c|c|c|c|c|c|c|c|c|c|c|}                                                                                                                                 
\hline                                                                                                                                                                     
 & sf & rf & rest & dt & dCylDp & ctrlStress & shearperc & dens & mush & mupsh & rhob \\                                                                                   
\hline                                                                                                                                                                     
sf & 1 & 5.549787e-03 & -3.818461e-04 & -1.268763e-15 & -1.628657e-02 & 1.282025e-15 & 4.517397e-03 & 0 & 3.838826e-02 & 8.725701e-01 & -8.393464e-02 \\                   
\hline                                                                                                                                                                     
rf & 5.549787e-03 & 1 & -1.523330e-03 & -2.349289e-15 & -5.968531e-02 & 2.322503e-15 & 1.802162e-02 & 3.348007e-18 & 5.891756e-01 & 3.370233e-01 & -3.101856e-02 \\        
\hline                                                                                                                                                                     
rest & -3.818461e-04 & -1.523330e-03 & 1 & -1.555718e-15 & -2.758674e-01 & 1.568359e-15 & 8.090707e-02 & 6.680307e-18 & 1.551852e-01 & -5.671687e-03 & -1.712429e-02 \\    
\hline                                                                                                                                                                     
dt & -1.268763e-15 & -2.349289e-15 & -1.555718e-15 & 1 & -1.026312e-16 & -1.000000e+00 & -2.681936e-17 & 0 & 6.168810e-16 & -4.320958e-15 & 1.243669e-14 \\                
\hline                                                                                                                                                                     
dCylDp & -1.628657e-02 & -5.968531e-02 & -2.758674e-01 & -1.026312e-16 & 1 & 7.853515e-17 & -2.939311e-01 & 2.688281e-17 & -2.879551e-01 & -1.916393e-01 & 9.603603e-02 \\ 
\hline                                                                                                                                                                     
ctrlStress & 1.282025e-15 & 2.322503e-15 & 1.568359e-15 & -1 & 7.853515e-17 & 1 & -3.731389e-17 & 0 & -6.100950e-16 & 4.292811e-15 & -1.234126e-14 \\                      
\hline                                                                                                                                                                     
shearperc & 4.517397e-03 & 1.802162e-02 & 8.090707e-02 & -2.681936e-17 & -2.939311e-01 & -3.731389e-17 & 1 & -3.512479e-17 & 5.730199e-02 & 5.380657e-02 & -5.095294e-03 \\
\hline                                                                                                                                                                     
dens & 0 & 3.348007e-18 & 6.680307e-18 & 0 & 2.688281e-17 & 0 & -3.512479e-17 & 1 & -4.980664e-02 & 5.709445e-02 & 9.900341e-01 \\                                         
\hline                                                                                                                                                                     
mush & 3.838826e-02 & 5.891756e-01 & 1.551852e-01 & 6.168810e-16 & -2.879551e-01 & -6.100950e-16 & 5.730199e-02 & -4.980664e-02 & 1 & 2.603411e-01 & -9.516313e-02 \\      
\hline                                                                                                                                                                     
mupsh & 8.725701e-01 & 3.370233e-01 & -5.671687e-03 & -4.320958e-15 & -1.916393e-01 & 4.292811e-15 & 5.380657e-02 & 5.709445e-02 & 2.603411e-01 & 1 & -4.329071e-02 \\     
\hline                                                                                                                                                                     
rhob & -8.393464e-02 & -3.101856e-02 & -1.712429e-02 & 1.243669e-14 & 9.603603e-02 & -1.234126e-14 & -5.095294e-03 & 9.900341e-01 & -9.516313e-02 & -4.329071e-02 & 1 \\   
\hline                                                                                                                                                                     
\end{tabular}                                                                                                                                                              
\caption{MyTableCaption}                                                                                                                                                   
\label{table:MyTableLabel}                                                                                                                                                 
\end{table}               
\improvement{underline that is good that the input parameters are not
correlated}
