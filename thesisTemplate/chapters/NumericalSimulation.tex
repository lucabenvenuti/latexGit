% !TEX encoding = UTF-8
% !TEX TS-program = pdflatex
% !TEX root = ../Tesi.tex
% !TEX spellcheck = en-EN

%************************************************
\chapter{Numerical Simulation}
\label{cap:numericalsimulation}
%************************************************

\lipsum[1]


\section{Angle of Repose}
\label{sec:aorsim}


In $AoR$ simulations we sought to replicate meticulously the experimental setup, 
considering both the plate and the liftable boundary, with the same domain size consideration as before. 
The particles had the same properties as in the shear-cell simulation. 
The first phase was identical to that of the shear-cell simulation. 
After lifting the boundary, the particles formed a heap.
An image post-processing software was used to obtain the average slope.


\section{Shear Cell}
\label{sec:scsimulation}
%************************************************

For each particle i inside the domain, a Discrete Element Method ($DEM$) code
follows the trajectory and calculates the force that particle i exerts on particle j.
The main forces involved are: gravity, contact forces due to collisions, and
further interactions such as electrostatic, Van der Waals, cohesive forces and fluid-solid interactions in multiphase flows. For the raw material used in this work 
Di Renzo and Di Maio \cite{RefWorks:145} suggested using the non-linear
Hertzian model without cohesion for the particle-particle and particle-wall contacts. 
This granular model uses the following formula for the contact force between two granular particles (Eq. \ref{eq:forceij}):
%************************************************
\begin{equation}
 F_{ij} = 
\begin{cases}
F_{n,ij} + F_{t,ij} = \left( k_n \delta_{n,ij} + \gamma_n v_{n,ij} \right) + \left( k_t \delta_{t,ij} + \gamma_t v_{t,ij} \right) & \text{if } r < d ,\\
0    & \text{if } r > d ,\\
\end{cases}
 \label{eq:forceij}
\end{equation}

%************************************************
where the subscript $n$ stands for normal and $t$ for tangential. 
Here, $k$ and $\gamma$ are respectively the stiffness and damping coefficients, 
while $\delta$ and $v$ are the displacement and the velocity, $r$ is the
distance between two particles of radii $R_i$ and $R_j$, and $d = R_i + R_j $ is
the contact distance.
Both the normal and the tangential
force comprise two terms, a spring force and a damping force. 
The tangential (or shear) force is a ``history'' effect that accounts for the
tangential displacement (``tangential overlap'') between the particles for the
duration of contact.
In the paper by Wensrich and Katterfeld \cite{RefWorks:87}, further details on
the method can be found.\\
$LIGGGHTS$, the simulation toolbox we used, meets all the requirements of
modelling the shear tester described in \ref{subsec:srsctexperiment}. 
First, it is capable of importing triangulated meshes of the two rings and a top lid. 
Since the real setup had a wall thickness, contact forces acting on a mesh are summed and can be saved, 
and thus shear force calculation is available out of the box. 
Second, the code can move a mesh with constant 
velocity as required for the measurement. 
To determine the shear stresses, the bulk solid had to be stressed with 
user-defined normal stresses. 
Therefore, a stress-controlled wall (servo-wall in $LIGGGHTS$) was applied to the lid. \\
Although the geometry differs, the $SSC$ was designed to obtain the same values for the shear stresses as the 
Jenike shear cell tester ($JSCT$), but with improved automation and reliability,
see Schulze \cite{RefWorks:118}. 
For this reason, the simulation setup was
based on the $JSCT$.
As suggested by Aigner et al. \cite{RefWorks:139} and Benvenuti et al. \cite{RefWorks:173}, 
the diameter and the height of the rings operated in the simulations had to be sufficiently large to avoid relevant wall effects. 
However, a larger domain increases the number of particles and thus
simulation time.
For this reason, we considered the cylinder dimension ($dCylDp$), which is
proportional to the mean particle diameter, as an additional $DEM$
parameter to be investigated. \\
A simulation run comprised four phases. 
First, the shear cell was filled with the granulate material, and it was allowed
to settle.
Then, the top lid was lowered and the first normal stress applied to the bulk
solid.
As in the experiment, the servo-wall allows calculating the position of the lid
while the first particle is being touched. 
The distance between the lid and the bottom of the domain multiplied by the 
simulation area gave the total volume.
Since the software already provided the total mass, we were able to calculate:
%************************************************
\begin{equation}
\rho_b = \frac{mass}{volume}.
 \label{eq:rhob}
\end{equation}

%************************************************
Next, the ring moved by a distance $l=0.1875 \cdot radius ~of ~the ~ring$, and
the required pre-shear force was measured.
Finally, the normal load was reduced to a fraction of the initial load, 
the ring was moved again by a distance $l$, and the shear force was recorded. 
Unlike in the original experiment, the bottom ring was moved to facilitate the numerical simulation. 
The velocity of the ring displacement, and consequently the total simulation time, 
was determined by a trade-off between minimizing the normal load oscillation and
optimizing computation time. 
The former requires (relatively) low velocity, and the latter requires high
speed.
We chose a constant velocity of $3*(mean ~ particle ~ radius)/seconds$, as the
best trade-off.\\
The normal stresses (pre-shear and shear phases) applied in each simulation were
the same as in the experiments, see \ref{subsec:srsctexperiment}.
The corresponding $\tau_{psh}$ and $\tau_{sh}$ were calculated - as in the experiments - from the mean of the plateau.\\