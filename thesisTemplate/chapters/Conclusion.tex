% !TEX encoding = UTF-8
% !TEX TS-program = pdflatex
% !TEX root = ../Tesi.tex
% !TEX spellcheck = en-EN

%************************************************
\part{Conclusion}
\label{par:conclusion}
%************************************************

We have presented a two-step method for \acs{DEM} simulation parameter
identification, and used the identified parameters for large scale numerical
simulations.
In the first step, an artificial neural network is
trained using dedicated \acs{DEM} simulations in order to predict bulk
behaviours as function of a set of \acs{DEM} simulation parameters.
In the second step, this artificial neural network is then used
to predict the bulk behaviour of a huge number of additional \acs{DEM} parameter
sets.
The main findings of this study can be summarized as follows:
\begin{itemize}
  \item{An artificial neural network can be trained by a limited number of
  dedicated \acs{DEM} simulations.
  The trained artificial neural network is then able to predict
  granular bulk behaviour.}
  \item{This prediction of granular bulk behaviour is much more efficient
  than computationally expensive \acs{DEM} simulations.
  Thus, the macroscopic output associated with a huge number of parameter sets
  can be studied.}
  \item{If the predictions of the artificial neural network are compared to a bulk experiment,
  valid sets of \acs{DEM} simulation parameters can be readily deduced for a
  specific granular material.}
  \item{This \acs{DEM} parameter identification method can be applied to
  arbitrary bulk experiments.
  Combining two artificial neural networks which predict two different bulk
  behaviours leads to winnowing the set of valid \acs{DEM} simulation parameters.}
  \item{In turn, an empty intersection between the networks would be a clear
  indication that the chosen contact law is not appropriate.}
  \item{Artificial neural networks can establish the complicated relationship
  between microscopic \acs{DEM} parameters of a given contact law and macroscopic
  properties of granular materials with less \acs{DEM} simulations than direct
  procedures.}
  \item{Additional numerical simulations for additional materials with different
  size distributions and stress conditions can be used to train additional \acs{ANNs}. 
  These predict the numerical bulk for the investigated material. Again,
  confrontation with experimental measumerements provides valid microscopic
  values.}
  \item{The valid values obtained are used for a large scale simulation. It
  represents the movements of sinter particles along a chute. The simulation,
  tuned with the given parameters, successfully demonstrated the effectiveness
  of the chute as a segregation system.}
  \item{Similarly, the valid values for all the investigated materials were
  used in a blast furnace simulation. Especially, the raceway area was
  investigated. We evaluated the effect of the variation of the input velocity
  at the tuyere. The numerical results are consistent with the literature data.}
\end{itemize}
