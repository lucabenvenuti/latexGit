% !TEX encoding = UTF-8
% !TEX TS-program = pdflatex
% !TEX root = ../Tesi.tex
% !TEX spellcheck = en-EN

%************************************************
\part{Conclusion}
\label{par:conclusion}
%************************************************

We have presented a two-step method for \acs{DEM} simulation parameter
identification. 
\wrong{3 steps: 2 step id + 1 step app}
In the first step, an artificial neural network is 
trained using dedicated \acs{DEM} simulations in order to predict bulk 
behaviours as function of a set of \acs{DEM} simulation parameters. 
In the second step, this artificial neural network is then used 
to predict the bulk behaviour of a huge number of additional \acs{DEM} parameter
sets.
The main findings of this study can be summarized as follows:
\begin{itemize}
  \item{An artificial neural network can be trained by a limited number of
  dedicated \acs{DEM} simulations.
  		The trained artificial neural network is then able to predict
  		granular bulk behaviour.}
  \item{This prediction of granular bulk behaviour is much more efficient
  		than computationally expensive \acs{DEM} simulations.
  		Thus, the macroscopic output associated with a huge number of parameter sets
  		can be studied.}
  \item{If the predictions of the artificial neural network are compared to a bulk experiment, 
  		valid sets of \acs{DEM} simulation parameters can be readily deduced for a
  		specific granular material.}
  \item{This \acs{DEM} parameter identification method can be applied to
  arbitrary bulk experiments.
  		Combining two artificial neural networks which predict two different bulk
  		behaviours leads to winnowing the set of valid \acs{DEM} simulation parameters.}
   \item{\info{polidispersity}}
   \item{\info{application 1}}
   \item{\info{application 2}}
\end{itemize}
:-)
