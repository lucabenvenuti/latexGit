% !TEX encoding = UTF-8
% !TEX TS-program = pdflatex
% !TEX root = ../Tesi.tex
% !TEX spellcheck = en-EN

%************************************************
\chapter{Influence of variations of input parameters}
\label{cap:influence}
%************************************************

All the results in this chapter are pure \acs{DEM}, so they shoould be
materials agnostic, given the same particle distribution.

\section{DEM Simulations}
\label{sec:simulations}

For sinter fine, 546 shear cell and 81 static \acs{AoR} simulations 
\wrong{write down all the simulations performed at the end.}
were run with
the parameter combinations described in Table
\ref{tab:10DEMVariableinputvalues}.
The computational time amounted to 1 hour with 32 AMD cores for a benchmark
shear-cell simulation and to 9 hours for a benchmark \acs{AoR} simulation, both with
50,000 particles.
Simulations with larger \acs{dCylDp} required more time (e.g., about 12 hours for
the shear cell with 400,000 particles ). \\

\begin{table}[h]
\centering
\begin{tabular}{ccccc}
\hline
    \acs{mus} & \acs{mur} & \acs{CoR} & \acs{rhop} & \acs{dCylDp} \\
    	$[-]$  & $[-]$   & $[-]$   & $[kg/m3]$ & $[-]$ \\
    \hline
    0.4 / 0.6 / 0.8 & 0.4 / 0.6 / 0.8 & 0.5 / 0.7 / 0.9 & 2500 / 3000 / 3500 & 20 / 36 / 38 / 40 \\

\hline
\end{tabular}
\caption[DEM variable input values]{DEM variable input values for training the
Neural Networks}
\label{tab:10DEMVariableinputvalues}
\end{table}.

\info{Some examples on how the numerical bulk values change when the DEM input
values change.}
\info{or sub-chapter with PCA?}

\section{PCA analysis}
\label{sec:pcaanalysis}

The linear relationship between the
training values can be seen in Table \ref{tab:06inputRelationshipTable}.
Sliding friction (\acs{mus}), rolling friction (\acs{mur}) and particle density (\acs{rhop})
had the greatest influence on, respectively, the coefficient of pre-shear
(\acs{mupsh}), the angle of repose  (\acs{AoR}) and the bulk density (\acs{rhob}). Notably, \acs{rhop}
was not used as a training parameter for \acs{AoR} bulk behaviour. 
\begin{table}[h]
\centering
\scalebox{1.0}{
\begin{tabular}{c|cccccccc}
\hline
          & $\mu_s$ & $\mu_r$ & $COR$ & $\rho_p$ & $\mu_{sh}$ & $\mu_{psh}$ & $\rho_{b}$ & $AOR$ \\
          \hline
    $\mu_s$ & 100.00 & 0.55  & 0.04  & 0.00  & 3.84  & 87.26 & 8.39  & 49.48 \\
    $\mu_r$ & 0.55  & 100.00 & 0.15  & 0.00  & 58.92 & 33.70 & 3.10  & 60.20 \\
    $COR$ & 0.04  & 0.15  & 100.00 & 0.00  & 15.52 & 0.57  & 1.71  & 0.00 \\
    $\rho_p$ & 0.00  & 0.00  & 0.00  & 100.00 & 4.98  & 5.71  & 99.00 & 0.00 \\
    $\mu_{sh}$ & 3.84  & 58.92 & 15.52 & 4.98  & 100.00 & 26.03 & 9.52  & 0.00 \\
    $\mu_{psh}$ & \textbf{87.26} & 33.70 & 0.57  & 5.71  & 26.03 & 100.00 & 4.33 
    & 0.00
    \\
    $\rho_{b}$ & 8.39  & 3.10  & 1.71  & \textbf{99.00} & 9.52  & 4.33  & 100.00
    & 0.00 \\
    $AOR$ & 49.48 & \textbf{60.20} & 0.00  & 0.00  & 0.00  & 0.00  & 0.00  &
    100.00 \\
    
\hline
\end{tabular}}
\caption{Values of linear relationship between considered variables multiplied
for 100}
\label{tab:06inputRelationshipTable}
\end{table}
\improvement{underline that is good that the input parameters are not
correlated}
