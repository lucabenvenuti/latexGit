% !TEX encoding = UTF-8
% !TEX TS-program = pdflatex
% !TEX root = ../Tesi.tex
% !TEX spellcheck = en-EN

%************************************************
\chapter{Aim}
\label{cap:aim}
%************************************************

In our study, we harnessed Artificial Neural Networks ($ANNs$) in order to
reduce the number of $DEM$ test simulations required. 
$ANNs$ have proven to be a versatile tool in analysing complex, non-linear
systems of multi-dimensional input streams (Vaferi et al. \cite{RefWorks:150}, Witten et
al. \cite{RefWorks:174} and Haykin \cite{RefWorks:158}).
In our case, we fed an $ANN$ with $DEM$ contact law parameters as input
and compared the output with the bulk behaviour 
predicted by a corresponding $DEM$ simulation. 
The difference between $ANN$ prediction and $DEM$ prediction is used to train our 
specific $ANN$ with a backward-propagation algorithm. 
After a training phase comprising a limited number of $DEM$ test simulations,
the $ANN$ can then be used as a stand-alone prediction tool for the bulk behaviour of a 
granular material in relation to $DEM$ contact law parameters. \\
In this study, we applied this parameter identification method to two different
granular bulk behaviours, namely the angle of repose ($AoR$) test and the
Schulze shear cell ($SSC$) test.
In both cases, we first trained a specific $ANN$ using a number of $DEM$ test
simulations before we identified valid sets of $DEM$ contact law parameters by
comparing the stand-alone $ANN$ predictions with corresponding bulk experiments. 
For both cases we obtained valid sets of contact law parameters, 
which we then compared to formulate a reliable contact law for a given
granular material.
We further show that the same $ANN$ can be used to characterize different granular materials, 
which have the same particle behaviour and can modelled with the same contact
law.