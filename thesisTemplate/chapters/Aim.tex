% !TEX encoding = UTF-8
% !TEX TS-program = pdflatex
% !TEX root = ../Tesi.tex
% !TEX spellcheck = en-EN

%************************************************
\chapter{Aim}
\label{cap:aim}
%************************************************

In our study, we harnessed Artificial Neural Networks (\acs{ANNs}) in order to
reduce the number of \acs{DEM} test simulations required. 
\acs{ANNs} have proven to be a versatile tool in analysing complex, non-linear
systems of multi-dimensional input streams (Vaferi et al. \cite{RefWorks:150}, Witten et
al. \cite{RefWorks:174} and Haykin \cite{RefWorks:158}).
In our case, we fed an \acs{ANN} with \acs{DEM} contact law parameters as input
and compared the output with the bulk behaviour 
predicted by a corresponding \acs{DEM} simulation. 
The difference between \acs{ANN} prediction and \acs{DEM} prediction is used to train our 
specific \acs{ANN} with a backward-propagation algorithm. 
After a training phase comprising a limited number of \acs{DEM} test simulations,
the \acs{ANN} can then be used as a stand-alone prediction tool for the bulk behaviour of a 
granular material in relation to \acs{DEM} contact law parameters. \\
In this study, we applied this parameter identification method to two different
granular bulk behaviours, namely the angle of repose (\acs{AoR}) test and the
Schulze shear cell (\acs{SCT}) test.
In both cases, we first trained a specific \acs{ANN} using a number of \acs{DEM} test
simulations before we identified valid sets of \acs{DEM} contact law parameters by
comparing the stand-alone \acs{ANN} predictions with corresponding bulk experiments. 
For both cases we obtained valid sets of contact law parameters, 
which we then compared to formulate a reliable contact law for a given
granular material.
We further show that the same \acs{ANN} can be used to characterize different granular materials, 
which have the same particle behaviour and can modelled with the same contact
law.

\improvement{add some words on the applications}
\improvement{add some cool pictures}