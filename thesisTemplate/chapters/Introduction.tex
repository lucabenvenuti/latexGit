% !TEX encoding = UTF-8
% !TEX TS-program = pdflatex
% !TEX root = ../Tesi.tex
% !TEX spellcheck = en-EN

%************************************************
\part{Introduction}
\label{par:introduction}
%************************************************

Particles in various forms - ranging from raw materials to food grains and pharmaceutical powders - 
play a major role in a variety of industries, including process industry and metallurgy. 
In his book, \citet{RefWorks:117} stated that "between 1 and 10\% of all the energy is used in 
comminution, i.e. the processes of crushing, grinding, milling, micronising". 

Discrete Element Methods ($DEMs$) , "a special class of numerical schemes for
simulating the behavior of discrete, interacting bodies", are widely used to 
simulate particle behaviour in these granular processes
(\citet{RefWorks:130}).\\
In their original formulation of $DEM$, \citet{RefWorks:172} allowed two 
particles to slightly overlap upon contact, and consequently they proposed
repulsive forces in relation to this overlap distance.
The force that particle i exerts on particle j is defined as:
\begin{equation}
m \ddot{x}_{ij} + c \dot{x}_{ij} + k x_{ij} =  F_{ij} .
\label{equ:newtonlaw}
\end{equation}


%\begin{table}[h]
\centering
\begin{tabular}{l}
\hline 
    Radius \ac{R} (m)   \\ [5pt]

	Size distribution (-) \\ [5pt]

    Young's modulus \ac{E} (Pa)  \\ [5pt]

    Poisson's ratio \ac{nu} (-) \\ 
     Time step \ac{deltat} (s) \\ [5pt]
        \hline
     Coefficient of sliding friction \ac{mus} (-)\\  [5pt]
    Coefficient of rolling friction \ac{mur} (-) \\ [5pt]
    Coefficient of restitution \ac{CoR} (-)   \\ [5pt]
     Particle density $\ac{rhop} = \frac{mass ~ of ~ one ~ particle}{volume ~ of
     ~ one ~ particle}$ ($kg/m^3$)  \\ [5pt]
     Geometry factor \ac{dCylDp} (-)  \\ [5pt]
   
\hline
\end{tabular}
\caption[DEM parameters]{DEM parameters. The upper parameters were
identical in all simulations. The lower parameters were constant in each
simulation, but were varied between simulations.}
\label{tab:08DEMparameters}
\end{table}



