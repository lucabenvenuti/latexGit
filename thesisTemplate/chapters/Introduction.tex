% !TEX encoding = UTF-8
% !TEX TS-program = pdflatex
% !TEX root = ../Tesi.tex
% !TEX spellcheck = it-IT

%************************************************
\part{Introduction}
\label{par:introduction}
%************************************************

\lipsum[1]


Particles in various forms - ranging from raw materials to food grains and pharmaceutical powders - 
play a major role in a variety of industries. 
Discrete Element Methods ($DEMs$) are widely used to simulate
particle behaviour in these granular processes (Cleary and Sawley \cite{RefWorks:130}).\\
In their original formulation of $DEM$, Cundall and Strack \cite{RefWorks:172} allowed two 
particles to slightly overlap upon contact, and consequently they proposed
repulsive forces in relation to this overlap distance.
Their fundamental modelling concept has since been widely accepted in the
literature and their soft-sphere contact law has been developed further by
numerous researchers (Vu-Quoc and Zhang \cite{RefWorks:148} and Di Renzo and Di Maio \cite{RefWorks:145}). 
With increasing computational resources, $DEM$ simulations have become very
popular giving rise to the development of commercial (e.g., $PFC3D$, used by
Wensrich and Katterfeld \cite{RefWorks:87}) and open-source software (e.g.,
$LIGGGHTS$, Kloss et al. \cite{RefWorks:136}, Aigner et al. \cite{RefWorks:139}).
Soft-sphere $DEM$ simulations of thousands of particles have been proven to 
faithfully model particle bulk behaviour (Hohner et al. \cite{RefWorks:86}). \\
In these macroscopic $DEM$ simulations, the contact law kernel between a 
pair of particles determines the global bulk behaviour of the granular material (Ai et al. \cite{RefWorks:131}). 
As a consequence, defining a correct contact law is of crucial importance for the predictive 
capability of $DEM$ simulations. 
Since $DEM$ contact laws are based 
on a set of semi-empirical parameters, correct contact law 
parameters must be defined for a given granular material
or $DEM$ simulations will fail (Combarros et al. \cite{RefWorks:177}). \\
Identifying $DEM$ contact law parameters is not a trivial task. 
Due to the huge number of particles in a granular material, it
may be impractical to identify valid parameter sets by performing bilateral 
particle collision experiments. 
Furthermore, some contact law parameters such as the coefficient of rolling
friction are purely empirical and cannot be determined by direct 
particle-to-particle measurements (Wensrich and Katterfeld \cite{RefWorks:87}).
Therefore, $DEM$ contact law parameters (Table \ref{tab:08DEMparameters}) are
commonly determined by comparing the macroscopic outcome of large-scale $DEM$
simulations with bulk experiments (Alenzi et al. \cite{RefWorks:91}). 
If $DEM$ simulation results disagree with bulk measurements, the set of contact
law parameters must be adjusted until reasonable agreement is achieved.\\
However, this purely forward methodology of parameter identification is limited by 
the multi-dimensionality of the parameter space and the associated computational costs of the required 
$DEM$ test simulations. 
Moreover, one parameter set which is valid for one bulk behaviour (e.g., angle
of repose) might fail for another (e.g., shear tester). \\
There are yet ways to determine contact parameters directly by measuring
material properties or by performing particle based experiments, see e.g. Combarros et al. \cite{RefWorks:177}, 
Paulick et al. \cite{RefWorks:181}, and Lommen et al. \cite{RefWorks:186}. 
However, these methodologies are laborious, 
since they have to be performed for every new granular material prior to a $DEM$
simulations. 
Especially for the already cited rolling friction parameter, it is arduous to
link the rolling friction parameter to the non-sphericity of the particle. Clearly, there is a
need for an efficient method for identifying $DEM$ contact law parameters, given
a specific particle behaviour.
In our study, we harnessed Artificial Neural Networks ($ANNs$) in order to
reduce the number of $DEM$ test simulations required. 
$ANNs$ have proven to be a versatile tool in analysing complex, non-linear
systems of multi-dimensional input streams (Vaferi et al. \cite{RefWorks:150}, Witten et
al. \cite{RefWorks:174} and Haykin \cite{RefWorks:158}).
In our case, we fed an $ANN$ with $DEM$ contact law parameters as input
and compared the output with the bulk behaviour 
predicted by a corresponding $DEM$ simulation. 
The difference between $ANN$ prediction and $DEM$ prediction is used to train our 
specific $ANN$ with a backward-propagation algorithm (described further below). 
After a training phase comprising a limited number of $DEM$ test simulations,
the $ANN$ can then be used as a stand-alone prediction tool for the bulk behaviour of a 
granular material in relation to $DEM$ contact law parameters. \\
In this study, we applied this parameter identification method to two different
granular bulk behaviours, namely the angle of repose ($AoR$) test and the
Schulze shear cell ($SSC$) test.
In both cases, we first trained a specific $ANN$ using a number of $DEM$ test
simulations before we identified valid sets of $DEM$ contact law parameters by
comparing the stand-alone $ANN$ predictions with corresponding bulk experiments. 
For both cases we obtained valid sets of contact law parameters, 
which we then compared to formulate a reliable contact law for a given
granular material.
We further show that the same $ANN$ can be used to characterize different granular materials, 
which have the same particle behaviour and can modelled with the same contact
law. \\
In the next section we define some prerequisites including $DEM$ contact law
definitions, a general description of the $ANN$ functionality, and the proposed
method of $DEM$ contact law parameter identification.
We then describe how it is applied to characterize the $DEM$ contact law
parameters of sinter fines.