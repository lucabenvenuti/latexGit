% !TEX encoding = UTF-8
% !TEX TS-program = pdflatex
% !TEX root = ../Tesi.tex
% !TEX spellcheck = en-EN

%************************************************
\part{Introduction}
\label{par:introduction}
%************************************************

Particles in various forms - ranging from raw materials to food grains and pharmaceutical powders - 
play a major role in a variety of industries, including process industry and metallurgy. 
In his book, \citet{RefWorks:117} stated that "between 1 and 10\% of all the energy is used in 
comminution, i.e. the processes of crushing, grinding, milling, micronising". 
Many methods have been developed to study particles.
For instance, Discrete Element Methods ($DEMs$), "a special class of numerical
schemes for simulating the behavior of discrete, interacting bodies", are widely used to 
simulate particle behaviour in these granular processes
(\citet{RefWorks:130}).\\ 
\begin{figure}[!htb]
\centering
\includegraphics[width=.50\columnwidth]{images/133gravel}
\caption[Gravel]{Gravel.}
\label{fig:133gravel}
\end{figure}
\begin{figure}[!htb]
\centering
\includegraphics[width=.50\columnwidth]{images/050steel}
\caption[Steel]{Steel.}
\label{fig:050steel}
\end{figure}
In fact, gravels, Fig. \ref{fig:054bsgmaterials}, or granular particles in
general, are far from being a well-defined and easy to characterize material,
for instance a steel beam, Fig.
\ref{fig:050steel}. For continuous materials simulation
parameters are readily available.
In case of a pile of particles the sum of discrete particle properties determines the pile's macroscopic behavior 
(e.g. angle of repose).
In discrete particle simulations particle based parameters (e.g. contact parameters) determine the macroscopic behavior 
of the ensemble.
Unfortunately, particles are not uniform and particle based simulation
parameters are difficult to obtain, and also depends on the numerical shape
(polyhedral, multi-spheres, and simple spheres).
A set of experimental and numerical solutions, together with artificial neural networks, can improve the accuracy 
and the range of applicability of the characterization of particles properties, and reduce the computational costs.
Discrete Element Method requires parameters for the individual contact, but characterize every particle is prohibitive.
We need to find average contact parameters that lead to the expected bulk effect.
We could start with an example of piled particles, more specifically called the
drained angle of repose. This angle of repose \ref{fig:060aor},
characteristic of the bulk macroscopic behaviour of the ensemble, is originated from the microscopic
characteristics of each particle.
\begin{figure}[htbp]
  %\null\hfill
  \subfloat[Silibeads angle of repose.]{
	  \includegraphics[width=.48\columnwidth]{images/058silibeads}
	  \label{fig:058silibeads}
  }
  \quad
 % \hfill
  \subfloat[Sinter pellets  angle of repose.]{
	  \includegraphics[width=.48\columnwidth]{images/059sinterpellets}
	  \label{fig:059sinterpellets}
  }
 % \hfill\null
  \caption{Angle of repose identification.}
  \label{fig:060aor}
\end{figure}
Measurement of a bulk parameter value, through calibration we obtain the
individual contact parameters:
\begin{enumerate}
\item{Chose initial set of parameters}
\item{DEM simulation}
\item{Compare macroscopic DEM simulation results with experiments}
\item{Choose new parameters}
\end{enumerate}
By our calibration procedure we obtain valid sets of particle based simulation parameters.
Ok, but that's very time consuming, because in each control loop we have to
perform a complete $DEM$ simulation in our case we would need 9.900 days on a 32
core machine. It is not necessary to evaluate a huge number of parameter sets,
rather we should try to evaluate the \textbf{sensitivity} 
of the macroscopic bulk behavior with respect to individual particle based parameters.
This can be realized efficiently by artificial neural networks Fig.
\ref{fig:048neuron0}. 
After the original work of \citet{RefWorks:189} we know that the human brain is
composed of neurons and their connections. They receive inputs from receptive
nerves and together elaborate an output response (e.g. to remove the hand from
an hot surface).
\begin{figure}[!htb]
\centering
\includegraphics[width=.50\columnwidth]{images/048neuron0}
\caption[Biological inspiration]{Biological inspiration \cite{RefWorks:158}.}
\label{fig:048neuron0}
\end{figure}
Similarly, in the artificial neural network we trained a series of neurons had
as inputs particle based parameters and as output the corresponding $DEM$
simulation results.
With both data the network is trained (i.e. individual neurons are
weighted).
Later, the trained neural network can be used to predict additional valid sets of particle based simulation parameters. 

\begin{enumerate}
\item{Train neural network by 500 dedicated DEM simulations (time consuming)}
\item{Test another 6.250.000 combinations by the neural network (very fast)}
\item{Check if predictions of neural network are correct (by comparing with experimental  values)}
\end{enumerate}

Typically, less than 1\% of the tested parameter sets lead to correct
macroscopic results (i.e. 6.000 to 60.000 valid parameter sets).
By this assessment of particle based simulation parameters we obtain valuable information about the dependence 
of bulk solid behavior on individual particle properties.
First we can determine the validity range, mean and variance band of each input
parameter. Next we can determine a probability density function for each input
parameter. Then, we can investigate mutual dependencies.
This calibration procedure is universal in a sense that the same artificial neural network can be harnessed for 
different macroscopic bulk behaviors.
This effort is really necessary because the predictive capability of any DEM simulation strongly 
depends on the validity of the particle 
based simulation parameters.
\improvement{add some words on the applications}
