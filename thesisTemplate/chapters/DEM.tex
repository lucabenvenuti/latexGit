% !TEX encoding = UTF-8
% !TEX TS-program = pdflatex
% !TEX root = ../Tesi.tex
% !TEX spellcheck = en-EN

%************************************************
\chapter{Discrete Element Method}
\label{cap:dem}
%************************************************

The discrete element method (\acs{DEM}) is derived from Molecular Dynamics. \\ 
Being a Lagrangian method, for each particle $i$ inside the domain, a \acs{DEM}
code follows the trajectory and calculates the force that particle $i$ exerts on
particle $j$.
The main forces involved are: gravity, contact forces due to collisions, and
further interactions such as electrostatic, Van der Waals, cohesive forces and fluid-solid interactions in 
multiphase flows. \\
Two approaches can be implemented for this method:
\begin{itemize}
  \item{hard spheres,}
  \item{soft spheres.}
\end{itemize} 
The hard-sphere approach is well-suited for binary collisions, easy to
calculate, but cannot predict multi-particle interactions.
It is mainly used for dilute flows, rather then in the dense flows we
investigated.\\
For the simulations performed in a coupled particle-fluid environment, see
Chapter \ref{cap:blastfurnace}, we used an \textit{unresolved
\acs{CFD}-\acs{DEM} approach}, see \ref{sec:unresolvedcfddem}.

%\section{DEM}
%\label{sec:dem}
% \section{Hard spheres approach}
% \label{sec:hardspheresapproach}
% \info{words on hard spheres approach}

\section{Soft sphere approach}
\label{sec:softspheresapproach}

%\info{could use tikz for this image}
\begin{figure}[!h]
\centering
\includegraphics[width=.6\columnwidth]{images/109twospheres}
\caption[Two spheres]{Two colliding spheres with the soft-sphere approach.}
\label{fig:109twospheres}
\end{figure}
The soft sphere approach considers the deformation of particles when they
collide with each other. From the vector in Fig. \ref{fig:109twospheres} we can
define after time \textit{t} the normal and tangential overlaps \acs{xin} and
\acs{xit}:
\begin{equation}
\xi_n = R_i + R_j - \mathbf{n}(\mathbf{r}_j - \mathbf{r}_i),
 \label{eq:xin}
\end{equation}

\begin{equation}
\xi_t = \int_{0}^{t}{\mathit{dt'}|\mathbf{v}_t(\mathit{t'})|} .
 \label{eq:xit}
\end{equation}

where the subscript \acs{n} stands for normal and \acs{t} for tangential. 
At this point we can divide the contact models into:
\begin{itemize}
  \item{linear, e.g. Hooke,}
  \item{non-linear, e.g. Hertz.}
\end{itemize}

\subsection{Hertz}
\label{subsec:hertz}

For the raw material used in this work 
Di Renzo and Di Maio \cite{RefWorks:145} suggested using the non-linear
Hertzian model without cohesion for the particle-particle and particle-wall contacts. 
For an elasto-perfectly-plastic-viscous model the contact forces, normal and
tangential, are:
\begin{equation}
\mathbf{F}_{n} = k_n \xi_n \mathbf{n} + C_n \mathbf{v}_n , 
\label{eq:fn}
\end{equation}

\begin{equation}
\mathbf{F}_t = 
 \begin{cases}
k_t \xi_t \mathbf{t} + \gamma_t \mathbf{v}_t  & \text{if } |k_t \xi_t \mathbf{t}
+ \gamma_t \mathbf{v}_t| \leq \mu_s |\mathbf{F}_n| ,\\
 \mu_s |\mathbf{F}_n| \mathbf{t} & \text{else.}
\end{cases}
 \label{eq:ft}
\end{equation}
%************************************************
The Coulomb's law of friction is the condition in the tangential force
formulation.\\
Here, \acs{k} and \acs{gamma} are respectively the stiffness and damping
coefficients, while \textbf{v} is the velocity.
Both the normal and the tangential
force comprise two terms, a spring force and a damping force. 
The tangential (or shear) force is a ``history'' effect that accounts for the
tangential displacement (``tangential overlap'') between the particles for the
duration of contact.
The \acs{kn}, \acs{kt}, \acs{gamman}, and \acs{gammat} coefficients are
calculated from the material properties as follows:
%************************************************
\begin{equation}
\begin{aligned}
	k_n &= \frac{4}{3} E_{eq} \sqrt{R_{eq} \delta_n} ,\\
	\gamma_n &= 2 \sqrt{\frac{5}{6}} \beta \sqrt{S_n m_{eq}} ,\\
	k_t &= 8 G_{eq} \sqrt{R_{eq}} \delta_n ,\\
	\gamma_t &= 2 \sqrt{\frac{5}{6}} \beta \sqrt{S_t m_{eq}} .
\end{aligned}
\label{eq:hertz}
\end{equation}

%************************************************
In addition to the equations \ref{eq:hertz} the following relations (Eqns. \ref{eq:equivProp2}) are required:
%************************************************
\begin{equation}
\begin{aligned}
 \frac{1}{E_{eq}} & = \frac{1-\nu_i^2}{E_i} + \frac{1-\nu_j^2}{E_j} ,\\
 \frac{1}{G_{eq}} & = \frac{2(2+\nu_i)(1-\nu_i)}{E_i} + \frac{2(2+\nu_j)(1-\nu_j)}{E_j} ,\\
 \frac{1}{R_{eq}} &= \frac{1}{R_i} + \frac{1}{R_j} ,\\
 \frac{1}{m_{eq}} &= \frac{1}{m_i} + \frac{1}{m_j} ,\\
 \beta & = \frac{\ln(e)}{\sqrt{ln^2(e)+\pi^2}} ,\\
 S_n & = 2 E_{eq} \sqrt{R_{eq} \delta_n} ,\\
 S_t & = 8 G_{eq} \sqrt{R_{eq} \delta_n} ,\\
 k_r & = k_t R_{eq}^2 .\\
\end{aligned}
\label{eq:equivProp2}
\end{equation}


%************************************************

The coefficient of sliding friction \acs{mus} is 
one of the particle-based \acs{DEM} parameter we investigated, 
another being the coefficient of rolling friction (\acs{mur}),
which can be simulated with these models:
\begin{itemize}
  \item{constant direct torque (CDT),}
  \item{elasto-plastic-spring-dashpot.}
\end{itemize}

In the paper by Wensrich and Katterfeld \cite{RefWorks:87}, further details on
the methods can be found.\\

% \subsection{Hooke}
% \label{subsec:hooke}


%This granular model uses the following formula for the contact force between
% two granular particles (Eq. \ref{eq:forceij}):
% %************************************************
% \begin{equation}
 F_{ij} = 
\begin{cases}
F_{n,ij} + F_{t,ij} = \left( k_n \delta_{n,ij} + \gamma_n v_{n,ij} \right) + \left( k_t \delta_{t,ij} + \gamma_t v_{t,ij} \right) & \text{if } r < d ,\\
0    & \text{if } r > d ,\\
\end{cases}
 \label{eq:forceij}
\end{equation}

% %************************************************


% , $r$ is the
% distance between two particles of radii $R_i$ and $R_j$, and $d = R_i + R_j $ is
% % the contact distance.
% 
% 
% 
% 
% In the contact law we used, 
% the tangential component of the contact force between two generic particles
% ($F_t$) is truncated to fulfil:
% % %************************************************
% % \begin{equation}
F_{t} \leq \mu_s F_{n},
 \label{eq:force_t}
\end{equation}

% % %************************************************
% 
% where $F_n$ is the normal component and 

\subsection{Elasto-plastic-spring-dashpot 2 model}
\label{subsec:epsd2}

For coarse non-spherical particles, \acs{mur} is a critical parameter and describes
inter-particle friction in medium to dense granular flow simulations. 
It is proportional to the 
torque counteracting the rotation of the particle. 
The \acs{mur} parameter enters the 
equations according to the elasto-rolling resistance model presented by Wensrich and 
Katterfeld \cite{RefWorks:87} and Ai et al. \cite{RefWorks:131} 
based on the work of Jiang et al. \cite{RefWorks:143}. 
A scheme of the particles interaction can be seen in Fig.
\ref{fig:128springdashpot}.\\
\begin{figure}[!htb]
\centering
\includegraphics[width=.40\columnwidth]{images/128springdashpot}
\caption[Spring dashpot]{Spring dashpot model.}
\label{fig:128springdashpot}
\end{figure}
The model is called EPSD2 in \acs{LIGGGHTS} and is appropriate for both one-way and cyclical rolling cases.
The maximum magnitude of rolling resistance torque is (Eq. \ref{eq:trmax}):

%************************************************
\begin{equation}
T_{r~max} = \mu_r R_r |\tilde{F_n}| .
 \label{eq:trmax}
\end{equation}

%************************************************

where $R_r$ is the equivalent radius and $F_n$ the normal force.
The last two particle-based \acs{DEM} parameters we investigated were 
the particle density (\acs{rhop})
and the coefficient of restitution (\acs{CoR}) as defined by
\citet{RefWorks:131}.
Together with the others key parameters they are listed in Table
\ref{tab:08DEMparameters}.\\

\begin{table}[h]
\centering
\begin{tabular}{l}
\toprule
    Radius \acs{R} (m)   \\ [5pt]

	Size distribution (-) \\ [5pt]

    Young's modulus \acs{E} (Pa)  \\ [5pt]

    Poisson's ratio \acs{nu} (-) \\ 
     Time step \acs{deltat} (s) \\ [5pt]
\midrule
     Coefficient of sliding friction \acs{mus} (-)\\  [5pt]
    Coefficient of rolling friction \acs{mur} (-) \\ [5pt]
    Coefficient of restitution \acs{CoR} (-)   \\ [5pt]
     Particle density $\acs{rhop} = \frac{mass ~ of ~ one ~ particle}{volume ~ of
     ~ one ~ particle}$ ($kg/m^3$)  \\ [5pt]
     Geometry factor $\acs{dCylDp} = \frac{diameter ~ of ~ the ~
     cylinder}{diameter ~ of ~ one ~ particle}$ (-)  \\ [5pt]
   
\bottomrule
\end{tabular}
\caption[DEM parameters]{DEM parameters. The upper parameters were
identical in all simulations.
%  The lower parameters were constant in each
% simulation, but were varied between simulations.
}
\label{tab:08DEMparameters}
\end{table}



\section{Literature Values}
\label{sec:literaturevalues}

There are yet ways to determine contact parameters directly by measuring
material properties or by performing particle based experiments, see e.g.
\citet{RefWorks:177}, \citet{RefWorks:181}, and \citet{RefWorks:186}.
\citet{RefWorks:140} and \citet{RefWorks:190}
are two of the most prominent examples of parameters identification by
theoretical analysis. 

\subsection{Limitations}
\label{subsec:limitations}

The available literature data, however, focused on glass spheres, while
non-sphericity was one of the fields we wanted to investigate. 
Further, it is related to all the microscopic parameters we analysed. 
Also the procedure of \citet{RefWorks:177}, although on
industrial samples (aluminium oxide), still had mostly spherical particles. 
In addition, these methodologies are laborious, 
since they have to be performed for every new granular material prior to a $DEM$
simulations. 
Especially for the already cited rolling friction parameter, it is arduous to
link the rolling friction parameter to the non-sphericity of the particle. 


\section{Unresolved CFD-DEM}
\label{sec:unresolvedcfddem}
%************************************************

The existence of particles immersed in a fluid influences its behaviour. 
Keeping aside the short-scale flow field, the local averages inside the
Navier-Stokes fluid equations could account for the particles.
This is called the \textit{unresolved \acs{CFD}-\acs{DEM} approach}, and can be
seen in Fig. \ref{fig:130unresolvedcfddem}.\\
\begin{figure}[!htb]
\centering
\includegraphics[width=.40\columnwidth]{images/130unresolvedcfddem}
\caption[Unresolved CFD-DEM]{Unresolved CFD-DEM \cite{Refworks:202}.}
\label{fig:130unresolvedcfddem}
\end{figure}
On the hypothesis of having \textit{g(r)} a convenient averaging
function, which is non-negative, smooth, asymptotically zero for large $r$, normalized, \citet{RefWorks:201}
defines the local void fraction ($\varepsilon(\mathbf{r},t)$), the volume of the
fluid ($V_f(t)$) over the total volume, as:
\begin{equation}
\varepsilon(\mathbf{r},t) = \int_{V_f(t)}{g(|\mathbf{r} -
\mathbf{r'}|) \dd^3 r' }.
 \label{eq:voidfraction}
\end{equation}

where $\mathbf{r}$ is the vector containing the position of a single particle
and $\mathbf{r'}$ the position of the its closest neighbour particle.
We define local mean values of fluid point properties as:
\begin{equation}
\bar{a}(\mathbf{r},t) = \frac{1}{\varepsilon(\mathbf{r},t)} \int_{V_f(t)}{d^3 r'
g(|\mathbf{r} - \mathbf{r'}|) a(\mathbf{r'},t)},
 \label{eq:a}
\end{equation}

which came from a slowly varying a rapidly fluctuating part:
\begin{equation}
\label{eq:a2}
a(\mathbf{r},t) = \bar{a}(\mathbf{r'},t) + a'(\mathbf{r},t).
\end{equation}

Particle properties ($p$), on the solid volume $V_s(t)$ and the solid fraction
$\phi(\mathbf{r},t)$, are expressed as field quantities as:
\begin{equation}
\begin{align*}
\bar{b}(\mathbf{r},t) &= \frac{1}{\phi(\mathbf{r},t)} \int_{V_f(t)}{
g(|\mathbf{r} - \mathbf{r'}|) b(\mathbf{r'},t) \dd^3 r'} \\
& \approx
\frac{1}{\phi(\mathbf{r},t)} \sum_{p}{g(\mathbf{r} - \mathbf{r}_p)
V_p b_p(t)}.
\end{align*}
 \label{eq:b}
\end{equation}

The approximation is valid when $g(r)$ varies slowly over the particles length
scale and its surface $S_p$, and the averages of the derivatives can be written
as:
\begin{equation}
\begin{align*}
\nabla_{\mathbf{r}} \varepsilon (\mathbf{r},t) \bar{a}(\mathbf{r},t) 
&=
\int_{V_f(t)}{
a(\mathbf{r'}, t) \nabla_{\mathbf{r}} g(|\mathbf{r} - \mathbf{r'}| )  \dd^3 r'} \\
&=
- \int_{S_f(t)}{ a(\mathbf{S},t) g(|\mathbf{r} - \mathbf{S}|) \dd \mathbf{S} + 
\varepsilon (\mathbf{r},t) \overline{\nabla a}(\mathbf{r},t)} \\
& \approx
\sum_{p}{\int_{S_p(t)}{ a(\mathbf{S},t) g(|\mathbf{r} - \mathbf{S}|)  \dd
\mathbf{S}}} + \varepsilon (\mathbf{r},t) \overline{\nabla a}(\mathbf{r},t).
\end{align*}
 \label{eq:divepsa}
\end{equation}

Further, for time derivatives we can write:
\begin{equation}
\begin{align*}
\frac{\partial}{\partial t}\varepsilon (\mathbf{r},t) \bar{a}(\mathbf{r},t) 
&=
\int_{V_f(t)}{
\frac{\partial a(\mathbf{r'}, t)}{\partial t} g(|\mathbf{r}
- \mathbf{r'}| )  \dd^3 r'} \\
&+ \lim \frac{1}{\Delta t} \left[
\int_{V_f(t + \Delta t)}{ a(\mathbf{r'}, t) g(|\mathbf{r} - \mathbf{S}|)  \dd^3
r'} - \int_{V_f(t)}{ a(\mathbf{r'}, t) g(|\mathbf{r} - \mathbf{S}|)  \dd^3 r'}
\right ]\\
& \approx
\varepsilon (\mathbf{r},t) \overline{\dot{a}}(\mathbf{r},t) - 
\sum_{p}{\int_{S_p(t)}{\mathbf{v_s} a(\mathbf{S},t) g(|\mathbf{r} - \mathbf{S}|) 
\dd \mathbf{S}}}.
\end{align*}
 \label{eq:ddtepsa}
\end{equation}

\begin{figure}[!htb]
\centering
\includegraphics[width=.30\columnwidth]{images/131stokessphere}
\caption[Particle in a fluid]{Particle in a fluid \cite{Refworks:37}.}
\label{fig:131stokessphere}
\end{figure}
We could now write in Eq. \ref{eq:divepsa} that $a = \rho \mathbf{u}$,
respectively the density and the velocity of the fluid surronding the particle, as can be
seen in Fig. \ref{fig:131stokessphere}. So, it becomes:
\begin{equation}
\nabla_{\mathbf{r}} \varepsilon (\mathbf{r},t) \overline{\rho \mathbf{u}}(\mathbf{r},t) 
 \approx
\sum_{p}{\int_{S_p(t)}{ \rho \mathbf{u}(\mathbf{S},t) g(|\mathbf{r} - \mathbf{S}|)  \dd
\mathbf{S}}} + \varepsilon (\mathbf{r},t) \overline{\nabla {\rho
\mathbf{u}}}(\mathbf{r},t).
 \label{eq:divepsrhou}
\end{equation}

Similarly, for Eq. \ref{eq:ddtepsa}, we consider $a = \rho$, and we write:
\begin{equation}
\frac{\partial}{\partial t}\varepsilon (\mathbf{r},t) \bar{\rho}(\mathbf{r},t) 
\approx
\varepsilon (\mathbf{r},t) \overline{\dot{\rho}}(\mathbf{r},t) - 
\sum_{p}{\int_{S_p(t)}{\mathbf{v_s} \rho(\mathbf{S},t) g(|\mathbf{r} - \mathbf{S}|) 
\dd \mathbf{S}}}.
 \label{eq:ddtepsrho}
\end{equation}

At this point, it is easy to demonstrate that by merging Eq. \ref{eq:divepsrhou}
and Eq. \ref{eq:ddtepsrho} we obtain the \textit{volume-averaged continuity
equation}:
\begin{equation}
\frac{\partial}{\partial t}\varepsilon (\mathbf{r},t) \overline{\rho}(\mathbf{r},t) +
\nabla \cdot \varepsilon (\mathbf{r},t) \overline{\rho
\mathbf{u}}(\mathbf{r},t)  = 0.
 \label{eq:continuity}
\end{equation}

In case of incompressible flow $\frac{\partial \rho}{\partial t} = 0$. Thus,
Eq.
\ref{eq:continuity} becomes:
\begin{equation}
\frac{\partial}{\partial t}\varepsilon (\mathbf{r},t)  +
\nabla \cdot \varepsilon (\mathbf{r},t) \overline{
\mathbf{u}}(\mathbf{r},t)  = 0.
 \label{eq:continuityincompressible}
\end{equation}

In the same way, for the Navier-Stokes momentum equation, including the stress
tensor $\tau$ \cite{Refworks:202}:
\begin{equation}
\int_{V_f(t)}{\left[g(|\mathbf{r} - \mathbf{r'}|) \rho_f \left(\frac{\partial
\mathbf{u}}{\partial t} + ( \mathbf{u} \cdot \nabla) \mathbf{u} \right) \right]
\dd^3 r'} = 
\int_{V_f(t)}{\left[g(|\mathbf{r} - \mathbf{r'}|) \nabla \cdot \tau +
\rho_f \mathbf{g} \right] \dd^3 r'} ,
 \label{eq:nsmomentumequation}
\end{equation}

which can be written:
\begin{equation}
\rho_f \varepsilon (\mathbf{r}, t) \left(\frac{\partial
\mathbf{\overline{u}}}{\partial t} + ( \mathbf{\overline{u}} \cdot \nabla)
\mathbf{\overline{u}} \right) = 
\nabla \cdot (\overline{\tau} - \overline{R}) -\sum_{p}{g(|\mathbf{r} -
\mathbf{r_p}|)\mathbf{F}_{p-f}(\mathbf{\overline{u}}, \varepsilon,
\mathbf{v}_p, d_p) + \rho_f \varepsilon (\mathbf{r}, t) \mathbf{g}} ,
 \label{eq:nsmomentumequation2}
\end{equation}

with $\mathbf{F}_{p-f}$ as the interaction force between the particle and the
fluid, or viceversa, considering Newton's third law.
This force is dependant on the particle
velocity $\mathbf{v}_{p}$ and the diameter of the particle $d_p$. Further:
\begin{equation}
\overline{R}(\mathbf{r}) = 
\rho_f 
\int_{V_f(t)}{g(|\mathbf{r} - \mathbf{r'}|)  
\mathbf{u'}(\mathbf{r'}, t) \circ \mathbf{u'}(\mathbf{r'}, t)
\dd^3 r'} .
 \label{eq:rr}
\end{equation}

To complete the system, we add the particles equation of motion:
\begin{equation}
m_p \frac{\partial \mathbf{v}_p}{\partial t} = 
\mathbf{F}_{p-f}(\mathbf{\overline{u}}, \varepsilon,
\mathbf{v}_p, d_p) + m_p \mathbf{g} + \Phi_{p-p}(\mathbf{r}_p, \{\mathbf{r}_k
\}),
\label{equ:equationofmotion}
\end{equation}

where $m_p$ is the mass of the particle, $\mathbf{g}$ the acceleration due to
gravity, and $\Phi_{p-p}$ is a force field, which acts on particle $p$ because
of the remaining particles.
This \textit{unresolved \acs{CFD}-\acs{DEM} approach} works as long as the exact
trajectories of each particle can be negliged.\\
For completeness, we could say that also the equation of motion can be averaged
over space, obtaining:
\begin{equation}
\frac{\partial}{\partial t}\phi (\mathbf{r},t) \rho_p (\mathbf{r},t) +
\nabla \cdot \phi (\mathbf{r},t) \overline{\rho_p
\mathbf{v}_p}(\mathbf{r},t)  = 0.
 \label{eq:eomaveraged}
\end{equation}

If the spheres have the same density, we can write:
\begin{equation}
\frac{\partial}{\partial t}\phi (\mathbf{r},t) +
\nabla \cdot \phi (\mathbf{r},t) \overline{
\mathbf{v}_p}(\mathbf{r},t)  = 0,
 \label{eq:eomaveraged2}
\end{equation}

Finally:
\begin{equation}
\begin{align*}
\rho_p \phi (\mathbf{r}, t) \left(\frac{\partial
\mathbf{\overline{v}}_p}{\partial t} + ( \mathbf{\overline{v}}_p \cdot \nabla)
\mathbf{\overline{v}}_p \right) 
& = 
\sum_{p}{g(|\mathbf{r} - \mathbf{r_p}|)\Phi_{p-p}}
- \nabla \cdot S \\ 
& + \sum_{p}{g(|\mathbf{r} - \mathbf{r_p}|)\mathbf{F}_{p-f}} 
+ \phi(\mathbf{r}, t) \rho_p \mathbf{g} ,
\end{align*}
 \label{eq:eom3}
\end{equation}

where:
\begin{equation}
S(\mathbf{r}) = 
\sum_{p}{m_p g(|\mathbf{r} - \mathbf{r}_p|)  
\mathbf{v'}_p \circ \mathbf{v'}_p} .
 \label{eq:sr}
\end{equation}

The main advantage of this model is to allow a greater number of particles to be
considered, since they are treated as an additional averaged fluid: this is
called the \textit{two-fluid model}.
Hovewer, this model work poorly in a dense system like our, see
\citet{Refworks:202}, and was thus not considered further.

\subsection{Governing Equations}
\label{subsec:governingequations}
From the equations derived in the previous paragraph, the software uses the
following simplified governing equations \cite{Refworks:202}, which can be
divided into fluid and solid phase, given for both the same gravitational
acceleration $g$.
First, in the fluid phase the \textit{continuity equation} can be written as:
\begin{equation}
\frac{\partial \alpha_f}{\partial t}   +
\nabla \cdot (\alpha_f u)   = 0 ,
 \label{eq:continuityincompressiblebis}
\end{equation}

where $\alpha_f$ is the fluid fraction, and $u$ is the fluid velocity, while the
\textit{momentum equation} is:
\begin{equation}
\frac{\partial \alpha_f \rho_f u}{\partial t}   +
\nabla \cdot (\alpha_f \rho_f u u) - \nabla \cdot (\alpha_f \tau)  = 
 - \alpha_f \nabla p + R_{sl} + \alpha_f \rho_f g .
 \label{eq:nsmomentumequation3}
\end{equation}

Here, $\rho_f$ is the fluid density, $\tau$ the shear stresses, $p$ the fluid
pressure, and with $R_{sl}$, the force exchange term, given by:
\begin{equation}
R_{sl} = \frac{1}{\Delta V} \sum{f_{d,i}},
 \label{eq:rsl}
\end{equation}

where $f_{d,i}$ are the drag forces for every particle in the computational
cell, of which $\Delta V$ is the volume. \\
Then, for the solid phase, we can write the \textit{equation of motion} for each
of the particles involved as:
\begin{equation}
m_p \frac {\dd u_p}{\dd t} = m_p g + f_{p,f} + \sum_{N_p}{f_{p,p}} +
\sum_{N_w}{f_{p,w}} ,
 \label{eq:eom4}
\end{equation}

where $u_p$ is the velocity of the particle, $f_{p,p}$ the
particle-particle interaction forces, while $f_{p,w}$ are the particle-wall
interaction forces. The interaction force between the particle and the
fluid is:
\begin{equation}
\label{eq:forceparticlefluid}
f_{p,f} = f_d + f_{\nabla p} + f_{\nabla \tau} ,
\end{equation}

where $f_d$ is the drag force, $f_{\nabla p}$ the pressure gradient force,
and $f_{\nabla \tau}$ the viscous force.\\
Closure of the system can be obtained with initial and
boundary conditions for both phases.

\section{Time Discretization}
\label{sec:timestep}

According to the original article by \citet{RefWorks:172}, which defined first
the \acs{DEM}, 
it is necessary to computationally resolve the equations for particle contacts
in discrete time steps, the length of which should be carefully considered.
These time steps are small fractions of the total simulation time.\\
For pure \acs{DEM} simulations, and for for the \acs{DEM} portions of the
coupled simulations, \citet{RefWorks:172} suggest to fix a time step smaller
than the 20\% of the Rayleigh time, defined as:
\begin{equation}
\label{eq:rayleightime}
t_r = \pi r \cdot \frac{\sqrt{\frac{\rho_p}{G}}}{0.1631 \cdot \nu +
0.8766},
\end{equation}

%dt_r = pi*r*sqrt(rho/G)/(0.1631*nu+0.8766);
where $r$ is the radius of the smallest particle, $\rho_p$ its density, $G$ its
shear modulus, and $\nu$ its Poisson ratio.
For the \acs{CFD} portions of the coupled simulations, \citet{RefWorks:201}
states that the time step restrictions can be more easily complied.
Thus, we used:
\begin{equation}
\label{eq:timestepcfd.tex}
\Delta t_{CFD} = 10 \cdot \Delta t_{DEM}.
\end{equation}

%dt_r = pi*r*sqrt(rho/G)/(0.1631*nu+0.8766);

\section{Coupled Simulations}
\label{sec:coupledsimulations}

\begin{figure}[!htb]
\centering
	\begin{tikzpicture}[auto, node distance=2cm ]
	    \node [input, name=input] {};
 	    \node [sum, right=1.5cm of input] (one) {DEM step};
 	    \node [block, below of=one] (two) {Interpolation of fluid velocity};
 	    \node [block, below of=two] (three) {Interpolation of fluid pressure};
 	    \node [block, below of=three] (four) {Calculation of forces};
 	    \node [output, right=1.5cm of four] (five) {five};
 	    \node [block, right=4.5cm of one] (six) {Calculation of void fraction};
 	    \node [sum, right=1.5cm of one] (seven) {CFD step};
 	    \node [block, below of=six] (eight) {Calculation of void
 	    fraction derivative}; 
 	    \node [block, below of=eight] (nine) {Calculation of momentum}; 
 	    \node [block, below of=nine] (ten) {New velocity field};
 	    \node [block, below of=ten] (eleven) {New pressure};
 	    \node [block, below of=eleven] (twelve) {New void
 	    fraction derivative}; 	 
 	    \node [relation, below=1.0cm of twelve] (thirteen) {Convergence?}; 
 	    \node [output, right=2.0cm of thirteen] (fourteen) {five};
 	    \node [output, below of=thirteen] (fifteen) {five}; 	    
 	    \node [output, right=2.0cm of ten] (sixteen) {five}; 
 	    \node [sum, below=1.0cm of thirteen] (seventeen) {back to start}; 	    	    	  	    	     	     	    
% 	
% 	    % Once the nodes are placed, connecting them is easy. 
 	    \draw [draw,->] (input) -- node {} (one);
 	    \draw [draw,->] (one) -- node {} (two);
 	    \draw [draw,->] (two) -- node {} (three);
 	    \draw [draw,->] (three) -- node {} (four);
 	    \draw [draw,->] (four) -- node {} (five); 	    
 	    \draw [draw,->] (five) -| node {} (seven);
 	    \draw [draw,->] (seven) -- node {} (six);
 	    \draw [draw,->] (six) -- node {} (eight); 	    
 	    \draw [draw,->] (eight) -- node {} (nine); 
 	    \draw [draw,->] (nine) -- node {} (ten);
 	    \draw [draw,->] (ten) -- node {} (eleven); 	  
 	    \draw [draw,->] (eleven) -- node {} (twelve);
 	    \draw [draw,->] (twelve) -- node {} (thirteen);
 	    \draw [draw,->] (thirteen) -- node {$no$} (fourteen);
 	    \draw [draw,->] (thirteen) -- node {$yes$} (seventeen); 	     	   	     	    	  	     	     	 	    
 	    \draw [draw,->] (fourteen) -- node {} (sixteen);
 	    \draw [draw,->] (sixteen) -- node {} (ten); 	  	     	 	    
    
	\end{tikzpicture}
\caption[Computational CFD-DEM steps]{Computational CFD-DEM steps in a
simulation. These steps are iterated many times until convergence and duration
are both achieved.}
\label{fig:132cfdemstep}
\end{figure}
% \wrong{use this tikz heavy image}

The work of the unresolved \acs{CFD}-\acs{DEM} software is
illustrated in Fig. \ref{fig:132cfdemstep}, and can be
found in details in \citet{RefWorks:201}.
In our simulations, the whole diagram represents only one \acs{CFD} step and 10
\acs{DEM} steps. 
In fact, the \textit{Calculation of forces} block should be read as
\textit{10 \acs{DEM} time steps for the calculation of the particle forces}.
This is necessary because of the severe time discretization requirements
inherent the method.
