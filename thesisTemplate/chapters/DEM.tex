% !TEX encoding = UTF-8
% !TEX TS-program = pdflatex
% !TEX root = ../Tesi.tex
% !TEX spellcheck = en-EN

%************************************************
\chapter{Discrete Element Method}
\label{cap:dem}
%************************************************

For each particle i inside the domain, a Discrete Element Method ($DEM$) code
follows the trajectory and calculates the force that particle i exerts on particle j.
The main forces involved are: gravity, contact forces due to collisions, and
further interactions such as electrostatic, Van der Waals, cohesive forces and fluid-solid interactions in multiphase flows. For the raw material used in this work 
Di Renzo and Di Maio \cite{RefWorks:145} suggested using the non-linear
Hertzian model without cohesion for the particle-particle and particle-wall contacts. 
This granular model uses the following formula for the contact force between two granular particles (Eq. \ref{eq:forceij}):
%************************************************
\begin{equation}
 F_{ij} = 
\begin{cases}
F_{n,ij} + F_{t,ij} = \left( k_n \delta_{n,ij} + \gamma_n v_{n,ij} \right) + \left( k_t \delta_{t,ij} + \gamma_t v_{t,ij} \right) & \text{if } r < d ,\\
0    & \text{if } r > d ,\\
\end{cases}
 \label{eq:forceij}
\end{equation}

%************************************************
where the subscript $n$ stands for normal and $t$ for tangential. 
Here, $k$ and $\gamma$ are respectively the stiffness and damping coefficients, 
while $\delta$ and $v$ are the displacement and the velocity, $r$ is the
distance between two particles of radii $R_i$ and $R_j$, and $d = R_i + R_j $ is
the contact distance.
Both the normal and the tangential
force comprise two terms, a spring force and a damping force. 
The tangential (or shear) force is a ``history'' effect that accounts for the
tangential displacement (``tangential overlap'') between the particles for the
duration of contact.
In the paper by Wensrich and Katterfeld \cite{RefWorks:87}, further details on
the method can be found.\\

We decided to utilize a single
contact law for all the simulations performed.

The time step was between $1.29 \%$ and $1.53 \%$ of the Rayleigh time, which
also depends on the particle density ($\rho_p$).
Furthermore, we locked the size distribution, which was obtained by experimental
sieving, see Table \ref{tab:09DEMFixedinputvalues}.
In the contact law we used, 
the tangential component of the contact force between two generic particles
($F_t$) is truncated to fulfil:
%************************************************
\begin{equation}
F_{t,ij} \leq \mu_s F_{n,ij},
 \label{eq:force_t}
\end{equation}

%************************************************

where $F_n$ is the normal component and $\mu_s$ is the coefficient of sliding
friction, one of the particle-based $DEM$ parameter we investigated, 
another being the coefficient of rolling friction ($\mu_r$). 
For coarse non-spherical particles, this is a critical parameter and describes
inter-particle friction in medium to dense granular flow simulations. It is proportional to the 
torque counteracting the rotation of the particle. The $\mu_r$ parameter enters the 
equations according to the elasto-rolling resistance model presented by Wensrich and 
Katterfeld \cite{RefWorks:87} and Ai et al. \cite{RefWorks:131} 
based on the work of Jiang et al. \cite{RefWorks:143}. 
The model is called $EPSD2$ in $LIGGGHTS$ and is appropriate for both one-way and cyclical rolling cases.
The maximum magnitude of rolling resistance torque is (Eq. \ref{eq:trmax}):

%************************************************
\begin{equation}
T_{r~max} = \mu_r R_r |\tilde{F_n}| ~,
 \label{eq:trmax}
\end{equation}

%************************************************

where $R_r$ is the equivalent radius and $F_n$ the normal force.
The last two particle-based $DEM$ parameters we investigated were $\rho_p$
and the coefficient of restitution ($COR$) as defined by Ai. et al.
\cite{RefWorks:131}.\\

These coefficients, $COR$, $\mu_s$, $\mu_r$,
$\rho_p$ and $dCylDp$ (the cylinder dimension, proportional to the mean
particle diameter), as indicated in Table \ref{tab:10DEMVariableinputvalues}, 
were constant in each simulation, but their combination differed between
simulations.
Further, $dCylDp$ was used to evaluate the wall effect, but only $~10\%$ of the
simulations had a $dCylDp$ larger than $20$ (additional information can be found
in \ref{subsec:srsctsimulation}).
The normal stress $\sigma_n$ and its
percentage during the incipient flow condition $\tau_{\%}$
varied to replicate twelve shear-cell load conditions. 
The complete description of the shear-cell simulations can be found in \ref{subsec:srsctsimulation}, 
and the $AoR$ simulation is presented in \ref{subsec:aorsimulation}.
A Matlab script allowed us to extract from the simulation output the numerical
values representative of bulk behaviour (hereafter called \textit{bulk values}),
see Table \ref{tab:14bulkvalues},
for each $DEM$ simulation parameter combination.


, which consists of
 bulk density ($\rho_b$),
 coefficient of internal friction in the pre-shear phase ($\mu_{psh}$),
 coefficient of internal friction in the shear phase ($\mu_{sh}$),
 and angle of repose ($AoR$).
 
 
 
The first bulk value ($\rho_b$) was provided directly. 
For correctly performed simulations, see \ref{subsec:srsctsimulation}, we
observed a stress path as in Fig. \ref{fig:21simexample}.
First, the $\sigma_n$ was kept constant while the coefficient of internal
friction ($\mu_{ie}$) initially increased and then reached a plateau.
The second bulk value ($\mu_{psh}$) was calculated as the average of the
$\mu_{ie}$ in this plateau.
The $\sigma_n$ was then automatically reduced, in our example to $80 \%$ of
its initial value.
Subsequently, a second plateau developed.
We obtained the third
value ($\mu_{sh}$) as the average of $\mu_{ie}$ in this second plateau.
The stress path accords with the experimental one, especially the plateaux.\\
In the $AoR$ tests the average of the repose angles provided us with the fourth
bulk value, allowing us to define the numerical bulk behaviour.

\section{Literature Values}
\label{sec:literaturevalues}

The $DEM$ parameters for the Young's modulus ($E$) and the Poisson's coefficient
($\nu$) were taken from the literature, see \cite{RefWorks:175} 
and \cite{RefWorks:176}; however we reduced the former to increase the time step
($\Delta t$), following the recommendations of Ai et al. \cite{RefWorks:131}.

