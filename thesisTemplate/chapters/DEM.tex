% !TEX encoding = UTF-8
% !TEX TS-program = pdflatex
% !TEX root = ../Tesi.tex
% !TEX spellcheck = en-EN

%************************************************
\chapter{Discrete Element Method}
\label{cap:dem}
%************************************************

For each particle i inside the domain, a Discrete Element Method ($DEM$) code
follows the trajectory and calculates the force that particle i exerts on particle j.
The main forces involved are: gravity, contact forces due to collisions, and
further interactions such as electrostatic, Van der Waals, cohesive forces and fluid-solid interactions in multiphase flows. For the raw material used in this work 
Di Renzo and Di Maio \cite{RefWorks:145} suggested using the non-linear
Hertzian model without cohesion for the particle-particle and particle-wall contacts. 
This granular model uses the following formula for the contact force between two granular particles (Eq. \ref{eq:forceij}):
%************************************************
\begin{equation}
 F_{ij} = 
\begin{cases}
F_{n,ij} + F_{t,ij} = \left( k_n \delta_{n,ij} + \gamma_n v_{n,ij} \right) + \left( k_t \delta_{t,ij} + \gamma_t v_{t,ij} \right) & \text{if } r < d ,\\
0    & \text{if } r > d ,\\
\end{cases}
 \label{eq:forceij}
\end{equation}

%************************************************
where the subscript $n$ stands for normal and $t$ for tangential. 
Here, $k$ and $\gamma$ are respectively the stiffness and damping coefficients, 
while $\delta$ and $v$ are the displacement and the velocity, $r$ is the
distance between two particles of radii $R_i$ and $R_j$, and $d = R_i + R_j $ is
the contact distance.
Both the normal and the tangential
force comprise two terms, a spring force and a damping force. 
The tangential (or shear) force is a ``history'' effect that accounts for the
tangential displacement (``tangential overlap'') between the particles for the
duration of contact.
In the paper by Wensrich and Katterfeld \cite{RefWorks:87}, further details on
the method can be found.\\

\section{Literature Values}
\label{sec:literaturevalues}

\lipsum[1]

