% !TEX encoding = UTF-8
% !TEX TS-program = pdflatex
% !TEX root = ../Tesi.tex
% !TEX spellcheck = en-EN

%************************************************
\chapter{Blast Furnace}
\label{cap:blastfurnace}
%************************************************

\section{Blast furnace design}
\label{sec:bfdesign}

Well explained in the vast literature (\cite{RefWorks:203}), 
iron making consists in separating the iron from its chemical combination with
oxygen. 
As of now, the blast furnace (Fig. \ref{fig:125blastfurnace}) is considered the
most efficient process.

\begin{figure}[htbp]
\centering 
  \subfloat[Blast furnace scheme \cite{RefWorks:200}.]
  {
	  \includegraphics[width=.54\columnwidth]{images/125blastfurnace}
	  \label{fig:125blastfurnace}  }
  \quad
    \subfloat[Blast furnace section layout (Voestalpine Stahl GmbH).]
    {
	  \includegraphics[width=.40\columnwidth]{images/068racewaylayout}
	  \label{fig:068racewaylayout}  }
  \\
  \caption{Schematics of the blast furnace investigation.}
  \label{fig:319raceway}
\end{figure}

\subsection{Structure}
\label{subsec:bfstructure}

The blast furnace is a massive vertical apparatus, built with consistent amounts
of refractory materials and a robust steel shell.
The areas in front of each tuyere, which can be seen in Fig.
\ref{fig:125blastfurnace}, is called $raceway$ and is the most active. 
We dedicated this chapter to the investigation of its behaviour.

\subsection{Simulation prerequisites}
\label{subsec:simulationprerequisites}

Given its axial-symmetric geometry and according to \citet{RefWorks:208}, we
considered only a wedge and applied periodic boundary conditions.
Three tuyeres and a sufficiently high space over and under them were used to
grant uniform conditions.
As in \citet{RefWorks:208}, 15,000 cells were created for the mesh, shown in
Fig. \ref{fig:273layoutbf}. 
The tuyeres were considered as inlets. Air is inserted with a velocity of 100
m/s, a density of $1.1885 kg/m^3$ and a kinematic viscosity of $1.5 \cdot
10^{-5} m^2/s$.\\
We performed two of simulations of this volume to investigate
the effect of the variation of the \acl{mus} (\acs{mus}):
\begin{enumerate}
  \item{\acs{mus} = 0.1, or reduced,}
  \item{\acs{mus} = 0.9, or real, obtained from the parameter identification.}
\end{enumerate}
The remaining \acs{DEM} parameters can be found in Table
\ref{tab:36BFDEMvalues}, derived from the parameters identified in the previous
Chapter \ref{cap:anntraining}.
The procedure to obtain the equivalent \acs{DEM} parameters from those of the
materials involved can be found in \citet{RefWorks:208}, together with the
particle distribution, which drove us into using approximately 160,000 particles
for each simulation.
This amount allowed us to completely filled the setup.
On top, a servo-wall, similar to the \acs{SCT} simulation, consented to apply
the correct load conditions inside the system.

\begin{table}[h]
\centering
\begin{tabular}{cccc}
\hline
    Young's & Poisson's & \acs{CoR} & \acs{mur}\\
   modulus & ratio & & \\
    $[MPa]$ & $[-]$ & [-] & [-] \\
    \hline
    10    & 0.40 & 0.5 & 0.4 \\


\hline
\end{tabular}
\caption{DEM fixed input values for the blast furnace simulation.}
\label{tab:36BFDEMvalues}
\end{table}
\begin{figure}[!htb]
\centering
\includegraphics[width=.80\columnwidth]{images/273layoutbf}
\caption[Blast furnace simulation layout]{Blast furnace simulation layout \cite{RefWorks:208}.}
\label{fig:273layoutbf}
\end{figure}

\section{Results}
\label{sec:resultsbf}

In accordance to the literature (\cite{RefWorks:203, RefWorks:208}) we focused
on the gases, or fluid, velocity and the particles velocity in a vertical slice
of the simulation domain of the two simulations.
Further, we investigated the penetration depth of the raceway 

\subsection{Fluid velocity}
\label{subsec:fluidvelocity}

The change of sliding friction coefficients bear no effect on the
velocity of the gases. 

\begin{figure}[htbp]
\centering 
  \subfloat[\acs{mus} = 0.1, spatial average.]
  {
	  \includegraphics[width=.21\columnwidth]{images/288u_average_lf_stat}
	  \label{fig:288u_average_lf_stat}
  }
  \quad
    \subfloat[\acs{mus} = 0.1, velocity vectors.]
    {
	  \includegraphics[width=.21\columnwidth]{images/287u_average_lf_arrow}
	  \label{fig:287u_average_lf_arrow}
  }
  \quad
    \subfloat[\acs{mus} = 0.1, stream lines.]
    {
	  \includegraphics[width=.21\columnwidth]{images/289u_average_lf_stream}
	  \label{fig:289u_average_lf_stream}
  }
  \quad
  \subfloat[Legend.]
  {
	  \includegraphics[width=.21\columnwidth]{images/274u_average_legend}
	  \label{fig:274u_average_legend}
  }
  \\
  \subfloat[\acs{mus} = 0.9, spatial average.]
  {
	  \includegraphics[width=.21\columnwidth]{images/276u_average_hf_stat}
	  \label{fig:276u_average_hf_stat}
  }
  \quad
    \subfloat[\acs{mus} = 0.9, velocity vectors.]
    {
	  \includegraphics[width=.21\columnwidth]{images/275u_average_arrow}
	  \label{fig:275u_average_arrow}
  }
  \quad
    \subfloat[\acs{mus} = 0.9, stream lines.]
    {
	  \includegraphics[width=.21\columnwidth]{images/277u_average_hf_stream}
	  \label{fig:277u_average_hf_stream}
  }
  \quad
  \subfloat[Legend.]
  {
	  \includegraphics[width=.21\columnwidth]{images/274u_average_legend}
	  \label{fig:274u_average_legend}
  }
  \\  
  \caption[Vertical slice of fluid velocity]{Vertical slice of fluid velocity.}
  \label{fig:299u_average_lf}
\end{figure}

\subsection{Particle velocity}
\label{subsec:particlevelocity}

However, the particles velocity is clearly, and logically, higher for low
friction particles. 
They could be moved by the flow more easily compared to high
friction particles, see Fig. \ref{fig:300us_average_lf}.


\begin{figure}[htbp]
\centering 
  \subfloat[\acs{mus} = 0.1, spatial average.]
  {
	  \includegraphics[width=.21\columnwidth]{images/291us_average_lf_stat}
	  \label{fig:291us_average_lf_stat}
  }
  \quad
    \subfloat[\acs{mus} = 0.1, velocity vectors.]
    {
	  \includegraphics[width=.21\columnwidth]{images/290us_average_lf_arrow}
	  \label{fig:290us_average_lf_arrow}
  }
  \quad
    \subfloat[\acs{mus} = 0.1, stream lines.]
    {
	  \includegraphics[width=.21\columnwidth]{images/292us_average_lf_stream}
	  \label{fig:292us_average_lf_stream}
  }
  \quad
  \subfloat[Legend.]
  {
	  \includegraphics[width=.21\columnwidth]{images/278us_average_hf_legend}
	  \label{fig:278us_average_hf_legend}
  }
  \\
  \subfloat[\acs{mus} = 0.9, spatial average.]
  {
	  \includegraphics[width=.223\columnwidth]{images/280us_average_hf_stat}
	  \label{fig:280us_average_hf_stat}
  }
  \quad
    \subfloat[\acs{mus} = 0.9, velocity vectors.]
    {
	  \includegraphics[width=.21\columnwidth]{images/279us_average_hf_arrow}
	  \label{fig:279us_average_hf_arrow}
  }
  \quad
    \subfloat[\acs{mus} = 0.9, stream lines.]
    {
	  \includegraphics[width=.21\columnwidth]{images/281us_average_hf_stream}
	  \label{fig:281us_average_hf_stream}
  }
  \quad
  \subfloat[Legend.]
  {
	  \includegraphics[width=.21\columnwidth]{images/278us_average_hf_legend}
	  \label{fig:278us_average_hf_legend}
  }
  \\  
  \caption[Vertical slice of particle velocity 1]{Vertical slice of fluid
  velocity 1. The effect of sliding friction over the particle velocity is
  consistent. The area with fast particles is almost 50\%
  larger in the low friction case.}
  \label{fig:300us_average_lf}
\end{figure}

\begin{figure}[htbp]
\centering 
  \subfloat[\acs{mus} = 0.1, \acs{mur} = 0.4, $v_{inlet}$ = 100 m/s .]
  {
	  \includegraphics[width=.21\columnwidth]{images/296ver_slice_4mslf}
	  \label{fig:296ver_slice_4mslf}
  }
  \quad
    \subfloat[\acs{mus} = 0.5, \acs{mur} = 0.4, $v_{inlet}$ = 100 m/s .]
    {
	  \includegraphics[width=.21\columnwidth]{images/295ver_slice_01mslf}
	  \label{fig:295ver_slice_01mslf}
  }
  \quad
    \subfloat[\acs{mus} = 0.9, \acs{mur} = 0.4, $v_{inlet}$ = 100 m/s .]
    {
	  \includegraphics[width=.21\columnwidth]{images/294ver_slice_001mslf}
	  \label{fig:294ver_slice_001mslf}
  }
  \quad
  \subfloat[\acs{mus} = 0.1, \acs{mur} = 0.4, $v_{inlet}$ = 200 m/s .]
  {
	  \includegraphics[width=.21\columnwidth]{images/293ver_slice_0003mslf}
	  \label{fig:293ver_slice_0003mslf}
  }
  \\
  \caption{298verticalslicelf .}
  \label{fig:298verticalslicelf}
\end{figure}

\subsection{Penetration depth}
\label{subsec:penetrationdepth}

\begin{figure}[htbp]
\centering 
  \subfloat[\acs{mus} = 0.1 .]
  {
	  \includegraphics[width=.28\columnwidth]{images/285hor_slice_01mslf}
	  \label{fig:285hor_slice_01mslf}
  }
  \quad
    \subfloat[\acs{mus} = 0.9 .]
    {
	  \includegraphics[width=.27\columnwidth]{images/271hor_slice_01mshf}
	  \label{fig:271hor_slice_01mshf}
  }
  \quad
    \subfloat[Legend and slice position.]
    {
	  \includegraphics[width=.34\columnwidth]{images/272slice}
	  \label{fig:272slice}
  }
  \\
  \caption{Raceway penetration depth on the horizontal.}
  \label{fig:286hor_slice_01ms}
\end{figure}