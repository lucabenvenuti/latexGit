% !TEX encoding = UTF-8
% !TEX TS-program = pdflatex
% !TEX root = ../Tesi.tex
% !TEX spellcheck = en-EN

%************************************************
\chapter{Experimental Characterization}
\label{cap:experimentalcharacterization}
%************************************************

\lipsum[1]



\section{Particle Distribution}
\label{sec:particledistribution}

\lipsum[1]



\section{Angle of Repose (p-p) - Small Scale}
\label{sec:aor}


A sample was deposited on a 20 cm diameter plate with liftable boundary called
static angle of repose ($AoR$) tester.
Once the particles were in position, the boundary was lifted, allowing some particles to drop. 
Once stabilized, the $AoR$ was measured eight times using a digital protractor at different positions of the heap. 
The average of the measurements gave the fourth bulk value.
Note that, since the experiments were performed only for larger-size bulk
solids, the compaction condition in the initial state was not critical to the final result.

\section{Angle of Repose (p-p) - Large Scale}
\label{sec:aorlargescale}


\lipsum[1]



\section{Schulze Ring Shear Cell tester (p-p)}
\label{sec:SRSCT}
%************************************************

A representative sample of bulk solid was placed in a shear cell of specified
dimensions ($external ~ radius = 100 ~ mm$, $internal ~ radius = 50 ~ mm$).
A normal load was applied to the cover. As soon as the lid touched the sample,
its position was calculated.
Together with the area of the ring, the total volume can be calculated, and subsequently the $bulk ~ density ~ (\rho_b)$ 
of the sample was obtained the first bulk value.
Then the specimen was pre-sheared until a steady-state shear value was reached.
The steady-state flow horizontal stress
is called steady-state flow/pre-shear stress.
If the normal stress is known, it provides (Eq. \ref{eq:phi_ps}) the angle of
internal friction of the pre-shear phase ($\phi_{e-psh}$), and consequently the
pre-shear coefficient of internal friction $ (\mu_{psh})$, the second
bulk value, see Schulze \cite{RefWorks:118}:
%************************************************
\begin{equation}
\begin{aligned}
\phi_{e-psh} &= \arctan \left(\frac{\tau_{psh}}{\sigma_{n,psh}} \right) ,\\
\mu_{psh} &=\tan(\phi_{e-psh}) .
\end{aligned}
 \label{eq:phi_ps}
\end{equation}

%************************************************
The normal stress and the angular velocity were then immediately reduced to zero. 
Subsequently, the specimen was sheared under a fraction ($shear-perc$) of the first normal load until the shear force 
reached a maximum and began to decrease. 
Both the pre-shear and shear phases were executed at constant velocity. 
We define the horizontal stress at the shear force peak as the maximum shear
stress, thus obtaining the incipient flow/shear coefficient of internal friction $
(\mu_{sh})$, third bulk value (Eq. \ref{eq:phi_s})\cite{RefWorks:118}:
%************************************************
\begin{equation}
\begin{aligned}
\phi_{e-sh} &= \arctan \left(\frac{\tau_{sh}}{\sigma_{n,sh}} \right) ,\\
\mu_{sh} &= \tan(\phi_{e-sh}) .
\end{aligned}
 \label{eq:phi_s}
\end{equation}

%************************************************
Three different pre-shear normal loads were applied in the experiment
(1,000, 2,000, and 10,000 Pa).
For each we used a normal load proportional to the initial one
(\textit{shear-perc}), increasing from stage one (40\%) to stage four (100\%)
with two escalating intermediate stages (60\% and 80\%) for a total of twelve load conditions.
Each experiment was performed on a fresh material sample. \\

\section{Jenike Shear Cell tester}
\label{sec:jsct}
%************************************************

\lipsum[1]