% !TEX encoding = UTF-8
% !TEX TS-program = pdflatex
% !TEX root = ../Tesi.tex
% !TEX spellcheck = en-EN

%************************************************
\chapter{Unresolved CFD-DEM}
\label{cap:unresolvedcfddem}
%************************************************

The existence of particles immersed in a fluid influences its behaviour. 
Keeping aside the short-scale flow field, the local averages inside the
Navier-Stokes fluid equations could account for the particles.
On the hypothesis of having \textit{g(r)} a convenient averaging function, which
is non-negative, smooth, asyntotically zero for large $r$, normalized, we can
define the local void fraction ($\varepsilon(\mathbf{r},t)$), the volume of the
fluid ($V_f(t)$) over the total volume, as:
\begin{equation}
\varepsilon(\mathbf{r},t) = \int_{V_f(t)}{d^3 r' g(|\mathbf{r} - \mathbf{r'}|)}.
 \label{eq:voidfraction}
\end{equation}

We define local mean values of fluid point properties as:
\begin{equation}
\bar{a}(\mathbf{r},t) = \frac{1}{\varepsilon(\mathbf{r},t)} \int_{V_f(t)}{d^3 r'
g(|\mathbf{r} - \mathbf{r'}|) a(\mathbf{r'},t)},
 \label{eq:a}
\end{equation}

which came from a slowly varying a rapidly fluctuating part:
\begin{equation}
\label{eq:a2}
a(\mathbf{r},t) = \bar{a}(\mathbf{r'},t) + a'(\mathbf{r},t).
\end{equation}

Particle properties, on the solid volume $V_s(t)$ and the solid fraction
$\phi(\mathbf{r},t)$, are expressed as field quantities as:
\begin{equation}
\begin{align*}
\bar{b}(\mathbf{r},t) &= \frac{1}{\phi(\mathbf{r},t)} \int_{V_f(t)}{
g(|\mathbf{r} - \mathbf{r'}|) b(\mathbf{r'},t) \dd^3 r'} \\
& \approx
\frac{1}{\phi(\mathbf{r},t)} \sum_{p}{g(\mathbf{r} - \mathbf{r}_p)
V_p b_p(t)}.
\end{align*}
 \label{eq:b}
\end{equation}

The approximation is valid when $g(r)$ varies slowly over the particles length
scale.
\begin{equation}
\begin{align*}
\nabla_{\mathbf{r}} \varepsilon (\mathbf{r},t) \bar{a}(\mathbf{r},t) 
&=
\int_{V_f(t)}{
a(\mathbf{r'}, t) \nabla_{\mathbf{r}} g(|\mathbf{r} - \mathbf{r'}| )  \dd^3 r'} \\
&=
- \int_{S_f(t)}{ a(\mathbf{S},t) g(|\mathbf{r} - \mathbf{S}|) \dd \mathbf{S} + 
\varepsilon (\mathbf{r},t) \overline{\nabla a}(\mathbf{r},t)} \\
& \approx
\sum_{p}{\int_{S_p(t)}{ a(\mathbf{S},t) g(|\mathbf{r} - \mathbf{S}|)  \dd
\mathbf{S}}} + \varepsilon (\mathbf{r},t) \overline{\nabla a}(\mathbf{r},t).
\end{align*}
 \label{eq:divepsa}
\end{equation}



\info{to be continued}